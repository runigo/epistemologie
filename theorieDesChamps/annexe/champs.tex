\chapter{Champs}


%%%%%%%%%%%%%%%%%%%%%
\section{Champ gravitationnel}
%%%%%%%%%%%%%%%%%%%%%
La force gravitationnel exercé par $m_1$ sur $m_2$ est

\[
\overrightarrow{F_{m_1/m_2}} = G . \frac{m_1.m_2}{(d_{m_1m_2})^2} . \overrightarrow{u_{m_2m_1}}
\]

On définie le champ gravitationnel créé par $m_1$ au point ou se trouve $m_2$ par

\[
\overrightarrow{g_{m_1/m_2}} = G . \frac{m_1}{(d_{m_1m_2})^2} . \overrightarrow{u_{m_2m_1}}
\]

La force gravitationnel exercé par le champ $\overrightarrow{g_{m_1/m_2}}$ sur $m_2$ est alors

\[
\overrightarrow{F_{\overrightarrow{g_{m_1/m_2}}/m_2}} = \overrightarrow{g_{m_1/m_2}} . m_2
\]




%%%%%%%%%%%%%%%%%%%%%
\section{Champ électrique}
%%%%%%%%%%%%%%%%%%%%%

La force électrique exercé par $Q_1$ sur $Q_2$ est

\[
\overrightarrow{F_{Q_1/Q_2}} = k . \frac{Q_1.Q_2}{(d_{Q_1Q_2})^2} . \overrightarrow{u_{Q_2Q_1}}
\]

On définie le champ électrique créé par $Q_1$ au point ou se trouve $Q_2$ par

\[
\overrightarrow{E_{Q_1/Q_2}} = k . \frac{Q_1}{(d_{Q_1Q_2})^2} . \overrightarrow{u_{Q_2Q_1}}
\]

La force électrique exercé par $\overrightarrow{g_{Q_1/Q_2}}$ sur $Q_2$ est alors

\[
\overrightarrow{F_{\overrightarrow{E_{Q_1/Q_2}}/Q_2}} = \overrightarrow{E_{Q_1/Q_2}} . Q_2
\]

Le champ créé par une charge électrique est à priori un outil purement mathématique, un artifice de calcul bien pratique. L'existence de ce "champ électrique" est à priori hypothétique. Néanmoins, son existence permet d'interpréter la transmission de "l'information de présence" entre les charges, de lever l'hypothèse d'une transmission d'information instantanée et immatérielle entre les charges.

%%%%%%%%%%%%%%%%%%%%%%%%%%%%%%%%%%%%%%%%%%%%%%%%%%%%%%%%%%%%%%%%%%%%%%%%%%%%
