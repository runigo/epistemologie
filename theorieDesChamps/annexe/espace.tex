
%%%%%%%%%%%%%%%%%%%%%
\chapter{Espace vectoriel}
%%%%%%%%%%%%%%%%%%%%%

%%%%%%%%%%%%%%%%%%%%%%%%%
\section{Ensemble et application}
%%%%%%%%%%%%%%%%%%%%%%%%%
%$\mathcal{}$
Un ensemble est une collection d'objets. Ces objets sont appelés éléments (a) de l'ensemble ($\mathcal{A}$) :
\[
 a \in \mathcal{A}
\]


Une application ($f$) met en relation chaque élément ($a$) d'un ensemble ($\mathcal{A}$, dit de départ) avec un élément ($b$) d'un autre ensemble ($\mathcal{B}$, dit d'arrivé) :
\begin{align*}
f :\ \ \ \ \ \ \ \ \ \mathcal{A} \ \  & \rightarrow \ \ \ \mathcal{B} \\
a \ \ & \mapsto \ \ b = f(a)
\end{align*}

Une loi de composition est une application qui associe deux éléments (éventuellement du même ensemble) à un troisième élément. 
\begin{align*}
f :\ \ \ \ \ \ \ \ \ \mathcal{A} \times \mathcal{B} \ \  & \rightarrow \ \ \ \mathcal{C} \\
(a,b) \ \ & \mapsto \ \ c = f(a,b)
\end{align*}

Une loi de composition est dite interne si $\mathcal{A} = \mathcal{B} = \mathcal{C}$, externe sinon.



%%%%%%%%%%%%%%%%%%%%%%%%%
\section{Espace vectoriel}
%%%%%%%%%%%%%%%%%%%%%%%%%
%
Un espace vectoriel est un ensemble (ses éléments sont appelés vecteur), possédant une loi de composition interne (la somme de deux vecteurs d'un espace vectoriel appartient à cet espace) et une loi de composition externe (la multiplication par un scalaire d'un vecteur d'un espace vectoriel appartient à cet espace).


%%%%%%%%%%%%%%%%%%%%%%%%%%%%%%%%%%%%%%%%%%%%%%%%%%%%%%%%%%%%%%%%%%%%%%%%%%%%%%%%%%%%%
