
%%%%%%%%%%%%%%%%%%%%%
\chapter{Transformation de fourier}
%%%%%%%%%%%%%%%%%%%%%

%%%%%%%%%%%%%%%%%%%%%%%%%
\section{Série de fourier}
%%%%%%%%%%%%%%%%%%%%%%%%%
%
Une fonction périodique (de période $T$) est égale à une somme discrète de sinusoïde :
\[
f_T(x)=a_0 + \sum_{n=1}^\infty \left( a_n \cos \frac{2 n \pi x}{T} + b_n \sin \frac{2 n \pi x}{T} \right)
\]
$a_n$ et $b_n$ sont les coefficient de fourier de $f_T(x)$.

%%%%%%%%%%%%%%%%%%%%%%%%%
\section{Transformé de fourier}
%%%%%%%%%%%%%%%%%%%%%%%%%
%
Une fonction quelconque est égale à une somme continue de sinusoïde :
\[
f(x) = \int_{-\infty}^\infty e^{2 i \pi \nu x}\widehat{f}(\nu) d\nu
\]
$\widehat{f}(\nu)$ est la transformé de fourier de $f(x)$.

%%%%%%%%%%%%%%%%%%%%%%%%%%%%%%%%%%%%%%%%%%%%%%%%%%%%%%%%%%%%%%%%%%%%%%%%%%%%%%%%%%%%%
