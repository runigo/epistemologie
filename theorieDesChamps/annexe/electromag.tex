
%%%%%%%%%%%%%%%%%%%%%
\chapter{Électromagnétisme}
%%%%%%%%%%%%%%%%%%%%%

{\footnotesize
D'après https://glq2200.clberube.org/chapitres/docs/mag-theorie-maxwell
}

\section{Loi de Faraday}
Un champ électrique (courant) est induit le long de boucles qui entourent la surface traversée par un champ magnétique variable dans le temps.

Forme différentielle:

\[
\nabla \times \boldsymbol{E} = - \frac{\partial{\boldsymbol{B}}}{\partial{t}}\]

Forme intégrale:

\[
\oint \boldsymbol{E} \cdot \mathrm{d}\boldsymbol{l} = -\frac{\mathrm{d}}{\mathrm{d}t} \iint_S \boldsymbol{B}\cdot \mathrm{d}\boldsymbol{S}
\]

\section{Loi d’Ampère}
Un courant qui perce une surface génère un champ magnétique qui tourne autour de celle-ci.

Forme différentielle :

\[
\nabla \times \boldsymbol{B} = \mu_0 \left( \boldsymbol{J} + \epsilon_0 \frac{\partial{\boldsymbol{E}}}{\partial{t}} \right)
\]


Forme intégrale :
\[\oint \boldsymbol{B} \cdot \mathrm{d}\boldsymbol{l} = \mu_0 \iint_S \boldsymbol{J}\cdot\mathrm{d}\boldsymbol{S} + \mu_0 \epsilon_0 \frac{\mathrm{d}}{\mathrm{d}t} \iint_S \boldsymbol{E}\cdot \mathrm{d}\boldsymbol{S}\]
Version macroscopique :
\[\oint \boldsymbol{H} \cdot \mathrm{d}\boldsymbol{l} = \iint_S \left( \boldsymbol{J} + \frac{\partial\boldsymbol{D}}{\partial t} \right) \cdot \mathrm{d}\boldsymbol{S}\]

\section{Loi de Gauss}
Le champ électrique qui sort d’une surface fermée est proportionnel à la charge contenue dans le volume.

Forme différentielle :
\[\nabla \cdot \boldsymbol{E} = \frac{\rho}{\epsilon_0}\]
Forme intégrale :
\[\iint_{S} \boldsymbol{E} \cdot \mathrm{d}\boldsymbol{S} = \frac{1}{\epsilon_0} \iiint_V \rho \,\mathrm{d}V\]
Version macroscopique :
\[\iint_{S} \boldsymbol{D} \cdot \mathrm{d}\boldsymbol{S} = \iiint_V \rho_f \,\mathrm{d}V\]

\section{Loi de Gauss pour le magnétisme}
Le champ magnétique qui sort d’une surface fermée est toujours nul. Il n’y a pas de «charges» magnétiques dans le volume ainsi formé (les monopôles magnétiques n’existent pas).

Forme différentielle :
\[\nabla \cdot \boldsymbol{B} = 0\]
Forme intégrale :
\[\iint_{S} \boldsymbol{B} \cdot\mathrm{d}\boldsymbol{S} = 0\]


%%%%%%%%%%%%%%%%%%%%%%%%%
%\section{Équation de Maxwell}
%%%%%%%%%%%%%%%%%%%%%%%%%
%
%%%%%%%%%%%%%%%%%%%%%%%%%
%\section{}
%%%%%%%%%%%%%%%%%%%%%%%%%
%
%%%%%%%%%%%%%%%%%%%%%%%%%%%%%%%%%%%%%%%%%%%%%%%%%%%%%%%%%%%%%%%%%%%%%%%%%%%%%%%%%%%%%
