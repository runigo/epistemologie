
%%%%%%%%%%%%%%%%%%%%%
\chapter{Relativité}
%%%%%%%%%%%%%%%%%%%%%

%%%%%%%%%%%%%%%%%%%%%%%%%
\section{Transformation de Lorentz}
%%%%%%%%%%%%%%%%%%%%%%%%%
%
$R$ et $R'$ étant deux référentiels galiléen, $V$ étant la vitesse de déplacement de $R'$ dans $R$ (Le déplacement ayant lieu suivant la direction $Ox$), un évènement de coordonnées $(x,y,z,t)$ dans $R$ a pour coordonnées dans $R'$ :
\begin{align*}
x' & = \frac{x - V.t}{\sqrt{1-V^2/c^2}} \\
y' & = y \\
z' & = z \\
t' & = \frac{t - V.x/c^2}{\sqrt{1-V^2/c^2}} \\
\end{align*}

%%%%%%%%%%%%%%%%%%%%%%%%%
\section{Interprétation relativiste du champ magnétique}
%%%%%%%%%%%%%%%%%%%%%%%%%
On considère deux fils (1 et 2) rectiligne et parrallèle parcourues par les courants $i_1$ et $i_2$.

Pour les électrons constituants $i_1$ (appelés $\{e^-_1\}$), les électrons constituants $i_2$ (appelés $\{e^-_2\}$) sont en mouvement par rapports au protons (appelés $\{p^+_2\}$) du fil 2.

$\{e^-_2\}$ et $\{p^+_2\}$, étant en mouvement par rapport au fil 1, ils subissent une {\it contraction des longueurs}. Leurs vitesses étant différentes, leurs {\it contractions des longueurs} sont différentes. De cette différence de {\it contraction de longueurs}, le fil 1 semble voir le fil 2 comme possédant une densité de charge non nulle.

Il en résulte que $\{e^-_1\}$ voient apparaître une densité de charge différente de zéro dans le fil 2. Il en est de même pour $\{p^+_1\}$.

Il en résulte une force "électrique-relativiste" entre les deux fils, que l'on identifie à la force magnétique.
%
%%%%%%%%%%%%%%%%%%%%%%%%%%%%%%%%%%%%%%%%%%%%%%%%%%%%%%%%%%%%%%%%%%%%%%%%%%%%%%%%%%%%%
