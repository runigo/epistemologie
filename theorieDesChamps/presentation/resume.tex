\begin{center}
\Large
Introduction
\normalsize
\end{center}
\vspace{3cm}

La physique décrit le comportement de la matière. Pour cela, elle mesure des grandeurs (masse, vitesse, température, ...), et énonce des lois mathématiques reliant ces grandeurs.

Historiquement, ces lois ont évolué, de nouvelles lois remplaçant les anciennes. Également, de nouvelles grandeurs ont été inventées (énergie, entropie, fonction d'onde, ...).

Les champs sont des grandeurs physique qui ont montrés leur pertinence afin de décrire la matière. Les champs sont des objets mathématiques permettant une expression élégante des lois. Les lois des théories des champs décrivent l'évolution spatio-temporel des champs.

\vspace{3cm}

L'objectif de ce présent document est d'ammener le lecteur à une vision élémentaire des théories des champs. Le parti pris pour cela a été de se limiter à quelques exemples significatifs
% afin de préciser la notion de champs
% et de limiter les aspects mathématiques.
% et de privilégier les schémas.

Les premiers chapitres montrent à travers quelques exemples comment les champs se sont introduit en physique et permettent une première approche de la notion de champs.

Les chapitres suivants abordent de manière élémentaire quelques aspects des théories des champs.

%tente d'ammener le lecteur à une initiation à la généalogie de la notion de champs en sciences p

