\chapter{Forces à distance}
%
Une force modélise une action mécanique : un homme pousse un chariot, le chariot se met en mouvement. L'action est de pousser, c'est une action de contact, son effet est la mise en mouvement. Cette action peut être modélisée par une force : l'homme applique une force sur le chariot.

\begin{center}
\includegraphics[scale=0.6]{./forces/chariotPousse}
\end{center}

Une force est représentée par un vecteur et un point d'application : Le point d'application est le point de contact (A), la force est horizontal, vers la droite et possède une valeur (mesurée en newton).

\begin{center}
\includegraphics[scale=0.6]{./forces/chariotVecteur}
\end{center}

Nous étudions dans les paragraphes suivants, trois forces à distance (il n'y a pas de contact entre les corps).


%%%%%%%%%%%%%%%%%%%%%
\section{Champ gravitationnel}
%%%%%%%%%%%%%%%%%%%%%
%
%\subsection{Définition}


%%%%%%%%%%%%%%%%%%%%%%%%%%%%%%%%%%%%%%%%%%%%%%%%%%%%%%%%%%%%%%%%%%%%%%%%%%%%

%

%%%%%%%%%%%%%%%%%%%%%
\section{Champ électrostatique}
%%%%%%%%%%%%%%%%%%%%%
%
\subsection{Modèle}
Le champ électrostatique est un intermédiaire entre la présence d'une charge électrique $Q_1$ et la force exercé par celle-ci sur une charge électrique $Q_2$ :
%L'expérience du pendule électrostatique peut se modéliser par des {\it forces électrostatiques} s'exerçant entre les particules chargées. Ce modèle suppose une {\it action à distance}.
%La notion de champs permet de modéliser cette action entre les charges électriques : une . Le champ exerce une force sur les charges électriques.

% Dans un second temps, elle peut se modéliser par  en disant qu'une particule chargée crée un champ électrostatique en tout point de l'espace et qu'une particule chargée placé dans un champ électrostatique subit une force.

\begin{itemize}[leftmargin=2.3cm, label=\ding{32}, itemsep=5pt]
\item La charge électrique $Q_1$ crée un champ électrique dans tout l'espace.
\item Le champ électrique exerce une force sur la charge électrique $Q_2$.
\end{itemize}

\subsection{Représentation mathématique}
Une charge électrique $Q_1$ crée un champ électrique $\overrightarrow{E}$ dans tout l'espace. 

\begin{center}

\tikzstyle arrowstyle=[scale=1] %taille des flèches
\tikzstyle directed=[postaction={decorate,decoration={markings,
mark=at position .65 with {\arrow[arrowstyle]{stealth}}}}]%définition d'un trait avec une flèche en son milieu

\begin{minipage}[c]{.45\linewidth}
\begin{center}
Représentation des lignes de champs : le champ électrique est un vecteur ($\overrightarrow{E}$) qui dépend de la position par rapport à la charge électrique.
\end{center}
\end{minipage}
\hfill
\begin{minipage}[c]{.45\linewidth}
\begin{tikzpicture}
\foreach \t in {30,60, ...,360} %boucle
\draw[red,thick, directed] (0pt,0pt) -- (\t:3cm);%trait tracé en coordonnées polaires
\fill circle (1mm) node [below right] {\colorbox{white}{+q}};%point central
\end{tikzpicture}
\end{minipage}


\setlength{\unitlength}{1cm}
\begin{minipage}[c]{.45\linewidth}
\begin{picture}(10,3)
\put(0.5,1.0){\circle{0.3}}
\put(0.3,0.3){$Q_1$}
\put(5.5,1.0){\vector(-1,0){1.36}}
\put(3.7,1.3){$\overrightarrow{E}$}
\end{picture}
\end{minipage}
\hfill
\begin{minipage}[c]{.45\linewidth}
\begin{center}
\end{center}
\end{minipage}

\end{center}

Le champ électrique $\overrightarrow{E}$ exerce une force $\overrightarrow{F}_{\overrightarrow{E}/Q_2}$ sur la charge électrique $Q_2$

\setlength{\unitlength}{1cm}
\begin{picture}(10,3)
\put(5.5,1.0){\circle{0.5}}
\put(5.3,0.2){$Q_2$}
\put(5.5,1.0){\vector(-1,0){1.36}}
\put(3.7,1.3){$\overrightarrow{F}_{\overrightarrow{E}/Q_2}$}
\end{picture}

Le champ créé par une charge électrique est à priori un outil purement mathématique, un artifice de calcul bien pratique. L'existence de ce "champ électrique" est à priori hypothétique. Néanmoins, son existence permet d'interpréter la transmission de "l'information de présence" entre les charges, de lever l'hypothèse d'une transmission d'information instantanée et immatérielle entre les charges.

%%%%%%%%%%%%%%%%%%%%%%%%%%%%%%%%%%%%%%%%%%%%%%%%%%%%%%%%%%%%%%%%%%%%%%%%%%%%

%

%%%%%%%%%%%%%%%%%%%%%
\section{Force magnétique}
%%%%%%%%%%%%%%%%%%%%%
%

La force magnétique est la force qui s'exerce entre les \textbf{\textit {aimants}}.
Les aimants sont des matériaux possédant des propriétés magnétiques.
Certain métaux peuvent être aimantés par la proximité d'un aimant.

\subsection{Pôles magnétiques}

Un aimant est toujours {\it orienté}, il possède un pôle dit \textbf{\textit {nord}} et un pôle
dit \textbf{\textit {sud}}.


\subsection{Action magnétique}

La force magnétique entre deux aimants est attractive et répulsive, elle exerce un {\it couple} :
entre deux aimants, les pôles opposés s'attirent, les pôles identiques se repoussent. 

Ci-dessous, deux aimants sont schématisés, les pôles nord en rouge et les pôles sud en bleu. On ne représente que les forces que l'aimant 2 exerce sur l'aimant 1.

\begin{center}
\begin{tikzpicture}[scale=1]
\fill [blue] (0,0) rectangle (0.3,1.5);
\fill [red] (0,1.5) rectangle (0.3,3);
\begin{scope}[rotate=30,yshift=-4.2cm]%
\fill [red] (8,0) rectangle (8.3,1.5);
\fill [blue] (8,1.5) rectangle (8.3,3);
\end{scope}
\thicklines
\put(0.15,0.15){\vector(1,0){1.76}}
\put(2,0.3){$\overrightarrow{F}_{N_2/S_1}$}

\put(0.15,2.85){\vector(1,0){2.26}}
\put(2.6,3){$\overrightarrow{F}_{S_2/N_1}$}

\draw [thick] [->] ((0.15,(0.15) --++(-1.5,-0.7) node [left] {$\overrightarrow{F}_{N_2/N_1}$};

\draw [thick] [->] ((0.15,(2.85) --++(-0.8,0.3) node [left] {$\overrightarrow{F}_{S_2/S_1}$};

\draw node at (0.3,-0.9) {aimant 1};
\draw node at (8.3,-0.9) {aimant 2};
\end{tikzpicture}
\end{center}
%%%%%%%%%%%%%%%%%%%%%%%%%%%%%%%%%%%%%%%%%%%%%%%%%%%%%%%%%%%%%%%%%%%%%%%%%%%%

%
%\input{.//.tex}
