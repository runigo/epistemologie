\section{Action mécanique}
%
Un homme pousse un chariot, le chariot se met en mouvement. L'homme exerce une action mécanique, c'est une action dite {\it de contact} (il y a un contact entre l'acteur de l'action et le receveur), son effet est la mise en mouvement du chariot.

\begin{center}
\includegraphics[scale=0.6]{./forces/chariotPousse}
\end{center}

Une action mécanique est modélisée par une force. La force est un ensemble de caractéristique (intensité, direction, sens, point d'application), elle est représentée par un vecteur {\it partant} du point d'application.

\begin{center}
\includegraphics[scale=0.6]{./forces/chariotPousseForce}

$\overrightarrow{F_{H/C}}$ : Force exercée par l'Homme sur le Chariot.
\end{center}

Certaines actions mécaniques se produisent sans contact mécanique entre l'acteur et le receveur. Elles sont dites {\it à distance}. Les paragraphes suivants donnent des exemples.

