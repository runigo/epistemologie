
%%%%%%%%%%%%%%%%%%%%%
\section{Force gravitationnelle}
%%%%%%%%%%%%%%%%%%%%%
%

L'interaction gravitationnelle est une action à distance entre les corps \textbf{\textit {massique}} : les corps ayant une
\textbf{\textit {masse}} s'attirent entre eux, la force gravitationnelle modélise l'interaction gravitationnelle. 

La masse est toujours positive, la force gravitationnelle est toujours attractive.

La terre exerce sur la lune une action mécanique dont l'effet est d'{\it incurver} sa trajectoire, sans cette action la lune s'éloignerait de la terre.

\setlength{\unitlength}{1cm}
%
\begin{center}
\mbox{%\fbox{
\begin{picture}(17,3)
\put(2,2){\circle{1.52}}
\put(1.6,0.8){Terre}
\thicklines
\put(2,2){\vector(1,0){2.76}}
\put(4.3,2.3){$\overrightarrow{F}_{L/T}$}
\put(15,2){\circle{0.5}}
\put(14.6,1.2){Lune}
\thicklines
\put(15,2){\vector(-1,0){2.76}}
\put(12,2.3){$\overrightarrow{F}_{T/L}$}
\end{picture}}

$\overrightarrow{F_{L/T}}$ : Force gravitationnelle exercée par la lune sur la terre,

 $\overrightarrow{F_{T/L}}$ : Force gravitationnelle exercée par la terre sur la lune.
\end{center}


%%%%%%%%%%%%%%%%%%%%%%%%%%%%%%%%%%%%%%%%%%%%%%%%%%%%%%%%%%%%%%%%%%%%%%%%%%%%
