
%%%%%%%%%%%%%%%%%%%%%
\section{Force électrique}
%%%%%%%%%%%%%%%%%%%%%
%
\subsection{Constitution de la matière}
Au niveau microscopique, la matière est constituée de particules : électrons, protons et neutrons. Les électrons ont une charge électrique négative, les protons ont une charge électrique positive, les neutrons sont électriquement neutre.

\subsection{Interaction électrique}
L'interaction électrique s'exerçe entre les corps \textbf{\textit {chargées électriquement}}.
%Les \textbf{\textit {charges électriques}} peuvent être positive ou négative.
%La loi de Coulomb indique que des corps chargés électriquement exercent entre eux des forces. 
Des charges électriques de même signe se repoussent, des charges de signes opposées s'attirent.

Cette interaction permet la cohésion de la matière : formation des atomes, des molécules, puis des objets macroscopique.

L'interaction électrique est mise en évidence lors des expérience de triboélectricité (électrisation par frottement).

\subsection{Force de Coulomb}
La force de Coulomb modélise l'interaction électrique entre deux corps chargés électriquement (appelés charge électrique $Q_1$ et charge électrique $Q_2$).
%le sens de cette force dépend du signe de ces charges (positive ou négative).


La charge électrique $Q_1$ exerce la force $\overrightarrow{F}_{Q_1/Q_2}$ sur la charge électrique $Q_2$.
La charge électrique $Q_2$ exerce la force $\overrightarrow{F}_{Q_2/Q_1}$ sur la charge électrique $Q_1$.
%respecte le principe de réaction, la charge électrique $Q_2$ exerce une force de Coulomb $\overrightarrow{F}_{Q_2/Q_1}$ sur la charge électrique $Q_1$.

Si les charges $Q_2$ et $Q_1$ sont de signes opposées,
\begin{center}
\setlength{\unitlength}{1cm}
\begin{picture}(10,3)
\put(0.5,1.0){\circle{0.3}}
\put(0.3,0.3){$Q_1$}
\put(0.5,1.0){\vector(1,0){2.16}}
\put(2,1.3){$\overrightarrow{F}_{Q_2/Q_1}$}
\put(9.5,1.0){\circle{0.5}}
\put(9.3,0.2){$Q_2$}
\put(9.5,1.0){\vector(-1,0){2.16}}
\put(6.9,1.3){$\overrightarrow{F}_{Q_1/Q_2}$}
\end{picture}
\end{center}

Si les charges $Q_2$ et $Q_1$ sont de même signe,

\begin{center}
\setlength{\unitlength}{1cm}
\begin{picture}(10,3)
\put(0.5,1.0){\circle{0.3}}
\put(0.3,0.3){$Q_1$}
\put(0.5,1.0){\vector(-1,0){2.16}}
\put(-2.2,1.3){$\overrightarrow{F}_{Q_2/Q_1}$}
\put(9.5,1.0){\circle{0.5}}
\put(9.3,0.2){$Q_2$}
\put(9.5,1.0){\vector(1,0){2.16}}
\put(11,1.3){$\overrightarrow{F}_{Q_1/Q_2}$}
\end{picture}
\end{center}
%On peut alors se demander comment l'information de la présence de $Q_1$ parvient à $Q_2$, y a-t-il quelque chose qui se propage entre les charges ? Cette question peut être simplifiée en disant que les charges créent un champ dans tout l'espace et qu'elle sont sensibles à ce champ.
%\end{minipage}\hfill\begin{minipage}[c]{.45\linewidth}

%%%%%%%%%%%%%%%%%%%%%%%%%%%%%%%%%%%%%%%%%%%%%%%%%%%%%%%%%%%%%%%%%%%%%%%%%%%%
