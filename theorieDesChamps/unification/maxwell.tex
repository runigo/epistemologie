
%%%%%%%%%%%%%%%%%%%%%
\section{Unification électromagnétique}
%%%%%%%%%%%%%%%%%%%%%
%
\textsc{Oersted}, 1820 : Un courant électrique crée un champ magnétique.

\textsc{Faraday}, 1831 : Un aimant en mouvement crée un champ électrique.

\textsc{Maxwell} 1864 : Un champ magnétique variable au cours du temps crée un champ électrique, un champ électrique variable au cours du temps crée un champ magnétique.

\subsection{Champ électrique et champ magnétique}

Au début du XIX$^\text{ème}$ siècle, les expériences de Hans Christian Oersted montrent un lien intime entre les forces magnétiques et les forces électriques.

Les équations de maxwell vont établir un lien entre les champs magnétique et électrique : non seuleument ces champs sont créés par les charges électrique, les aimants et les charges en mouvement, mais un champ électrique est créé par un champ magnétique variable au cours du temps et un champ magnétique est créé par un champ électrique variable au cours du temps.

%\begin{center}
\[
\overrightarrow{E} \text{ variable au cours du temps } \Rightarrow \text{ création de } \overrightarrow{B}
\]
\[
\overrightarrow{B} \text{ variable au cours du temps } \Rightarrow \text{ création de } \overrightarrow{E}
\]
%\end{center}

%\vspace{0.2cm}
%%%%%%%%%%%%%%%%%%%%%%%%%%%%%%%%%%%%%%%%%%%%%%%%%%%%%%%%%%%%%%%%%%%%%%%%%%%%%


%%%%%%%%%%%%%%%%%%%%%%%%%%%%%%%%%%%%%%%%%%%%%%%%%%%%%%%%%%%%%%%%%%%%%%%%
