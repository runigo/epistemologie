
%%%%%%%%%%%%%%%%%%%%%
\section{Champ gravitationnel}
%%%%%%%%%%%%%%%%%%%%%
%
Le champ gravitationnel est défini comme un {\it intermédiaire de calcul} entre la présence d'une masse (p.e. la terre) et la force exercé par celle-ci sur une autre masse (p.e. la pomme).

\subsection{Vision "force de Newton"}

\begin{minipage}[c]{.45\linewidth}
\begin{center}
\includegraphics[scale=.6]{./champs/pomme2}
\end{center}
\end{minipage}
\hfill
\begin{minipage}[c]{.45\linewidth}
La pomme est soumise à l'attraction gravitationnelle de la terre. Cette attraction gravitationnelle est modélisée par une action mécanique (le poids), celle-ci est représentés par le vecteur $\overrightarrow{P}$.
\end{minipage}

\vfill
\subsection{Vision "champ gravitationnel"}

\begin{minipage}[c]{.45\linewidth}
La terre crée un champ de gravitation dans tout l'espace. C'est un champ vectoriel, aussi appelé accélération de la pesanteur, représenté par le vecteur $\overrightarrow{g}$.
\end{minipage}
\hfill
\begin{minipage}[c]{.45\linewidth}
\begin{center}
\includegraphics[scale=.6]{./champs/pommierG2}
\end{center}
\end{minipage}

\vfill

\begin{minipage}[c]{.45\linewidth}
\begin{center}
\includegraphics[scale=.6]{./champs/pommeG2}
\end{center}
\end{minipage}
\hfill
\begin{minipage}[c]{.45\linewidth}
La pomme est soumise à l'accélération de la pesanteur (le champ $\overrightarrow{g}$). L'action de ce champ est modélisée par une action mécanique (le poids), le poids est représenté par le vecteur $\overrightarrow{P}$.
\end{minipage}

\vfill
\newpage


%%%%%%%%%%%%%%%%%%%%%%%%%%%%%%%%%%%%%%%%%%%%%%%%%%%%%%%%%%%%%%%%%%%%%%%%%%%%
