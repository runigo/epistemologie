
%%%%%%%%%%%%%%%%%%%%%
\section{Champs quantiques}
%%%%%%%%%%%%%%%%%%%%%
%
Une particule élémentaire est décrite par une fonction d'onde. Un ensemble de particules élémentaires identiques sont décrites par la superposition, la somme, des fonctions d'ondes individuelles. Ce champ total, pour un type de particule est à même de décrire l'ensemble de ces particules.

Ainsi, l'ensemble des électrons est décrit par le champ électron, l'ensemble des photons est décrit par le champ photon. Ces deux champs sont présent dans l'espace temps, la détection d'un photon ou d'un électron est due à un échange élémentaire d'énergie entre ces deux champs.

Lorsqu'un photon se matérialise en une paire électron-positron : un transfert d'énergie s'oppère du champ photon vers le champ électron. Lorsqu'un électron rencontre un positron, ils s'anihilent donnant un photon :  un transfert d'énergie s'oppère du champ électron vers le champ photon.

Une particule élémentaire peut alors être considérée comme l'évènement : transfert d'énergie entre deux champs quantique, la matière classique ne serait donc que la manifestation de ces évenements.

Les évènement de transfert d'énergie entre deux champs quantique caractérise le \textsf{\textit {couplage}} entre les champs.

L'existence (ou la détection) de la force électrique entre un électron et un proton serait la signature du couplage entre le champ électron et le champ photon d'une part et du couplage entre le champ proton et le champ photon. Le champ photon est le médiateur de la force électrique entre les particules chargées.
%%%%%%%%%%%%%%%%%%%%%%%%%%%%%%%%%%%%%%%%%%%%%%%%%%%%%%%%%%%%%%%%%%%%%%%%
