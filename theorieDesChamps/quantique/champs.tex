
%%%%%%%%%%%%%%%%%%%%%
\section{Champs quantiques}
%%%%%%%%%%%%%%%%%%%%%
%
Une particule élémentaire est décrite par une fonction d'onde. Un ensemble de particules élémentaires identiques sont décrites par la superposition, la somme, des fonctions d'ondes individuelles. Ce champ total, pour un type de particule est à même de décrire l'ensemble de ces particules.

Ainsi, l'ensemble des électrons est décrit par le champ électron, l'ensemble des photons est décrit par le champ photon. Ces deux champs sont présent dans l'espace temps, la détection d'un photon ou d'un électron est due à un échange élémentaire d'énergie entre ces deux champs.

Lorsqu'un photon se matérialise en une paire électron-positron, le champ photon fournit de l'énergie au champ électron. Lorsqu'un électron rencontre un positron, ils s'anihilent donnant un photon : le champ électron fournit de l'énergie au champ photon.

Une particule élémentaire peut alors être considérée comme l'évènement : échange d'énergie entre deux champs quantique, la matière classique n'est donc que la manifestation de ces évenements.
%%%%%%%%%%%%%%%%%%%%%%%%%%%%%%%%%%%%%%%%%%%%%%%%%%%%%%%%%%%%%%%%%%%%%%%%
