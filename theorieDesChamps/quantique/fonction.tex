
%%%%%%%%%%%%%%%%%%%%%
\section{Fonction d'onde}
%%%%%%%%%%%%%%%%%%%%%
%
Dans la théorie quantique, les particules sont décrite par leur fonction d'onde.
La fonction d'onde est relié à la probabilité de détecter la particule en un point donné.
Il s'agit donc d'un champ, attribuant une "valeur" en chaque point de l'espace.
{\footnotesize Cette valeur est un nombre complexe, la probabilité de trouver la particule en un point est égale au module de ce nombre mis au carré}.
Le schéma suivant représente le champ de probabilité de présence de l'électron dans un atome d'hydrogène selon son niveau d'énergie.

\begin{center}
\includegraphics[scale=0.7]{./quantique/hydrogene}
\end{center}


%%%%%%%%%%%%%%%%%%%%%%%%%%%%%%%%%%%%%%%%%%%%%%%%%%%%%%%%%%%%%%%%%%%%%%%%
