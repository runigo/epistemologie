
%%%%%%%%%%%%%%%%%%%%%
\section{Pente et altitude}
%%%%%%%%%%%%%%%%%%%%%
%
Afin de représenter le relief sur une carte en 2 dimension, les lignes de niveaux sont habituellement utilisées. Il existe néanmoins une autre représentation : le champs de pente. Il s'agit d'un champ vectoriel, en chaque point de la carte, on représente la pente par un vecteur dirigé dans le sens de la pente et dont la longueur est proportionnel à la pente (d'autant plus grande que la pente est grande)

\vspace{0.97cm}
\begin{minipage}{0.45\textwidth}
Considérons le relief ci-contre, constitué de deux collines, représenté en coupe. La première colline, de sommet B possède un coté très pentu (point A). La seconde est moins haute (sommet F). Entre les deux, le col se trouve en D.
\end{minipage}
\hfill
\begin{minipage}{0.45\textwidth}
\begin{center}
\includegraphics[scale=0.7]{./potentiel/collines}
\end{center}
\end{minipage}

\vspace{0.97cm}
\begin{minipage}{0.45\textwidth}
\begin{center}
\includegraphics[scale=0.7]{./potentiel/potentiel}
\end{center}
\end{minipage}
\hfill
\begin{minipage}{0.45\textwidth}
On représente ici les deux collines grâce à des lignes de niveau. Ces lignes sont plus proche les unes des autres au point A.
\end{minipage}

\vspace{0.97cm}
\begin{minipage}{0.45\textwidth}
Enfin on représente ici le "champ de pente", au point A la pente est forte, aux points B, D et F la pente est nulle.
\end{minipage}
\hfill
\begin{minipage}{0.45\textwidth}
\begin{center}
\includegraphics[scale=0.7]{./potentiel/vecteur}
\end{center}
\end{minipage}

\vspace{0.97cm}
On remarque alors que le vecteur "champs de pente" est perpendiculaire au ligne de niveau du champs "d'altitude".

%%%%%%%%%%%%%%%%%%%%%%%%%%%%%%%%%%%%%%%%%%%%%%%%%%%%%%%%%%%%%%%%%%%%%%%%%%%%
