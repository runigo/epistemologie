Un train se trouve dans un tunnel. On observe que les extrémité de ce train se trouvent au même moment aux deux extrémités du tunnel
	schéma
On en déduit que ce train possède la même longueur que ce tunnel.

Considérons à présent le train roulant à très grande vitesse et les deux évènements : 
l'avant du train se trouve à la sortie E2
l'arrière du train se trouve à l'entrée E1

Étant donné la contraction des longueurs, le train semble plus court que le tunnel, et dans le référentiel du tunnel E1 a lieu avant E2.

Étant donné la contraction des longueurs, le tunnel semble plus court que le train, et dans le référentiel du train E2 a lieu avant E1.
