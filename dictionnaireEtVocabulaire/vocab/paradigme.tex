\section{Paradigme}
\subsection{Kuhn}


\subsubsection{Sciences normale}
Recherche fermement accréditée par une ou plusieurs découverte scientifiques passées, découvertes que tel groupe de scientifique considère comme suffisantes pour fournir le point de départ d'autres travaux.


\subsubsection{Paradigme}
Découvertes qui ont en commun deux caractéristiques :

\begin{itemize}[leftmargin=2cm, label=\ding{32}, itemsep=10pt]
\item Découvertes suffisamment remarquables pour soustraire un groupe cohérent d'adeptes à d'autres formes d'activité scientifique concurrentes.
\item Ouvrent des perspectives suffisamment vastes pour fournir à ce nouveau groupe de chercheurs toutes sortes de problèmes à résoudre.
\end{itemize}


\subsection{Cuvilliers}


{\bf Paradigme} — [G. {\it paradeigma}, exemple] — {\bf 1.} (En grammaire).
Modèle, exemple type : « {\it Aimer} est le paradigme des verbes du premier
groupe. »

— {\textsf{\textit {Méta.}} {\bf 2.} Type exemplaire. {\it Chez Platon}, appliqué à
l’Idée$^1$ « Peut-être se trouve-t-il au ciel un paradigme pour celui qui,
l'ayant aperçu, veut s'y établir » ({\it République}, 592 b). Cf. {\it
Exemplarisme}*.

