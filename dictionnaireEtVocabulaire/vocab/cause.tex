\section{Les mots}
\subsection{Cause}

\begin{itemize}[leftmargin=2cm, label=\ding{32}, itemsep=20pt]
\item Usage : les pluies causent les inondations
\item Champ : responsable, coupable
\item Abstraction : cause {\bf et} effet. {\it Corrélation, habitude, influence, lien}
\item Formulation/équation : la cause a lieu {\bf avant} l'effet, temps et temporalité.
\end{itemize}

\begin{itemize}[leftmargin=2cm, label=\ding{32}, itemsep=10pt]
\item Usage = sens commun.
\item Champ lexical = discipline, section.
\item Abstraction = monde des idées.
\item Formulation/équation = lien physique.
\end{itemize}

\subsection{Energie}

\begin{itemize}[leftmargin=2cm, label=\ding{32}, itemsep=20pt]%\end{itemize}
\item Usage : Je n'ai plus d'énergie pour discuter.
\item Champ : Les énergies renouvelables sont écologiques.
\item Abstraction : Énergie {\bf et} entropie.
\item Formulation/équation : Dans un système isolé, l'énergie se conserve, l'entropie augmente.
\end{itemize}

Au son sens commun, l'énergie est synonyme de force. Au sens physique, l'énergie n'est pas une force.
