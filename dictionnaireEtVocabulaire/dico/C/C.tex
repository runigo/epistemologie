\chapter{C}
%
\section{Causalité}

Relation entre des phénomènes : deux phénomènes sont relié par
la causalité si l'un est la cause de l'autre.

Chez Hume, la croyance en une causalité est provoqué par l'habitude de voir deux phénomènes, l'un se produisant {\it systématiquement} avant l'autre. Le premier phénomène apparaissant est alors appelé {\it cause}, le second {\it effet}.

{\footnotesize
\begin{itemize}[leftmargin=1cm, label=\ding{32}, itemsep=1pt]
\item {\bf \textsc{Étymologie} :} latin {\it causa}, cause et
procés. latin {\it effectus}, résultat, effet, de {\it facere}, faire.
\item {\bf \textsc{Corrélats} :} déterminisme, temps.
\end{itemize}
}

%
\section{Causalité (principe de {\bf --})}

La cause précède l'effet.

Nous ne connaissons pas d'expérience remettant en cause ce
principe, la physique quantique décrit un univers causal.
{\footnotesize
\begin{itemize}[leftmargin=1cm, label=\ding{32}, itemsep=1pt]
\item {\bf \textsc{Corrélats} :} déterminisme, temps.
\end{itemize}
}

%
\section{Corpuscule}

petit grain de matière.
{\footnotesize
\begin{itemize}[leftmargin=1cm, label=\ding{32}, itemsep=1pt]
\item {\bf \textsc{Étymologie} :} latin {\it corpusculum}, de {\it corpus}, corps législatif, politique.
% 1495, J. de Vignay.
\item {\bf \textsc{Terme voisin} :} atome.
\item {\bf \textsc{Terme opposé} :} onde.
\item {\bf \textsc{Corrélats} :} substance.
\end{itemize}
}

