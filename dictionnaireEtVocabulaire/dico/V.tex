\chapter{V}
%%%%%%%%%%%%%%%%%%%
\section{Vibration}
%%%%%%%%%%%%%%%%%%%
Mouvement périodique.
{\footnotesize
\begin{itemize}[leftmargin=1cm, label=\ding{32}, itemsep=1pt]
\item {\bf \textsc{Étymologie} :} latin {\it vibrare}, agiter, brandir.
\item {\bf \textsc{Corrélats} : ondulatoire} .
\end{itemize}
}

%%%%%%%%%%%%%%%%%%%
\section{Vitesse}
%%%%%%%%%%%%%%%%%%%
{\footnotesize
\begin{itemize}[leftmargin=1cm, label=\ding{32}, itemsep=1pt]
\item {\bf \textsc{Étymologie} :} latin {\it },.
\item {\bf \textsc{Sens ordinaire} :} .
\item {\bf \textsc{Terme voisin} :} .
\item {\bf \textsc{Terme opposé} :} .
\item {\bf \textsc{Corrélats} :} .
\end{itemize}
}

%%%%%%%%%%%%%%%%%%%
\section{Vitesse de la lumière}
%%%%%%%%%%%%%%%%%%%

ANALOGIE
Lorsqu'en refroidissant, l'eau passe de l'état liquide à l'état solide, il apparaît une rigidité qui n'avait pas d'existence dans l'eau liquide.
Lors de l'expansion de l'univers, lors de la transition de phase, il apparaît une massité qui n'avait pas d'existence dans l'univers primordiale.



{\footnotesize
\begin{itemize}[leftmargin=1cm, label=\ding{32}, itemsep=1pt]
\item {\bf \textsc{Étymologie} :} latin {\it },.
\item {\bf \textsc{Sens ordinaire} :} .
\item {\bf \textsc{Terme voisin} :} .
\item {\bf \textsc{Terme opposé} :} .
\item {\bf \textsc{Corrélats} :} .
\end{itemize}
}
