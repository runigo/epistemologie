\chapter{T}
%%{\bf }{\it }{\bf --} « » {\footnotesize X}$^\text{e}$
\section{Temps}

La causalité quantique et l'irréversibilité thermodynamique
donnent deux {\it visions} du temps.

Pour le physicien, la durée est une grandeur physique, le
temps est l' « axe mathématique » de cette grandeur.

temps propre, durée de vie (de demi-vie)

{\footnotesize
\begin{itemize}[leftmargin=1cm, label=\ding{32}, itemsep=1pt]
\item {\bf \textsc{Étymologie} :} latin {\it },.
\item {\bf \textsc{Sens ordinaire} :} .
\item {\bf \textsc{Spiritualisme} :} .
\end{itemize}

\begin{itemize}[leftmargin=1cm, label=\ding{32}, itemsep=1pt]
\item {\bf \textsc{Terme voisin} : durée} .
\item {\bf \textsc{Terme opposé} :} .
\item {\bf \textsc{Corrélats} :} .
\end{itemize}
}

%%%%%%%%%%%%%%%%%%%
\section{Transition de phase}
%%%%%%%%%%%%%%%%%%%

Évolution brusque d'une grandeur thermodynamique ({\it température}, {\it pression}, {\it volume}).

Exemple de l'eau qui gèle : Lorsque l'on refroidie de l'eau liquide, il arrive un moment ou elle devient solide. Le passage de l'eau liquide à l'eau solide est une transition de phase : La température diminue de façon continue et si la pression reste constante, le volume augmente brusquement.

L'eau liquide est caractérisée par ses grandeurs physiques : {\it viscosité}, {\it tension superficielle}, {\it masse volumique}.

L'eau solide (la glace) est caractérisée par ses grandeurs physiques : {\it constante élastique}, {\it masse volumique}.

La glace possède une certaine viscosité mais à une echelle de temps très grande (on l'observe dans les glaciers). La viscosité de la glace est 10 millions de milliard de fois plus grande que celle de l'eau liquide. 

% eau : 10^-3 Pa s  glace : 10^13 Pa s  Vapeur d'eau 10,5 10^-5 Pa s

ANALOGIE

Lorsqu'en refroidissant, l'eau passe de l'état liquide à l'état solide, il apparaît une rigidité qui n'avait pas d'existence dans l'eau liquide. Apparait alors une constante de rigidité (tenseur des constantes élastiques)

Lors de l'expansion de l'univers, lors de la transition de phase, il apparaît une massité qui n'avait pas d'existence dans l'univers primordiale. Apparait alors une constante d'inertie (vitesse de la lumière, constante de la gravitation)

{\footnotesize
\begin{itemize}[leftmargin=1cm, label=\ding{32}, itemsep=1pt]
\item {\bf \textsc{Étymologie} :} latin {\it },.
\item {\bf \textsc{Sens ordinaire} :} .
\item {\bf \textsc{Terme voisin} :} .
\item {\bf \textsc{Terme opposé} :} .
\item {\bf \textsc{Corrélats} :} .
\end{itemize}
}
