\chapter{N}
%%{\bf }{\it }{\bf --} « » {\footnotesize X}$^\text{e}$
%%%%%%%%%%%%%%%%
\section{Neutre}
%%%%%%%%%%%%%%%%
Se dit d'un corps ne portant pas de charge électrique. Plus précisement, dont la charge électrique totale est nulle.
{\footnotesize
\begin{itemize}[leftmargin=1cm, label=\ding{32}, itemsep=1pt]
\item {\bf \textsc{Étymologie} :} latin {\it neuter}, ni l'un ni l'autre.
%{\bf neutron} 1932, Joliot.
%\item {\bf \textsc{Terme voisin} :} .
\item {\bf \textsc{Terme opposé} :} chargé électriquement.
\item {\bf \textsc{Corrélats} :} charge électrique.
\end{itemize}
}
%%%%%%%%%%%%%%%%%%
\section{Neutrino}
%%%%%%%%%%%%%%%%%%
Particule élémentaire.
Le neutrino est électriquement neutre.

En 2019, sa masse aurait été inférieur à 1,1 eV.
En 2022, sa masse serait inférieur à 0,8 eV.
En 2025, il serait possible qu'on en sache davantage sur sa masse
{\footnotesize
(https://www.quebecscience.qc.ca/sciences/mysterieuse-masse-neutrinos-katrin/)
}

{\footnotesize
\begin{itemize}[leftmargin=1cm, label=\ding{32}, itemsep=1pt]
\item {\bf \textsc{Étymologie} :} petit neutre (de l'italien ?) vers 1940.
\end{itemize}
}
%%%%%%%%%%%%%%%%%%
\section{Neutron}
%%%%%%%%%%%%%%%%%%
Particule pas si élémentaire (constitué de trois quarks).
Le neutron est électriquement neutre.
Sa masse est proche de celle du proton.
C'est un des constituants du noyau des atomes.
{\footnotesize
\begin{itemize}[leftmargin=1cm, label=\ding{32}, itemsep=1pt]
\item {\bf \textsc{Étymologie} :} neutre, {\bf neutron} 1932, Joliot.
\item {\bf \textsc{Corrélats} : proton, électron} .
\end{itemize}
}
%%%%%%%%%%%%%%%%
\section{Noumène}
%%%%%%%%%%%%%%%%
Objet de la raison, réalité intelligible {\bf --} Chose en soi.
{\footnotesize
\begin{itemize}[leftmargin=1cm, label=\ding{32}, itemsep=1pt]
\item {\bf \textsc{Étymologie} :} grec {\it nooumena}, choses pensées, {\it noumenon}, ce qui est pensée, {\it noein}, penser.
\item {\bf \textsc{Terme opposé} :} phénomène.
\end{itemize}
}
