\chapter{G}
%%{\bf }{\it }{\bf --}{\footnotesize X}$^\text{e}$
\section{Grandeur physique}

Résultat d'une {\it mesure expérimentale}.

Variable mathématique apparaissant dans les équations de la physique.

\subsection{exemples}
Considérons une boule de pétanque roulant sur une surface lisse.

Nous disposons d'appareils de mesure afin de déterminer son diamètre, sa masse, sa position et sa vitesse.

(le diamètre, la masse, la position et la vitesse sont des grandeurs physiques)

Les appareils mesurent ces grandeurs physiques : m = 700 g, d = 7,5 cm, v = 0,24 m/s, x = 0,24 t m

On constate que la masse, le diamètre, la vitesse sont constantes alors que la position dépend du temps.


