\chapter{M}
%%{\bf }{\it }{\bf --} « » {\footnotesize X}$^\text{e}$
\section{Modèle}
Le fonctionnement d'un ordinateur peut servir de modèle afin de décrire le fonctionnement d'un humain. Il ne s'agit que d'un modèle, il ne prétend pas s'identifier à l'objet décrit. Il prétend donner une description relativement fidèle. Un bon paradigmme précise les limites de son modèle.
{\footnotesize
\begin{itemize}[leftmargin=1cm, label=\ding{32}, itemsep=1pt]
\item {\bf \textsc{} :} latin {\it modulus}, mesure.
\end{itemize}
}
