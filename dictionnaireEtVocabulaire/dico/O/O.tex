\chapter{}
%%{\bf }{\it }{\bf --} « » {\footnotesize X}$^\text{e}$
%%%%%%%%%%%%%%%%%%%%
\section{Observable}
%%%%%%%%%%%%%%%%%%%%
%\section{Physique quantique}
Opérateur mathématique associé à une grandeur physique.

L'observation d'un événement quantique correspond à une opération réalisé sur des quantons, changeant l'état de ces quantons.

{\footnotesize
\begin{itemize}[leftmargin=1cm, label=\ding{32}, itemsep=1pt]
\item {\bf \textsc{Étymologie} :} latin {\it observare} observer, veiller sur, respecter. {\it ob-} au-devant, {\it servare}, préserver, garder.
\item {\bf \textsc{Corrélats} :} Mesure.
\end{itemize}
}
%%%%%%%%%%%%%%
\section{Onde}
%%%%%%%%%%%%%%
Perturbation se propageant dans un milieux matériel.
{\footnotesize
\begin{itemize}[leftmargin=1cm, label=\ding{32}, itemsep=1pt]
\item {\bf \textsc{Étymologie} :} latin {\it unda}, vague, masse d'eau agité.
\item {\bf \textsc{Terme opposé :} corpuscule} .
\item {\bf \textsc{Corrélats :} dualité} .
\end{itemize}
}
