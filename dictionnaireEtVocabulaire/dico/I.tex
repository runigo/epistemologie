\chapter{I}
%%{\bf }{\it }{\bf --}{\footnotesize X}$^\text{e}$
\section{Incertitude}

Sécuriser son territoire, réduire l'incertitude,
prendre une assurance, car il y aurait un principe
d'incertitude macroscopique. Par analogie, on utilise 
abusivement ce terme en physique quantique.

\section{Irréversible}

La physique statistique (ou thermodynamique) décrit les transformation des systèmes isolé comme irréversible : leur entropie augmente.

Équation de Boltzmann

{\footnotesize
\begin{itemize}[leftmargin=1cm, label=\ding{32}, itemsep=1pt]
\item {\bf \textsc{Étymologie} :} latin ir, privé de et {\it reversus}, participe passé de {\it revertere}, retourner, {\it vertere}, tourner.
\item {\bf \textsc{Humour étymologique} :} Nous prenons un virage irréversible.
\item {\bf \textsc{Terme opposé} :} réversible.
\item {\bf \textsc{Corrélats} :} temps.
\end{itemize}
}
