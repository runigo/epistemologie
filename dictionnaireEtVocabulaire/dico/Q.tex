\chapter{Q}
%%{\bf }{\it }{\bf --} « » {\footnotesize X}$^\text{e}$
%%%%%%%%%%%%%%%%%%%%
\section{Quanton}
%%%%%%%%%%%%%%%%%%%%
En physique quantique, un quanton est une entité décrite par son {\it état}.
L'état d'un quanton est « constitué » de {\it fonction d'onde} « obéissant » à l'équation de Schrödinger.

%%%%%%%%%%%%%%%%%%%%
\section{Quantifier}
%%%%%%%%%%%%%%%%%%%%
{\footnotesize
\begin{itemize}[leftmargin=1cm, label=\ding{32}, itemsep=1pt]
\item {\bf \textsc{Étymologie} :} latin {\it quantus}, combien, et du suffixe d'action {\it -ifier}.
% 1906, Larousse
\item {\bf \textsc{Sens ordinaire} :} mesurer, évaluer une quantité.
\item {\bf \textsc{Terme voisin} :} échantilloner.
\item {\bf \textsc{Terme opposé} :} continu.
\item {\bf \textsc{Corrélats} :} quantique.
\end{itemize}
}
%%%%%%%%%%%%%%%%%%%%
\section{Quantique}
%%%%%%%%%%%%%%%%%%%%

Théorie physique unifiant les théories corpusculaires de la matière avec les théorie ondulatoire de la lumière : matière et lumière ne sont plus ni corpusculaire ni ondulatoire, mais obéissent à une même équation, ont un comportement semblable.

{\footnotesize
\begin{itemize}[leftmargin=1cm, label=\ding{32}, itemsep=1pt]
\item {\bf \textsc{Étymologie} :} 1930, Larousse.
\item {\bf \textsc{Terme opposé} :} continu.
\item {\bf \textsc{Corrélats} :} quanton.
\end{itemize}
}
%%%%%%%%%%%%%%%%%%%%
\section{Quantum}
%%%%%%%%%%%%%%%%%%%%

 1764, Voltaire, singulier de {\it quantus}, combien grand.

%%%%%%%%%%%%%%%%%%%%
\section{Quanta} {\bf }
%%%%%%%%%%%%%%%%%%%%

Unité d'énergie minimal d'échange entre la lumière et la matière.

{\footnotesize
\begin{itemize}[leftmargin=1cm, label=\ding{32}, itemsep=1pt]
\item {\bf \textsc{Étymologie} :} 1900, Planck, pluriel de {\it quantum}.
\item {\bf \textsc{Terme opposé} : continu} .
\item {\bf \textsc{Corrélats} : quantique} .
\end{itemize}
}
