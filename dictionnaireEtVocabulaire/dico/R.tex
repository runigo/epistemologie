\chapter{R}
%%{\bf }{\it }{\bf --}{\footnotesize X}$^\text{e}$
\section{Raisonnement}

\subsection{Raisonnement analytique}

Exemple : Tous les corps sont étendus : étendus est contenu dans l'idée de corps, a priori ()

\subsection{Raisonnement synthétique}

Exemple : Tous les corps sont pesants :  pesants n'est pas présent dans l'idée de corps, à postériori (C'est l'expérience qui nous l'apprend).

{\footnotesize
\begin{itemize}[leftmargin=1cm, label=\ding{32}, itemsep=1pt]
\item {\bf \textsc{Étymologie} :} latin {\it },.
\item {\bf \textsc{Sens ordinaire} :} .
\item {\bf \textsc{Spiritualisme} :} .
\end{itemize}

\begin{itemize}[leftmargin=1cm, label=\ding{32}, itemsep=1pt]
\item {\bf \textsc{Terme voisin} :} .
\item {\bf \textsc{Terme opposé} :} .
\item {\bf \textsc{Corrélats} :} .
\end{itemize}
}
