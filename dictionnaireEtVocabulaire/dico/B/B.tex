\chapter{}
%%{\bf }{\it }{\bf --} « » {\footnotesize X}$^\text{e}$
\section{Bigbang}

Lors des premiers instants du bigbang l'expansion de l'univers se serait produite à des vitesses supérieur à celle de la lumière.
Quelques instants plus tard, une transition de phase se serait produite transformant l'univers primordiale en un univers constitué de photon et d'autre particules.
Certaine de ces particule ayant une masse.
Cette transition de phase (ou une autre ?) aurait produit la masse en plus de fixer une vitesse limite.





{\footnotesize
\begin{itemize}[leftmargin=1cm, label=\ding{32}, itemsep=1pt]
\item {\bf \textsc{Étymologie} :} latin {\it },.
\item {\bf \textsc{Sens ordinaire} :} .
\item {\bf \textsc{Spiritualisme} :} .
\end{itemize}

\begin{itemize}[leftmargin=1cm, label=\ding{32}, itemsep=1pt]
\item {\bf \textsc{Terme voisin} :} .
\item {\bf \textsc{Terme opposé} :} .
\item {\bf \textsc{Corrélats} :} .
\end{itemize}
}

%%%%%%%%%%%%%%%%%%%
\section{}
%%%%%%%%%%%%%%%%%%%
{\footnotesize
\begin{itemize}[leftmargin=1cm, label=\ding{32}, itemsep=1pt]
\item {\bf \textsc{Étymologie} :} latin {\it },.
\item {\bf \textsc{Sens ordinaire} :} .
\item {\bf \textsc{Terme voisin} :} .
\item {\bf \textsc{Terme opposé} :} .
\item {\bf \textsc{Corrélats} :} .
\end{itemize}
}
