
%%%%%%%%%%%%%%%%%%%%%
%\section{Propriétés des ondes électromagnétique}
\section{Unification électromagnétique}
%%%%%%%%%%%%%%%%%%%%%

\subsection{Champ électrique et champ magnétique}

Au début du XIX$^\text{ème}$ siècle, les expériences de Hans Christian Oersted montrent un lien intime entre les forces magnétiques et les forces électriques : les 

Les équations de maxwell vont établir un lien entre les champs magnétique et électrique : non seuleument ces champs sont créés par les charges électrique, les aimants et les charges en mouvement, mais un champ électrique est créé par un champ magnétique variable au cours du temps et un champ magnétique est créé par un champ électrique variable au cours du temps.

%\begin{center}
\[
\overrightarrow{E} \text{ variable au cours du temps } \Rightarrow \text{ création de } \overrightarrow{B}
\]
\[
\overrightarrow{B} \text{ variable au cours du temps } \Rightarrow \text{ création de } \overrightarrow{E}
\]
%\end{center}

%\vspace{0.2cm}

\subsection{Ondes électromagnétiques et lumière}

De ces dernières propriétés découlent l'existence d'ondes électromagnétiques dont la vitesse peut être calculée en fonction des constantes électriques et magnétiques apparaissant dans les équations de Maxwell. La valeur calculée, correspondant à la valeur de la vitesse de la lumière alors connue à l'époque.

Ces ondes se propage dans le vide et dans les milieux transparent.Leur identification avec la lumière achèvera l'unification des phénomènes électriques, magnétiques et lumineux.

%%%%%%%%%%%%%%%%%%%%%%%%%%%%%%%%%%%%%%%%%%%%%%%%%%%%%%%%%%%%%%%%%%%%%%%%%%%%%
