Newton aurait-il confondu le temps avec la date ? Les équations de la mécanique qu'il énnonce contiennent une variable t. Newton aurait appelé cette variable "temps". On doit distinguer la date (l'heure) avec la durée (époque, période). Ces deux notions peuvent être apellé temps. Le physicien distingue la date de la durée : une durée est une différence de deux dates, une date est une grandeur dépendant d'une origine.

Dans le langage commun, nous appelons "la date" la durée qui s'est écoulé depuis la naissance du Christ. Nous appelons "l'heure" la durée qui s'est écoulé depuis minuit. Grâce au choix d'une origine, la date et l'heure permettent de qualifier, d'indiquer un instant. 

Doit on distinguer "l'instant" d'un "laps de temps" ? 
