\section{Pratique de la philosophie}
%{\bf }{\bf --}{\it } {\footnotesize XIX}$^\text{e}$

\subsection{noumène}

\begin{itemize}[leftmargin=1cm, label=\ding{32}, itemsep=1pt]
\item {\bf \textsc{Étymologie} :} terme formé par
Kant à partir du  {\it grec}, noumenon,
«chose pensée ».
\item {\bf \textsc{Chez Kant} :} réalité intelligible ; objet de la raison
que peut connaître un esprit doué
d’intuition intellectuelle — par
exemple Dieu —, mais pas
l'homme, qui ne connaît que les
phénomènes, c'est-à-dire les objets
de son intuition sensible.
\end{itemize}

Parce que le noumène est humainement
inconnaissable, il est parfois assimilé à
la chose en soi, que Kant oppose également
au phénomène. Or, le noumène
n'est pas la chose en soi, laquelle est
l'objet tel qu'il est en lui-même, sans
référence à la connaissance que quiconque
— homme ou Dieu — peut en
prendre. Le noumène est au contraire un
objet de pensée, ou de connaissance
rationnelle, une essence intelligible, au
sens platonicien, même si celle-ci est
inaccessible à l'esprit humain. C'est
pourquoi, chez Kant, ce sont surtout les
idées métaphysiques (l'âme, l'univers,
Dieu) qui ont une réalité « nouménale ».

\begin{itemize}[leftmargin=1cm, label=\ding{32}, itemsep=1pt]
\item {\bf \textsc{Terme voisin} :} chose en soi;
essence ; intelligible.
\item {\bf \textsc{Terme opposé} :} phénomène.
\item {\bf \textsc{Corrélats} :} métaphysique.
\end{itemize}

