\section{Des observables aux grandeurs physiques,
des phénomènes à la loi}
%{\bf }{\bf --}{\it } {\footnotesize XIX}$^\text{e}$
%14 M. À. TONNELAT
%\footnote{}size[]
%1.

L'observation immédiate nous introduit au cœur d’une situation
complexe. L'univers est « tout d’une pièce, comme un océan »
%{\footnotesize [{\sc Leibniz}, {\it Opera Philosophica}, éd. Erdmann, p. 506]}.
{\footnotesize [{\sc Leibniz}, {\it Opera Philosophica}]}.
Les ressemblances et les répétitions que nous pouvons y discerner sont inextricablement
mêlées à des caractères particuliers et changeants. « Si nos
faces n’étaient semblables,
% écrivait Montaigne, 
%{\footnotesize [{\sc Montaigne}, {\it Essais}, Paris, Flammarion, vol. IV, p. 194]}
on ne saurait discerner
l’homme de la bête ; si elles n’étaient dissemblables on ne saurait discerner
l’homme de l’homme. »
{\footnotesize [{\sc Montaigne}, {\it Essais}]}
La perception puis l’expérimentation nous
livrent un résultat brut, complexe, mais somme toute inéluctable : les
\textbf{\textit {observables}}{\it }. C’est l’enchaînement entre les diverses observables qui constitue
le \textbf{\textit {phénomène}}{\it }. Au sens étymologique, le phénomène s’identifie aux
apparences. Rien {\it d’illusoire ou de fallacieux ne doit être associé à ce
qualificatif d’« apparent »}. Il signifie tout simplement qu’observables et
phénomènes sont liés, par définition, à l’observation et à ses modalités.
Ils en demeurent nécessairement tributaires; ils n’ont de sens que par
rapport à l’observateur même et à son état de mouvement.

C’est un lent travail de systématisation et de discernement qui conduit
de l’idée immédiate, mais fragmentaire, d’observable à la notion
élaborée, mais synthétique, de grandeur physique. Partant de la perception
immédiate du phénomène, il mène à l’enchaînement nécessaire que
constitue la loi.

Ce travail consiste à construire l’objet à partir de ses traces dans
l'expérience de l’observateur, c’est-à-dire à partir d’observables et de
mesures relatives. La correspondance peut sembler très simple dans
certains cas. Par exemple, la donnée du diamètre apparent du soleil
jointe à la connaissance de la distance qui nous sépare de celui-ci permettra
de déterminer les dimensions « vraies » du soleil. Ainsi, au sens
de la Mécanique classique, un jeu d’observables (dépendant par définition
des modalités de l’observation) peut être équivalent à des données
sur les dimensions « intrinsèques » de l’objet
%{\footnotesize [Nous assimilerons pour le moment « objet » et grandeur intrinsèque. Ce
%rapprochement est sans ambiguïté dans la physique classique]}.
Néanmoins, pour être
capable de reconstruire l’objet sans ambiguïté, il faut être assuré de
disposer d’un jeu « complet » d’observables. Une assurance trop vite
%HISTOIRE DU PRINCIPE DE RELATIVITÉ 15
acquise engendre les prétentions des sophistes. Elle entraîne la Mécanique
classique elle-même vers des illusions que devra dissiper la
Relativité Restreinte.

Même à ce stade, le passage de l’observable à l’objet s’accompagne
d’inévitables critères théoriques. Ils interviennent {\it a fortiori} dès que
l’« objet » est lié à des manifestations plus indirectes. La hauteur de la
colonne barométrique est une observable qui manifeste un « objet » très
élaboré : la pression atmosphérique.

Ainsi la construction d’un objet constitue toujours une idéalisation.
Mais la solidité et la nécessité de celle-ci présente d’incontestables privilèges :
l’objet est indépendant de l’observateur et de son mouvement. Il
est affranchi du {\it hic} et du {\it nunc}
{\footnotesize [de l'ici et du maintenant, n.d.n]}
inhérents à toute observation.

Les observables et le phénomène qui les relie partagent la solidité
mais aussi les vicissitudes d’une situation de fait; l’objet de la loi physique
bénéficie du prestige mais aussi de la fragilité d’une situation de
droit. Il peut servir de thème explicatif pour ordonner l’ensemble de
l’expérience sensible
%{\footnotesize [L'idée de grandeur physique est utilisée ici dans le sens classique, et non dans
%l’acception assez particulière de la Mécanique Quantique. En Mécanique quantique
%l’objet se réduit à des données statistiques et n’est plus localisable dans
%l’espace-temps si son état de mouvement est bien déterminé.
%Nous nous limiterons, dans tout ce qui suit, aux conceptions purement classiques]}
;
il constitue un élément de l’édifice scientifique.

La distinction entre {\it observable} et {\it grandeur physique} nous semble
essentielle pour interpréter les progrès de l’esprit relativiste. Insistons
sur le fait qu’elle est entièrement indépendante d’une ontologie sous-jacente.
%{\footnotesize [Nous éviterons de lier univoquement et de façon exclusive le critère de
%« réalité » qui, pour un physicien, comporte en fait un jugement de valeur soit
%à la notion d’{\it observable} sous prétexte qu’il s’agit d’une donnée brute et affranchie
%de toute élaboration théorique (ce qui est d’ailleurs inexact), soit, au contraire,
%à la notion d’{\it objet} en arguant du fait qu’elle est affranchie des contingences et des
%modalités inhérentes à toute expérience particulière.
%Attacher le critère de « réalité » à l’un ou bien à l’autre de ces points de vue
%c’est déjà prendre parti sur la signification de la Relativité.]}
Ainsi, pour Ptolémée, le soleil mobile autour de la terre
entraînant son cortège d’astres errants constituait une grandeur physique
authentique ou, si l’on préfère, la représentation qui rendait compte de
l’ensemble de l’expérience sensible et des mesures de son époque. Il se
substitue à la perception de cet objet minuscule, chaque jour nouveau,
qui apparaît à l'Orient. Une grandeur physique est donc toujours {\it objective}
en ce sens qu’elle satisfait — même de façon embryonnaire — un
ensemble de lois, qu’elle groupe — même pour un temps limité — une
série d’observations. On peut, sans inconvénient, réduire ces « idéalisations »
%TONNELAT. — Histoire du principe de relativité. 2
%16 M. À. TONNELAT
à de purs jeux d’ombres : il suffira pour qu’elles deviennent
objets de physique, que ces ombres satisfassent l’optique d’Euclide.
A cette condition, mais à cette condition seulement, la caverne des
magiciens devient laboratoire.

Par définition même, les observables et leur association en cohortes
de phénomènes constituent des données immédiates; elles s’intègrent
à une situation d’ensemble qui comprend, en particulier, les conditions
de l’observation. Nous appellerons {\it relativisme} cette dépendance vis-à-vis
de l’observation ou, si l’on préfère, cette subordination à l’observateur.
{\it Une observable, un phénomène constituent des données essentiellement relatives.
Il n’y aurait pas plus de sens à nier le relativisme d’une observable ou
d’un phénomène qu’à contester leur existence.}

Au contraire, la {\it grandeur} ou la {\it loi physique} constitue, par construction,
des {\it absolus} en ce sens qu’{\it elles sont indépendantes des modalités de
l'observation}. Bien entendu, cet absolu est pratiquement fonction de la
science du moment et de l’ensemble des connaissances de l’époque.
Néanmoins, nier l’existence de cet objet, de cette loi, ou plutôt l’identifier
aux impressions fluctuantes d’un instant c’est abolir sa raison d’être,
c’est construire, à côté de la perception, un rêve capricieux et incohérent,
un rêve inutile.

Cette assimilation fallacieuse de l’objet à l’observable, de la loi au
phénomène est cependant une tentation de tous les temps. Un Relativisme
philosophique étend indûment à l’objet, à la loi, cette subordination à
la perception immédiate, ce « relativisme » scientifique qui caractérise
l’observable et le phénomène.


%{\footnotesize [Nous assimilerons pour le moment « objet » et grandeur intrinsèque. Ce
%rapprochement est sans ambiguïté dans la physique classique]}.
