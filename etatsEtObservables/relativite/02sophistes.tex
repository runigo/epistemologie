\section{Le Relativisme des Sophistes}
%{\bf }{\bf --}{\it } {\footnotesize XIX}$^\text{e}$

« L'homme est la mesure de toutes choses affirmait déjà Protagoras
d’Abdère
%{\footnotesize [XÉNOPHANE, Fragment 1]};
de celles qui sont, en tant qu’elles sont, de celles qui ne
sont pas en tant qu’elles ne sont pas. » C’est ainsi que, pour Xénophane,
les astres s’allument à l’Orient, s’éteignent à l’Occident. Leur vraie
grandeur est celle que manifeste l’observation immédiate et le soleil,
« chaque jour nouveau »
%{\footnotesize [PROTAGORAS, Fragment 6]}
a « la largeur d’un pied d’homme ».
%{\footnotesize [XÉNOPHANE, Fragment 3]}.

%17
Ces affirmations naïves caractérisent un Relativisme intégral, naturel
tant qu’il s’agit d’observables arbitraires dont l’enchaînement constitue
un phénomène, entièrement stérile s’il est question de grandeurs intrinsèques
liées par une loi. Les opinions de Xénophane conduisent alors
inévitablement au sophisme de Gorgias : rien ne peut être connu et toute
Science est parfaitement vaine et inutile.

Des opinions parentes du sophisme mais infiniment plus nuancées
ont des résurgences dans un certain humanisme de tous les temps. Elles
apparaissent alors comme une sorte de tentation qui voudrait opposer
les desseins d’une Science sèche ou trop ambitieuse au devenir harmonieux
de l’homme. Ceux-là doivent alors être subordonnés à celui-ci dès
que l’on veut jouer les uns contre l’autre.

Les enseignements de Socrate ne sont pas exempts d’une telle illusion;
le réquisitoire antinewtonien de Gœthe n’en est pas non plus dépourvu.
Au {\footnotesize XVII}$^\text{e}$ siècle, on soupçonnera les lunettes de véhiculer des visions
illusoires ; bien plus tard, on accusera le prisme de dénaturer la lumière
qui baignait la campagne italienne, la Science de défigurer une instructive
et intuitive vérité : « Toute philosophie de la nature, écrit Gœthe, est
seulement de l’anthropomorphisme : l’homme, pour sauvegarder son
unité personnelle, répartit sur toute chose ce qu’il n’est pas. Retranscrit
dans son unité, ceci ne fait qu’un avec lui-même. »

En fait, le Relativisme intégral est une retombée de l’Humanisme. Au
contraire, le développement authentique de celui-ci ne peut se réaliser
sans une abdication qui dans la Science, dans le style, dans l’Art permet
de se servir du particulier puis de le dépasser pour retrouver ainsi ce qui
peut être commun à tous.

Il n’est pas rare, nous l’avons dit, de constater que le Relativisme de
Protagoras a été confondu avec les prémisses de la théorie d’Einstein
dont le sens est diamétralement opposé. On a cru que l’homme ou, plus
exactement, que l’observateur était la mesure de toute chose, que les
objets, à l’instar des phénomènes, devenaient inféodés à son temps, à son
espace, à son système de référence. Se refusant à distinguer les observables
— qui, bien entendu, dépendent de l’observation — des structures intrinsèques
qu’elles permettent de mettre en évidence, on s’est refusé à reconnaître
l’invariance de celles-ci qui se traduit par la covariance de la loi.

C’est ainsi que J. Petzoldt
%{\footnotesize [J. PETZOLDT, Die Stellung der Relativitätstheorie, p. 15]}
estimant que la Relativité est un corollaire
du positivisme de Mach, rattache l’une et l’autre au subjectivisme de
%18
Protagoras « Ç’a été le malheur de l’humanité, écrit-il, que l'opinion saine
de ce phisolosophe n’ait pu prévaloir contre les doctrines de Platon et
d’Aristote ; le Moyen Age lui eût été épargné ».

Bien au contraire, le Principe de Relativité fut de tout temps, comme
le remarque Meyerson,
%{\footnotesize[E. MEYERSON, La déduction Relativiste, p. 68)]}
celui de la non-relativité du réel ». Il se situe
dans la ligne de cette « émancipation anthropomorphique » que Planck
reconnaît comme l’évolution nécessaire de toute théorie physique.
Néanmoins, ce passage à l’« objectif » ne doit pas se confondre avec une
assimilation passive aux données particulières des apparences immédiates.

