\section{La solution Keplerienne :
Système Mécanique ou Attraction}
%{\bf }{\bf --}{\it } {\footnotesize XIX}$^\text{e}$

Copernicien convaincu, Kepler va rester, à certains égards, plus
proche d’Aristote que de Descartes et de Galilée. Pourtant sa méthode
d’obtention des trajectoires planétaires a toujours sa place dans l’astronomie
moderne. Les hypothèses s’appuient habilement sur les précises
observations de Tycho et suppléent aux lacunes dans le calcul des indiscernables
« Les travaux de Kepler, écrit Einstein \footnote{A. Einstein. Johannes KEPLER, Essays in Science, trad. 1934, p. 27.}, montrent que la
connaissance ne peut se déduire de la seule expérience. Une comparaison
de ce que l’esprit a conçu avec ce qu’il observe est nécessaire. »

Néanmoins, des survivances aristotéliciennes interviennent dans maints
passages de son œuvre. Il est particulièrement instructif de comparer, à
cet égard, l’argumentation de l’anticopernicien Tycho (opposée à Rothmann)
et celle du copernicien Kepler (opposée à Tycho) : l’une et l’autre
écartent délibérément l’hypothèse d’un univers infini et prétendent ainsi
soit confondre, soit défendre Copernic.

Kepler abandonne le concept de « lieux naturels » mais la notion de
système mécanique reste toujours nécessairement liée à celle d’action
physique. Les corps « séparés » de la terre ne l’accompagnent pas en
raison d’une « faculté animale » comme chez Copernic, mais en vertu
d’une action d’origine magnétique. Il est inutile d’attribuer à toutes les
choses terrestres une « même âme motrice » comme le pensait Copernic
%34
car «la force de la faculté corporelle (que nous appelons gravité ou
force magnétique) se fait valoir dans les mouvements corporels en entraînant
les corps attirés par la terre et en les faisant ainsi participer au
mouvement de celle-ci » \footnote{KEPLER, Astronomia Nova, Opera, V, IL, p. 152.
C’est à Kepler que nous devons le concept d’inertie. Mais l’inertie keplerienne
traduit la résistance du corps grave au mouvement c’est-à-dire sa tendance au repos,
ce qui est très différent de sa résistance à l’accélération c’est-à-dire une résistance
contre la mise en mouvement (conception moderne).

Tout mouvement implique donc un moteur (précisément à cause de l’inertie).
Sinon il s’userait.

A. Koyré remarque que l’inertie keplerienne joue le même rôle que la résistance
externe du milieu dans la physique d’Aristote. Ainsi le mouvement de corps dépourvus
d'inertie, serait instantané (cf. A. KoyRÉ, Études galiléennes, note 3, III-4.

Dans ce qui précède, on ne peut donc dire que les graves participent au mouve-
ment de la terre en raison de leur inertie. Mais au contraire, étant doués d’inertie,
une force doit expliquer leur entraînement.
}.

Il existe ainsi une sorte de « raptus » par la terre, «raptus » que
pourrait figurer une infinité de chaînes invisibles.

L’ami de Kepler, Fabricius, lui pose la question suivante : « Par
quel raisonnement veux-tu, partisan de Copernic, répondre à l’argument
de Tycho sur le tir du canon? Certes, si le canon tire vers l’Orient, le
boulet, grâce au mouvement plus rapide de la terre trouvera son lieu de
repos plutôt que vers l’Occident et ne pourra pas du tout se mouvoir
vers l’Orient. Cet argument possède une force herculéenne contre le
mouvement diurne de la terre \footnote{KEPLER, Astronomia Nova. In commentaria de Motibus Martis, n° 21 (Opera,
éd. Frich, vol. IIT, p. 458.}. »

Tout corps matériel est en lui-même, par nature, immobile et destiné
au repos, dans quelque lieu qu’il soit. « Car le repos, de même que
les ténèbres, est une espèce de privation qui n’exige pas de création mais
appartient aux choses créées comme une certaine trace du néant; le mouvement,
par contre, est quelque chose de positif comme la lumière. Ainsi,
si la pierre se meut localement elle ne le fait pas en tant qu’elle est matérielle
mais en tant qu’elle. est poussée et attirée intrinsèquement par
quelque chose. C’est pourquoi je ne définis la gravité, c’est-à-dire cette
force qui meut la pierre intrinsèquement, que comme une force magnétique
unissant les semblables, qui est numériquement la même dans le
grand et le petit corps et prend les mêmes dimensions que les corps.
Par conséquent, si une pierre était placée près de la terre, une pierre dont
la masse aurait des dimensions comparables à celles de la terre... alors
il adviendrait que non seulement la pierre irait vers la terre mais aussi la
%35
terre vers la pierre ; et elles diviseraient l’espace qui les sépare dans la
proportion inverse de leurs poids \footnote{KEPLER. op. cit., p. 458.}. »

Ainsi, pour Kepler, la pierre suit la terre parce que la terre l’attire.
Ce n’est pas un éfat mécanique (notre inertie actuelle) mais une force
physique réelle qui s’avère nécessaire pour vaincre (au sens de Kepler)
l’inertie de la pierre. Le système mécanique selon Bruno devient, avec
Kepler, un authentique système physique, siège d’interactions.

Les conceptions de Kepler sur la nature des forces d’attraction ont
d’ailleurs varié dans une large mesure : il est passé d’un vitalisme ou, si
l’on veut, d’un animisme cosmique à une conception plus physique de
l'attraction. Pourtant, il n’a jamais reconnu l’équivalence du repos et du
mouvement et, pour lui, l’inertie reste nécessairement une résistance au
mouvement, une tendance au repos.

Ainsi, par rapport à un observateur immobile sur la rive, il existe une
différence entre la projection d’un objet vers l’avant et vers l’arrière
d’un bateau. La distance franchie, la force du jet seront différentes. Si,
pour le navigateur, tout se passe de la même manière c’est pour la raison
suivante : le raptus de la terre s’exerce de façon analogue sur le navire
(par contact) et sur le projectile (par attraction). Tout au plus pourrait-on
concevoir une légère différence si le projectile était lancé très loin dunavire.

« Que l’on raisonne de la même manière mutatus mutandis dans le
cas du canon — … Par rapport à l’espace du monde le boulet est imparti
d’un mouvement plus violent vers lorient. Mais cet espace composé
(mondial) n’a rien à voir avec l’espace que les hommes peuvent mesurer
sur la terre. Sur la terre, l’espace parcouru par le boulet est, dans les deux
cas à peu près le même, car la force est la même et les liens magnétiques
sont les mêmes. En admettant même que la différence soit perceptible,
il n’en reste pas moins vrai que la possibilité d’en faire l’expérience est
inexistante. Qui donc pourra m'’assurer que la force de l’explosion de la
poudre a été la même dans les deux cas que toutes les autres circonstances
ont été pareilles? \footnote{Kepler. Épitome Astronomiae Copernicae, I, vol. VI, p. 182.} ».

