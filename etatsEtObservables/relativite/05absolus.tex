\section{La notion d’Observateur « privilégié »
et les absolus de la Cinématique et de la Physique}
%{\bf }{\bf --}{\it } {\footnotesize XIX}$^\text{e}$

La Mécanique d’Aristote fait reposer le caractère absolu du mouvement
sur une convention, sans doute arbitraire, mais liée à l’ensemble
de sa physique et essentielle à sa philosophie. Les opinions qui voudront
se rallier à la physique d’Aristote sans admettre cette cosmologie comme
un tout vont essayer, au contraire, d’étayer l’hypothèse du mouvement
absolu par des motifs extrinsèques : argumentation d’inspiration religieuse
en faveur d’un fixisme terrestre, considérations souvent naïves et anthropomorphiques
confondant mouvement et travail, travail et effort.

Il est très difficile de dissocier les éléments d’une théorie aussi organique
que celle d’Aristote, aussi logique que celle de Descartes sans
aboutir à l’écroulement de l’édifice tout entier. Le résultat, toutefois,
reste instructif car il révèle la fragilité des ressorts naturels, de l’emboîtage
des arguments que liait, sans les cimenter les uns aux autres, l’unité d’une
philosophie.

Le corollaire de l’idée de mouvement absolu est la notion d’observateur
absolument immobile. Aristote ne se souciait pas d’incarner cet
observateur dans un ensemble expérimental. Les difficultés de la Physique
du Moyen Age consisteront, au contraire, à rattacher le caractère absolu
de l’immobilité et du mouvement à un contexte expérimental donné ou
imaginé.

L’observateur absolu est alors un observateur éminemment privilégié
et susceptible de mettre en évidence sa situation exceptionnelle :

— Soit en comparant les incidences de son mouvement avec celles du
mouvement d’autres observateurs sur la description d’un même phénomène.
Si une description semble « privilégiée », parce que plus simple et
%22
plus efficace, elle révèle alors un absolu indépendant du phénomène lui-même,
donné une fois pour toutes et commun à tous. Elle peut donc
servir de critère cinématique pour le choix du système de référence dans
lequel on retranscrit cette observation. Elle serait ainsi susceptible de
mettre en évidence un absolu d’ordre cinématique \footnote{(1) Cf. note (1) et (2) p. 52.} optique ou encore
descriptif. Celui-ci caractériserait en effet le privilège de la version d’un
certain observateur en ce qui concerne la description d’un phénomène
donné.

— Soit en étudiant la modification éventuelle qu’apporterait le mouvement
d’un référentiel à un phénomène qui lui serait lié. Les observateurs
doivent alors comparer des phénomènes du même type dans leurs référentiels
respectifs. Toute description privilégiée révèle alors l’influence
effective du mouvement sur le déroulement de phénomènes qui seraient
identiques dans des systèmes immobiles. Elle traduit une modification
physique de l’enchaînement intrinsèque des observables et sert à mettre
en évidence un absolu d’ordre physique.

\subsection{Absolu descriptif}

L'idée d’un observateur privilégié par rapport à une certaine classe
de mouvements est pratiquement indispensable pour mettre en évidence
les lois de ces mouvements. Le déplacement d’une bille sur un manège
en rotation se décrira, de préférence, dans le système de référence d’un
observateur immobile sur ce manège. La régularité du mouvement des
astres conduit tout naturellement à postuler un fixisme terrestre.

La simplicité particulière de l’expression des lois physiques, simplicité
relative à un certain système de référence, corrobore tout naturellement
le caractère privilégié de ce système par rapport au phénomène étudié.
De ce point de vue, on peut soutenir, avec Henri Poincaré, que la
terre était fixe dans la mesure où les trajectoires des planètes, rapportées
au système de référence lié à la terre, constituaient des courbes simples.
Le caractère « privilégié » du mouvement circulaire représentait a priori
une excellente hypothèse de travail. La prolifération des épicycles et des
déférents allait devenir la véritable objection contre les privilèges du
repérage adopté.
%23

En effet, cette identification de l’objet et des apparences, identification
que le Sophisme de Protagoras permettait à quiconque, une cinématique
des mouvements absolus la réserve à un seul. Un observateur
privilégié — ou plutôt une classe d’observateurs privilégiés, — voit, en
quelque sorte, le mouvement en vraie grandeur. Les « apparences » se
confondent pour lui, mais pour lui seul, avec la réalité.

Insistons sur le fait que cet absolu descriptif instaure une hiérarchie
des situations et des mouvements au sein de la seule cinématique. Aucun
« effet physique » du mouvement n’est requis, aucune dynamique ne doit
intervenir. Dans un univers structuré, le relativisme s’ordonne : il ne
saurait alors fluctuer au gré des sentiments ou des impressions fugitives.
Il caractérise des structures naturelles indifférentes aux forces de la
future dynamique. Il entérine des privilèges nés d’une situation de droit
car ils accompagnent une implicite cosmologie. C’est l’ordonnance dont
se réclament les juges de Galilée.

Mais, bien avant Galilée, le pouvoir de la cinématique est très différent.
Bien entendu « Ignorer le mouvement c’est ignorer la nature » tel est
le Principe de l’École, mais on ajoute, aussitôt : « Les mathématiques ne
peuvent se rapporter au mouvement car il n’y est pas question d’une
fin ou d’un bien. »

\subsection{Absolu physique}

On peut supposer aussi qu’un mouvement «vrai» se manifeste par
des effets qui peuvent être décelés au moyen d’expériences réalisables à
l’intérieur d’un système de référence lié à ce mouvement (expériences
internes). Cette conviction familière semble intervenir implicitement dans
la détection de nombreux mouvements réels : les cahots du wagon
révèlent au voyageur que son déplacement n’est pas un rêve et que le
paysage ne fuit pas. « Quand je suis tranquille et qu’un autre, s’éloignant
d’un mille, est rouge de fatigue, c’est lui qui se meut et moi qui me repose »
affirmait à Descartes Thomas Morus dans une boutade qui a toujours
paru exprimer le fameux bon sens populaire.

Newton fera des effets physiques du mouvement le critère indispensable
et d’ailleurs suffisant de son caractère absolu. Toutefois, bien avant
la création d’une dynamique, le caractère absolu du mouvement était
confusément associé à la production de phénomènes physiques dont ce
mouvement devenait plus ou moins responsable. L’immobilité absolue
%24
aurait au contraire conservé au phénomène son déroulement authentique
sans lui associer aucune manifestation parasite.

Notons dès maintenant que ces effets physiques, attribués au mouvement,
sont liés en fait, non au mouvement lui-même mais à ses modifications.
Les cahots du wagon, dus aux chocs contre les rails ne constituent
pas une «expérience interne» et manifestent seulement un
mouvement relatif qui n’a jamais été douteux. Au contraire, les accélérations
ressenties au départ, à l’arrivée, lors des changements de régime
constituent des expériences internes authentiques. Elles manifestent
alors des modifications de cette vitesse et non pas son existence absolue.
D'une manière analogue, la fatigue du promeneur qu’évoque Thomas
Morus, est l’effet d’un certain effort et se rapporte à la succession des
accélérations et des décélérations qu’entraîne la marche à pied. Un
glissement idéal uniforme sur un sol parfaitement lisse et plan n’entraînerait
aucun effort, aucune usure, aucun effet.

Il ne s’agit pas jusqu’à présent de contester le caractère absolu de
changements d’état de mouvement. Il s’agit de savoir si le mouvement
«en lui-même », mouvement dont un déplacement rectiligne et uniforme
sans début, sans fin, sans retour, donne une image idéale peut entraîner
des phénomènes physiques susceptibles de le mettre en évidence. Peut-il
aussi acquérir une réalité physique qui le distingue intrinsèquement du
repos et lui confère un caractère absolu?

La philosophie d’Aristote ne faisait que postuler une réponse affirmative.
Les Mécaniciens, jusqu’au xvie siècle, de Ptolémée à Benedetti
vont essayer de justifier cette réponse.

