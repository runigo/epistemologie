\section{Inertie et Relativité avant Galilée}
%{\bf }{\bf --}{\it } {\footnotesize XIX}$^\text{e}$

La notion d’observateurs équivalents, de référentiels physiquement
semblables conditionne, nous le verrons, l’idée même de Relativité. Des
systèmes physiques équivalents sont, pour nous, des systèmes soustraits
%36
aux contraintes ; des observateurs similaires constituent des observateurs
libres. Les critères de cette liberté tels qu’ils seront exprimés par un
principe d’inertie sont, tout d’abord, bien loin d’être évidents.

Pour Aristote, le mouvement (altération et changement) est lié à la
nature du mobile. Une conséquence logique de cette conception est que,
pour un lieu donné et pour un mouvement donné, il ne peut exister qu’un
seul mouvement naturel.

Cette nature du mouvement nécessite, nous l’avons vu, l’édification
d’un cosmos et se traduit par les conséquences suivantes :

Tant qu’il s’agit du monde céleste, un déplacement rectiligne est
inconcevable car il ne pourrait, dans un cosmos fini, être éternel. Cet
argument va rester d’un grand poids pour Galilée lui-même :

« Ayant établi ce principe, on peut conclure immédiatement que,
si tous les corps cosmiques doivent être mobiles par leur nature, il est
impossible que leur mouvement soit rectiligne ou autre que circulaire ;
et la raison en est assez facile et manifeste : puisque ce qui se meut de
façon rectiligne change de lieu et, continuant de se mouvoir, s’éloigne
toujours davantage du terme d’où il était parti. il s’ensuivrait que, dès
le commencement, il n’était pas dans son lieu naturel et que, par conséquent,
les parties du monde n'étaient pas disposées dans un ordre parfait.
Mais nous avons admis qu’elles étaient parfaitement ordonnées ; donc,
il est impossible qu’elles soient déterminées par leur nature à changer de
lieu et, par conséquent, à se mouvoir en ligne droite \footnote{Galilée, Dialogo I (Opere, vol. VIN), p. 43. « Moto retto di sua natura
infinito. Moto terro impossibile esser nel mondo ben ordinato ».}. »

Au contraire, pour Aristote, le mouvement rectiligne est possible dans
le monde sublunaire mais, tandis que le mouvement rectiligne vertical est
naturel (de haut en bas ou de bas en haut, suivant le lieu naturel du corps
considéré), tous les autres mouvements rectilignes sont forcés ou violents.

Au xive siècle, une telle conclusion était toujours celle des Maîtres
péripatéticiens d'Oxford. Ils cherchaient néanmoins à mathématiser
Aristote \footnote{Thomas Bradwardine par exemple (1328). Swineshead, maître de Nicolas
Oresme).}. Des propositions telles que: « La vitesse est l’excès de la
force motrice sur la résistance » ne sont pas sans mérite par l’esprit même
qu’elles cherchent à instaurer, mais elles montrent qu’une authentique
inertie est encore bien lointaine.

Les nominalistes parisiens, c’est-à-dire les disciples de Jean Buridan \footnote{Jean BurIDAN, né à Béthune vers 1300, recteur de l’Université de Paris
en 1327. Mort à Paris en 1358 (?).}
%37
critiquent vigoureusement le dire d’Aristote selon lequel la vitesse d’un
mobile serait inversement proportionnelle à la résistance.

Cette assertion, on le sait, est contredite, dans l’immédiat, par l’expérience.
Elle donne lieu au célèbre problème du jet : le mobile continue à
progresser même quand la force qui a permis de le lancer a disparu.

Buridan, en particulier, critique à la fois la doctrine de l’antipéristasis
et celle d’Aristote ;

D'après la première, l’air devrait se précipiter dans le vide laissé
par le mouvement du mobile et pourrait ainsi le pousser vers l’avant.
Mais, dit Buridan, si une charrette transporte des ballots de paille, les fétus
de paille sont, dans le mouvement projetés vers l’arrière — et non vers
l’avant comme cela devrait avoir lieu s’ils subissent une poussée à l’arrière.

D'après Aristote, une telle poussée vers l’avant se transmettrait au
projectile par l’intermédiaire de l’air et constituerait ainsi une réaction
du milieu.

Au contraire, Buridan va supposer que si le mobile peut poursuivre
son parcours après la disparition de toute intervention extérieure, c’est
qu’il a emmagasiné une vis impressa, une force spécifique : l’impetus \footnote{Il est indispensable de rappeler qu’une physique de l’impetus avait eu
sous forme bien souvent épisodique un long passé.
Jean Philopon (517) supposait que le jet était alimenté par une certaine énergie
motrice incorporelle cédée au projectile. Cette doctrine semble avoir été assez
largement admise à Bagdad (Avicenne).

On la retrouve ensuite chez Thierry de Chartres et dans la scolastique du
xime siècle (Saint Bonaventure, Saint Thomas).

Néanmoins c’est bien Buridan qui garde le mérite d’un exposé systématique
de la théorie et d’un examen critique de ses conséquences (chute des corps, rebondissement
d’une bille, etc...).
}.

«Il faut donc admettre que le moteur, en mouvant le mobile, lui
imprime un certain élan (impetus), une certaine force motrice dans le
sens où le moteur le mouvait. C’est par cet impetus qu’est mue la pierre
après que,celui qui l’a lancée a cessé de la mouvoir ; mais, à cause de la
résistance de l’air et du poids de la pierre qui l’attire en un sens contraire
à celui où l’impetus la pousse, cet impetus va sans cesse décroissant. »

Les corps reçoivent l’impetus en proportion de la quantité de matière
qu’ils contiennent : l’impetus serait donc, dans les conceptions actuelles,
proportionnel à la masse inerte mais ce serait une sorte de masse inerte
active \footnote{On distingue actuellement la masse grave active qui crée le champ de gravitation
et la masse passive qui le subit. Avant la Relativité générale, l’une et l’autre
de ces masses restaient théoriquement distinctes de la « masse inerte » qui, d’après

la loi F: = my, manifeste la résistance d’un corps d’épreuve à toute accélération.}.
%38

Duhem voit dans la notion d’impetus une préfiguration de celle
d'inertie.

Nous souscrivons, pleinement au contraire, à l’opinion de A. Koyré
qui pense qu’une préfiguration de la notion d’inertie ne peut s’introduire
tant que l’on conserve la théorie aristotélicienne d’un « mouvement
processus », d’un « mouvement altération ». Tant qu’il est ainsi conçu,
— et Buridan admet ces principes, — il reste nécessaire qu’une action
quelconque entretienne ce mouvement. Buridan, selon A. Koyré \footnote{A. Koyré. Études galiléennes, vol. I, cf. pp. 10 à 16.},
« substitue à une force extrinsèque une qualité intrinsèque ». L’explication
est sans doute supérieure mais elle n’engendre pas « l’idée d’inertie ».
Pour cela, il est nécessaire que le mouvement devienne d’abord un éfat,
état qui n’est pas intrinsèquement différent du repos et qui, par conséquent,
ne requiert aucune action spécifique pour persévérer.

C’est encore en utilisant l’impetus, cette pseudo-inertie, que Giordano
Bruno veut justifier un authentique Principe de Relativité.

Il imagine deux hommes: le premier lié à un navire en marche ; le
second immobile sur la rive. Leur situation est telle qu’ils peuvent, à un
instant donné, «avoir leurs mains en un même point de l’air. Ils laissent
alors, l’un et l’autre, tomber une pierre sans lui donner aucun élan: La
pierre lâchée par le premier tombera à la verticale. Au contraire, abandonnée
par le second, elle tombera vers\l’arrière. «Ce qui ne provient de

 

.  F1G. 8. — Les trajectoires de deux pierres lâchées d’un même point sans
vitesse initiale, ne coïncident pas si les origines appartiennent à des mécaniques
différentes.
%39

rien d’autre, explique Bruno, que de ce que la pierre qui part de la main
de celui qui est porté par le navire, et par conséquent se meut selon le
mouvement de celui-ci possède une certaine vertu imprimée que ne possède
pas l’autre. »

A cette explication par l’impetus, Bruno surajoute un commentaire
relativiste qu’il juge équivalent.

« De cette diversité nous ne pouvons donner aucune raison, sinon
celle que les choses qui sont rattachées au navire se meuvent avec celui-ci;
et que l’une des pierres porte avec elle la vertu du moteur, tandis que
l’autre n’y a pas participation. D’où l’on voit très clairement que la
pierre ne reçoit la vertu d’aller en droite ligne, ni du point dont elle part,
ni du point où elle va, mais de l’efficace de la vertu qui lui fut imprimée \footnote{Giordano Bruno, La Cena de le Ceneri, III. 5. Opere Italiane, 1830, v. 1,
p. 171.}. »

Cette argumentation suppose, en somme, que l’impetus vient combler
la différence entre le mouvement du navire et le mouvement des corps
« séparés » qui se meuvent avec lui. L’impetus est utilisé par Bruno
comme une sorte de garantie de la validité de la notion de système
mécanique. En même temps, il permet d’entériner une relativité du
mouvement.

Mais la remarque précédente de A. Koyré s’applique encore : Si la
notion d’impetus s’accorde avec les idées d’inertie, de « système mécanique »,
de Relativité, elle ne saurait fonder ni les unes ni les autres et
ne leur ést nullement équivalente. Une authentique inertie, une authentique
Relativité n’ont aucun besoin d’avoir recours à des « vis Impressa »
pour les justifier. Leur nature originale est, précisément, de s’affranchir de
tout substrat qu’on pourrait leur surajouter. C’est pourquoi l’impetus
qui restera, nous le verrons, impuissant à fonder les notions d’inertie et
de Relativité dans la physique de Galilée ne réussit pas davantage à
justifier la notion de système mécanique dans celle de Bruno.

Quant au fameux exemple de la pierre tombant du haut du mât d’un
navire, il demeure encore une simple expérience de pensée à laquelle se
réfèrent les partisans du mouvement absolu. Ces affirmations vont se
heurter aux objections de Galilée :

« Vous dites : puisque, lorsque le navire est immobile, la pierre tombe

 au pied du mât, et lorsqu'il se meut, elle tombe loin du pied; inversement

du fait que la pierre tombe au pied du mât, on infère que le navire est
%40 M. À. TONNELAT
immobile... et ainsi, de la chute de la pierre près du pied de la tour
s’infère l’immobilité de la terre. N’est-ce pas là votre raisonnement ?... \footnote{Dialogo II, Opere ltaliane, V-VIX, p. 169.} »

Des observations réelles ont-elles été tentées pour justifier ce principe?
Elles pourraient l’avoir été mais seraient bien inutiles. Dans les Dialogues,
Simplicio — représentant les aristotéliciens — et Salviati — parlant
pour Galilée — reprennent le classique débat de la pierre tombant du
haut du mât :

Salviati \footnote{Dialogo II, p. 169.} : « Très bien. Avez-vous déjà fait cette expérience avec le
navire? »

Simplicio : « Je ne l’ai pas faite; mais je crois bien que les auteurs
qui la produisent comme argument l’ont soigneusement observée;
d’ailleurs la cause de la différence se reconnaît avec tant de clarté qu’elle
ne laisse aucun doute. »

Personne n’a jamais fait cette expérience, réplique Salviati \footnote{Dialogo II, p. 171. HAE

Galilée a raison : personne n’a jamais fait cette expérience. Néanmoins Antonio

Rocco après la publication des Dialogues n’hésite pas à écrire :

« Che un sasso cadente dall’albero della nave corrente venga direttamente al piede
dell’albero, io non lo credo; e quando le vedessi, m’ingenierei trovarli altra ragione
che la rivoluzione della terra! »

(Antonio Rocco, Esercitazioni filosofiche di Antonio Rocco, Opere, vol. VII,

p. 677).}. Tous
les auteurs se sont référés uniquement à l’autorité de leurs prédécesseurs.
S’ils l'avaient faite, ils auraient vu que la pierre tombant toujours au
pied du mât, on ne peut rien en conclure ni pour ni contre le mouvement
du navire pas plus qu’on ne peut conclure si une pierre tombe au pied
d’une haute tour ni pour ni contre le mouvement de la terre!

Et Simplicio de poursuivre :

Simplicio : «Et vous, cette expérience, l’avez-vous faite pour parler
avec autant d’assurance? Car si ni vous ni les autres ne l’ont faite, la
discussion est oiseuse, puisque là où il s’agit de choses aussi éloignées
de la raison humaine, l’expérience seule peut apporter une décision \footnote{Dialogo IL, p. 169.}. »

Salviati : « Et moi, sans expérience, je suis sûr que l’effet s’ensuivra
comme je vous le dis puisqu'il est nécessaire qu’il s’ensuive; et
j'ajoute en plus que vous-même, vous savez qu’il ne peut s’ensuivre
autrement ; bien que vous prétendiez que vous ne le savez pas. Mais je
%41
suis un si bon accoucheur des cerveaux que je vous le ferai confesser de

vive force \footnote{Dialogo II, p. 171. Cf. Lettre à Ingoli, Opere VI, pp. 542-546.}. »
Ainsi, écrit A. Koyré, alors que l’on voit habituellement dans Galilée

« l'observateur prudent et sagace » \footnote{E. JoUGuET, Lectures de Mécanique, Paris, 1924, vol. I, p. 111.}, le « fondateur de la méthode
expérimentale » \footnote{E. MaAcH, Die Mechanik, p. 127.} «ce n’est pas le porte-parole de Galilée, Salviati,
mais l’aristotélicien Simplicio qui est présenté comme champion de
l'expérience et c’est Salviati, au contraire, qui en proclame l’inutilité \footnote{À. KoYRÉ, Galilée et la loi d’inertie (Act. Scient. et Industr. Hermann,
Paris, 1939, p. 217, IL, 67).}... »
Pour ces choses « si peu éloignées de la raison humaine » le ‘rôle de la
méthode socratique est d’organiser l’expérience après une déduction et
une analyse appropriée.

Ajoutons que cette fameuse expérience de pensée ne fut réalisée qu’en
1641 par Gassendi \footnote{« M. Gassendi ayant été toujours très curieux de chercher à justifier par
les expériences la vérité les spéculations que la philosophie lui propose, et se trouvant
à Marseille avec Monseigneur le Comte d’Allais en 1641 fit voir sur une galère
qui sortit exprès en mer, par ordre de ce Prince, plus illustre par l’amour et la connaissance
qu’il a des bonnes choses que par la grandeur de sa naissance, qu’une
pierre lâchée du plus haut du mât, tandis que la galère vogue avec toute la force
et la vitesse possible, ne tombe point ailleurs qu’elle ne le ferait si la même galère
était arrêtée et immobile: si bien que, soit qu’elle aille, ou qu’elle n’aille pas, la
pierre tombe toujours le long du mât, à son pied et du même côté. Cette expérience
faite en présence de Monseigneur le Comte d’Allais et d’un grand nombre de
personnes qui y assistaient, semble tenir quelque chose du paradoxe, à beaucoup
qui ne l’avaient point vue; ce qui fut cause que M. Gassendi composa un traité
« De motu impresso a motore translato » que nous vîimes de lui la même année
en forme de lettre écrite à M. du Puy. »

(Recueil de Lettres des Sieurs Morin, De La Roche, De Nevre et Gassendi, et
suite de l’apologie du Sieur Gassendi touchant la question De motu Impresso a
Motore Translato. À Paris, chez Augustin Courbé, MDCL, Préface).

Cité par À. Koyré. Études galiléennes, III-64.}. Elle semble avoir connu alors un très grand
retentissement mais son accomplissement n’était plus nécessaire pour
aider la notion d’inertie et le Principe de Relativité à se résoudre en
idées claires, à triompher aussi de l’autorité des vieux dogmes.

