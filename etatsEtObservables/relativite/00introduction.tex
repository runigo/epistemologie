

% Author and supervisor
\begin{minipage}{0.5\textwidth}
\begin{flushleft} \large
%
\end{flushleft}
\end{minipage}
\begin{minipage}{0.5\textwidth}
%\begin{flushright}
Toute philosophie de la nature est seulement de
l’anthropomorphisme. L’homme pour sauvegarder son
unité personnelle répartit sur toute chose ce qu’il n’est pas.

\hfill {\sc W. G{\oe}the}

%\end{flushright}
\end{minipage}


\section{Introduction}
%{\bf }{\bf --}{\it } {\footnotesize XIX}$^\text{e}$
%CHAPITRE PREMIER

%RELATIVISME ET RELATIVITÉ
%DE L'ANTIQUITÉ A LA RENAISSANCE

L'idée de Relativité, comme la plupart des concepts de la Physique
moderne est une notion vivante dont le sens s’est précisé petit à petit
au cours d’un développement tourmenté et incertain. Au risque de
décevoir certains spécialistes, nous affirmerons donc qu’il n’existe pas
une authentique et imperfectible Relativité dont nous nous proposerions
de rechercher l’esquisse dans les premiers développements des théories
scientifiques. Aucune ébauche imparfaite mais prometteuse, n’attend sous
le voile des ignorances et des préjugés une sorte d’investiture. Cette idée
même est antirelativiste.

La Relativité générale, nous le verrons, est une théorie physique, celle
des phénomènes de gravitation; la Relativité Restreinte constitue une
nouvelle cinématique; la Relativité, sans autre qualificatif, est presque
un état d’esprit souvent confondu avec l’exigence même d’une explication
 rationnelle dans les Sciences.

Bien loin d’être « instinctive », la notion de Relativité, comme celle
d'inertie qui lui est souvent associée, a demandé un renversement des
évidences, une « mutation de l’intellect humain ». Ce qui était inconcevable
devient naturel, puis familier. Née dans la confusion de l’aristotélisme
finissant, rénovée par les contradictions attachées à un insaisissable
éther, l’idée de Relativité semble chaque fois liée davantage à ce
qui la suit qu’à ce qui la précède. Vision novatrice, elle éclaire son propre
chemin et même, dans une large mesure, en définit les méandres et en
détermine l’approfondissement.

