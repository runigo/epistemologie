\section{La notion de système Mécanique. Giordano Bruno}
%{\bf }{\bf --}{\it } {\footnotesize XIX}$^\text{e}$
%30

\footnote{Giordano BRUNO (1568-1600), admet le caractère infini de l’univers copernicien. Emprisonné par l’inquisition en 1593, Bruno est brûlé à Rome le 17 février}

Pour Giordano Bruno, la nature du mouvement des graves pose un
problème d’une importance comparable à celle que soulèvera plus tard,
la propagation des actions électromagnétiques. La finalité qui s’attache
au mouvement est alors inséparable des propriétés intrinsèques du lieu
et, par conséquent, d’une forme de cosmologie.

« Le lieu est un néant et n’exerce aucune force » explique le copernicien
Gilbert. C’est Giordano Bruno qui, en un sens quasi moderne,
précise le concept de système mécanique ou de solide de référence.

Selon Bruno, un système mécanique est formé par un ensemble de
corps liés non par une même nature mais par leur participation à un
mouvement commun. Cette notion de système mécanique est donc
relative à un mouvement, soit réel, soit hypothétique. Par exemple, la
pierre qui tombe du haut du mât d’un navire et ce navire lui-même
forment deux systèmes mécaniques distincts en ce qui concerne la chute
de la pierre par rapport au navire; ils constituent par contre un système
mécanique unique si l’on considère le mouvement du bateau par rapport
à la rive (fig. 7).


FiG. 7. — La pierre tombant du haut du mât aboutit au pied de celui-ci
quel que soit le mouvement du bateau par rapport à la rive.
. «Toutes les choses qui se trouvent sur la terre se meuvent avec la terre. La
pierre jetée de la hune reviendra en bas de quelque façon que le navire se meuve. »

Giordano BRUNO (La Cena de le Ceneri).

%31

Une conclusion s’impose donc à Bruno : il est impossible de déceler
le mouvement d’un système mécanique par des expériences réalisées à
bord de ce système lui-même.

« Toutes les choses qui se trouvent sur la terre se meuvent avec la
terre. La pierre jetée vers la hune reviendra en bas de quelque manière
que le navire se meuve » \footnote{Giordano BRUNO, La Cena de le Ceneri NII, 5. Opere Italiane, 1830, p. 170.
« Con la terra dunque si muovano tutte le cose che si trovano in terra. »} affirme Bruno à partir de cet exemple
fameux. D’une manière analogue, il est impossible de faire apparaître
dans l’air calme, une dissymétrie dans la trajectoire d’un boulet selon
qu'il est tiré vers l’Est ou vers l’Ouest.

Pour Giordano Bruno, la participation des corps au mouvement de
la terre ne s’explique plus du tout par le recours à une même « nature »,
mais par leur appartenance à un même système Mécanique. Ainsi se
trouve résolu, du même coup, le problème dela chute des graves (sur lequel
nous reviendrons). Les lieux se déterminent par rapport à un système
Mécanique et, comme tels, ne caractérisent plus intrinsèquement le
cosmos. L’« infinitisme » de l'Univers de Bruno, va d’ailleurs lier ses
thèmes physique, cosmologique, métaphysique, et rejeter l’organisation
médiévale du cosmos aristotélicien.

Dans tous les exemples proposés par Tycho, discutés par Bruno,
exemples qui seront de nouveau repris par Kepler, la source (c’est-à-dire
le haut du mât dans le premier exemple, le canon dans le second) et
l'observateur appartiennent, l’un et l’autre, au même système mécanique
en ce a concerne l’effet qu’il s’agit de mettre en évidence. Certes, la
pierre tombe du haut du mât mais, par rapport à la translation dans le
sens de la marche qu’il s’agirait de déceler, le navire, le mât et la pierre
forment un ensemble solidaire ; ils se comportent comme un tout pour un
observateur de la rive. D’une manière analogue le boulet de canon se
meut par rapport à la terre et, en ce qui concerne ce mouvement, constitue
un système mécanique distinct. Néanmoins, pour un observateur
terrestre, les systèmes physiques liés à chaque boulet sont identiques à
l'orientation près, que le boulet soit tiré vers l’Est ou vers l’Ouest \footnote{Les effets du mouvement de rotation de la terre sur elle-même sont assimilés
ici, bien entendu, à ceux d’un mouvement rectiligne et uniforme. Aucun effet
ES 

d’accélération de Coriolis n’est supposé intervenir. Cette accélération, en w /\ v

(@, vitesse angulaire de la terre, D = vitesse du mobile) introduit, au contraire,

une dissymétrie dans le mouvement Est-Ouest des projectiles.}.
Aucune dissymétrie ne peut donc lui apparaître.
%32

Ainsi, renouvelant les vieux problèmes, un aviateur immobile ne
verrait pas la terre défiler sous les ailes de son appareil. Un avion dépourvu
de vitesse relative forme avec la terre un même système mécanique, qu’il
soit au sol ou dans les airs \footnote{Ou plutôt «en altitude ». Un entraînement de l’avion par l’intermédiaire
de l’air ne joue bien entendu aucun rôle. La notion de système mécanique est indépendante
de l’existence d’un milieu.}. Autrement dit, sa vitesse demeure celle
de la terre et aucune expérience réalisée à bord de l’appareil ne peut
déceler le mouvement terrestre.

La notion d’expérience interne, celle de système mécanique sont
strictement équivalentes pour entraîner cette inaptitude à déceler tout
mouvement étranger. Elles sont toutefois introduites dans des sens assez
différents que nous devons préciser ici:

1) Si la source et l’observateur appartiennent au même système
mécanique et si, de plus, le mobile qui va de l’une à l’autre est constamment
solidaire de ce système par rapport à un mouvement commun,
toute expérience réalisée sur ce mobile est par définition une expérience
interne « vraie » ou, si l’on préfère, une expérience interne « au sens
fort ». Par définition, elle ne peut mettre en évidence un mouvement
auquel source, observateur, mobile participent également.

Tel est le cas de la pierre tombant du haut de la hune, des boulets
tirés vers l’est ou vers l’ouest. Ce ne sera jamais le cas des expériences
réalisées sur le futur éther, sauf si l’on admet que celui-ci est entraîné
totalement par le mouvement de la terre.

2) Au contraire, si la source et l’observateur appartiennent au même
système mais si le mobile observé parvient, à un instant quelconque de
l’expérience, à se désolidariser du mouvement commun, la notion d’expérience
interne au sens strict et, avec elle, la notion de système mécanique
ne peut être maintenue. On pourra parler encore d’expérience interne
en ce sens que l’observateur et la source appartiennent au même système
physique mais il s’agit alors d’une « expérience interne apparente » Ou,
si l’on préfère, d’une expérience interne «au sens faible ». Bien entendu
on peut et même on doit déceler par une expérience interne de ce genre
le mouvement relatif auquel source et observateur participent tous les
deux.

Tel serait le cas des expériences classiques si la pierre au lieu de tomber
du haut du mât, venait à ricocher sur la rive, si le boulet de canon
était freiné par l’air dont on veut déterminer la vitesse. Tel sera toujours
%33
le cas des pseudo-expériences internes destinées à détecter le mouvement
de la terre par rapport à un éther immobile ou partiellement entraîné.
La source et l’observateur appartiennent au même système mais la
lumière émise est solidaire de l’éther, système par rapport auquel il s’agit
de déceler un mouvement. Un résultat positif est alors non seulement
compatible avec les postulats de Bruno et de Galilée mais il est impliqué
par eux.

Au contraire, dans le cas d’authentiques expériences internes, le
refus qu’oppose la nature à toutes les tentatives destinées à mettre en
évidence le mouvement d’un système mécanique par des expériences
réalisées à bord de celui-ci, dénie à ce mouvement un caractère physique
intrinsèque.

