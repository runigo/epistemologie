\subsection{La révolution Copernicienne, prélude à l’idée d’Équivalence}
%{\bf }{\bf --}{\it } {\footnotesize XIX}$^\text{e}$

Sans postuler, au sens actuel, aucun principe Relativiste, l’astronomie
issue de la théorie de Copernic allait porter une atteinte décisive à la
hiérarchisation du cosmos et favoriser un nouvel état d’esprit.

\subsection{Du cosmos d’ Aristote à celui de Copernic}

Aristote suppose que tout corps simple possède nécessairement un
mouvement simple. Or, parmi les mouvements simples, nous devons
distinguer les mouvements rectilignes et les mouvements circulaires. Les
mouvements rectilignes, dirigés vers le bas, c’est-à-dire vers le centre,

%26

caractérisent la terre et l’eau c’est-à-dire les graves; les mouvements
rectilignes, dirigés vers le haut, sont l’apanage de l’air et du feu. Ainsi,
les mouvements rectilignes se distribuent entre les quatre éléments. Les
mouvements circulaires vont au contraire caractériser les corps célestes \footnote{ARISTOTE, De Celo, L I, 2; Physique, II, 1 et V, 2.}.
L’enseignement d’Aristote va influencer Ptolémée d’Alexandrie :

« Si donc, écrit celui-ci, la terre tournait en une révolution quotidienne...
ce mouvement devrait être extrêmement véhément et d’une
vitesse insurpassable. Or les choses mues par une rotation violente
semblent être totalement inaptes à se réunir mais plutôt devoir se disperser
à moins qu’elles ne soient maintenues en liaison par quelque force.
Et depuis longtemps déjà, la terre dispersée aurait dépassé le ciel même
(rien n’est plus ridicule); à plus forte raison, les êtres animés et toutes les
autres masses séparées qui ne pourraient aucunement demeurer stables.
Mais, en outre, les choses tombant librement n’arriveraient pas non plus
en perpendiculaire au lieu qui leur fut destiné, lieu entre temps retiré
avec telle rapidité de dessous d’elles. Et nous verrions ainsi toujours se
porter vers l’occident les nuages et toutes les choses qui flottent dans
l’air \footnote{PTOLÉMÉE, A/mageste, I. 7.}. »

\subsection{L’argumentation Copernicienne}

Le système de Copernic, exposé dans le De Revolutionibus, paraît
en 1543. Copernic, croit-on, en aurait reçu sur son lit de mort, le premier
exemplaire imprimé. La publication de l’ouvrage présente d’ailleurs
une histoire tourmentée.

Copernic aurait eu très tôt l’idée de son système mais l’aurait tenue
secrète pendant quatre fois neuf ans, si l’on en croit une lettre-Préface \footnote{Lettre-préface au pape Paul III.}.
Vers 1512 circulait déjà, dans un cercle restreint, un résumé des principes
de la nouvelle théorie \footnote{De hypothesibus motuum cœlestium a se constitutis commentariolus.}. En dépit de l’absence d’objections venues de
milieux ecclésiastiques, en dépit même d’encouragements émanant d’éminentes
personnalités, Copernic opte pour la prudence et diffère la publication
de son système du monde. Vers 1540, il confie son ouvrage à
Rhétieus qui en donne aussitôt une « Narratio Prima » \footnote{Celle-ci est résumée dans une lettre à Johann Schôner, Dantzig, 1540.}. Le succès
est tel que Rhétieus se décide à faire publier l’œuvre entière sous la
responsabilité du théologien luthérien Osiander. Effrayé par l’audace
%27
des idées coperniciennes, Osiander réussit à en édulcorer la portée
dans une célèbre préface qui passa longtemps pour exprimer les opinions
de Copernic lui-même. On y expliquait que le but de la science, particulièrement
de l’astronomie, se réduisait à retrouver les apparences et non
pas à connaître les causes. Celles-ci ne peuvent que lui échapper. Le
système de Copernic représente ainsi une méthode simple qui ne prétend
nullement accéder à une vérité.

Bien entendu, l’opinion de Copernic ne fut jamais celle qu’exposa
Osiander. Le monde de Copernic ne peut être assimilé à un artifice de
calcul, fût-il des plus subtils. Certes, il ne prétend pas donner une description
complète ni définitive du cosmos mais le réalisme simple de
Copernic pense accèder à un aspect important d’une réalité objective.
Elle se traduit par le calcul mais ne se réduit aucunement à celui-ci. Ainsi
est pleinement justifiée la sévérité du jugement de Joseph Bertrand \footnote{Joseph BERTRAND, Les fondateurs de l’astronomie moderne (Paris, 1865, p. 51).} à
l’égard de la préface d’Osiander : « Ces lignes dans lesquelles la prudence
simule le scepticismé sont la négation de la science ».

En dépit de la simplicité logique du système, l’argumentation copernicienne
est souvent malaisée. Elle ne peut s’affranchir d’emblée d’un
vocabulaire aristotélicien. Qu’opposer, en particulier aux objections de
Ptolémée : la rotation de la terre ne devrait-elle pas entraîner une dispersion
des corps qui lui sont liés?

La réponse de Copernic consiste à postuler que le mouvement de la
terre est « naturel » et non pas « violent », que les corps de provenance
terrestre mais séparés de la terre lui sont néanmoins physiquement
reliés. « Quant aux choses qui tombent ou qui s’élèvent, leur mouvement
doit être double et généralement composé de rectiligne et de circulaire \footnote{N. CoperniC, De Revolutionibus, 1543.}. »
Le circulaire s’unit au rectiligne « comme la maladie à l’animal »
écrira-t-il, prévoyant ainsi les difficultés qui vont différer l’avènement
du Principe d’inertie.

Pourtant, sur ce point particulier, l’argumentation de Copernic
reste faible : si le mouvement d'Occident en Orient est « naturel aux
choses terrestres », on ne voit pas pourquoi il est sans influence sur les
mouvements des graves. Néanmoins le raisonnement de Copernic contient
en lui-même le germe d’une application générale de la Mécanique
et l’amenuisement de la division du cosmos en régions supra et sublunaires.

