\subsection{Les Réticences. Tycho Brahé}
%{\bf }{\bf --}{\it } {\footnotesize XIX}$^\text{e}$
%28

Cette évidente régression de l’espace absolu va se heurter aux objections
de Tycho Brahé. Celui-ci connaît parfaitement le système de
Copernic à l’époque (1588) où il énonce ses propres idées \footnote{De mundi aetheri recentioribus phænonxenisi liber secundus. On sait que le
système de Tycho consiste à faire tourner les planètes autour du soleil et celui-ci
autour de la terre.}. Il se
rallie à l’interprétation des conceptions coperniciennes que lui propose
la Préface d’Osiander. Il s’agit d’un simple artifice destiné à faciliter la
description mathématique des apparences.

Sans doute l’attitude de Tycho est-elle surtout dictée par les incompatibilités
qu’il pressent entre la Bible et les énoncés strictement coperniciens.
Il faut néanmoins ajouter que pour l’habile observateur qu’est
Tycho, l’impossibilité de mettre en évidence une parallaxe des fixes joue
également contre l’hypothèse du mouvement de la terre \footnote{Cf. par exemple la polémique entre Tycho et le copernicien Rothmann.TYCHONIS BRAHÉ, Astronomicarum Epistolarum biber. Urenienburg, MDXCII,
p. 188, éd. Dreyer, 1919, p. 218. Notons que Tychonis (1546-1601) était très exactement
le contemporain de Bruno (1548-1600).}. « Il est faux,
écrit-il que deux mouvements l’un circulaire, l’autre rectiligne, puissent
se composer sans se gêner comme le prétendait Copernic. Il est faux
« que la partie séparée d’un tout en conserve la vertu. Bien au contraire,
on peut dire qu’elle ne le fait jamais » \footnote{Tycho, op. cit., p. 189-219.}.

C’est Tycho qui propose la célèbre expérience des boulets de canon
faisant ainsi appel à des procédés très modernes inspirés par l’invention
alors très récente du canon (fig. 6).

« Or que se passerait-il, je te le demande écrit-il, si d’un grand canon,
on tirait un boulet vers l'Orient... ; et puis, du même canon, et du même
lieu on en tirait un autre vers l'Occident? Peut-on croire que l’un et
l’autre franchiraient sur la terre des espaces égaux? »

L’aristotélicien Tycho ne le pense pas « Par suite du mouvement
diurne extrêmement rapide de la terre (s’il y en avait un), l’obus tiré vers
l’Orient ne pourrait jamais franchir autant d’espace sur la surface de la
terre, la terre (de son mouvement propre) venant au-devant de lui, que
celui qui de la même manière serait lancé vers l'Occident ».


