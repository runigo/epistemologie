\subsection{La Physique d’Aristote et la notion de mouvement absolu}
%{\bf }{\bf --}{\it } {\footnotesize XIX}$^\text{e}$
%20

La physique d’Aristote se présente comme une entreprise beaucoup
trop structurée pour ne pas avoir aperçu, comme on le croit trop souvent,
que tout mouvement local devait être à la fois relatif et absolu. Il est
relatif parce que la retranscription d’un mouvement quelconque dans
l’expérience d’un autre observateur procède d’une comparaison et n’a
aucun sens intrinsèque. Il est absolu parce que cet indéniable relativisme
de fait n’atteint pas l’essence du mouvement qui est altération et changement.

Cette affirmation nous apparaît comme un postulat gratuit dans le
domaine de notre physique. Par contre, elle découle nécessairement de
la métaphysique d’Aristote. En effet, sa Mécanique est accrochée à la
conception du premier moteur. Elle en retire son caractère essentiel. Les
attributs du mouvement ne peuvent donc constituer une convention
accessoire ni un postulat surajouté.

Une description authentique du mouvement requiert ainsi un terme
absolument immobile, origine ou aboutissement. Le mouvement d’une
pierre est un phénomène original qui possède sa finalité propre. On
pourrait dire qu’il est suspendu à la physique du cosmos tout entier.
La notion expérimentale de mouvement relatif, même s’il s’agit de déplacement
rectiligne et uniforme, n’épuise donc aucunement l’essence du
mouvement.

En admettant les conceptions aristotéliciennes, le caractère absolu du
mouvement n’est pas forcément lié à l’apparition de manifestations qui
traduiraient sa présence par des caractères secondaires mais, somme
toute, accessoires. Les « effets physiques du mouvement », inséparables
de la dynamique newtonienne, ont une interprétation bien différente
dans la physique d’Aristote. L’état de mouvement est intrinsèquement
différent de l’état de repos comme l’état solide est différent de l’état
liquide. Tout changement (accélération, décélération) dans le régime du
mouvement, tout « effet » dont la nature est étrangère au mouvement
lui-même, manifeste l’intervention de contraintes. Pourtant, les actions
violentes ne sont nullement nécessaires pour définir le mouvement
absolu. Celui-ci se reconnaît plutôt dans une unité d'intention qui donne
à l’expérience une interprétation quasi-organique. Ainsi un voyageur
contemple un paysage fugitif et mouvant qu’il « sait » pourtant immobile,
%21
non pas en raison des accélérations au départ et à l’arrivée, mais par le
fait que départ et arrivée constituent le début et la fin d’une intention
que régit une finalité. Tout mouvement implique cette intention, cette
finalité qui dépasse à vrai dire une physique du cosmos, mais dont la
cinématique (vitesse relative) laisse échapper la véritable nature.

Au cours de cette longue période qui va s’achever à la Renaissance,
avec les travaux de Kepler, de Bruno, de Galilée et de Descartes, le caractère
absolu du mouvement est plus ou moins lié à des conceptions de ce
genre.

