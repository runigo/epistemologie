\section{Mouvement relatif. Mouvement absolu}
%{\bf }{\bf --}{\it } {\footnotesize XIX}$^\text{e}$

\subsection{Mouvement relatif}
Les observables les plus immédiates que manifeste
l’expérience se rapportent à la position et aux mouvements des corps.
Dire qu’il s’agit d’observables signifie ipso facto que ces notions se
rapportent à l’observateur \footnote{Dans cet ouvrage nous utiliserons à maintes reprises le mot « observateur ».
I ne doit pas nous faire illusion. Ce terme ne recouvre pas l’idée d’un être pensant
susceptible d’ériger ses perceptions en opinions subjectives, de les déformer éventuellement.

L’observateur est le support désincarné, mais cependant matériel, auquel est lié
un système de référence et, avec celui-ci, une possibilité de mesure. L’observateur
est muni de règles et d’horloges mais dénué de passion. Il effectue des mesures à
l’aide de ses instruments, il observe des coïncidences. Son « humanité » s’arrête là.

Nous emploierons toujours le mot « observateur » dans ce sens qui ne comporte
ni subjectivisme, ni jugement de valeur. Nous ne reviendrons plus sur l’absence
de tout caractère anthropomorphique de tels observateurs.}. Il ne s’agit aucunement des impressions
purement subjectives de celui-ci. Pour éviter tout abus de langage, nous
dirons que ces notions traduisent immédiatement les mesures réalisables
dans un système de référence lié à l’observateur.

Le célèbre exemple du mouvement d’une bille sur le pont d’un
bateau en marche, mouvement différent suivant qu’on l’observe du pont
ou de la rive, ne donne lieu à aucune discussion tant qu'il s’agit des
apparences pures et simples de ce mouvement. Les notions de vitesse
relative, de mouvement relatif inhérentes à toute mesure n’ont jamais
prêté à discussion, que le physicien se nomme Aristote, Galilée ou Einstein.
La description d’un mouvement « relatif » est réalisée par des expériences
portant d’un système sur l’autre.

Il va de soi que ce mouvement relatif (évaluation des coordonnées du
point « mobile » dans le système « fixe ») s’accompagne d’une totale
réciprocité.


HISTOIRE DU PRINCIPE DE RELATIVITÉ 19

Il est plus important de souligner que la détermination d’un mouvement
relatif ne peut s’effectuer que par rapport à un système de référence
matériel. 1 exige l’intervention d’un « solide de référence ». En effet,
toute expérience portant d’un système sur l’autre exige une interaction
par l’intermédiaire d’un signal lumineux ou d’un mobile matériel. C’est
en ce sens matérialiste que doit être comprise la notion de mouvement
relatif.

\subsection{Mouvement absolu}

Le mouvement relatif se rapporte à un repère arbitraire. Il ne peut
nous renseigner sur le lieu absolu du mobile ni sur la direction intrinsèque
de son mouvement. On peut penser — et telle a été pendant
longtemps l’opinion admise — que le mouvement relatif ne pouvait
exprimer qu’une apparence cinématique. Il manifesterait — sans épuiser
sa nature — un véritable changement de lieu qui aurait constitué le
mouvement absolu.

« Tout d’abord, écrit Maxwell \footnote{MaxweLL, Matter and Motion, Préface.}, nous pouvions penser que notre
condition d’êtres conscients impliquait, comme éléments essentiels du
savoir, la connaissance absolue du lieu dans lequel nous nous trouvions
et de la direction de notre mouvement. Mais cette opinion qui était
indubitablement celle de nombreux Sages de l’Antiquité a disparu de
plus en plus des conceptions du physicien. Dans l’espace, il n’existe pas
de bornes kilométriques ; une portion d’espace est exactement semblable
à une autre, de sorte que nous ne pouvons pas savoir où nous sommes.
Nous nous trouvons dans une mer sans vagues et sans étoiles, sans
boussole ni soleil, sans vent ni marée, et nous ne pouvons dire dans
quelle direction nous allons. Nous n’avons pas de données que nous
pourrions exploiter pour faire un calcul. Nous pouvons, bien entendu,
déterminer notre mouvement par comparaisons avec les corps voisins
mais nous ignorons ce qu’est le mouvement de ces corps dans l’espace. »

La détermination du mouvement absolu dans l’espace absolu ne peut
donc recevoir, selon certains auteurs, qu’une solution « analogue à celle
du mouvement perpétuel et de la quadrature du cercle » \footnote{CassIReR, Einstein’s theory of Relativity, Dover Public. 1953, p. 408.}. Il fallait
transformer cet « aspect négatif », cette « limitation du savoir » en un
principe de connaissance. L’histoire de la Relativité est celle de cette
acquisition.

