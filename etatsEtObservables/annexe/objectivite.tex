\subsection{Encyclopédie}
%{\bf noumène} : terme (en grec {\it to nooumenon},

{\bf objectivité} : caractère de ce qui est objectif.
La notion est corrélative de celle de
subjectivité et lui est souvent opposée.
Elle sert à caractériser en général tous les
processus qui se déroulent indépendamment
de l'intervention d’un sujet. Est
objectif ce qui relève donc de l’expérience
externe au sujet ou de l’expérimentation
scientifique. On notera qu’on tend à présenter
la science moderne comme un
effort de connaissance objective de la réalité.
On peut distinguer entre deux
grandes conceptions de l'objectivité
l'une consiste à poser une indépendance
absolue des phénomènes objectifs relativement
au sujet qui les observe ou les
conçoit, c’est en ce sens que le mathématicien
et astronome Ampère peut déclarer
que « les lois mathématiques du mouvement
des astres réglaient leur mouvement
depuis que le monde existe et bien avant
que Kepler ne les ait démontrées ».
L'autre, qui en tenant compte du caractère
construit de l’objectivité scientifique,
considère que le sujet participe en
quelque manière de la détermination de
l’objet. Ainsi le phénomène, au sens kantien,
est objectif en ce sens qu’il procède
de l’application des formes pures de l’intuition
sensible (espace, temps) et des
catégories pures de l’entendement à la
matière empirique. Ainsi par le jeu des
catégories, l’objectivité des objets relève
de l’activité du sujet qui détermine {\it a
priori} la forme de l’objet. C’est seulement
ainsi selon Kant que l’on peut poser l’universalité
stricte des lois physiques et se
soustraire aux critiques relatives à l’induction
du type de celle de Hume. Si l’idée
qu’un entendement dont les catégories
fixées une fois pour toutes dicteraient par
avance au donné son mode de manifestation
et fonderaient dans l’activité du sujet
l’objectivité a suscité une postérité chez
les philosophes, postérité dont témoignent
les divers courants de philosophie
transcendantale des {\footnotesize XIX}$^\text{e}$ s. et du {\footnotesize XX}$^\text{e}$ s.,
elle a plutôt été retenue par les épistémologues
et les philosophes des sciences
pour ses aspects constructivistes. C’est
ainsi que s’est imposée l’idée que l’objectivité
scientifique résulterait d’un processus
d’objectivation par le sujet des concepts
dont il fait usage dans le domaine scientifique
considéré. Cette conception permet
de donner aux constructions des sciences
humaines (sociologie, économie, psychologie)
une « objectivité », celle de ses
concepts fondamentaux et des relations
qu'on peut établir entre ces concepts, analogue
à l’objectivité reconnue dans les
sciences de la nature.

%—> Kant e transcendantal

