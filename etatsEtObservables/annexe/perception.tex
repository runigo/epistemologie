\subsection{Encyclopédie}
%1235
%perception
%{\bf }{\bf --}{\it }
Le français autorise un usage
assez large du terme de perception. On
dit par exemple qu’une personne est
« mal perçue » par son entourage professionnel
pour signifier que celui-ci la prend
en mauvaise part, ou encore que ce qu’a
dit un orateur n’a pas été « perçu » pour
dire qu’il n’a pas été compris. De l’œuvre
d’un artiste, on pourra dire qu’elle est le
reflet « d’une perception esthétique originale »,
voulant ainsi désigner la spécificité
du point de vue de l’artiste sur le réel. En
un autre sens, le plus courant sans doute,
le médecin dira de celui dont le sens de
la vue ou de l’ouïe est déficient, qu’il est
victime de « troubles de la perception ».
La perception désigne alors la faculté de
prendre connaissance de son environnement
au moyen des sens. Cela suppose à
tout le moins une certaine activité du
sujet percevant dont les organes sensoriels
doivent en outre être conformés de
manière à pouvoir recevoir des impressions.
À cela s'ajoute l'ambiguïté du
terme de perception, dans la mesure où il
peut désigner aussi bien la représentation
du perçu ou « percept » que l'{\it acte} par
lequel ce contenu de perception est
donné, conduit à poser la question de la
part d’activité du sujet percevant et ce faisant
celle des liens entre l’âme et le corps,
entre le psychologique et le physiologique.
C’est pourquoi, bien que voisine de
celle de sensation, la notion de perception
ne se confond toutefois pas avec elle : la
perception connote davantage l’activité
organisatrice du sujet qui perçoit des
impressions sensorielles. D’ailleurs l’étymologie
latine, {\it perceptio}, qui désigne au
sens premier l’action de recueillir puis,
par métaphore, la saisie cognitive ou la
connaissance ajoute à cette valeur active
une connotation intellectuelle. Le terme
de perception s’est progressivement
imposé comme synonyme d’idée dans le
latin philosophique et par suite dans le
français de l’Âge classique. Chez Descartes
notamment, {\it perceptio} est même
Synonyme de {\it conceptio} ou d’{\it inspectio
mentis} (inspection de l'esprit) et désigne
de manière générique tous les actes de
l'intelligence. Ainsi la vérité est-elle définie
comme « une {\it perception} claire et distincte »,
et les cartésiens (Malebranche,
Arnauld, Spinoza, Leibniz) utilisent perception
concurremment avec les termes
d'idée ou de conception. Aussi le sens qui
s’est imposé par la suite, notamment en
%
psychologie, pour désigner et l'impression
qui se produit en présence d’un objet et
la faculté qui permet cette prise de
conscience ne doit-il pas occulter totalement
les emplois, sans doute un peu vieillis
mais fréquents dans la langue
philosophique, où perception est synonyme
d’idée ou de représentation intellectuelle.

\subsubsection{Perception, sensation connaissance}

La perception, notamment lorsqu'elle
est assimilée à la sensation, est souvent
présentée comme l’une des caractéristiques
des organismes vivants. Déjà, Aristote
considérait que le mouvement
({\it kinésis}) et la sensation (gr. {\it aisthèsis})
étaient les propriétés minimales de tout
être vivant (Aristote, {\it De l'âme} I). En ce
sens, on peut dire d’un animal voire d’une
plante qu’ils perçoivent les modifications
de leur environnement. La perception
(l'acte de la sensation) a donc d’abord
pour fonction de rendre possible la survie
de l’animal ou de la plante en lui procurant
toutes les informations utiles à sa
subsistance. Dans le cas de l’homme, cette
fonction utilitaire de la perception-sensation
ne suffit pas pour autant à en faire
une connaissance fiable du réel dans la
mesure où la sensation semble d’abord
être caractérisée par sa mobilité et une
certaine instabilité : c’est la question abordée
par Platon, notamment dans le {\it Théétête}.
Elle pose aussi le problème classique
de l’articulation entre la condition d’universalité
à laquelle est astreinte toute
connaissance scientifique et la nécessaire
singularité de toute perception. Si les
informations fournies par la perception
sont indispensables pour s'orienter dans
le monde, l’expérience des nombreuses
illusions ou erreurs des sens dont les sceptiques
ont dressé une liste canonique dans
laquelle les philosophes classiques puiseront
leurs exemples (tour carrée paraissant
ronde de loin, bâton droit plongé
dans l’eau paraissant tordu, taille comparative
de la Lune et du Soleil) montre
qu’on ne peut toujours se fier aux données
de la perception. Les sens sont-ils en
eux-mêmes trompeurs ou ne doit-on pas
plutôt incriminer le jugement que l’entendement
forme à partir des données des
sens ? Pour Lucrèce, et toute une tradition
issue du sensualisme épicurien, la
véracité des sens repose sur l’impossibilité
de prouver qu'ils nous trompent : si les
%perception
sens nous trompaient, la raison qui, selon
Lucrèce, est « tout entière issue de la sensation »
serait tout aussi mensongère. Ne
peut-on néanmoins tirer quelque connaissance
certaine de l’analyse rationnelle de
la perception ? Dans la seconde des {\it Méditations
métaphysiques}, Descartes montre
qu'on ne peut se fier aux qualités sensibles
perçues et toujours instables pour
connaître la nature corporelle. L'analyse
du « morceau de cire » a certes pour fonction
de confirmer que l’esprit est plus aisé
à connaître que le corps mais aussi que,
contrairement à ce que l’on croit d’ordinaire,
ce ne sont pas les sens qui perçoivent
mais l’entendement qui seul permet
de donner à l’objet son unité. Aussi Descartes
définit-il la perception non comme
une « vision » ou un « attouchement »
mais comme une « inspection de l'esprit ».
Point n’est cependant nécessaire pour
soutenir le caractère non trompeur des
sens d’affirmer avec Lucrèce que la raison
provient des sens, ni de montrer avec
Descartes qu’on ne peut rien conclure de
certains des qualités sensibles : il suffit de
constater que les sens sont en deçà du vrai
et du faux, « si l’on peut dire que les sens
ne trompent pas, ce n’est point parce
qu'ils jugent toujours juste mais qu’ils ne
jugent pas du tout ».

\subsubsection{Perception et jugement}

La question générale de la valeur des
jugements fondés sur la perception sensible
pose au fond le rôle de la pensée ou
de l’entendement dans la perception.
C’est l’un des aspects discutés au {\footnotesize XVIII}$^\text{e}$ s.
dans le cadre des analyses suscitées par
le problème de Molyneux : l’aveugle de
naissance qui recouvre la vue peut-il discerner
le cube du globe sans une « instruction
préalable » ? Ce fameux problème
posé par le mathématicien irlandais William
Molyneux (1656-1698) était résumé
par Locke ainsi : « Supposez un aveugle
de naissance qui soit présentement
homme fait, auquel on ait appris à distinguer
par l’attouchement un cube et un
globe du même métal, et à peu près de la
même grosseur [...]. Supposez que, le cube
et le globe étant posés sur une table, cet
aveugle vienne à jouir de la vue : on
demande si, en les voyant sans les toucher,
il pourrait les discerner et dire quel
est le globe et quel est le cube » (John
Locke, {\it Essai philosophique concernant
%{\bf }{\bf --}{\it } 1236
l'entendement humain}, livre II, c. 9, \S 8).
Formulée dans les termes de la philosophie
française du {\footnotesize XIX}$^\text{e}$ s., la question posée
est de savoir jusqu’à quel point on peut
dire que « percevoir, c’est interpréter »
(Jules Lagneau) et que la perception est
une « fonction d’entendement » (Alain) ?
Ces conceptions intellectualistes de la
perception, en privilégiant le rôle de la
pensée, risquent cependant de minimiser
le rôle du corps dans la perception. Ainsi
reliée aux sens, la perception paraît impliquer
l'activité synthétique d’une
conscience qui lie en elle les représentations.
Le rôle de l’entendement dans cette
activité de liaison a conduit Kant à poser
une distinction entre les « jugements de
perception », qui n’ont de validité que
subjective et n’impliquent pas un rapport
à l'objet, et « jugements d'expérience »
qui supposent que le jugement soit soumis
à des conditions de nécessité et d’universalité
qui seules peuvent lui donner une
validité objective.

\subsubsection{Le mécanisme de la perception}

Affirmer que toute perception s’accomplit
en fonction de certaines règles ou de
certaines lois est une chose, décrire le
mécanisme qui régit la perception en est
une autre. L’un des aspects du problème
porte sur la nature du rapport entre la
perception et la réalité extérieure qui est
perçue : la perception donne-t-elle des
images de la réalité, en est-elle le signe ou
nous met-elle en présence de la réalité
elle-même ? La représentation la plus
naturelle, quand on évoque le mécanisme
de la perception, consiste sans doute dans
le fait de postuler une ressemblance entre
la réalité et les représentations que l’on
s’en fait par la perception. Cette tendance
à comparer les contenus perçus à des
images mentales fait selon Descartes obstacle
à l'analyse du mécanisme de la perception,
aussi bien d’un point de vue
physiologique que psychologique. À une
conception du mécanisme de la sensation,
fondé sur la ressemblance, il convient
selon lui de substituer une conception
reposant sur une analogie entre la sensation
et le signe. À moins que l’on ne
considère que c’est manquer le phénomène
de la perception que de l’envisager
en termes d’image ou de signe. Au {\footnotesize XX}$^\text{e}$ s.,
le courant phénoménologique de Husserl
combat fermement des théories qu’il
considère « non seulement inexactes, mais
% 1237
absurdes ». Pour Husserl, la perception
est un acte de conscience {\it sui generis} qu’on
ne saurait rendre intelligible en l’assimilant
à une conscience d'image ou à une
conscience de signe. En termes husserliens,
il y a une différence d’essence
infranchissable entre d’un côté la perception
qui saisit la chose extérieure « dans
sa corporéité », pour ainsi dire « en chair
et en os », et de l’autre les vécus de
conscience qui reposent sur une représentation
symbolique, par image ou par
signe, qui ne vise pas la chose directement.
Seule la perception saisit la chose
dans sa présence corporelle, contrairement
à l’imagination ou au souvenir, qui
sont des actes ou des vécus de conscience
qui rendent présent à la conscience
quelque chose qui est absent.

Quoi qu’il en soit, il semble donc que
l’on peut définir la perception comme un
acte spécifique de la conscience qui se distingue
du souvenir ou de la représentation
imaginaire, certes par sa capacité à se
maintenir et à persister indépendamment
de la volonté, mais aussi par cette caractéristique
qui veut que nous ne percevions
les choses que « par esquisses » et dans ce
qu'on peut appeler un champ perceptif :
« Le “quelque chose” perceptif est toujours
au milieu d’autre chose, il fait toujours
partie d’un “champ” » (Merleau-Ponty,
{\it Phénoménologie de la perception}).
L'idée que la perception est un acte qui
doit être envisagé dans sa totalité et sa
spécificité est aussi l’un des arguments
majeurs des tenants de la {\it Gestaltpsychologie}
ou {\it psychologie de la forme}. Issue du
travail d’un groupe de psychologues allemands
du début du {\footnotesize XX}$^\text{e}$ s. (école de Berlin,
Kôhler, Wertheimer, Koffka, Kræœchter
von Kahlpot) qui s’opposaient à la psychologie
classique issue de l’empirisme et
de l’associationnisme, cette théorie considérait
que les formes ou structures ({\it Gestalten})
sont des données premières et
immédiates de la perception. Aussi, la
perception n’est pas l’organisation après
coup de données sensorielles élémentaires
ou pures, mais doit être comprise à
la lumière de la notion de forme. Toutefois,
bien  qu’étroitement liée à la
conscience et à la psychologie, la question
de la perception ouvre aussi sur des
aspects plus fondamentaux comme celui
de la nature de notre rapport à la réalité.

\subsubsection{Perception et existence}

Poussée à l'extrême, la thèse qui
consiste à poser que la réalité extérieure
telle que nous la percevons n’est que l’élaboration
après coup des impressions sensibles
à pu conduire l’évêque anglican
Berkeley à affirmer l’équivalence entre
{\it être} (lat. {\it esse}) et {\it être perçu} (lat. {\it percipi}).
Le {\it esse est percipi} apparaît donc comme
un cas limite de l’idéalisme, ou plus précisément
de l’immatérialisme : l'existence
d’une chose, ou d’une idée consiste dans
le fait qu’elle est perçue. Cette doctrine
affirme qu’« une cerise [...] n’est rien
qu'un assemblage de qualités sensibles et
d'idées perçues par divers sens », idées
qui sont réunies en une seule chose et
sous un seul nom par un acte de l’intelligence
lequel ne permet pas d’en poser
l'existence en dehors de cette perception.
Cette équivalence entre l’existence d’une
chose et le fait qu’on en perçoive les qualités
ne peut se réfuter qu’au prix de l’affirmation
de l’hétérogénéité radicale
entre l’acte de percevoir qui relève des
sens et celui de penser qui relève de l’entendement.
Aussi Kant peut-il affirmer
que « la perception qui procure au
concept sa matière est le seul caractère de
la réalité » (Kant, {\it Critique de la raison
pure}).

Que la perception permette ou non de
décider de l'existence du perçu, elle
concerne aussi la question de l’existence
en cela qu’elle peut modifier profondément
notre rapport sensible à la réalité.
C’est pourquoi il est important de se
demander si on peut éduquer les sens.
Non certes développer les capacités des
organes des sens mais, par exemple, éduquer
l’enfant dans son apprentissage de la
réalité, ou enrichir la sensibilité de
l’homme par la culture artistique. Beaucoup
d'artistes ont d’ailleurs affirmé qu'ils
n'avaient d’autre but que d’affiner leurs
sens, fût-ce à la manière de Rimbaud dans
la lettre dite « du Voyant » en travaillant
à leur dérèglement : « [...] Maintenant, je
m'encrapule le plus possible. Pourquoi ?
Je veux être poète, et je travaille à me
rendre {\it Voyant} : vous ne comprendrez pas
du tout, et je ne saurais presque vous
expliquer. Il s’agit d’arriver à l'inconnu
par le dérèglement de {\it tous les sens} »,
{\it Lettre à Georges Izambard}, Charleville,
mai 1871 ; ou en s’efforçant à la manière
de Matisse de voir toutes choses comme
si on les voyait pour la première fois.

\subsubsection{Perception subliminale}
Perception dont
le sujet percevant n’est pas conscient. En
effet, les stimuli qui la constituent n’atteignent
pas son seuil de perception. La perception
subliminale agit sur le sujet sur le
plan inconscient et préconscient. Cette
propriété a été exploitée par des spécialistes
de la propagande et de la publicité.
Ces utilisateurs de la perception subliminale
ont été désignés comme les techniciens
de la « persuasion occulte ».
Cependant le succès obtenu a été
moindre que celui attendu.

