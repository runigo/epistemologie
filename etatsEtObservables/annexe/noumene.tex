\subsection{Encyclopédie}
{\bf noumène} : terme (en grec {\it to nooumenon},
participe substantivé de {\it noeisthai}, « penser »,
« être pensé ») qui signifie « chose
pensée », « objet de la pensée ». Chez Platon,
le noumène est synonyme de forme
intelligible ou Idée, c’est-à-dire qu’il
désigne ce qui n’est pas du ressort de l’apparence
« visible et tangible », mais qui ne
peut être saisi que par l’entendement
({\it République} VI, 509d ; {\it Timée} 30d, 51d).
La distinction phénomène-noumène se
trouve également dans la célèbre définition
du scepticisme donnée par Sextus
Empiricus au début des ses {\it Hypotyposes
pyrrhoniennes} (I, 8-9) : « Le scepticisme
est la capacité d’opposer {\it phénomènes} et
{\it noumènes} de toutes les façons possibles »,
où il désigne par « phénomènes » les données
des sens et par « noumènes » les
%1169
contenus de la pensée ou ce qui relève de
l'opinion (par exemple, qu’il y ait une
Providence divine).

Chez Kant, le noumène est l’objet intelligible,
qu’il oppose à l’objet de la sensibilité :
« l’objet de la sensibilité est le
sensible ; ce qui ne contient rien qui ne
puisse être connu par l'intelligence est
l’intelligible. Les écoles des Anciens
appelaient le premier {\it phénomène}, le
second, {\it noumène} ({\it Dissertation} de 1770,
{\it De mundi sensibilis atque intelligibilis
forma et principiis}, \S 3). Kant distingue
une signification positive et une signification
négative du noumène. Dans le premier cas,
noumène désigne l’objet d’une
intuition, non pas sensible, mais purement
intellectuelle, intuition dont par ailleurs
l’homme est dépourvu et dont il ne peut
même pas comprendre la possibilité ; dans
le second cas, le mot désigne une chose,
abstraction faite de l'intuition sensible
que nous en avons, c’est-à-dire vue
comme pure négation de toute détermination
sensible ({\it Critique de la raison pure},
Appendice de l’Analytique transcendantale).
Le concept de noumène est donc
« problématique », et désigne une « limite »;
en effet, il ne contient pas de
contradiction (il n’y a pas de contradiction en lui),
mais il ne peut correspondre
à aucune vérité connue. C’est-à-dire qu’il
s’agit d’une « représentation vide », à
laquelle on ne peut appliquer les catégories
de l’entendement et dont l'unique
fonction est celle de « tracer les limites de
notre connaissance sensible » (Remarque
à l’Amphibologie de l’Analytique transcendantale).
Ce n’est que dans le domaine
de la pratique que le concept de noumène
trouve une application légitime, comme
concept qui désigne la libre volonté
({\it causa noumenon}). Le terme « noumène »
a été aussi utilisé par la suite par
Arthur Schopenhauer, dans une acception
proche de celle de Kant, pour indiquer
la « volonté » aveugle, universelle et
absolue, essence réelle du monde illusoire
des représentations phénoméniques.

