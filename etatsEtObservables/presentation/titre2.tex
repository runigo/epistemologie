\begin{titlepage}
%
~\\[1cm]

\begin{center}
%\includegraphics[scale=0.5]{./presentation/chambreABulle}
\end{center}

\textsc{\Large }\\[0.5cm]

% Title \\[0.4cm]
\HRule

\begin{center}
{\huge \bfseries  Observable et sciences des phénomènes\\ }
 
%{\Large Extraits de sources bibliographiques\\}%\\[0.4cm]
\end{center}

\HRule \\[1.5cm]

\vspace{1cm}
\begin{itemize}[leftmargin=1cm, label=\ding{32}, itemsep=2pt]
\item {\bf Objet :} recueil de sources sur le sujet.
\item {\bf Contenu :} définitions, histoire, épistémologie.
\item {\bf Public concerné :} curieux en tout genre.
\end{itemize}
\vspace{2cm}


% Author and supervisor
\begin{minipage}{0.4\textwidth}
\begin{flushleft} \large
%\emph{Auteur:}\\
%Stephan \textsc{Runigo}
\end{flushleft}
\end{minipage}
\begin{minipage}{0.4\textwidth}
\begin{flushright} \large
\emph{Numérisation:}\\
Stephan \textsc{Runigo}
\end{flushright}
\end{minipage}

\vspace{2.5cm}

Ce document reproduit quelques passages d'ouvrages.

\vfill

Le premier chapitre concerne la notion d'observable, présentée dans un ouvrage d'histoire des sciences ({\sc Tonnelat}, {\it histoire du principe de relativité}, 1971).

\vfill

Les chapitres suivant rassemblent des articles de philosophies concernant la notion de phénomène ({\it encyclopédie de la philosophie}, {\sc Garzanti}, 2002, et {\it La pratique de la philosophie}, {\sc Hatier}, 2000).

\vspace{1.5cm}

{\large \today}

\end{titlepage}
