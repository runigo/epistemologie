\section{Pratique de la philosophie}
%{\bf }{\bf --}{\it } {\footnotesize XIX}$^\text{e}$

\subsection{phénomène}
%PHÉNOMÈNE (n. m.)

\begin{itemize}[leftmargin=1cm, label=\ding{32}, itemsep=1pt]
\item {\bf \textsc{Étymologie} :} grec {\it phainomenon},
de {\it phainestai}, « être visible »,
« briller », de {\it phôs}, « la lumière ».
\item {\bf \textsc{Sens ordinaire} :} Ce qui se
montre, c’est-à-dire ce qui se manifeste
à la conscience soit directement
(phénomènes affectifs et
psychologiques), soit par l’intermédiaire
des sens (phénomènes sensibles).
\item {\bf \textsc{Épistémologie} :} toutes les
données de l'expérience en tant
qu’elles font l’objet des sciences de
la nature.
\item {\bf \textsc{Psychologie} :} ensemble des données de la
conscience, accessibles par l’intuition.
\item {\bf \textsc{Chez Hume} :} ensemble des
affections de l'esprit, constitué des
impressions directement issues de
l'expérience, et des idées qui en
sont les copies atténuées.
\item {\bf \textsc{Chez Kant} :} tout ce qui est objet d’expérience
possible, et se caractérise par
une forme et une matière : « Ce qui,
dans le phénomène, correspond à
la sensation, je l'appelle matière de
ce phénomène ; mais ce qui fait que
le divers qu'il y a en lui est ordonné
suivant certains rapports, je le
nomme forme du phénomène. »
\item {\bf \textsc{Phénoménologie} :} le  phénomène
(ce qui apparaît à la
conscience) est objet d’intuition ou
de connaissance immédiate, en
même temps que manifestation de
l'essence (cf. Husserl et Merleau-Ponty).
\end{itemize}

Contrairement à Kant, qui considérait le
phénomène comme une manifestation
sensible — dans l’espace et le temps —
d’une « chose en soi » définitivement inaccessible,
les philosophes qui se réclament
de la « phénoménologie » estiment que,
dans le phénomène, ce sont les choses
elles-mêmes qui se révèlent : le projet
phénoménologique consiste précisément
en cet effort pour laisser se dévoiler — à
partir de l'intuition immédiate, de l'expérience
concrète — le « monde » situé en
deçà de la science. La « vision des
essences », dans le phénomène, est donc
possible grâce à la méthode phénoménologique,
qui nous permet de rétablir
une relation originaire avec les choses
« en chair et en os ».

\begin{itemize}[leftmargin=1cm, label=\ding{32}, itemsep=1pt]
\item {\bf \textsc{Terme voisin} :} apparence.
\item {\bf \textsc{Terme opposé} :} chose en soi: noumène.
\item {\bf \textsc{Corrélats} :} époché ; essence ; intentionnalité ;
phénoménisme ; phénoménologie..
\end{itemize}
 
