\section{Pratique de la philosophie}
%{\bf }{\bf --}{\it } {\footnotesize XIX}$^\text{e}$

\subsection{phénomène}
%PHÉNOMÈNE (n. m.)

\begin{itemize}[leftmargin=1cm, label=\ding{32}, itemsep=1pt]
\item {\bf \textsc{Étymologie} :} grec {\it phainomenon},
de {\it phainestai}, « être visible »,
« briller », de {\it phôs}, « la lumière ».
\item {\bf \textsc{Sens ordinaire} :} Ce qui se
montre, c’est-à-dire ce qui se manifeste
à la conscience soit directement
(phénomènes affectifs et
psychologiques), soit par l’intermédiaire
des sens (phénomènes sensibles).
\item {\bf \textsc{Épistémologie} :} toutes les
données de l'expérience en tant
qu’elles font l’objet des sciences de
la nature.
\item {\bf \textsc{Psychologie} :} ensemble des données de la
conscience, accessibles par l’intuition.
\item {\bf \textsc{Chez Hume} :} ensemble des
affections de l'esprit, constitué des
impressions directement issues de
l'expérience, et des idées qui en
sont les copies atténuées.
\item {\bf \textsc{Chez Kant} :} tout ce qui est objet d’expérience
possible, et se caractérise par
une forme et une matière : « Ce qui,
dans le phénomène, correspond à
la sensation, je l'appelle matière de
ce phénomène ; mais ce qui fait que
le divers qu'il y a en lui est ordonné
suivant certains rapports, je le
nomme forme du phénomène. »
\item {\bf \textsc{Phénoménologie} :} le  phénomène
(ce qui apparaît à la
conscience) est objet d’intuition ou
de connaissance immédiate, en
même temps que manifestation de
l'essence (cf. Husserl et Merleau-Ponty).
\end{itemize}

Contrairement à Kant, qui considérait le
phénomène comme une manifestation
sensible — dans l’espace et le temps —
d’une « chose en soi » définitivement inaccessible,
les philosophes qui se réclament
de la « phénoménologie » estiment que,
dans le phénomène, ce sont les choses
elles-mêmes qui se révèlent : le projet
phénoménologique consiste précisément
en cet effort pour laisser se dévoiler — à
partir de l'intuition immédiate, de l'expérience
concrète — le « monde » situé en
deçà de la science. La « vision des
essences », dans le phénomène, est donc
possible grâce à la méthode phénoménologique,
qui nous permet de rétablir
une relation originaire avec les choses
« en chair et en os ».

\begin{itemize}[leftmargin=1cm, label=\ding{32}, itemsep=1pt]
\item {\bf \textsc{Terme voisin} :} apparence.
\item {\bf \textsc{Terme opposé} :} chose en soi: noumène.
\item {\bf \textsc{Corrélats} :} époché ; essence ; intentionnalité ;
phénoménisme ; phénoménologie..
\end{itemize}
 
\subsection{phénoménisme}
%(n. m.)
\begin{itemize}[leftmargin=1cm, label=\ding{32}, itemsep=1pt]
\item {\bf \textsc{Étymologie} :} formé à partir de
phénomène.
\item {\bf \textsc{Philosophie de la connaissance} :} doctrine soutenue
dès l'Antiquité par des sophistes
comme Protagoras, ainsi que par
%342
l'école sceptique (cf. Scepticisme) ;
elle affirme que nous ne connaissons
que les apparences et qu'il faut
donc, dans nos jugements, remplacer
le verbe être par le verbe sembler
(ne pas dire par exemple du
miel qu’il est doux », comme si
c'était une propriété objective, mais
qu'« il me semble doux »: à un
autre, il pourra apparaître autrement).
\end{itemize}

Le phénoménisme définit volontiers la
connaissance comme le fait de « sauver
les apparences», c’est-à-dire de
construire des explications qui soient en
accord avec les phénomènes observables,
sans se demander si ces explications
sont vraies en elles-mêmes et
atteignent le cœur du réel. Cette définition
fut reprise dans la philosophie
contemporaine des sciences par le positivisme
d’Auguste Comte, et surtout le
conventionnalisme de Pierre Duhem.

\begin{itemize}[leftmargin=1cm, label=\ding{32}, itemsep=1pt]
\item {\bf \textsc{Corrélats} :} phénomène.
\end{itemize}

\subsection{phénoménologie}
%(n. f.) ÉTYM.: .

\begin{itemize}[leftmargin=1cm, label=\ding{32}, itemsep=1pt]
\item {\bf \textsc{Étymologie} :} terme forgé à partir
du grec {\it phainomenon}, « ce qui se
montre », et {\it logos}, « discours »,
« science », signifiant littéralement
« science des phénomènes ».
\end{itemize}

Le terme de phénoménologie apparaît
chez Jean Henri Lambert en 1734, avec
le sens de « doctrine de l'apparence ». Il
est ensuite repris par Kant et surtout
Hegel qui publie en 1807 une Phénoménologie
de l'esprit. Cette dernière est
l’histoire du développement progressif
de la conscience, depuis la simple sensation
jusqu’à la raison universelle ou
« savoir absolu ». Mais c'est avec Husserl,
à l’orée du {\footnotesize XX}$^\text{e}$ siècle, que la phénoménologie
naît vraiment, moins
comme une école attachée à une doctrine
où un système, que comme un
mouvement de pensée qui se donne la
tâche, toujours renouvelée, de décrire ce
qui apparaît en tant qu'il apparaît, grâce
à une méthode : la « méthode phénoménologique ».

\vspace{0.5cm}
{\bf Les enjeux de la phénoménologie}

Il s’agit pour Husserl de reprendre, en la
radicalisant, l'interrogation philosophique
première tournée du côté de l’essence de
ce qui se manifeste, c’est à dire des « phénomènes ».
De son point de vue, la phénoménologie
%342
est en quelque manière le
nom moderne de la philosophie.

« Retourner aux choses mêmes » : tel est
le mot d'ordre par lequel Husserl
exprime l'exigence de respecter ce qui
se manifeste en tant qu'il se manifeste,
par opposition à la démarche de la
métaphysique traditionnelle, suspecte à
ses yeux de trahir les phénomènes en
les dévaluant comme simple apparence
trompeuse, sous prétexte d'en saisir
l'essence. Cette science des phénomènes
qu'est la phénoménologie est
donc essentiellement descriptive (et non
déductive): elle est description des
essences — car s'il n'y avait pas d’essence,
le réel s’effondrerait — mais
l'être, l'essence, n’est nulle part ailleurs
que dans les phénomènes: autant
d’apparaître, autant d'être.

« Théorie de la connaissance », la phénoménologie
se donne la tâche plus vaste
d'élucider tout ce qui se présente à la
conscience et, par là, toutes les manières
de se présenter à la conscience (des jugements
ou des actes de volonté, mais aussi
des actes de perception). Dans l’expression
« retour aux choses mêmes », il ne
faut donc pas entendre « chose » au sens
de la chose spatio-temporelle qui se présente
dans la perception, mais plus largement
comme « ce qui est présent » à la
pensée. Il n’en reste pas moins que la
perception, en tant qu'elle est notre
ouverture la plus immédiate et la plus
fondamentale à ce qui se manifeste, est,
pour le phénoménologue, fondatrice de
tous les autres actes : en un sens, toute
phénoménologie est une « phénoménologie
de la perception ».

\vspace{0.5cm}
{\bf La méthode phénoménologique}

Il s’agit donc de faire apparaître ce qui
n'apparaît jamais dans les phénomènes,
leur acte même d’apparaître, et leur
manière d'apparaître (pour prendre une
image : la mise en scène, que le spectateur
« ne voit pas » dans un spectacle).
La méthode employée pour faire apparaître
la structure, d’abord inapparente,
de toute apparition, est appelée époché
ou « mise entre parenthèses », ou
« réduction  transcendantale ». Elle
consiste à suspendre toute croyance
immédiate et naïve en l'existence des
choses, toute thèse naturelle — notons
d’ailleurs que la science n'échappe pas
à ce type d'adhésion naïve. L'époché
montre alors que, de la même manière
qu'il n’y a de spectacle que pour un
regard, il n’y a de phénomène que pour
une conscience. Elle dévoile donc
%343
comme première et essentielle la corrélation
entre la conscience et le monde :
tout phénomène doit être rapporté à
l'acte de conscience qui le vise. Cette
corrélation est proprement le sens du
phénomène (ce que nous appelions tout
à l'heure sa « mise en scène »). Absorbée
dans sa croyance aux choses, c’est son
propre travail de visée que la
conscience, dans sa naïveté, ignore, et
que le phénoménologue lui révèle,
reconduisant — la réduction doit être
entendue en son sens étymologique de
reconduction — les phénomènes à la
conscience comme à leur source de
constitution. La conscience doit dès lors
être comprise comme un pur acte de « se
jeter vers », qu'Husserl nomme intentionnalité :
« Toute conscience est
conscience de quelque chose. » La
réduction est dite « transcendantale »,
en ce qu’elle dévoile la conscience
comme ce sans quoi les phénomènes
n'auraient aucun sens et aucun être.

Reconduisant tout phénomène vers la
conscience, la phénoménologie husserlienne
est un idéalisme, même si, dans
ses derniers écrits, Husserl prend en
compte l'inscription de la conscience
dans le « monde de la vie », ce monde
que je trouve toujours déjà là. Husserl
ébranle ainsi la maîtrise que la
conscience semble avoir sur tous les
phénomènes, sans pour autant la
remettre en question.

D'une manière générale, il y a une tension
interne à la phénoménologie prise
entre l'exigence de décrire un monde qui
précède la conscience, et l'exigence de
montrer que, fondamentalement, c’est la
conscience qui constitue tout sens.

C'est, d'une certaine façon, en interrogeant
et en mettant en cause le caractère
absolu de la conscience, que vont se
développer les phénoménologies postérieures
à Husserl : soit en s'intéressant
à cette proximité primordiale avec le
monde (Merleau-Ponty) ; soit en tentant
de dégager une origine plus profonde
que la conscience elle-même qui, du
coup, se fait en partie passive, s'échappe
à elle-même (Levinas).

Remarquons enfin que la phénoménologie,
en tant que méthode, n’a cessé de
dialoguer avec d’autres disciplines, et en
particulier avec les sciences humaines.
Puisqu’elle veut montrer que la
conscience n’est jamais un phénomène,
mais ce qui rend possible les phénomènes,
elle s’est toujours vigoureusement
opposée au « psychologisme » qui
tend à réifier la conscience, à en faire un
%
objet de la nature explicable grâce à des
lois scientifiques. Elle a pu fournir par
ailleurs des outils descriptifs très sûrs à
certains théoriciens de la psychiatrie ou
de la sociologie. On note aussi avec
intérêt la tentative, pourtant paradoxale,
des sciences cognitives — qui tentent
de faire servir la description husserlienne
de l'intentionnalité à l'étude
scientifique d’une conscience pensée
comme objective et naturelle.

\begin{itemize}[leftmargin=1cm, label=\ding{32}, itemsep=1pt]
\item {\bf \textsc{Terme voisin} :} phénomène.
\item {\bf \textsc{Corrélats} :} conscience ; ego ;
épochê;  intentionnalité ;  phénomène ;
sens ; transcendantal.
\end{itemize}

