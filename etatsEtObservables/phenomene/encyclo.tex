
%%%%%%%%%%%%%%%%%%%%%%%%%%%%%%%%%%%%%%%%%%
\section{Encyclopédie de la philosophie}
%%%%%%%%%%%%%%%%%%%%%%%%%%%%%%%%%%%%%%%%%%
\subsection{phénomène}

Ce qui se manifeste ou apparaît
(en gr. {\it to phainomenon}, participe substantivé
de {\it phainesthai}, « se manifester »,
%1246
« apparaître »). Pour Aristote, est phénomène
tout ce qui se manifeste directement
aux sens, en particulier les données
de l’observation empirique, par exemple
les phénomènes célestes de l’astronomie
({\it Traité du ciel} III 4, 303a 22-23) ; au sens
figuré, est phénomène ce qui se pense et
se dit, les avis et les opinions communes
({\it Éthique à Nicomaque} VII 1, 1145b 2-6:
cf. Platon, {\it République} VII, 517b). Est
aussi phénomène pour Platon la façon
d’être défectueuse des choses sensibles
par rapport aux espèces idéales, c'est-à-dire
la simple apparence opposée à la réalité
ou vérité de l’étant ({\it La République}, X
596e 4) ; dans cette acception, le terme est
aussi utilisé par Aristote, comme prédicat
dans des expressions du type « un savoir
apparent », « un bien apparent ». Le scepticisme
grec pose à nouveau le dilemme
qui oppose le phénomène à la « chose évidente »
des dogmatiques. Celle-ci est la
chose sensible qui se manifeste aux sens,
le sujet extérieur de la sensation et de la
représentation. Le phénomène des sceptiques
est au contraire l'affection pure
involontaire (par exemple avoir froid ou
chaud) et ce qui est « phénomène pour
nous » et dont il est impossible, de
remettre en cause son apparence. En effet
« personne vraisemblablement ne contestera
que le sujet (extérieur) apparaît de
telle ou de telle autre façon, on s’intéressera
au contraire de savoir s’il est tel qu’il
apparaît » (Sextus Empiricus, {\it Hypothyposes
pyrrhoniennes} I, 22). Le renouveau
moderne du scepticisme classique attribué
à David Hume conféra au concept de
phénomène un sens psychologiquement
plus prononcé; Hume appréhende en
effet le caractère phénoménal de la réalité
en relation directe à la tendance associative
de l'esprit humain. D'où le phénoménisme
empirique qui connut d'ultérieurs
développements avec John Stuart Mill,
Avenarius et Ernst Mach, de même avec
la phase « soliptique » de Bertrand Russell
et avec l'identification des données
empiriques au vécu immédiat du premier
Rudolf Carnap ({\it La Construction logique
du monde}, 1928). On parle aussi de phénoménisme
à propos de Kant mais de
façon impropre, car le phénomène kantien,
opposé à la chose en soi ou noumène,
n’est pas le produit de simples
associations psychologiques; c’est au
contraire la donnée immédiate de la
connaissance sensible qui est le produit de
%1247
la synthèse des formes {\it a priori}, spatio-temporelles
de l'intuition ({\it Critique de la
raison pure}, livre II, chap. III). Par conséquent
si le contenu phénoménal est accidentel
et contingent, la forme du
phénomène est au contraire universelle et
nécessaire et peut donner lieu à un savoir
relatif à l’expérience. Après Kant, le
concept de phénomène a connu deux
importants développements avec Hegel
({\it La Phénoménologie de l'esprit}, 1807) et
avec la phénoménologie de Husserl. Dans
ces deux cas, l'opposition kantienne entre
phénomènes et choses en soi est abandonnée :
la « chose » s’identifie totalement à
sa manifestation ou à sa phénoménalisation.
La chose phénoménologique expérimentée
par Hegel et par Husserl est
d’ailleurs différente : il s’agit pour le premier
des manifestations de la vie spirituelle
dans ses « figures » historico-dialectiques,
et pour le second des objets
et des structures intentionnelles de la
conscience transcendantale. Cette
seconde acception a connu un développement
ultérieur avec Heidegger, qui entendait
la phénoménologie comme
« ontologie fondamentale » de l'être de
l'étant ({\it Être et Temps}, \S 7).

