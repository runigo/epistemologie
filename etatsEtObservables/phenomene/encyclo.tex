
%%%%%%%%%%%%%%%%%%%%%%%%%%%%%%%%%%%%%%%%%%
\section{Encyclopédie de la philosophie}
%%%%%%%%%%%%%%%%%%%%%%%%%%%%%%%%%%%%%%%%%%
\subsection{phénomène}

Ce qui se manifeste ou apparaît
(en gr. {\it to phainomenon}, participe substantivé
de {\it phainesthai}, « se manifester »,
%1246
« apparaître »). Pour Aristote, est phénomène
tout ce qui se manifeste directement
aux sens, en particulier les données
de l’observation empirique, par exemple
les phénomènes célestes de l’astronomie
({\it Traité du ciel} III 4, 303a 22-23) ; au sens
figuré, est phénomène ce qui se pense et
se dit, les avis et les opinions communes
({\it Éthique à Nicomaque} VII 1, 1145b 2-6:
cf. Platon, {\it République} VII, 517b). Est
aussi phénomène pour Platon la façon
d’être défectueuse des choses sensibles
par rapport aux espèces idéales, c'est-à-dire
la simple apparence opposée à la réalité
ou vérité de l’étant ({\it La République}, X
596e 4) ; dans cette acception, le terme est
aussi utilisé par Aristote, comme prédicat
dans des expressions du type « un savoir
apparent », « un bien apparent ». Le scepticisme
grec pose à nouveau le dilemme
qui oppose le phénomène à la « chose évidente »
des dogmatiques. Celle-ci est la
chose sensible qui se manifeste aux sens,
le sujet extérieur de la sensation et de la
représentation. Le phénomène des sceptiques
est au contraire l'affection pure
involontaire (par exemple avoir froid ou
chaud) et ce qui est « phénomène pour
nous » et dont il est impossible, de
remettre en cause son apparence. En effet
« personne vraisemblablement ne contestera
que le sujet (extérieur) apparaît de
telle ou de telle autre façon, on s’intéressera
au contraire de savoir s’il est tel qu’il
apparaît » (Sextus Empiricus, {\it Hypothyposes
pyrrhoniennes} I, 22). Le renouveau
moderne du scepticisme classique attribué
à David Hume conféra au concept de
phénomène un sens psychologiquement
plus prononcé; Hume appréhende en
effet le caractère phénoménal de la réalité
en relation directe à la tendance associative
de l'esprit humain. D'où le phénoménisme
empirique qui connut d'ultérieurs
développements avec John Stuart Mill,
Avenarius et Ernst Mach, de même avec
la phase « soliptique » de Bertrand Russell
et avec l'identification des données
empiriques au vécu immédiat du premier
Rudolf Carnap ({\it La Construction logique
du monde}, 1928). On parle aussi de phénoménisme
à propos de Kant mais de
façon impropre, car le phénomène kantien,
opposé à la chose en soi ou noumène,
n’est pas le produit de simples
associations psychologiques; c’est au
contraire la donnée immédiate de la
connaissance sensible qui est le produit de
%1247
la synthèse des formes {\it a priori}, spatio-temporelles
de l'intuition ({\it Critique de la
raison pure}, livre II, chap. III). Par conséquent
si le contenu phénoménal est accidentel
et contingent, la forme du
phénomène est au contraire universelle et
nécessaire et peut donner lieu à un savoir
relatif à l’expérience. Après Kant, le
concept de phénomène a connu deux
importants développements avec Hegel
({\it La Phénoménologie de l'esprit}, 1807) et
avec la phénoménologie de Husserl. Dans
ces deux cas, l'opposition kantienne entre
phénomènes et choses en soi est abandonnée :
la « chose » s’identifie totalement à
sa manifestation ou à sa phénoménalisation.
La chose phénoménologique expérimentée
par Hegel et par Husserl est
d’ailleurs différente : il s’agit pour le premier
des manifestations de la vie spirituelle
dans ses « figures » historico-dialectiques,
et pour le second des objets
et des structures intentionnelles de la
conscience transcendantale. Cette
seconde acception a connu un développement
ultérieur avec Heidegger, qui entendait
la phénoménologie comme
« ontologie fondamentale » de l'être de
l'étant ({\it Être et Temps}, \S 7).

\subsection{phénoménisme}
Toute doctrine philosophique
pour laquelle la réalité n’existe
pas en soi, mais seulement comme phénomène
situé dans l’espace et dans le temps.
Le phénoménisme peut être soit une
théorie de la connaissance soit, dans une
acception plus radicale, une doctrine
ontologique. En tant que théorie de la
connaissance il trouve sa plus claire explication
dans l’esthétique transcendantale
de Kant, selon qui l’espace et le temps,
qui sont des formes {\it a priori} de la sensibilité,
« se rapportent aux objets qu'autant
qu'ils sont considérés comme des phénomènes
et non comme des choses en soi »
({\it Critique de la raison pure}, Esthétique
transcendantale, sect. II, 7). Kant affirme
qu’on ne peut rien connaître d’autre que
les phénomènes, mais il n’exclut pas
l'existence d’objets réels, mais non
connaissables par la sensibilité, c’est-à-dire
les choses en soi ou noumènes. Le
phénoménisme  gnoséologique  kantien
s'oppose ainsi au phénoménisme ontologique
développé par l’empirisme anglais
%phénoménisme
— et en particulier par George Berkeley —
pour qui les phénomènes perçus par le
sujet ne sont pas que les seules données
connaissables mais également les seules
choses existantes. Le  phénoménisme
ontologique représentait le double désavantage
d’exclure radicalement de la
connaissance tout phénomène non perceptible,
et de se diriger vers un idéalisme
niant toute consistance autonome de la
matière. Résolues par la formulation gnoséologique
du phénoménisme kantien, les
apories de Berkeley furent de nouveau
affrontées en termes ontologiques par
l’empiriocriticisme de Ernst Mach (cf. en
particulier {\it L'Analyse des sensations et le
rapport du physique au psychique}, 1886).
Comme Berkeley, Mach affirme qu'il
n'existe aucune vérité en dehors des phénomènes ;
mais si Berkeley soutenait que
le phénomène était perçu par l'esprit du
sujet connaissant (se dirigeant vers un
immatérialisme idéaliste), Mach conçoit
au contraire l’expérience comme une donnée
pure, antérieure à la différentiation
entre phénomènes physiques et phénomènes
psychiques. Cette radicalisation du
statut de l'expérience permet à Mach de
soutenir que ce ne sont pas les objets
matériels qui influent fortuitement sur la
sensibilité générant des sensations, mais
que ce sont des ensembles de sensations
qui génèrent les objets, tant matériels que
psychiques. Une solution différente, substantiellement
inspirée de l’empirisme
radical de John Stuart Mill, est proposée
par Bertrand Russell dans {\it Mysticisme et
logique} (1917). Russell apporte une
réponse au paradoxe des phénomènes
non perçus, en postulant des entités intermédiaires,
les « perceptibles » ({\it sensibilia}),
qui ont un statut identique à celui des
données perçues, mais ne sont pas nécessairement
présentes au sujet connaissant.
La solution de Mach séduit au contraire
le phénoménisme hypothéqué par l’empirisme
logique. Dans {\it La Construction
logique du monde} (1928), Rudolf Carnap
a pour but la définition de processus au
moyen desquels toutes les assertions
concernant des objets pourraient être traduites
en assertions concernant des expériences
immédiates ayant la même valeur
de vérité. Dans l'hypothèse de Carnap, les
objets sont donc « construits » à partir de
données de l’esprit, qui acquièrent une
fonction tant normative que fondatrice.

\subsection{phénoménologie}
Terme qui, en philosophie,
a été forgé par Johann Heinrich
Lambert. Dans le {\it Neues Organon} (1764),
Lambert fait succéder la doctrine de l’apparence
(lat. {\it phenomenologia}, all. {\it Lehre
des Scheins}) à la doctrine de la vérité ({\it alethiologia}),
et lui assigne le rôle de découvrir
les causes subjectives et objectives du
caractère illusoire des objets de la sensibilité
({\it phénomènes}). Comme d’autres
termes de la langue scolaire, le mot a été
formé par la réunion de deux termes
d’origine grecque, {\it phainomenon}, littéralement
« ce qui apparaît », « ce qui est visible »,
et {\it logos}, « discours » ou « raison ».
Tous les usages ultérieurs du terme
conserveront ce trait caractéristique : il
s’agit en phénoménologie de décrire les
choses (les phénomènes) telles qu'elles
apparaissent, ce qui veut dire aussi s’efforcer
de décrire la {\it manière} ou le {\it mode}
de leur manifestation.

Cependant, c’est Kant qui le premier
s’efforce de montrer le lien entre la phénoménologie
et la métaphysique. Cette
dernière sera précédée et introduite,
comme l'écrit Kant en 1770 dans sa correspondance
avec Lambert, d’une
« science tout à fait particulière, bien que
négative ({\it Phaenomenologia  generalis}),
afin de déterminer la validité et les limites
des principes de la sensibilité ». Kant lui-même
reprend à son compte le terme de
phénoménologie dans ses {\it Premiers Principes
métaphysiques de la science de la
nature} (1786), pour ramener le concept
physique de mouvement aux catégories
de la modalité, c’est-à-dire en en étudiant
les caractéristiques « en relation au genre
de représentation, et donc comme phénomène
des sens externes » (Préface). En
1807, Hegel reprend le terme dans une
acception nouvelle et beaucoup plus
large : la phénoménologie devint dès lors
la « science de l’expérience de la
conscience », c’est-à-dire la description
dialectique du « chemin de la conscience
naturelle, qui court vers le savoir ». La
phénoménologie montre par conséquent
le « {\it devenir} de la science en général ou
du savoir » tel un « chemin de l’âme qui
parcourt la série de ses formations (ou
figures de la conscience) comme des stations
prescrites par sa nature afin qu’elle
devienne limpide à son esprit et qu’elle
atteigne, au moyen de la pleine expérience
de soi, la connaissance de ce qu’elle
est en soi et pour soi (savoir absolu) »
%1248{\it }
(Préface de {\it La Phénoménologie de l'esprit}).
Les « phénomènes » dont parle
Hegel ne sont donc plus de simples « apparences »
de la connaissance sensible : ils
sont les manifestations historiques
concrètes de l’évolution du savoir
humain, rendu objet d’une « science de
l'esprit » d'ensemble.

À partir du {\footnotesize XX}$^\text{e}$ siècle, le terme de « phénoménologie »
renvoie, plus qu'aux antécédents
historiques, à la doctrine et à la
méthode inaugurées par Edmund Husserl
et développées par ses nombreux disciples
et épigones. L'idée husserlienne de la
méthode phénoménologique naît à la fin
du {\footnotesize XIX}$^\text{e}$ s. dans le cadre de la polémique
sur l’origine et de la nature des
concepts  logico-mathématiques, qui
opposait les tenants du logicisme (Gottlob
Frege) et ceux du psychologisme (Franz
von Brentano). Brentano ramène tout
concept à l’activité « intentionnelle » de la
conscience : tout acte psychique, tout
vécu psychique est, toujours, conscience
{\it de} quelque chose; par conséquent la
façon dont la conscience appréhende les
objets, ou les « intentionne », détermine
en même temps le caractère des objets
intentionnés. Frege repoussa cependant
cette réduction de la logique à la psychologie,
car elle confondait la genèse psychique
d’un concept avec sa nature
universelle et formelle, essentiellement
non psychique (la logique et les mathématiques
ne s'intéressent pas à l’émergence
de la notion du chiffre 3 d’un point de vue
psychologique, mais au signifié idéal de
cette notion, qui est pour soi indifférente
au fait d’être conçue, ou non, par telle ou
telle autre conscience empirique). Husserl
repoussa à son tour le psychologisme
(« Prolégomènes » aux {\it Recherches
logiques}), mais introduisit en même temps
le problème de « rendre les idées
logiques, les concepts et les lois, clairs et
distincts du point de vue de la théorie de
la connaissance ». Les concepts logiques
« doivent avoir pour origine les intuitions »,
c’est-à-dire « l'expérience vécue »
concrète ({\it Erlebnis}) de la conscience, sans
pour autant nier leur nature idéale et universelle.
Il faut pour cela instituer une
« phénoménologie des vécus logiques »
qui — au lieu de « poser de façon ingénue
comme existants les objets intentionnels
conformément à leur signification (selon
le procédé des psychologistes), au lieu de
les déterminer ou de les adopter comme
%1249
hypothèses, et d’en tirer des conséquences,
etc. (selon le procédé des logiciens purs) »
— se propose de « {\it réfléchir},
c’est-à-dire de rendre objets les actes
intentionnels mêmes et leur contenu de
sens immanent » ({\it Recherches logiques},
Introduction au vol. II, 1901). Dès lors
Husserl définit la phénoménologie
comme un « retour aux {\it choses mêmes} » ;
ces dernières sont les « phénomènes »,
non pas en tant qu’« apparences » opposées
à d’hypothétiques « choses en soi »,
mais en tant que manifestations originaires
de la réalité dans la conscience. La
phénoménologie se propose donc de
décrire le phénomène « tel qu’il se donne »,
pour en cueillir la forme pure, ou
essence, ou idée ({\it êidos}). Le procédé phénoménologique
exige donc une préalable
« réduction eidétique » : tout jugement
commun doit être « suspendu » ({\it epoché}),
toute théorie doit être mise entre parenthèses,
afin que le phénomène émerge
tout comme il se livre naturellement dans
son essence. De cette façon la phénoménologie
se pose comme « science rigoureuse »
et comme « science première », en
opposition tant au naturalisme ingénu des
sciences naturelles et positivistes (à
commencer par la psychologie même),
qu’au formalisme abstrait de la logique,
au relativisme historiciste et aux philosophies
de la « vision du monde » ({\it Weltanschauung}).

Ces thèses fondamentales de la phénoménologique
furent appliquées en différents
domaines (en éthique avec Max
Scheler, en ontologie avec  Nicolai
Hartmann, Martin Heidegger, etc.), mais
Husserl estima par la suite qu’il était
nécessaire de développer la phénoménologie
eidétique au sens transcendantal.
Non seulement la phénoménologie n’est
pas une « science des données de faits »,
mais elle est une « science des essences »
({\it Idées directrices pour une phénoménologie
et une philosophie phénoménologiques
pures}, 1913, Introduction au I$^\text{er}$ vol.) ; de
plus elle trouve son fondement ultime et
« cartésien » dans l’originelle activité de la
conscience. Cette dernière (atteinte dans
sa pureté par une ultérieure « réduction
transcendantale » qui met entre parenthèses
la thèse de l'existence du monde
naturel, y compris l’homme comme « réalité
psychophysique ») est le « résidu phénoménologique »
qui ne peut être remis
en question, le lieu d’origine de tous les
sens possibles du monde, que la phénoménologie
doit regarder comme le terrain
privilégié de ses propres descriptions. Ce
« tournant idéaliste » est cependant rejeté
par la majorité des disciples de Husserl.

%—> eidos + épochè e phénoménolo-
%gique (école) + réduction eidétique

\subsection{phénoménologique (école) }
Courant de
pensée qui s’est formé autour de Edmund
Husserl, dans les premières années du
{\footnotesize XX}$^\text{e}$ s. et qui a trouvé un premier lieu
d'expression dans le {\it Jahrbuch für Philosophie
und phaenomenologische Forschung
(Revue annuelle de philosophie et de
recherche phénoménologique)}. L'influence
de Husserl est déterminante quant à l’expression
des buts et des diverses tendances
de la phénoménologie : celle-ci
désignant à la fois une méthode nouvelle
d'investigation et d’analyse des expériences
et des vécus de conscience. Toutefois,
la rigueur exigée par Husserl,
exprimée notamment dans des formules
comme l’idée d’une « philosophie comme
science rigoureuse », ne fait pas de la
phénoménologie une science dogmatique
qui suivrait une ligne de pensée homogène,
mais les travaux des phénoménologues
se présente plutôt comme une
série de développements et d'analyses,
parfois divergents et toujours susceptibles
de réactivation, de remise en chantier.
C’est pourquoi Husserl lui-même n’hésita
pas à prendre ses distances vis-à-vis de
ses propres disciples et, en dernier lieu,
à poursuivre ses objectifs dans une
totale solitude théorique. Les noyaux originels
de l'« école phénoménologique »
sont des groupes d'étudiants qui se réunissaient
dans les deux principaux
centres universitaires où enseignait Husserl :
Göttingen et Fribourg; mais seule
l’université de Göttingen a abrité un
« cercle » phénoménologique. Parmi ses
adhérents, certains provenaient de l’université
de Munich, où enseignait le psychologue
Theodor Lipps et où, sous sa
direction, avait déjà été créée l’{\it Akademisch-psychologischer
Verein}, un autre
groupe d'étudiants s'intéressant aux problèmes
de psychologie descriptive. La diffusion
des {\it Recherches logiques} (1900-1901),
contenant la critique husserlienne
du psychologisme, alors dominant dans
les questions de logique, attira l'attention
%
%
des étudiants de Lipps, qui reconnurent
dans le programme phénoménologique de
« retour aux choses mêmes » ({\it zu den
Sachen selbst}) un renouveau au sens
« réaliste » de la culture philosophique,
encore dominée par l’idéalisme. J. Daubert
et Adolf Reinach furent les premiers
à former le cercle de Göttingen, puis
Moritz A. Geiger, Theodor Conrad,
D. von Hildebrand, H. Conrad-Martius
et, après 1909, Alexandre Koyré, Jean
Hering, Roman W. Ingarden, F. Kaufmann,
Edith Stein. Mais entre Husserl et
ses disciples, une incompréhension s’est
installée assez vite, principalement due à
la nouvelle orientation de la recherche
husserlienne. Tandis que Husserl, justement
dans ces années-là, se consacrait à
la {\it constitution} transcendantale de l’objet,
et à la {\it réduction phénoménologique}
— thèmes qui devaient marquer le « tournant
transcendantal » de sa phénoménologie
(tournant qui devient public en 1913
avec la publication de première partie des
{\it Idées}) —, les amis du cercle de Gôttingen
s’orientaient toujours plus vers une ontologie
réaliste, qui s’intéressât au processus
de formation des phénomènes, indépendamment
de l’activité cognitive du sujet.
On trouve les mêmes réserves pour ce qui
est de l’acceptation de l’analyse des structures
transcendantales de la conscience
chez des auteurs comme Max Scheler et
Nicolai Hartmann (ce dernier pourtant ne
fit jamais partie à proprement parler de
l'« école »  husserlienne). Scheler et
Hartmann se servent de la méthode phénoménologique,
mais en rapport surtout à
la première période de la pensée de Husserl
(c’est-à-dire qu’ils se rangent aux
positions anti-psychologistes et anti-relativistes
exprimées dans les {\it Recherches
logiques}), récusant en partie le contenu
des {\it Idées}.

La Première Guerre mondiale et la
mutation de Husserl à Fribourg (seule
Stein le suivit dans sa nouvelle université)
décidèrent, en fait, de la dispersion du
groupe ; Reinach qui en était la figure la
plus représentative mourut à la guerre en
1917, sans avoir rien écrit de décisif sur
la phénoménologie. Même les travaux de
Alexander Pfänder n’ont pas rencontré
l'approbation du maître, bien qu'il ait été
l'un des plus actifs collaborateurs du {\it Jahrbuch}
et que Husserl l’ait désigné comme
le successeur de sa chaire de Fribourg,
avant de connaître Heidegger. À Fribourg,
%1250
les disciples de Husserl sont Oscar
Becker, Ludwig Landgrebe et Eugen
Fink; mais l'influence du jeune Martin
Heidegger, qui était alors l'assistant de
Husserl, fut décisive pour tous ces
auteurs. Heidegger qui, à la suite de {\it Être
et Temps} (1927), semble avoir insufflé une
nouvelle orientation à la phénoménologie,
en dispersant le noyau constitutif en
une multitude d’éléments. Chez Fink et
chez Landgrebe, en effet, le long travail
philologique sur les textes husserliens ne
prélude pas à l’approfondissement de la
thématique transcendantale, mais conduit
plutôt au développement de considérations
ontologiques assez comparables à
celles qu’on peut rencontrer dans la pensée
de Heidegger. Après les conférences
de Husserl à Paris, en 1929 (qui ont servi
de bases aux fameuses {\it Méditations cartésiennes}
de Husserl), l'influence de la phénoménologie
commença de s'étendre en
Europe, retenant l'attention d’auteurs tels
que Jean-Paul Sartre, Maurice Merleau-Ponty,
Paul Ricœur et Emmanuel Levinas
en France, Antonio Banfi et Enzo Paci en
Italie, Georges Gurvitch et M. Farber aux
Etats-Unis. Elève de Husserl à Fribourg
entre 1923 et 1924, Farber fonda en 1939
la {\it International Phaenomenological Society},
qui publia à partir de 1940 la revue {\it Philosophy
and Phaenomelogical Research}.
La recherche phénoménologique a aussi
influencé de différentes façons la psychologie
et la psychiatrie (Frederic J. Buytendijk,
Ludwig Binswanger, Merleau-Ponty,
Sartre), les arts, l'esthétique et la sociologie.

%—> Heidegger e Husserl e phénoménologie

