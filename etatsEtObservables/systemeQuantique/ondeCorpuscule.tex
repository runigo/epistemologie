
\section{Ondes et corpuscules}


En physique classique, il existe deux processus de transmission de l'information. Ces deux processus permettent une classification des phénomènes régi par le principe de causalité.
Analogie de l'étang, de la carabine et de l'éboulement

Un homme se trouve sur la rive d'un étang. Il tire à la carabine sur la rive opposé. Cela provoque un petit éboulement. Les deux événements “tir de carabine” et “éboulement” sont relié causalement. l'information entre ces deux évènements se propage suivant un processus de type “corpuscule”.

Un homme tire à la carabine sur la surface de l'étang. Une vague est alors créée. cette vague ne provoque pas un petit éboulement en un lieu particulier, mais quelques petits mouvement du sable de la rive. L'information reliant causalement l’événement “tir” des évènements “petits mouvements” s'est propagée de suivant le processus de type “onde”.
Quantons

En physique moderne, l'information se propage suivant un unique processus. Décrit par l'équation de Schrödinger, ce processus semble remettre en question le principe de causalité. Ce processus permet également de mettre en évidence le phénomène d'intrication.
Analogie avec l'étang

L'homme tire sur la surface de l'étang. cela crée une onde. Cet événement est “ponctuel”, il semble corpusculaire. La vague se propage jusqu'à la rive opposée, il s'agit d'un phénomène ondulatoire. Un éboulement a lieu en un endroit atteint par l'onde et simultanément, la totalité de la vague disparaît. Le phénomène est local, la “disparition” de la vague semble contredire le principe de causalité.

Les quantons sont fous mais ils sont tous fous de la même manière.

