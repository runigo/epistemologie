
\section{Fonction d'onde}


\subsection{Représentation x}
Un quanton étant dans un certain état, des capteurs serait susceptible de le détecter. Cette détection mesure la position du quanton. La probabilité d'observer le quanton à la position $x$ est relié à la fonction d'onde $\psi(x)$ par : 
\[
P(x)=|\psi(x)|^2
\]


\subsection{Représentation p}
Un quanton étant dans un certain état, des capteurs serait susceptible de mesurer sa longueur d'onde. Cette détection mesure la quantité de mouvement du quanton. La probabilité d'observer le quanton ayant la quantité de mouvement $p$ est relié à la fonction d'onde $\psi(p)$ par : 
\[
P(p)=|\psi(p)|^2
\]


\subsection{Lien entre les représentions p et x}

Les fonctions d'ondes en x et en p sont reliées par la transformation de fourier
\[
\psi(p) = TF[\psi(x)]
\]
