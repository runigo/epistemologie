\section{À propos du chat de Schrödinger}

\subsection{Schrödinger}
Avant 1900 : la science prévoit ce qui va advenir.

Après 1930 : Ce qui advient est probabiliste.

Conclusion : la science perd de sa valeur.

La Physique quantique nous dit que c'est imprévisible et que cela se vérifie par l'expérience.

\subsection{Boltzmann}
Il ne faut pas confondre la science avec la physique quantique, 

Parceque cette histoire de lumière et de chat de Schrödinger, c'est juste ce qui décrit correctement ces petites choses, mais après, tout ce qu'il y a après, les molécules, la chimie, la mécanique, la biochimie, les sciences du vivant, c'est pas le chat de Schrödinger qui nous le raconte, dès qu'il y en a plusieurs de ces petites choses, c'est la physique statistique qui nous le raconte,  la physique statistique montre comment une physique déterministe émerge d'une physique probabiliste.

Notre monde est déterministe,
% en dehors de Dieu et du libre-arbitre,
le déterminisme macroscopique est raconté par Boltzmann. Cette histoire se base sur la quantification du réèl microscopique.

Les inégalités d'Heisenberg établissent un lien entre la nature corpusculaire et ondulatoire de la matière.

Le quanton de Schrödinger satisfait à ces inégalités, les théorie de Boltzmann décrivent le comportement collectif.

Les inégalité d' Heisenberg ne sont que l'émergence de la nature mathématique des quantons.

%\subsection{Les mathématiques des quantons}%x : position%p : quantité de mouvement%Le déplacement d'un quanton dans 
