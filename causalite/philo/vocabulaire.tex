
\section{Vocabulaire}
\newpage
\subsection{cause}
	\begin{itemize}[leftmargin=1cm, label=\ding{32}, itemsep=11pt]

\ib{Causalité} — \si{Épist.} Rapport de cause*
à effet. — Principe de causalité :
« Tout a une cause et, dans les
mêmes conditions, la même cause
est suivie du même effet. »

\ib{Cause} — \si{Méta.} {\bf 1.} Force$^2$ productrice,
engendrant l'effet et se prolongeant
en lui. {\it cf.} {\it Efficace}* et {\it Occasionnelle}*. — \si{Épist.} {\bf 2.} Antécédent$^1$
constant (Hume) et inconditionnel
(J. S. Mill). — {\bf 3.} Phénomène lié au
phénomène considéré par une relation fonctionnelle : « La cause n’est
jamais vraiment empirique » (Bachelard) $->$ {\it Dans la science}, l'explication par les forces productrices
(sens 1) fait place de plus en plus à
l’explication par les relations fonctionnelles (sens 3). Aussi, tandis
que F. Bacon disait que « savoir
vraiment, c’est savoir par les causes »
(sens 2), A. Comte a pu écrire
(Cours, I) que la science renonce à
la recherche des causes (sens 1), ce
qui est d'ailleurs auj. discuté.

—— \si{Hist.} {\bf 4.} {\it Aristote} distingue
4 espèces de causes : a) la cause matérielle ({\it p. e.} dans une statue, la
matière dont elle est faite); — b) la
cause formelle (la figure que la statue
représente; {\it cf.} Formel); — c) la
cause efficiente, {\it i. e.} la cause au sens 1
(le sculpteur); — d) la cause finale$^1$
(le but : désir de la gloire ou du gain,
visé par le sculpteur).

— \si{Méta.} {\bf 5.} Cause première : voir
Premier$^4$.

	\end{itemize}
\subsection{effet}
	\begin{itemize}[leftmargin=1cm, label=\ding{32}, itemsep=11pt]

\ib{Effet} — \si{Épist.} {\bf 1.} Phénomène considéré comme produit par
une cause efficiente*. — \si{Psycho.} {\bf 2.} Loi de
l'{\it effet} : celle qui pose que « toutes
% 62
choses égales d’ailleurs, une réponse
est renforcée par le succès, affaiblie,
éliminée, ou remplacée à la suite de l'échec » (Lagache).

\ib{Efficace} — \si{Méta.} Qui produit réellement son effet : « Cause efficace »
({\it opp.} « occasionnelle* »).

\ib{Efficience} — \si{Épist.} {\bf 1.} ({\it Opp.} : {\it finalité}*).
Causalité efficiente* : « La
science ne peut s'intéresser à la finalité qu'après avoir épuisé tout son
effort dans la découverte de l’efficience » (F. Houssay).

— \si{Vulg.} {\bf 2.} [Angl. : {\it efficiency}].
Rendement, effet utile : « Le pragmatisme* est une théorie de
l’efficience de la connaissance. »

\ib{Efficiente (Cause)} — Celle qui « produit » l'effet (cf.
{\it efficace}*). Cette expression s'emploie  {\it auj.} comme
syn. de {\it cause}* tout court (aux
sens 1, 2 et même 3), et par {\it opp.}
à {\it cause finale} (cf. {\it Cause}$^4$). {\it Chez
Aristote}, au ctr., la cause efficiente
se subordonne à la cause finale :
c’est « l’activité qui sort du fond
même de l’être et tend à réaliser la fin » (Goblot).

	\end{itemize}

\subsection{hasard}
	\begin{itemize}[leftmargin=1cm, label=\ding{32}, itemsep=11pt]

\ib{Hasard} — \si{Vulg.} Ce qui n’est pas prévisible : {\bf 1.} soit qu’on
suppose dans les choses une indétermination$^2$ radicale ; — {\bf 2.} soit
qu'il s'agisse d'événements si complexes (cf. {\it Fortuit}*) qu’on ne puisse
en connaître toutes les conditions : « Il n’y a pas incompatibilité entre le
rôle de ce que nous appelons le hasard et l’établissement de lois
scientifiques » (Borel) ; — {\bf 3.} soit qu’on ignore le déterminisme$^1$ du
phénomène ; — {\bf 4.} soit que, se plaçant au point de vue de la finalité*,
on n’en aperçoive
% 86
pas les raisons d’être : « Ce qui est hasard à l’égard des hommes
est dessein à l'égard de Dieu » (Bossuet). $->$ Terme très équivoque.

	\end{itemize}
