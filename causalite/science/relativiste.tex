
\section{Physique relativiste}

Le postulat de la physique relativiste est : "Les lois de la physique sont les mêmes dans tout les référentiels". On peut en déduire : "La vitesse de la lumière est une constante", "La vitesse de la lumière est une vitesse limite". Autrement dit, rien ne peut aller plus vite que la lumière. La vitesse de la lumière se note c, sa valeur est 300 000 km/s.

Cette vitesse limite, restreint le principe de causalité. En effet, considérons deux évènement A et B, distant l'un de l'autre, et supposons que A est la cause de B. L'évènement B ne peut donc avoir lieu qu'après avoir été prévenu que l'évenement A a eu lieu. Il faut donc "un messager" prévenant B que A a eu lieu, ce messager ne pouvant aller plus vite que la lumière, B ne peut avoir lieu qu'après un certain laps temps incompressible le séparant de A

Ainsi, dans le paradigme de la relativiré restreinte, le principe de causalité est modifié : "L'effet à lieu après un temps incompressible de la cause".



%Dans le paradigme de la relativiré restreinte, un objet, possédant une masse, ne peut pas dépasser la vitesse de la lumière. La lumière se propage à la vitesse de la lumière. Ainsi, un photon, un signal lumineux, vont nécessairement à la vitesse

%\subsection{Relativisme}
%(À distinguer du relativisme : "Les opinions sont subjectifs").



