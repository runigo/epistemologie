
\section{Physique statistique}
%\newpage

Le principe "Les mêmes causes produisent les mêmes effets" est retrouvé grâce à la physique statistique : il ne s'agit plus d'un principe fondamental, que l'on choisit de poser, mais d'un théorème que l'on démontre, à partir d'une hypothèse atomique.

Second principe de la thermodynamique, Entropie -> flêche du temps, un pneu crevé se dégonfle, les glaçons fondent dans l'eau chaude.


\subsection{Thermodynamique classique}

cause : deux corps de température différente sont en contact thermique.
effet : de la chaleur passe du corps froid vers le corps chaud.


