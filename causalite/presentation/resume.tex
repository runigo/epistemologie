\begin{center}
\Large
Résumé
\normalsize
\end{center}
\vspace{3cm}
\begin{itemize}[leftmargin=1cm, label=\ding{32}, itemsep=21pt]
\item {\bf Objet : } Généalogie de la causalité.
\item {\bf Contenu : } Histoire du principe de causalité.
\item {\bf Public concerné : } Tous.
\end{itemize}

\vspace{3cm}

%\vspace{3cm}

%Causalité 

\begin{itemize}[leftmargin=1cm, label=\ding{32}, itemsep=21pt]
\item {\bf  } 
\end{itemize}
En philosophie :
\begin{itemize}[leftmargin=1cm, label=\ding{32}, itemsep=21pt]
\item {\bf Hume : } causalité $=$ habitude
\item {\bf Laplace : } Déterminisme absolue
\end{itemize}
En sciences :
\begin{itemize}[leftmargin=1cm, label=\ding{32}, itemsep=21pt]
\item {\bf Physique classique : } Les mêmes causes  produisent les mêmes effets
\item {\bf Quantique : } La cause précède l'effet
%\item {\bf Relativité : } Deux évenements relié causalement ont des cônes d'univers d'intersection non vide.
\item {\bf Relativité : } L'intersection des cônes d'univers de deux évenements relié causalement est non vide.
\end{itemize}
