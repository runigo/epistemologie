
%%%%%%%%%%%%%%%%%%%%%
\chapter{Étymologie}
%%%%%%%%%%%%%%%%%%%%%
%
D'après le dictionnaire étymologique de poche Larousse. Origine éthymologique et date de l'apparition du mot.

\begin{itemize}[leftmargin=1cm, label=\ding{32}, itemsep=3pt]
\item {\bf corpuscule} : 1495, J. de Vignay, du latin {\it corpusculum}, de {\it corpus}
%\item {\bf } : 
\item électrique : du latin scientifique {\it electricus}, de {\it electrum}, emprunté au grec {\it êlektron}, ambre jaune, d'après sa propriété d'attirer les corps légers quand on l'a frotté. {\bf électron} 1829, Boiste, Stoney.
\item {\bf énergie} : du bas latin {\it energia}, emprunté au grec {\it energeia}, force en action.
% {\bf énergétique} du grec  {\bf energetikos}
\item neutre : du latin {\it neuter}, ni l'un ni l'autre. {\bf neutron} 1932, Joliot.
\item {\bf neutrino} : petit neutre (de l'italien) ? {\bf neutrino} vers 1940.
\item {\bf onde} : du latin {\it unda}, vague, masse d'eau agité.
\item {\bf particule} : 1484, Chuquet, du latin {\it particula}, de {\it pars}, {\it partis}, partie.
\item {\bf proton} : du grec {\it prôton}, neutre de {\it prôtos}, premier. 1920, Rutherford.
\item {\bf photon} : du grec {\it phôs}, {\it phôtos}, lumière. 1923, Louis de Broglie.
%\item {\bf } : 
\item quantifier : 1906, Larousse, du latin {\it quantus}, combien, et du suffixe d'action {\it -ifier}. {\bf quantification}.
\item quantique : ?
\item quanton : ?
\item quantum : 1764, Voltaire, singulier de {\it quantus}, combien grand. {\bf quanta}, 1900, Planck, pluriel de {\it quantum}. {\bf quantique}, 1930, Larousse.
\end{itemize}

%\begin{itemize}[leftmargin=1cm, label=\ding{32}, itemsep=1pt]\end{itemize}




%%%%%%%%%%%%%%%%%%%%%%%%%%%%%%%%%%%%%%%%%%%%%%%%%%%%%%%%%%%%%%%%%%%%%%%%%%%%%%%%
