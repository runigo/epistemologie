
%%%%%%%%%%%%%%%%%%%%%%%%%%%%%%%%%%%%%%%%%%%%

\section{Thermodynamique classique}

 \subsection{Doctrine du phlogistique}

Développée au {\footnotesize XVII}$^\text{e}$ siècle, 


Modèle : le phlogiston est décrit comme une substance {\it fluide}, contenue dans les matières combustibles et libérée	 lors de la combustion.

Le volume de cendre restant d'un volume de bois brulé justifie ce modèle. La masse supérieur, des oxydes métalliques obtenus par combustion des métaux était expliqué par l'attribution d'une masse négative au phlogiston.

 \subsection{Révolution énergétique}

Au {\footnotesize XVIII}$^\text{e}$ siècle, Antoine-Laurent Lavoisier conclut à l'inexistence du phlogiston grace à l'utilisation systématique de la balance dans l'étude des réactions chimiques.

Plusieurs modèles de la chaleur s'affrontent : la chaleur est une substance, avec ou sans masse. La chaleur est un type de mouvement, une vibration.

\texttt{ Émilio Segré, p 222 et suivantes}

Nouveau paradigme : la chaleur est une forme microscopique de l'énergie.

Nouveau principe : l'énergie se conserve, de l'énergie thermique peut se convertir en énergie mécanique, de l'énergie mécanique peut se convertir en énergie thermique, 

 \subsection{Entropie}
L'irréversibilité des transformations thermodynamique conduit à l'invention d'une nouvelle grandeur physique, l'entropie, ainsi qu'à l'énoncé du second principe de la thermodynamique : au cours d'une transformation d'un système isolé, l'entropie augmente.


 % \subsubsection{}

%%%%%%%%%%%%%%%%%%%%%%%%%%%%%%%%%%%%%%%%%%%



