
%%%%%%%%%%%%%%%%%%%%%%%%%%%%%%%%%%%%%%%%%%%%

\section{Physique statistique}

La physique statistique décrit un système thermodynamique comme étant constitué d'un {\it grand nombre d'entité}. Ces entités interagissent de manière microscopique, en échangeant de l'énergie, décrire le {\it mouvement individuel} de chacune de ces entités étant une entreprise, quasi impossible en raison de leur grand nombre, mais également inutile.

  \subsection{Paradigme}

À l'équilibre, l'état d'un système statistique est l'état le plus probable.

  \subsubsection{Évolution}

Dans un système thermodynamique, une partie de l'énergie se trouve sous forme cinétique, une autre sous forme de masse. La thermodynamique se cantone à décrire l'évolution de la partie cinétique de l'énergie.

  \subsubsection{Démon de Maxwell}

 % \subsubsection{}

%%%%%%%%%%%%%%%%%%%%%%%%%%%%%%%%%%%%%%%%%%%



