
\section{Révolutions et paradigmes}
%%%%%%%%%%%%%%%%%%%%%%%%%%%%%%%%%%%%%%%%%%%%

Historiquement, la progression des sciences s'est réalisée par une succession de cycle :

\begin{center}
Science normale $\to$ crise $\to$ révolution $\to$ science normale.
\end{center}

L'évolution des sciences ne consiste pas en une accumulation de connaissance mais à une remise en cause de ses paradigmes.

\subsection{Les révolutions scientifiques en physique}

L'exemple de la physique permet une approche pédagogique des notions de paradigme et de révolution scientifique.

  \subsubsection{Évolution des lois scientifiques}

Les processus d'évolution des lois scientifiques sont divers et complexes.

La découverte d'une nouvelle loi peut découler d'une crise, mais elle peut également découler d'une unification de lois appartenant à des paradigmes différents, éventuellement incompatible entre eux.

Lorsque la solution a une crise consiste à changer de paradigmes, 

 Les crises produisent généralement de nouveaux paradigmes. Ceux-ci sont concurrents entre eux. Un processus darwinien fait alors émerger le nouveau paradigme, majoritairement admis par la communauté.


\subsection{La physique contemporaine}

La physique contemporaine offre une vision originale de l'espace et du temps.

Egalement, elle alimente le débat sur la nature granulaire ou continue du monde de la physique.

  \subsubsection{La relativité générale}

La {\it mécanique classique} décrit les mouvements au cours du temps dans un espace à 3 dimension. Le temps est décrit comme une dimension indépendante des 3 dimensions d'espaces.

La {\it relativité générale} décrit les mouvements dans un espace-temps à 4 dimensions dans lequel le temps n'est plus indépendant des dimensions spatiales.

  \subsubsection{La théorie quantique des champs}

La {\it physique quantique} décrit le mouvement des particules élémentaires par des fonctions d'ondes. La fonction d'onde fournit la probabilité de mesurer tel ou tel grandeur caractérisant la particule observée.

La {\it théorie quantique des champs} décrirait les particules comme étant des évènements, correspondant à l'interaction entre les champs. Les évènements observables serait probabilistes, causals, et conservatifs.

%nouveaux paradigmes concurrent.L'évolution peut être provoqué par une crise, 


%%%%%%%%%%%%%%%%%%%%%%%%%%%%%%%%%%%%%%%%%%%



