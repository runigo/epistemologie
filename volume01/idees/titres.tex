
\section{Titres et sous-titre}
%%%%%%%%%%%%%%%%%%%%%%%%%%%%%%%%%%%%%%%%%%%%

L'ouvrage dont le plan est décrit par ce document s'inspire d'une bibliographie et en particulier de deux livres :

\begin{itemize}[leftmargin=1cm, label=\ding{32}, itemsep=2pt]
\item {\it les physiciens classiques et leurs découvertes} d'Émilio Segré 
\item {\it la structure des révolutions scientifique} de Thomas S. Kuhn 
\end{itemize}

 Un clin d'{\oe}uil peut être fait à ces deux auteurs en sous-titrant notre ouvrage 

\begin{itemize}[leftmargin=1cm, label=\ding{32}, itemsep=2pt]
\item Les physiciens moderne et leurs découvertes
\item Les physiciens contemporain et leurs découvertes
\item Les physiciens post-classique et leurs découvertes
\item Les révolutions scientifiques de la physique.
\end{itemize}

Titres thématique :

\begin{itemize}[leftmargin=1cm, label=\ding{32}, itemsep=2pt]
\item L'espace-temps des physiciens
\item Introduction au monde des physiciens
%\item 
\end{itemize}


%Éloge de la discrimination

%%%%%%%%%%%%%%%%%%%%%%%%%%%%%%%%%%%%%%%%%%%



