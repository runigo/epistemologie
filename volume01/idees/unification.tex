
\section{Unification}
%%%%%%%%%%%%%%%%%%%%%%%%%%%%%%%%%%%%%%%%%%%%

Les processus d'évolution des lois scientifiques sont divers

Au cours de l'évolution historique de la physique, les changements de paradigmes ont
% généralement
 souvent
 réalisé l'unification de paradigmes existant.


L'unification a souvent été un guide, un fil rouge. Elle conduit à des lois de moins en moins nombreuses et de plus en plus universelles.

  \subsection{Aristote}

Unifie les phénomènes celestes : Les astres errant, la sphère des fixes, le Soleil et la lune suivent tous une trajectoire circulaire.

Unifie les phénomènes terrestres : Le feu, la terre, l'air et l'eau ont tendance à reprendre leur place en suivant une trajectoire rectiligne.

  \subsection{De Copernic à Newton en passant par Galilé et Tycho Brahé}

Unifient la mécanique céleste et la mécanique terrestre : Le mouvement naturel est le mouvement rectiligne uniforme, Tout les corps s'attirent en raison de leur masse et en raison inverse du carré de leur distance.

  \subsection{De Orsted à Maxwell}

Unification de l'électricité et du magnétisme : le champ électrique est créé par les charges électriques, le champ magnétique est créé par le mouvement des charges électriques.

Les équations de Maxwell unifient l'électromagnétisme et les phénomènes lumineux : Un champ électrique variable au cours du temps crée un champ magnétique, un champ magnétique variable au cours du temps crée un champ électrique. Il en découle l'existence d'ondes électromagnétiques : la lumière.

  \subsection{Thermodynamique}

Unification du phlogistique avec l'énergie.

  \subsection{De Lorentz à Einstein}

Unifient l'espace et le temps ? la quantité de mouvement et l'énergie ? les référentiels galiléen et les référentiels non-galiléen ?

Unification de l'espace-temps et de la (quantité de mouvement)-énergie ?

  \subsection{De Planck à Dirac en passant par de Broglie }

Unification de la mécanique corpusculaire et la mécanique ondulatoire

  \subsection{Modèle standard}

Unifie les forces électromagnétiques, les forces nucléaires et la radioactivité $\beta$


 % \subsection{}
%%%%%%%%%%%%%%%%%%%%%%%%%%%%%%%%%%%%%%%%%%%



