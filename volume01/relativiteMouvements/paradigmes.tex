
\section{Paradigmes}
%%%%%%%%%%%%%%%%%%%%%%%%%%%%%%%%%%%%%%%%%%%%

Historiquement, la progression des sciences s'est généralement réalisée par une succession de cycle :
\begin{center}
Science normale $\to$ crise $\to$ révolution $\to$ science normale.
\end{center}
L'évolution des sciences ne consiste pas tant en une accumulation de connaissance qu'à une remise en cause de ses paradigmes.


  \subsection{La mécanique classique}

Au {\footnotesize XVII}$^\text{e}$ siècle.

La mécanique classique se développe autour des notions de forces et de trajectoires.

La {\it relation fondamentale de la dynamique} permet de déterminer le mouvement d'un corps connaissant les forces qui s'exercent sur celui-ci. La détermination du mouvement consistant en la connaissance de la position du corps en fonction du temps. 

La mécanique classique suppose l'existence d'un temps universel, identique dans tout l'espace. Autrement dit, une horloge dans le train en mouvement indique la même date que l'horloge de la gare.

%au cours du temps

  \subsubsection{Principes}

L'unicité du temps dans l'univers est un principe de la mécanique classique.

Principe d'inertie : Le mouvement rectiligne uniforme est le mouvement naturel. Autrement dit

"Un corps soumis à aucune force perdure dans sont mouvement rectiligne uniforme". Autrement dit

Les lois de la physique restent les mêmes dans tout les référentiels galiléens.

  \subsubsection{Modèles}

Un solide est un corps qui conserve ses dimensions et sa masse au cours de son mouvement.

Ce mouvement se fait dans un espace temps dans lequel le temps est universel.

%Aristote distingue la mécanique celeste (parfaite, éternelle) de la mécanique terrestre (corrompue, mortelle). Ces deux mécaniques n'obéissent pas aux mêmes lois :

  \subsection{L'électromagnétisme}

  \subsubsection{Principes}

La loi de Coulomb donne la valeur de la force s'exerçant entre deux particules chargées électriquement.

Les équations de Maxwell donnent la valeur des champs (électrique et magnétique) dans l'espace en fonction des charges, de leurs mouvement, et du temps.

  \subsubsection{Modèles}

La matière est constituée de particules chargées électriquement, les particules chargées exercent des forces entre elles.


  \subsection{Le changement de paradigme}

La crise provient de la contradiction qui apparaît lorsque l'on étudie les ondes électromagnétique dans deux référentiels en mouvement l'un par rapport à l'aure. et l'universalité du temps de la mécanique classique.


Dans la section précédente, nous avons vu cette contradiction apparaître lorsque l'on a supposé la vitesse de la lumière identique dans tout les référentiels. C'est à dire en appliquant le principe de relativité à l'électromagnétisme : le principe de relativité implique que les lois sont les mêmes dans tous les référentiels, l'électromagnétisme implique que la lumière se propage à une vitesse constante. Ces deux implications sont contradictoire avec un temps universel.

 % \subsubsection{}

%%%%%%%%%%%%%%%%%%%%%%%%%%%%%%%%%%%%%%%%%%%



