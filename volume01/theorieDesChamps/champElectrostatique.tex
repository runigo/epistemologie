
%%%%%%%%%%%%%%%%%%%%%
\section{Champ électrostatique}
%%%%%%%%%%%%%%%%%%%%%
%
%\subsection{Définition}
L'expérience du pendule électrostatique peut se modéliser par des {\it forces électrostatiques} s'exerçant entre les particules chargées. Ce modèle suppose une {\it action à distance}. La notion de champs permet de modéliser cette action entre les charges électriques : une charge électrique crée un champ dans tout l'espace. Le champ exercent une force sur les charges électriques.

\begin{center}
Vision "Champ électrique"
\end{center}
Une charge électrique $Q_1$ crée un champ électrique $\overrightarrow{E}$ dans tout l'espace.

\begin{center}
\setlength{\unitlength}{1cm}
\begin{picture}(10,3)
\put(0.5,1.0){\circle{0.3}}
\put(0.3,0.3){$Q_1$}
\put(5.5,1.0){\vector(-1,0){1.36}}
\put(3.7,1.3){$\overrightarrow{E}$}
\end{picture}
\end{center}

Le champ électrique $\overrightarrow{E}$ exerce une force $\overrightarrow{F}_{\overrightarrow{E}/Q_2}$ sur la charge électrique $Q_2$

\setlength{\unitlength}{1cm}
\begin{picture}(10,3)
\put(5.5,1.0){\circle{0.5}}
\put(5.3,0.2){$Q_2$}
\put(5.5,1.0){\vector(-1,0){1.36}}
\put(3.7,1.3){$\overrightarrow{F}_{\overrightarrow{E}/Q_2}$}
\end{picture}

Le champ créé par une charge électrique est à priori un outil purement mathématique, un artifice de calcul bien pratique. L'existence de ce "champ électrique" est à priori hypothétique. Néanmoins, son existence permet d'interpréter la transmission de "l'information de présence" entre les charges, de lever l'hypothèse d'une transmission d'information instantanée et immatérielle entre les charges.

%%%%%%%%%%%%%%%%%%%%%%%%%%%%%%%%%%%%%%%%%%%%%%%%%%%%%%%%%%%%%%%%%%%%%%%%%%%%
