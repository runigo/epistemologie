
%%%%%%%%%%%%%%%%%%%%%
\section{Force électrostatique}
%%%%%%%%%%%%%%%%%%%%%
%

Les forces électrostatiques s'exerçent entre les particules chargées. Les charges électriques peuvent être positive ou négative.
%La loi de Coulomb indique que des corps chargés électriquement exercent entre eux des forces. 
Des charges électriques de même signe se repousent, des charges de signe opposées s'attirent.

%
Un batonnet en matière synthétique (règle en plastique) frottée à l'aide d'un chiffon, s'électrise. Il est alors capable d'attirer des petits bouts de papier.

\begin{center}
\texttt{FIGURE}
\end{center}

% Lors du frottements, des électrons sont arraché à la matière et s'accumulent sur l'un 
 Dans l'expérience du pendule électrostatique, le batonnet attire le pendule (constitué d'une petite boule de papier aluminium).
% (le pendule n'est pas chargé mais la force de Coulomb crée une polarisation),
Lorsque le pendule touche le batonnet, un transfert de charge a lieu, le pendule se charge électriquement, et est alors repoussé.

\begin{center}
\includegraphics[scale=0.9]{./theorieDesChamps/MascartTraiteDElectriciteStatique1876}
\end{center}

L'expérience du pendule électrostatique peut se modéliser par des {\it forces électrostatiques} s'exerçant entre les particules chargées. Ce modèle suppose une {\it action à distance}.

% Dans un second temps, elle peut se modéliser par  en disant qu'une particule chargée crée un champ électrostatique en tout point de l'espace et qu'une particule chargée placé dans un champ électrostatique subit une force.

%\begin{minipage}[c]{.45\linewidth}
\begin{center}
Vision "Force de Coulomb"
\end{center}
Une charge électrique $Q_1$ exerce une force de Coulomb $\overrightarrow{F}_{Q_1/Q_2}$ sur la charge électrique $Q_2$

\begin{center}
\setlength{\unitlength}{1cm}
\begin{picture}(10,3)
\put(0.5,1.0){\circle{0.3}}
\put(0.3,0.3){$Q_1$}
%\put(0.5,1.0){\vector(1,0){1.36}}
%\put(1.2,1.3){$\overrightarrow{F}_{Q_2/Q_1}$}
\put(5.5,1.0){\circle{0.5}}
\put(5.3,0.2){$Q_2$}
\put(5.5,1.0){\vector(-1,0){1.36}}
\put(3.7,1.3){$\overrightarrow{F}_{Q_1/Q_2}$}
\end{picture}
\end{center}
%On peut alors se demander comment l'information de la présence de $Q_1$ parvient à $Q_2$, y a-t-il quelque chose qui se propage entre les charges ? Cette question peut être simplifiée en disant que les charges créent un champ dans tout l'espace et qu'elle sont sensibles à ce champ.
%\end{minipage}\hfill\begin{minipage}[c]{.45\linewidth}

%%%%%%%%%%%%%%%%%%%%%%%%%%%%%%%%%%%%%%%%%%%%%%%%%%%%%%%%%%%%%%%%%%%%%%%%%%%%
