
%%%%%%%%%%%%%%%%%%%%%%%%%%%%%%%%%%%%%%%%%%%%

Historiquement, la progression des sciences s'est généralement réalisée par une succession de cycle :
\begin{center}
Science normale $\to$ crise $\to$ révolution $\to$ science normale.
\end{center}
L'évolution des sciences ne consiste pas tant en une accumulation de connaissance qu'à une remise en cause de ses paradigmes.

\section{Définitions}

  \subsection{Discipline scientifique}
Une discipline scientifique se définit par l'{\it ensemble des phénomènes} qu'elle étudie, un ou plusieurs {\it paradigmes}, ainsi qu'un ensemble d'{\it expériences}.

  \subsection{Paradigme et expérience}
Un paradigme scientifique est un ensemble de modèle, de lois et de principe. Un ensemble d'expériences est associé au paradigme, ces expériences explicitant la mise en application du paradigme.

  \subsection{Science normale}
En période de sciences normale, une discipline scientifique est généralement dominée par un paradigme. L'ensemble des résultats expérimentaux s'acroit.

  \subsection{Révolution scientifique}
Une révolution scientifique a lieu, dans une discipline scientifique, lorsque l'ancien paradigme dominant est remplacé par un nouveau.

  \subsection{Crise scientifique}
Une crise scientifique a lieu lorsqu'une expérience, concernant la discipline, vient contredire le paradigme.

	\begin{itemize}[leftmargin=1cm, label=\ding{32}, itemsep=1pt]
		\item Une nouvelle expérience vient contredire le paradigme
		\item Un paradigme contradictoire tend à s'imposer
		\item Les paradigmes de deux discipline rentrent en conflit du fait de l'apparition d'une contradiction.
	\end{itemize}

  \subsection{Résolution des crises scientifiques}
Une crise scientifique peut être résolue : 

	\begin{itemize}[leftmargin=1cm, label=\ding{32}, itemsep=1pt]
		\item Par un changement de paradigme.
		\item Par une nouvelle interprétation de l'expérience, compatible avec l'ancien paradigme.
		\item Par l'adjonction au paradigme, de règle, de modifications légères, permettant sa conservation.
	\end{itemize}



  \subsection{Les expériences de pensée}

Une expérience de pensée est une expérience que l'on peut décrire, imaginer, mais qui n'a pas forcément été réalisée et qui n'est pas forcément réalisable. Imaginable, une expérience de pensée a pour objectif de confronter un paradigme avec ses prévisions.



 % \subsubsection{}

%%%%%%%%%%%%%%%%%%%%%%%%%%%%%%%%%%%%%%%%%%%



