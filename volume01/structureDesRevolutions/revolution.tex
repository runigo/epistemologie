
%%%%%%%%%%%%%%%%%%%%%%%%%%%%%%%%%%%%%%%%%%%%

\section{Révolution scientifique}

L'évolution des sciences ne consiste pas tant en une accumulation de connaissance qu'à une remise en cause de ses paradigmes.


Historiquement, la progression des sciences s'est réalisée par une succession de cycle :
\begin{center}
Science normale $\to$ crise $\to$ révolution $\to$ science normale.
\end{center}


  \subsection{Crise}

Une expérience particulière, censé être décrite dans le paradigme, aboutit à un résultat en contradiction avec le paradigme.
Ou bien deux paradigmes contradictoires se trouvent en concurrence.

  \subsection{Résolution}


Soit l'expérience peut être remise en question, auquel cas le paradigme peut être conservé. Soit l'expérience remet en question le paradigme, auquel cas il fini par être changé.

 % \subsubsection{}

%%%%%%%%%%%%%%%%%%%%%%%%%%%%%%%%%%%%%%%%%%%



