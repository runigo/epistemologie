\begin{center}
\Large
Résumé
\normalsize
\end{center}
\vspace{3cm}
\begin{itemize}[leftmargin=1cm, label=\ding{32}, itemsep=21pt]
\item {\bf Objet : } Introduction à la structure des révolutions scientifiques, exemple de la physique.
\item {\bf Contenu : } Définitions, histoire de la physique, physique moderne.
\item {\bf Public concerné : } Néophyte.
\end{itemize}

\vspace{3cm}

Ce document est le plan d'un ouvrage qui, sans aspirer à une très grande rigueur mais se voulant pédagogique et abordable par le plus grand nombre, se propose d'atteindre deux objectif,
%Ce document est le plan d'un ouvrage dont l'objectif, sans aspirer à une très grande rigueur mais pédagogique et abordable par le plus grand nombre, sera double :

\begin{itemize}[leftmargin=1cm, label=\ding{32}, itemsep=2pt]
\item {\bf 1 } présenter la physique et son évolution en exploitant le concept de révolutions scientifiques.
\item {\bf 2 } offrir une initiation vulgarisatrice à quelques idées nouvelles apportées par la physique moderne,
\end{itemize}

Pour cela, l'ouvrage proposera au lecteur :

\begin{itemize}[leftmargin=1cm, label=\ding{32}, itemsep=2pt]
\item {\bf 1 } de  comprendre le processus de révolutions scientifique,
\item {\bf 2 } de découvrir quelques éléments de physique et d'histoire de la physique,
\item {\bf 3 } d'oppérer sur soi-même une révolution scientifique.
\end{itemize}

Le premier chapitre de ce document propose un plan détaillé de l'ouvrage. Le second chapitre propose un plan chronologique. Enfin le troisième chapitre développe quelques idées maîtresses.


%Les diagrammes de Feynman ont montré leur pertinence au niveau technique en électromagnétisme quantique et en chromodynamique. Ils demeurent néanmoins relativement simple à aborder et apportent au néophyte un regard original sur l'espace-temps et les quantons. L'objectif de ce document est l'initiation aux idées nouvelles apportées par la contemporaine théorie quantique des champs, il n'aspire pas à une totale rigueur mais se veut pédagogique et abordable par le plus grand nombre.

\vspace{3cm}

%Les deux premiers chapitres sont consacré à l'histoire des sciences et à l'évolution des paradigmes en rapport avec le mouvement. Le troisiemme chapitre introduit à la théorie des champs. Les chapitres suivants abordent les thèmes de la physique apportant une reflexion à propos du temps : relativité, quantique, causalité.
