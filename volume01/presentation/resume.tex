
\texttt{Version béta de cet ouvrage : le texte écrit en caractère "courrier" indique des parties non encore écrites.}

\vspace{.3cm}

\texttt{Version béta de cet ouvrage : Un certain nombre de paragraphe n'indique que les idées générales correspondantes et ne sont pas étoffées}

\begin{center}
\Large
Résumé
\normalsize
\end{center}
\vspace{3cm}
\begin{itemize}[leftmargin=1cm, label=\ding{32}, itemsep=21pt]
\item {\bf Objet : } Introduction à la structure des révolutions scientifiques, exemple de la physique.
\item {\bf Contenu : } Définitions, histoire de la physique, physique moderne.
\item {\bf Public concerné : } Néophyte.
\end{itemize}

\vspace{3cm}




Cet ouvrage, sans aspirer à une très grande rigueur mais se voulant pédagogique et abordable par le plus grand nombre, se propose d'offrir une vision illustrée, littéraire et épistémologique de la physique.

\vspace{.3cm}

Cet ouvrage ne se veut absolument pas exhaustif. Il entend piocher dans la physique et dans l'histoire de la physique, quelques exemples permettant d'aborder quelques concepts de l'épistémologie.

\vspace{.3cm}


% Pour ce faire,  une vision de la progression des sciences.
% un chapitre généraliste
% la physique, son histoire et son évolution.
% en exploitant le concept de révolutions scientifiques.
%L'objectif de l'ouvrage est de permettre au lecteur :

%\begin{itemize}[leftmargin=1cm, label=\ding{32}, itemsep=2pt]
%\item {\bf 1 } de  comprendre le processus de révolutions scientifique,
%\item {\bf 2 } de découvrir quelques éléments de physique et d'histoire de la physique,
%\item {\bf 2 } offrir une initiation vulgarisatrice à quelques idées nouvelles apportées par la physique moderne,
%\item {\bf 3 } éventuellement d'oppérer sur soi-même des révolutions scientifiques.
%\end{itemize}


\vspace{3cm}

%Le premier chapitre de ce document propose un plan détaillé de l'ouvrage. Le second chapitre propose un plan chronologique. Enfin le troisième chapitre développe quelques idées maîtresses.
%Les diagrammes de Feynman ont montré leur pertinence au niveau technique en électromagnétisme quantique et en chromodynamique. Ils demeurent néanmoins relativement simple à aborder et apportent au néophyte un regard original sur l'espace-temps et les quantons. L'objectif de ce document est l'initiation aux idées nouvelles apportées par la contemporaine théorie quantique des champs, il n'aspire pas à une totale rigueur mais se veut pédagogique et abordable par le plus grand nombre.

%%%%%%%%%%%%%%
Chapitrage
%%%%%%%%%%%%%%

\begin{itemize}[leftmargin=1cm, itemsep=1pt]%, label=\ding{32}
%\item 
%\item Cosmologie : observation, modèle, rationnalité
%\item 
%\item Mouvement : mesure, principe de relativité
%\item 
\item Thermodynamique et énergie : grandeurs et principes physiques, lois de conservation, unification (entre les différentes forme de l'énergie).
%\item 
\item Théorie des champs : modélisation mathématique, transmission de l'information, unification (entre l'électrostatique, le magnétisme et la lumière).
%\item 
\item Physique quantique : science moderne, espace-temps, , unification (entre l'électromagnétisme et les interactions "faible" et "forte").
%\item 
\item Structure des révolutions scientifiques : définitions des concepts épistémologiques.
%\item 
\end{itemize}

%Pour cela, chacun des chapitres sera articulé autour de trois idées principale : présenter un fait scientifique, présenter son histoire, mettre en évidence l'évolution des paradigmes associé.

%Les deux premiers chapitres sont consacré à l'histoire des sciences et à l'évolution des paradigmes en rapport avec le mouvement. Le troisiemme chapitre introduit à la théorie des champs. Les chapitres suivants abordent les thèmes de la physique apportant une reflexion à propos du temps : relativité, quantique, causalité.
