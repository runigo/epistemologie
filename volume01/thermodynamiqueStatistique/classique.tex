
%%%%%%%%%%%%%%%%%%%%%%%%%%%%%%%%%%%%%%%%%%%%

\section{Thermodynamique classique}

 \subsection{Doctrine du phlogistique}

Développé au {\footnotesize XVII}$^\text{e}$ siècle, 


Modèle : le phlogiston est une substance{\it fluide}, qui est contenu dans les matières combustibles et libéré lors de la combustion.

Le volume de cendre restant d'un volume de bois brulé justifie ce modèle. La masse supérieur, des oxydes métalliques obtenus par combustion des métaux était expliqué par l'attribution d'une masse négative au phlogiston.

 \subsubsection{Crise}

Au {\footnotesize XVIII}$^\text{e}$ siècle, Antoine-Laurent Lavoisier conclut à l'inexistence du phlogiston grace à l'utilisation systématique de la balance dans l'étude des réactions chimiques.

Plusieurs modèles de la chaleur s'affrontent : la chaleur est une substance, avec ou sans masse. La chaleur est un type de mouvement, une vibration.

{\sf Émilio Segré, p 222 et suivante}

 \subsection{Énergie mécanique et chaleur}

Modèle classique : la chaleur est une forme microscopique de l'énergie

 \subsection{Entropie et température}


 % \subsubsection{}

%%%%%%%%%%%%%%%%%%%%%%%%%%%%%%%%%%%%%%%%%%%



