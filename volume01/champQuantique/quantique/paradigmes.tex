
%%%%%%%%%%%%%%%%%%%%%
\section{Paradigmes}
%%%%%%%%%%%%%%%%%%%%%

\subsection{Ondes ou corpuscules ?}

%Démocrite suppose une composition atomique de la matière, déterminé par le seul principe de causalité. Il préfigure la pensée scientifique, ainsi que celle d'épicure, de Lucrèce et de Descartes.
La question de l'existence de l'atome (grain de matière indivisible) ou d'une matière continue, divisible à l'infini, a alimenté le débat scientifique depuis Démocrite (460-380 av JC). La lumière quand à elle, était souvent décrite par la notion de rayon lumineux, parfois par une notion de propagation.

En formulant le principe de Huygens, Christiaan Huygens (1629-1695), établissait le lien entre la propagation d'onde et les trajets des rayons lumineux.
L'optique de Newton (1661-1727) était une théorie corpusculaire jusqu'à un certain point. Newton mélange aux concepts corpusculaires d'autres concepts ("accès de réflexion facile","accès de transmission facile").
La démonstration du principe d'interférences par Thomas Young (1773-1829), imposa une interprétation ondulatoire des phénomènes lumineux.

Parallèlement, les développements de la chimie confirmaient la nature granulaire de la matière

Ainsi, la fin du XIX$^\text{\,e}$ siècle, la lumière était considérée comme un phénomène continu, alors que la matière était considéré comme granulaire. Tout comme Aristote séparait la mécanique céleste de la mécanique terrestre, les scientifiques séparait une mécanique granulaire d'une optique ondulatoire et continue.

Et tout comme la révolution copernicienne avait réconciliée les deux mécaniques d'Aristote, la révolution quantique allait tenter de réconcilier la mécanique granulaire de l'optique ondulatoire.



\subsection{Mécanique Newtoniennne}
%
La trajectoire d'un corps est décrite par ses coordonnées évoluant au cours du temps, obéissant au lois de la mécanique.
%
\subsection{Mécanique quantique}

L'évolution au cours du temps d'un quanton est décrite par sa fonction d'onde donnant l'amplitude de probabilité de mesurer ce quanton (de l'observer). La fonction d'onde evolue suivant l'équation de Schrödinger.

\subsection{Théorie quantique des champs}

Les différents champs échangent de l'énergie entre eux de manière quantique.

%%%%%%%%%%%%%%%%%%%%%%%%%%%%%%%%%%%%%%%%%%%%%%%%%%%%%%%%%%%%%%%%%%%%%%%%%%%%%%%%%%%%%
