
%%%%%%%%%%%%%%%%%%%%%
\section{Physique contemporaine}
%%%%%%%%%%%%%%%%%%%%%

\begin{tikzpicture}
    \def\horizontal {7.5}
    \def\vertical {1.3}
\draw[-latex,color=darkgray] (19.1*\horizontal,0) -- (20.5*\horizontal,0);
\draw[shift={(20.5*\horizontal,0)},color=darkgray,thin]
                                   node[below] {\footnotesize $temps$};
%1856-1951
  \draw[shift={(19.3*\horizontal,0)},color=darkgray,thin] (0pt,1pt) -- (0pt,-1pt)
                                   node[below] {\footnotesize Pétain};
%1946-
  \draw[shift={(20.17*\horizontal,0)},color=darkgray,thin] (0pt,1pt) -- (0pt,-1pt)
                                   node[below] {\footnotesize Trump};
%1918-1988
\draw (19.66*\horizontal,0.5*\vertical) node [rotate=30]{Feynman};
%1947-
\draw (19.86*\horizontal,0.5*\vertical) node [rotate=30]{Aspect};
\end{tikzpicture}
%
\subsection{Électrodynamique quantique}

Les champs échangent de l'énergie entre eux de manière quantique.

Les diagrammes de Feynman sont utilisé au niveau technique et ont montré leur pertinence.

\subsection{Intrication quantique}

L'expérience d'Alain Aspect semble en faveur de la nature probabiliste de la physique quantique, Dieu a bien l'air de jouer au dés.

%%%%%%%%%%%%%%%%%%%%%%%%%%%%%%%%%%%%%%%%%%%%%%%%%%%%%%%%%%%%%%%%%%%%%%%%%%%%%%%%
