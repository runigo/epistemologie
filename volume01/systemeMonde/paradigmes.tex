
\section{Paradigmes}
%%%%%%%%%%%%%%%%%%%%%%%%%%%%%%%%%%%%%%%%%%%%

Historiquement, la progression des sciences s'est généralement réalisée par une succession de cycle :
\begin{center}
Science normale $\to$ crise $\to$ révolution $\to$ science normale.
\end{center}
L'évolution des sciences ne consiste pas tant en une accumulation de connaissance qu'à une remise en cause de ses paradigmes.


  \subsection{Les grecs}
  \subsubsection{Modèles}

La civilisation grec verra l'émergence d'un certain nombre de modèle de cosmologie.


\begin{tabular}{lcr}
 & Terre fixe & Soleil fixe  \\
Terre plate & Thalès & \\
Terre sphérique & Pythagore & Aristarque \\
%\multicolumn{2}{c}{ } \\
%\multicolumn{2}{c}{  } \\
\end{tabular}

Le modèle d'une Terre fixe au centre de l'Univers sera prédominant jusqu'au {\footnotesize XVII}$^\text{e}$ siècle.


  \subsubsection{Principes}


\begin{tabular}{rl}
 & Principes \\
Thalès & Eau  \\
Pythagore & harmonie des sphères  \\
Anaximadre & apeiron \\
Aristote & mouvement circulaire \\
\end{tabular}

%Aristote distingue la mécanique celeste (parfaite, éternelle) de la mécanique terrestre (corrompue, mortelle). Ces deux mécaniques n'obéissent pas aux mêmes lois :

  \subsection{Ptolémée}

Les épicycles sont une modification du modèle, qui conserve la loi "le cercle est le mouvement naturel", et qui permet de se conformer aux observations. Cette modification du paradigme clos la crise déclanché par l'observation des mouvements de rétrogradation.



Ce nouveau paradigme sera utilisé par les astronomes jusqu'au {\footnotesize XVII}$^\text{e}$ siècle.
% Pour l'astronomie, cela sera une période de science normale jusqu'à la révolution copernicienne.

  \subsection{Révolution copernicienne}

Le modèle d' 



 % \subsubsection{}

%%%%%%%%%%%%%%%%%%%%%%%%%%%%%%%%%%%%%%%%%%%



