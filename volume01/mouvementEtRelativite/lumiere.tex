
%%%%%%%%%%%%%%%%%%%%%
\section{L'horloge a lumière}
%%%%%%%%%%%%%%%%%%%%%

À la fin du {\footnotesize XIX}$^\text{e}$ siècle, les physiciens identifient la lumière avec des ondes électromagnétiques. Les équations fournissent une valeur de la vitesse de la lumière. Dans le vide, cette valeur est constante et devait être la même dans tout les {\it référentiels galliléens}. L'indépendance de la vitesse de la lumière suivant les référentiels va révolutionner la vision de l'espace-temps des physiciens du début du {\footnotesize XX}$^\text{e}$ siècle.

Dans cette section, nous allons nous attacher à décrire une horloge, utilisant une propriété de la lumière, afin de montrer le changement introduit dans la physique par la vision de l'espace temps après Einstein.

\subsection{Définition}

La technologie fournit des horloges performantes. De façon générale, une horloge fait appel à un {\it phénomène périodique} (pendule, oscilateur à ressort, vibration atomique) et à un {\it compteur} (cadran à aiguille, électronique).

L'horloge à lumière est une horloge imaginaire dans laquel de la lumière effecturait des allers-retours entre deux miroirs, un dispositif permettant de compter ces allers-retours. L'affichage de l'horloge va nous permettre de mesurer des durées en "aller-retour".


\subsection{Application}

Imaginons une de ces horloges embarqué dans le train. Imaginons également que nous puissions enregistrer le chemin parcouru par la lumière.

Le chemin parcouru par la lumière entre les deux miroirs dans l'horloge en mouvement est plus grand que dans l'horloge immobile. Autrement dit, alors que l'horloge lié au référentiel terrestre affiche 10 allers-retours, l'horloge en mouvement affiche 9 allers-retours.

Les horloges en mouvement semblent prendre du retard sur les horloges immobiles. 

%%%%%%%%%%%%%%%%%%%%%%%%%%%%%%%%%%%%%%%%%%%%%%%%%%%%%%%%%%%%%%%%%%%%%%%%%%{\it }
