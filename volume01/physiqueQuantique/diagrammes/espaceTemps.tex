
%%%%%%%%%%%%%%%%%%%%%%%%%%%%%%%%%%%%%%%%%%%%

\section{Théorie des champs}

%%%%%%%%%%%%%%%%%%%%%%%%%%%%%%%%%%%%%%%%%%%

%\subsection{}

Finallement, le diagramme représentant un échange élémentaire d'énergie entre le champ électron et le champ photon est le suivant :

\begin{center}
\includegraphics[scale=2.5]{./diagrammes/feynmannElementaire}
\end{center}

Un tel échange se représente comme un processus unique qui apparaît tantôt comme une création, tantôt comme une anihilation, tantôt comme une émission, tantôt comme une absorption, suivant comment on place les axes de temps et d'espace.

%Ainsi, il semblerait que ce processus élémentaire d'interaction entre les champs se produirait sans existence prélable de temps et d'espace, mais faisant apparaître un temps et un espace à nos sens.

\subsection{Champs et quantons}

Les quantons n'apparaissent alors plus comme des "particules de matière" mais comme la "signature" des champs échangeant de l'énergie.

L'électromagnétisme quantique décrit donc les champs photon, électron et positron, comme trois champs "couplés". Ce couplage signifie qu'ils échangent de l'énergie entre eux. Cet échange se produit de façon quantique (quantifiée, discrète, "atomique").

La nature nous apparaitrait comme quantifiée car nous considérons comme "objet", non pas les champs, mais leurs échanges, leur couplage. La nature serait constituée de champs, continus, leur interaction discontinue nous faisant apparaître une nature "atomique".  

\subsection{Espace-temps et causalité}

Le diagramme élémentaire nous montre une unification des interactions entre les champs, dans laquelle l'espace et le temps joueraient un rôle "secondaire". On aurait plutôt à faire à des "évènements espacés" (échange quantique entre les champs) qui vérifiraient un "principe causal". L'espace-temps serait alors notre perception de ce monde d'évènements "espacés et causaux". en effet :

ce qui ressort de l'espace-temps des physiciens, c'est principalement l'invariance de la vitesse de la lumière, l'espace et le temps sont reliés, ce qui est fondamental semble être leur "liens", la vitesse limite.

La vitesse limite peut apparaître comme la "signature" d'un principe de causalité auquel obéirait les champs.
%%%%%%%%%%%%%%%%%%%%%%%%%%%%%%%%%%%%%%%%%%%%%%%%%%%%%%%%%%%%%%%%%%%%%%%%%%%%%
