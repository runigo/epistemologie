
%%%%%%%%%%%%%%%%%%%%%
\section{Étymologie}
%%%%%%%%%%%%%%%%%%%%%
%
D'après le dictionnaire étymologique de poche Larousse. Origine éthymologique et date de l'aparition du mot.
%\subsection{Physique}

%\subsection{Ondes et particules}
%\subsection{Particules élémentaire}
%\subsection{Lumière et matière}

\subsection{physique classique}
\begin{itemize}[leftmargin=1cm, label=\ding{32}, itemsep=1pt]
\item {\bf corpuscule} : 
\item {\bf particule} : 
\item {\bf onde} : 
\item {\bf } : 
\item {\bf } : 
\end{itemize}

\subsubsection{Matière et rayonnement}
\begin{itemize}[leftmargin=1cm, label=\ding{32}, itemsep=1pt]
\item électrique : du latin scientifique {\it electricus}, de {\it electrum}, emprunté au grec {\it êlektron}, ambre jaune, d'après sa propriété d'attirer les corps légers quand on l'a frotté. {\bf électron} 1829, Boiste, Stoney.
\item neutre : du latin {\it neuter}, ni l'un ni l'autre. {\bf neutron} 1932, Joliot.
\item {\bf proton} : du grec {\it prôton}, neutre de {\it prôtos}, premier. 1920, Rutherford.
\item {\bf photon} : du grec {\it phôs}, {\it phôtos}, lumière. 1923, Louis de Broglie.
\end{itemize}

\subsection{Physique nucléaire}

\begin{itemize}[leftmargin=1cm, label=\ding{32}, itemsep=1pt]
\item {\bf quark} : ?
\item {\bf gluon} : colle ?
\item {\bf higgs} : physicien écossais.
\item {\bf neutrino} : petit neutre (de l'italien) ? {\bf neutrino} vers 1940.
\end{itemize}



\subsection{Quanton}
%

\begin{itemize}[leftmargin=1cm, label=\ding{32}, itemsep=1pt]
\item quantique : quantité (dénombrable ?)
\item quanton
\end{itemize}




%%%%%%%%%%%%%%%%%%%%%%%%%%%%%%%%%%%%%%%%%%%%%%%%%%%%%%%%%%%%%%%%%%%%%%%%%%%%%%%%
