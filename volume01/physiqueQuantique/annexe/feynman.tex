%%%%%%%%%%%%%%%%%%%%%%%%%%%%%%%
\chapter{Diagrammes de Feynman}
%%%%%%%%%%%%%%%%%%%%%%%%%%%%%%%

\label{JeanLucDeziel}

%%%%%%%%%%%%%%%%%%%%
\section{Définition}
%%%%%%%%%%%%%%%%%%%%

Un diagramme de Feynman est généralement un graphe en 2 dimensions où l'un des axe est attribué à l'espace, l'autre au temps.

Les diagrammes de Feynman sont une façon très simple de représenter des interactions entre des particules. \cite{diagrammesFeynman}

Dans le cas où l'on s'interesse à l'interaction entre les électrons et les photons, il existe trois règles d'une simplicité remarquable dont on doit tenir compte afin de construire ce genre de diagramme :
\begin{itemize}[leftmargin=1cm, label=\ding{32}, itemsep=1pt]
\item Un photon va d'un point à un autre
\item Un électron va d'un point à un autre
\item Un électron émet ou absorbe un photon
\end{itemize}

%%%%%%%%%%%%%%%%%
\section{Exemple}
%%%%%%%%%%%%%%%%%

\tikzset{
    photon/.style={decorate, decoration={snake}, draw=red},
    electron/.style={draw=blue, postaction={decorate},
        decoration={markings,mark=at position .55 with {\arrow[draw=blue]{>}}}},
    gluon/.style={decorate, draw=magenta,
        decoration={coil,amplitude=4pt, segment length=5pt}} 
}

\begin{tikzpicture}[
        thick,
        % Set the overall layout of the tree
        level/.style={level distance=1.5cm},
        level 2/.style={sibling distance=2.6cm},
        level 3/.style={sibling distance=2cm}
    ]
    \coordinate
        child[grow=left]{
            child {
                node {$g$}
                % The 'edge from parent' is actually not needed because it is
                % implicitly added.
                edge from parent [gluon]
            }
            child {
                node {$g$}
                edge from parent [gluon]
            }
            edge from parent [gluon] node [above=3pt] {$g$}
        }
        % I have to insert a dummy child to get the tree to grow
        % correctly to the right.
        child[grow=right, level distance=0pt] {
        child  {
            child {
                child {
                    node {$\bar{d}$}
                    edge from parent [electron]
                }
                child {
                    node {$u$}
                    edge from parent [electron]
                }
                edge from parent [photon]
            }
            child {
                node {$b$}
                edge from parent [electron]
            }
            edge from parent [electron]
            node [below] {$t$}
        }
        child {
            child {
                node {$\bar{b}$}
                edge from parent [electron]
            }
            child {
                child {
                    node {$\bar{v}$}
                    edge from parent [electron]
                }
                child {
                    node {$e^{-}$}
                    edge from parent [electron]
                }
                edge from parent [photon]
            }
            edge from parent [electron]
            node [above] {$\bar{t}$}
        }
    };
\end{tikzpicture}

%%%%%%%%%%%%%%%%%%%%%%%%%%%%%%%%%%%%%%%%%%%%%%%%%%%%%%%%%%%%%%%%%%%%%%%%%%%%%%%%%%%%%%%%%%%%%
