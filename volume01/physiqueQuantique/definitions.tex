
%%%%%%%%%%%%%%%%%%%%%
\section{Définitions}
%%%%%%%%%%%%%%%%%%%%%
%
%\subsection{Ondes et particules}
\subsection{Ondes et corpuscules}
%\subsection{Lumière et matière}

Un corpuscule est un "grain de matière". Le mouvement d'un corpuscule peut être décrit par sa position en fonction du temps. Au cours de son mouvement, un corpuscule conserve une nature ponctuelle.

%un corpuscule peut être considérée peut supposer les particules ponctuelles (leur taille est petite).
%. Ce mouvement peut se traduire comme un transfert d'énergie
%Une onde est un transfert d'énergie sans transport de matière.

Une onde a besoin d'un milieux matériel pour se propager. Le mouvement de l'onde consiste en une déformation de ce milieux. Une onde est nécessairement étendue dans l'espace. Au cours du temps, une onde s'étend, elle tend à remplir tout l'espace.

%Le mouvement d'une onde est décrit par la donnée de son "amplitude".

Décrire la matière comme constituée de particules revient à supposer qu'elle a une nature granulaire : la matière n'est pas divisible à l'infini, il existe une limite où l'on observe des grains indivisibles, des particules élémentaires. Cette description s'oppose à une vision "continue" de la matière.

%\subsubsection{Exemple microscopique}

%Les électrons sont des particules chargés électriquement.'interaction entre deux électrons

\subsection{Quanton}
%

La physique moderne ne décrit plus les particules élémentaires comme des corpuscules.
Un quanton n'est pas une onde et n'est pas un corpuscule. C'est un objet dont le mouvement est décrit par l'équation de schrödinger.

Un quanton se déplace comme une onde et se dévoile comme un corpuscule.

%%%%%%%%%%%%%%%%%%%%%%%%%%%%%%%%%%%%%%%%%%%%%%%%%%%%%%%%%%%%%%%%%%%%%%%%%%%%%%%%
