
%%%%%%%%%%%%%%%%%%%%%%%%%%%%%%%%%%%%%%%%%%%%

\section{Initiation au diagrammes de Feynman \cite{diagrammesFeynman}}

%%%%%%%%%%%%%%%%%%%%%%%%%%%%%%%%%%%%%%%%%%%

\label{JeanLucDeziel}

\subsection{Définition}

Un diagramme de Feynman est généralement un graphe en 2 dimensions où l'un des axe est attribué à l'espace, l'autre au temps, et dans lequel on représente les interactions entre les champs.

%Les champs échangent de l'énergie entre eux de manière quantique, à travers les phénomènes de création (ou émission) et d'anihilation (ou absorption) des quantons.

%On suppose que l'on dispose de dispositif expérimental permettant de détecter les particules.

Une vision corpusculaire des quantons permet de se familiariser avec ces diagrammes. Les diagrammes représentent alors les "trajectoires" des quantons.

\subsection{Exemple et conventions}

L'électron et le photon sont des quantons. L'électron possède une masse ainsi qu'une charge électrique (il est l'un des principaux constituant de la matière avec le proton et le neutron). Le photon est la "particule de la lumière". Il n'a ni masse ni charge électrique et il se déplace à la vitesse de la lumière.

Dans les diagrammes, un électron est représenté par un segment muni d'une flêche, un photon est représenté par un segment ondulé.

Dans les diagrammes suivants, j'ai rajouté les chiffres 1, 2, 3 et 4 car ils facilitent la lecture chronologique du diagramme (ils n'appartiennent pas aux conventions de Feynman).

\vspace{0.9cm}
\begin{minipage}[c]{.45\linewidth}
\includegraphics[scale=2.5]{./quantique/feynmann01}
\end{minipage}
\hfill
\begin{minipage}[c]{.45\linewidth}
\begin{center}
Explication chronologique :
\end{center}
Un électron se déplace de 1 à 3, un photon se déplace de 2 à 3. En 3, l'électron absorbe le photon. Muni de ce regain d'énergie, l'électron se précipite en 4.
\end{minipage}


\subsection{Autre exemple}

%\vspace{1.1cm}
\begin{minipage}[c]{.45\linewidth}
\begin{center}
Explication chronologique :
\end{center}
Un électron se déplace de 1 à 3 tandis qu'un autre électron se déplace de 2 à 4.

\end{minipage}
\hfill
\begin{minipage}[c]{.45\linewidth}
\includegraphics[scale=2.5]{./quantique/electronElectron}
\end{minipage}

\vspace{0.9cm}
On suppose que l'on dispose d'un dispositif expérimental permettant de détecter les électrons en 3 et en 4 (ainsi bien entendu qu'un dispositif émettant les électrons en 1 et en 2). La détection des électrons en 3 et en 4 permet de savoir que deux électrons ont été émis en 1 et en 2, mais ne permet pas de savoir ce qui s'est passé au cours de leur déplacement. Autrement dit, le diagramme précédent ne représente qu'une possiblité de ce qui s'est passé. Les diagrammes suivants représentent d'autre possibilités conduisant aux mêmes mesures expérimentales, aux mêmes observations.

\begin{center}
\includegraphics[scale=2.5]{./quantique/quatreChemins}
\end{center}

\subsubsection{Validation expérimentale}

%On pourrait imaginer que les deux électrons observés en 3 et 4 proviennent du même dispositif "émetteur" (en 1 ou en 2). En pratique, on peut supposer que la détection de deux électrons en 3 et 4 presque simultanément implique qu'ils n'ont pas été émis par le même "émeteur".

La réalisation d'un grand nombre d'observation permet de mesurer statistiquement la probalité d'observer ces deux électrons.

La théorie permet de calculer la probabilité de réalisation de chacun des chemins possibles, leur somme permet de retrouver la valeur mesurée par l'expérience.

Les diagrammes permettent de faciliter les calculs, l'accord entre la théorie et l'observation montre l'intérêt de ces diagrammes.

%%%%%%%%%%%%%%%%%%%%%%%%%%%%%%%%%%%%%%%%%%%%%%%%%%%%%%%%%%%%%%%%%%%%%%%%%%%%%
