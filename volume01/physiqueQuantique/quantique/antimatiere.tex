
%%%%%%%%%%%%%%%%%%%%%%%%%%%%%%%%%%%%%%%%%%%%

\section{Matière et antimatière}

%%%%%%%%%%%%%%%%%%%%%%%%%%%%%%%%%%%%%%%%%%%

%\subsection{Matière et antimatière}

%L'équation de Schrödinger est symétrique par rapport au temps. Elle n'exclut pas l'existence de quanton remontant le temps. 
L'équation de Dirac prévoit l'existence de quanton d'énergie "négative". Cette négativité de l'énergie sera par la suite réinterprétée comme étant des particules dont l'énergie est positive mais pour lesquelles le temps est "inversé". L'expérience a montré l'existence de ces particules que l'on appelle des anti-particules.

Un anti-électron (appelé positron ou positon) possède la même masse qu'un électron et une charge opposé. Lorsqu'un électron rencontre un positron, ils s'anihilent, donnant un photon (dont l'énergie est égale à la somme des énergies de l'électron et du positron).

Un autre fait expérimental observé est l'apparition d'électron et d'anti-électron à la suite de l'émission de photon dont l'énergie est supérieur à l'énergie de masse ($2\times m_ec^{\,2}$) de l'ensemble électron-positron.

La flèche porté par le segment représentant un électron indique le "sens du temps". Ainsi, un anti-électron est simplement représenté par un segment fléché où la flèche remonte le temps par rapport à l'axe temporel du diagramme.

Les diagrammes suivant, où l'axe du temps est maintenant représenté horizontalement, représente l'anihilation d'un électron et d'un positron, et la création d'un électron et d'un positron.

\begin{minipage}[c]{.45\linewidth}
\begin{center}
\includegraphics[scale=2.5]{./diagrammes/electronPhotonAnihilation}
\end{center}
Un électron en 1 et un positron en 2 s'anihilent en 3, donnant un photon en 4.
\end{minipage}
\hfill
\begin{minipage}[c]{.45\linewidth}
\begin{center}
\includegraphics[scale=2.5]{./diagrammes/electronPhotonCreation}
\end{center}
Un photon en 1 se matérialise en 2 créant un électron en 3 et un anti-électron en 4.
\end{minipage}

%%%%%%%%%%%%%%%%%%%%%%%%%%%%%%%%%%%%%%%%%%%%%%%%%%%%%%%%%%%%%%%%%%%%%%%%%%%%%
