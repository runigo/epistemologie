\documentclass[11pt, a4paper]{report}
%\documentclass[11pt, a4paper]{article}

%====================== PACKAGES ======================


\usepackage[french]{babel}

\frenchbsetup{StandardLists=true}
\usepackage{enumitem}
\usepackage{pifont}

\usepackage[utf8x]{inputenc}
%\usepackage[latin1]{inputenc} % Feynman

%pour gérer les positionnement d'images
\usepackage{float}
\usepackage{amsmath}
\DeclareMathOperator{\dt}{dt}
\usepackage{graphicx}
\usepackage{tabularx}
\usepackage[colorinlistoftodos]{todonotes}
\usepackage{url}
%pour les informations sur un document compilé en PDF et les liens externes / internes
\usepackage[pdfborder=0]{hyperref}
\hypersetup{
	colorlinks = true
	}
%pour la mise en page des tableaux
\usepackage{array}
\usepackage{tabularx}
\usepackage{multirow}
\usepackage{multicol}
\setlength{\columnsep}{50pt}
%pour utiliser \floatbarrier
%\usepackage{placeins}
%\usepackage{floatrow}
%espacement entre les lignes
\usepackage{setspace}
%modifier la mise en page de l'abstract
\usepackage{abstract}
%police et mise en page (marges) du document
\usepackage[T1]{fontenc}
\usepackage[top=2cm, bottom=2cm, left=2cm, right=2cm]{geometry}
%Pour les galerie d'images
\usepackage{subfig}

\usepackage{pdfpages}
\usepackage{tikz}
%\usepackage{tikz}
\usetikzlibrary{trees}
\usetikzlibrary{decorations.pathmorphing}
\usetikzlibrary{decorations.markings}
\usetikzlibrary{decorations.pathreplacing,calligraphy}
\usetikzlibrary{decorations.pathmorphing,calc,shapes,shapes.geometric,patterns}
%\usetikzlibrary{decorations}
\usetikzlibrary{angles, quotes}
\usepackage{verbatim}

\usepackage{appendix}

\usepackage{comment}

\usepackage{xcolor}

\hypersetup{colorlinks=true,linkcolor=black}

\usepackage{makeidx}

%\PreviewEnvironment{tikzpicture}
%\setlength\PreviewBorder{0pt}%

%====================== INFORMATION ET REGLES ======================

%rajouter les numérotation pour les \paragraphe et \subparagraphe
\setcounter{secnumdepth}{4}
\setcounter{tocdepth}{4}

\hypersetup{							% Information sur le document
pdfauthor = {Stephan Runigo},			% Auteurs
pdftitle = {Documentation},			% Titre du document
pdfsubject = {Documentation},		% Sujet
pdfkeywords = {Document},	% Mots-clefs
pdfstartview={FitH}}	% ajuste la page à la largeur de l'écran
%pdfcreator = {MikTeX},% Logiciel qui a crée le document
%pdfproducer = {} % Société avec produit le logiciel
\makeindex % Créer un index
%======================== DEBUT DU DOCUMENT ========================
%
\begin{document}
%
%régler l'espacement entre les lignes
\newcommand{\HRule}{\rule{\linewidth}{0.5mm}}
%
% Titre, résumé, ... %
\begin{titlepage}
%
~\\[1cm]

\begin{center}
%\includegraphics[scale=0.5]{./presentation/chambreABulle}
\end{center}

\textsc{\Large }\\[0.5cm]

% Title \\[0.4cm]
\HRule

\begin{center}
{\huge \bfseries  Temps et Énergie\\ }
 
%{\Large et révolution quantique\\}%\\[0.4cm]
\end{center}

\HRule \\[1.5cm]

\vspace{3cm}
\begin{itemize}[leftmargin=1cm, label=\ding{32}, itemsep=2pt]
\item {\bf Objet :} initiation à l'épistémologie, vocabulaire et relation scientifique.
\item {\bf Contenu :} le temps et l'énergie en physique.
\item {\bf Niveau :} vulgarisation scientifique.
\end{itemize}
\vspace{3cm}


% Author and supervisor
\begin{minipage}{0.4\textwidth}
\begin{flushleft} \large
%\emph{Auteur:}\\
%Stephan \textsc{Runigo}
\end{flushleft}
\end{minipage}
\begin{minipage}{0.4\textwidth}
\begin{flushright} \large
\emph{Auteur:}\\
Stephan \textsc{Runigo}
\end{flushright}
\end{minipage}

\vfill

% Bottom of the page
{\large \today}

\end{titlepage}

%
% Table des matières
\tableofcontents
\thispagestyle{empty}
\setcounter{page}{0}
%
%espacement entre les lignes des tableaux
\renewcommand{\arraystretch}{1.5}
%
%====================== INCLUSION DES PARTIES ======================
%
~
\thispagestyle{empty}
%recommencer la numérotation des pages à "1"
\setcounter{page}{0}
\newpage
%
%%
\chapter{Cosmologie}
%

%%%%%%%%%%%%%%%%%%%%%%%%%%%%%%%%%%%%%%%%%%%%
\section{Les cosmologies antiques}

%%%%%%%%%%%%%%%%%%%%%%%%%%%%%%%%%%%%%%%%%%%%

  \subsection{Introduction}
Il n'est pas impossible qu'au sein de certaine tribu, des hommes, connaissant les volcans, immaginairent qu'un volcan géant, loin à l'est, crachait chaque matin une boule de feu. Ils pensaient alors que cette boule de feu traversait le ciel au cours de la journée pour plonger dans la mer à l'ouest provoquant la venue de la nuit. Ils étaient dans un paradigme. L'idée que cette boule de feu pourrait être la même chaque jour, qu'il s'agisse d'un même objet qui tournerait autour de la terre constitue un nouveau paradigme. L'idée de cette rotation autour de la terre d'un soleil unique est en contradiction avec l'histoire du volcan. Il n'est alors pas impossible que de longues discussions ont eu lieu.

Depuis, de nombreux paradigmes ont été inventé.
% les paradigmes dans lesquels se trouvent les hommes n'ont cesser d
  \subsection{Première cosmologie}

\begin{center}

\begin{tikzpicture}
    \def\horizontal {0.35}
    \def\vertical {1.3}
\begin{scope}
\draw[-latex,color=darkgray] (-37*\horizontal,0) -- (5*\horizontal,0);
\draw[shift={(5*\horizontal,0)},color=darkgray,thin]
                                   node[below] {\footnotesize $temps$};
%3500 av JC
  \draw[shift={(-35*\horizontal,0)},color=darkgray,thin] (0pt,1pt) -- (0pt,-1pt)
                                   node[above,text width=3cm,text centered]{ Invention de l'écriture en mésopotamie}
                                   node[below] {\footnotesize -3500 ans};
%776 av JC
  \draw[shift={(-12*\horizontal,0)},color=darkgray,thin] (0pt,1pt) -- (0pt,-1pt)
                                   node[above,text width=3cm,text centered]{ Invention de l'alphabet par les phéniciens}
                                   node[below] {\footnotesize -1200 ans};
  \draw[shift={(0,0)},color=darkgray,thin] (0pt,1pt) -- (0pt,-1pt)
                                   node[below] {\footnotesize 0}
                                   node[above,text width=3cm,text centered]{ Naissance de Jésus Christ};
%\draw (1.4*\horizontal,0.5*\vertical) node [rotate=30]{Ptolémé};
\end{scope}
\begin{scope}[yshift= -1.5cm]
    \def\vertical {0.3}
  \shade[bottom color=brown!10!gray!10!green, top color=white,shading angle={90},rounded corners=1pt]
 (-37*\horizontal,0) rectangle (-35*\horizontal, \vertical);
  \shade[bottom color=brown!10!gray!10!green, top color=brown!10!gray!10!green,shading angle={90},rounded corners=1pt]
 (-35*\horizontal,0) rectangle (-5*\horizontal, \vertical);
  \shade[bottom color=white, top color=brown!10!gray!10!green,shading angle={90},rounded corners=1pt]
 (-5*\horizontal,0) rectangle (-3*\horizontal, \vertical);
  \node[above] (P) at (-20*\horizontal,\vertical) {Civilisation summérienne}; % SUMÉRIENS

    \def\decalage {-1}
  \shade[bottom color=brown!10!gray!10!green, top color=white,shading angle={90},rounded corners=1pt]
 (-28*\horizontal, \decalage) rectangle (-25*\horizontal, \vertical + \decalage);
  \shade[bottom color=brown!10!gray!10!green, top color=brown!10!gray!10!green,shading angle={90},rounded corners=1pt]
 (-25*\horizontal, \decalage) rectangle (-2*\horizontal, \vertical + \decalage);
  \shade[bottom color=white, top color=brown!10!gray!10!green,shading angle={90},rounded corners=1pt]
 (-2*\horizontal, \decalage) rectangle (0*\horizontal, \vertical + \decalage);
  \node[above] (P) at (-15*\horizontal,\vertical  + \decalage) {Civilisation égyptienne}; % ÉGYPTIENS

    \def\decalage {-2}
  \shade[bottom color=brown!10!gray!10!green, top color=white,shading angle={90},rounded corners=1pt]
 (-10*\horizontal, \decalage) rectangle (-8*\horizontal, \vertical + \decalage);
  \shade[bottom color=brown!10!gray!10!green, top color=brown!10!gray!10!green,shading angle={90},rounded corners=1pt]
 (-8*\horizontal, \decalage) rectangle (1*\horizontal, \vertical + \decalage);
  \shade[bottom color=white, top color=brown!10!gray!10!green,shading angle={90},rounded corners=1pt]
 (1*\horizontal, \decalage) rectangle (3*\horizontal, \vertical + \decalage);
  \node[above] (P) at (-4*\horizontal,\vertical  + \decalage) {Civilisation grec}; % CRECS
\end{scope}
\end{tikzpicture}


%%%%%%%%%%%%%%%%%%%%%%%%%%%%%%%%%%%%%%%%%%%




\end{center}

L'observation du ciel il y a des milliers d'années a conduit les hommes à se familiariser avec les phénomènes celeste.
 Le lien entre le mouvement des étoiles et le retour des saisons était utilisé pour l'agriculture.
 L'astronomie était alors surtout un outil de mesure du temps.

 Sont apparues alors des descriptions du monde en plus de la connaissances du mouvement des astres.
Ces descriptions sont les premiers modèles de cosmologie. Ils n'étaient pas accompagné d'explication rationnelle.

Les sumérien décrivaient l'univers comme une sphère et la Terre comme un disque entouré par la mer. 

\begin{center}

\begin{tikzpicture}
    \def\rayon {3}
    \def\vertical {1.3}

 %\shade[bottom color=blue!40!brown!40!, top color=cyan!20!]
 %\draw (0,0) circle (5cm);

\shade[bottom color=blue!20!cyan!20!, top color=blue!40!cyan!40!]
  (-\rayon,0) arc (180:0:\rayon) -- (-\rayon,0); % CIEL

\shade[bottom color=cyan!20!blue, top color=blue!60!cyan!20!]
  (-\rayon,0) arc (180:360:\rayon) -- (-\rayon,0); % MER

\shade[bottom color=cyan!10!blue!60!, top color=cyan!20!blue!60!]
 (0,0) ellipse (3 and 1.5); % MER

%\shade[bottom color=green!20!brown!60!, top color=brown, decoration={random steps, segment length=2mm}, decorate] (0,0) ellipse (2 and 1); % TERRE
\fill[color=yellow!60!brown, decoration={random steps, segment length=2mm}, decorate, rounded corners]
 (0,-0.05) ellipse (2.05 and 1.05); % SABLE
\fill[color=brown, decoration={random steps, segment length=2mm}, decorate, rounded corners] (0,0.0) ellipse (2 and 0.95); % TERRE


  \fill [green!40!black, decoration={random steps, segment length=2mm}, decorate, rounded corners=1pt]
(0,0) ellipse (1.9 and 0.9); % SOUS VÉGÉTATION
  \fill [green!60!black, decoration={random steps, segment length=1mm}, decorate, rounded corners=1pt](0,0.1) ellipse (2 and 0.9); % SUR VÉGÉTATION

%  \fill [green!70!black, decoration={random steps, segment length=1mm}, decorate](0,0.2) ellipse (0.6 and 0.3);

 % \fill [green!\f!black, decoration={random steps, segment length=0.4mm}, decorate](0,0) ellipse (\n *2 and \n);
 % \fill [green!\f!black, decoration={random steps, segment length=0.4mm}, decorate](0,0) ellipse (\n *2 and \n);


%\draw (-\rayon,0) arc (180:0:\rayon);



%(x0,y0) arc (angledébut:anglefin:rayon)

\end{tikzpicture}


%%%%%%%%%%%%%%%%%%%%%%%%%%%%%%%%%%%%%%%%%%%




\end{center}

Les égyptiens associaient le ciel et la Terre à des divinité et developpèrent une astrologie, croyance en un pouvoir des astres sur les hommes.

  \subsection{Les grecs}
Au premier millénaire avant notre ère, en Grèce, apparaît une volonté de rechercher une explication rationnelle du monde.
Les premières tentatives d'apporter cet ordre furent le fait de philosophes ioniens du {\footnotesize VII}$^\text{e}$ siècle avant J-C.

Apparurent alors plusieurs systèmes du monde différents. Ils décrivaient un {\it modèle} associé à des {\it principes naturelles} afin d'expliquer les observations, plutôt que de faire appel à la magie ou à la volonté des dieux. 

Au {\footnotesize VI}$^\text{e}$ siècle avant J-C, Pythagore et ses disciples développe une théorie du mouvement des corps célestes, appelée Harmonie des Sphères :

Des sphères en rotation portaient les corps célestes. La Terre était sphérique au centre du monde. La dernière sphère portait les étoiles fixes.


%Ce paradigme était incapable d'expliquer les irrégularités dans le déplacement des planètes, en particulier le mouvement rétrograde.
%%%%%%%%%%%%%%%%%%%%%%%%%%%%%%%%%%%%%%%%%%%





%%%%%%%%%%%%%%%%%%%%%%%%%%%%%%%%%%%%%%%%%%%%
\section{Avant Ptolemé}

%%%%%%%%%%%%%%%%%%%%%%%%%%%%%%%%%%%%%%%%%%%%

  \subsection{Le système du monde chez les anciens}

Dans les siècles précédents Jésus Christ, les hommes décrivaient des systèmes du monde expliquant le mouvement des astres. Dans cette section nous évoquerons les idées suivantes :

	\begin{itemize}[leftmargin=1cm, label=\ding{32}, itemsep=0pt]
		\item Unicité du soleil
		\item Sphéricité de la Terre
		\item Sphère des fixes
		\item Astres errant
	\end{itemize}

 \subsubsection{Unicité du soleil}

 \subsubsection{Sphéricité de la Terre}

 \subsubsection{Sphère des fixes}

 \subsubsection{Astres errant}

%%%%%%%%%%%%%%%%%%%%%%%%%%%%%%%%%%%%%%%%%%%





%%%%%%%%%%%%%%%%%%%%%%%%%%%%%%%%%%%%%%%%%%%%

\section{Ptolémée}

L'observation précise des astres érants, montre un profond écart entre le paradigme aristotélicien et la réalité.

Il sagit d'une crise. Faut-il remettre en cause le paradigme ou l'expérience ?




 % \subsubsection{}

%%%%%%%%%%%%%%%%%%%%%%%%%%%%%%%%%%%%%%%%%%%





%%%%%%%%%%%%%%%%%%%%%%%%%%%%%%%%%%%%%%%%%%%%

\section{La révolution copernicienne}

Copernic décrit un système du monde dans lequel le soleil se trouve au centre et ou la Terre se retrouve à tourner autour du soleil en plus de tourner sur elle-même.


Il s'agit d'une crise. Quel est le bon paradigme ?

Cette crise dure en fait depuis Aristarque de Samos. Aurait-elle duré 14 siècles ?


  \subsection{Crise}

 % \subsubsection{}

%%%%%%%%%%%%%%%%%%%%%%%%%%%%%%%%%%%%%%%%%%%





%%%%%%%%%%%%%%%%%%%%%
\section{Paradigmes}
%%%%%%%%%%%%%%%%%%%%%

\subsection{Ondes ou corpuscules ?}

%Démocrite suppose une composition atomique de la matière, déterminé par le seul principe de causalité. Il préfigure la pensée scientifique, ainsi que celle d'épicure, de Lucrèce et de Descartes.
La question de l'existence de l'atome (grain de matière indivisible) ou d'une matière continue, divisible à l'infini, a alimenté le débat scientifique depuis Démocrite (460-380 av JC). La lumière quand à elle, était souvent décrite par la notion de rayon lumineux, parfois par une notion de propagation.

En formulant le principe de Huygens, Christiaan Huygens (1629-1695), établissait le lien entre la propagation d'onde et les trajets des rayons lumineux.
L'optique de Newton (1661-1727) était une théorie corpusculaire jusqu'à un certain point. Newton mélange aux concepts corpusculaires d'autres concepts ("accès de réflexion facile","accès de transmission facile").
La démonstration du principe d'interférences par Thomas Young (1773-1829), imposa une interprétation ondulatoire des phénomènes lumineux.

Parallèlement, les développements de la chimie confirmaient la nature granulaire de la matière

Ainsi, la fin du XIX$^\text{\,e}$ siècle, la lumière était considérée comme un phénomène continu, alors que la matière était considéré comme granulaire. Tout comme Aristote séparait la mécanique céleste de la mécanique terrestre, les scientifiques séparait une mécanique granulaire d'une optique ondulatoire et continue.

Et tout comme la révolution copernicienne avait réconciliée les deux mécaniques d'Aristote, la révolution quantique allait tenter de réconcilier la mécanique granulaire de l'optique ondulatoire.



\subsection{Mécanique Newtoniennne}
%
La trajectoire d'un corps est décrite par ses coordonnées évoluant au cours du temps, obéissant au lois de la mécanique.
%
\subsection{Mécanique quantique}

L'évolution au cours du temps d'un quanton est décrite par sa fonction d'onde donnant l'amplitude de probabilité de mesurer ce quanton (de l'observer). La fonction d'onde evolue suivant l'équation de Schrödinger.

\subsection{Théorie quantique des champs}

Les différents champs échangent de l'énergie entre eux de manière quantique.

%%%%%%%%%%%%%%%%%%%%%%%%%%%%%%%%%%%%%%%%%%%%%%%%%%%%%%%%%%%%%%%%%%%%%%%%%%%%%%%%%%%%%

%
%

%
%\chapter{Relativité du mouvement}
%

%%%%%%%%%%%%%%%%%%%%%
\section{Définitions}
%%%%%%%%%%%%%%%%%%%%%
%
%\subsection{Ondes et particules}
\subsection{Ondes et corpuscules}
%\subsection{Lumière et matière}

Un corpuscule est un "grain de matière". Le mouvement d'un corpuscule peut être décrit par sa position en fonction du temps. Au cours de son mouvement, un corpuscule conserve une nature ponctuelle.

%un corpuscule peut être considérée peut supposer les particules ponctuelles (leur taille est petite).
%. Ce mouvement peut se traduire comme un transfert d'énergie
%Une onde est un transfert d'énergie sans transport de matière.

Une onde a besoin d'un milieux matériel pour se propager. Le mouvement de l'onde consiste en une déformation de ce milieux. Une onde est nécessairement étendue dans l'espace. Au cours du temps, une onde s'étend, elle tend à remplir tout l'espace.

%Le mouvement d'une onde est décrit par la donnée de son "amplitude".

Décrire la matière comme constituée de particules revient à supposer qu'elle a une nature granulaire : la matière n'est pas divisible à l'infini, il existe une limite où l'on observe des grains indivisibles, des particules élémentaires. Cette description s'oppose à une vision "continue" de la matière.

%\subsubsection{Exemple microscopique}

%Les électrons sont des particules chargés électriquement.'interaction entre deux électrons

\subsection{Quanton}
%

La physique moderne ne décrit plus les particules élémentaires comme des corpuscules.
Un quanton n'est pas une onde et n'est pas un corpuscule. C'est un objet dont le mouvement est décrit par l'équation de schrödinger.

Un quanton se déplace comme une onde et se dévoile comme un corpuscule.

%%%%%%%%%%%%%%%%%%%%%%%%%%%%%%%%%%%%%%%%%%%%%%%%%%%%%%%%%%%%%%%%%%%%%%%%%%%%%%%%

%

%%%%%%%%%%%%%%%%%%%%%%%%%%%%%%%%%%%%%%%%%%%%

\section{Propriétés}

    \subsection{Mouvement rectiligne uniforme}

Lorsque nous sommes dans une automobile en mouvement rectiligne uniforme, nous ne ressentons pas le mouvement. Nous pouvons avoir l'impression que c'est le paysage qui est en mouvement.

 En revanche, lorsque l'automobile freine (mouvement non uniforme) ou prend un virage (mouvement non rectiligne), nous ressentons des {\it accélérations}.

De la même façon, dans un train roulant à vitesse constante sur une voie en ligne droite, les passagers ne ressentent pas le mouvement.

\begin{center}
%%%%%%%%%%%%%%%%%%%%%%%%%%%%%%%%%%%%%%%%%%%%%%%%%%%%%%
%%%%%%%%%%%%%%%%%%%%%%%%%%%%%%%%%%%%%%%%%%%%%%%%%%%%%%%%%%%%%%%%%%%%
\def\scl{1}%scaling factor of the picture


\begin{tikzpicture}[
  scale=\scl,
  beige/.style={color=gray!20!brown!40!yellow!20!},
  orange/.style={color=red!70!yellow!70!},
  wagon/.style={green!70!brown!20!black!75!,draw=black,thick},
 % toit/.style={black!70!brown!20!,draw=gray,thick},!80!gray!10!brown!20!yellow!10!
  %roue/.style={brown!20!black!70!,draw=black,thick},
  fenetre/.style={white,rounded corners = 2pt,draw=black, thick},
  porte/.style={red!55!black,draw=gray!20!, ultra thin},
  porteMotrice/.style={rounded corners = 1pt,draw=gray!60!, ultra thin},
  essieux/.style={gray!20!brown!30!black!50!,draw=gray!70!black, ultra thin},
  grisEssieux/.style={gray!20!brown!30!black!50!}
  ]

  \begin{scope}[xshift=0 cm,yshift=0 cm]%, scale = 0.3
%
%         LIAISONS
%
 % \fill[color=gray,draw=gray!20!, ultra thin] % 
 %(0.15, 0.9) rectangle (12.95, 2.2);

%
%     CORPS DE LA MOTRICE
%

  \fill[color=gray] % 
 (4.5, 2.9) -- (5.75, 2.85) -- (5.85, 2.7) -- (-5.85, 2.7) -- 
 (-5.75, 2.85) -- (-4.5, 2.9) -- cycle;

  \fill[beige,draw=gray!50!, ultra thin] % 
 (5.85, 2.7) -- (5.9, 2.55) -- (5.8, 2.1) -- (6.25, 1.95) -- (6.32, 1.75)
 -- (6.2, 0.95) --  (-6.2, 0.95) -- (-6.32, 1.75) -- (-6.25, 1.95)
 -- (-5.8, 2.1) -- (-5.9, 2.55) -- (-5.85, 2.7) -- cycle;

  \fill[orange] % 
 (4.45, 2.55) -- (3.85, 1.8) -- (6.32, 1.75) -- (6.35, 1.6) -- (3.35, 1.6) -- (4.1, 2.4) 
 -- (-4.1, 2.4) -- (-3.35, 1.6) -- (-6.27, 1.6) -- (-6.32, 1.75) -- (-3.85, 1.8) -- (-4.45, 2.55) -- cycle;

  %  FEUX DROITE
  \fill[color=gray!50!,draw=gray, ultra thin]
 (6.3,1.7) rectangle (6.38, 1.6);
  \fill[color=gray!50!,draw=gray, ultra thin]
 (6.3,1.55) rectangle (6.38, 1.45);
  \fill[color=brown!20!red!50!yellow!70!,draw=gray, ultra thin]
 (6.2,1.7) -- (6.3, 1.7) -- (6.3, 1.4) -- (6.1, 1.4) -- cycle;
  \fill[color=gray]
 (6.23,1.4) rectangle (6.28, 1.2);
  %  FEUX GAUCHE
  \fill[color=gray!50!,draw=gray, ultra thin]
 (-6.3,1.7) rectangle (-6.38, 1.6);
  \fill[color=gray!50!,draw=gray, ultra thin]
 (-6.3,1.55) rectangle (-6.38, 1.45);
  \fill[color=brown!20!red!50!yellow!70!,draw=gray, ultra thin]
 (-6.2,1.7) -- (-6.3, 1.7) -- (-6.3, 1.4) -- (-6.1, 1.4) -- cycle;
  \fill[color=gray]
 (-6.23,1.4) rectangle (-6.28, 1.2);

 % GRILLE
  \fill[color=gray!50!]
 (3.2,1.6) -- (3.2, 2.4) -- 
 (-3.2,2.4) -- (-3.2, 1.6) --  cycle;



\foreach \t in {-1,1}
{
      % PARE BRISE ET GRIS AUTOUR
      % gris
  \fill[color=gray] % 
 (\t*4.5, 2.55) -- (\t*5.9, 2.55) -- (\t*5.8, 2.1) -- (\t*4.15, 2.1) --  cycle;
      % vitre
  \shade[bottom color=gray!5!, top color=gray!90!, shading angle={90}, draw=black]
 (\t*5.8, 2.54) -- (\t*5.87, 2.54) -- (\t*5.8, 2.12) -- (\t*5.6, 2.12) --  cycle;
      % gris
  \fill[color=gray] % 
 (\t*4.5, 2.55) -- (\t*5.85, 2.55) -- (\t*5.65, 2.1) -- (\t*4.15, 2.1) --  cycle;
      % PORTES
    \draw[porteMotrice]
 (\t*4.55, 1.3) rectangle (\t*5, 2.6);
      % MARCHES
    \draw[draw=gray!40!,very thin]
 (\t*4.62, 1.03) rectangle (\t*4.93, 1.15);
      % RAMPES
    \draw[color=gray] (\t*4.5, 1.03) -- (\t*4.5, 2.3);
    \draw[color=gray] (\t*5.05, 1.03) -- (\t*5.05, 2.3);

  % FENÊTRE
  \shade[bottom color=white, top color=black!80!, shading angle={90}]
   (\t*4.65, 2.12) rectangle (\t*4.9, 2.52);
}

 % ESSIEUX 1

\def\y{0.2}
  \foreach \x in {3.64, -3.64}
   {
 \fill[gray] 
  (\x - 0.65, 0.2) rectangle (\x + 0.65, 0.95);
  }



%  ROUES

\def\hauteur{0.45}% de l'axe des roues
\foreach \x in {2.58, 4.7, -4.7, -2.58}
  {
 % gris essieux : gray!20!brown!30!black!50!
    \fill[gray!20!brown!20!black!60!] (\x, \hauteur) circle (0.45 cm);
    \fill[gray!20!brown!40!black!40!] (\x, \hauteur) circle (0.41 cm);
    \fill[gray!10!brown!30!black!60!] (\x, \hauteur) circle (0.38 cm);
    % RESSORT
  \foreach \t in {-1,1}
  {
     \draw[decorate, decoration={snake, segment length=1.5pt, amplitude=1.2mm}, black!70!, thick]
       (\x + \t * 0.28, 0.7) -- (\x + \t * 0.28, 0.3);
    % RESSORT fixation supérieur
     \fill[gray!20!brown!20!black!60!]
       (\x + \t * 0.28 - .13, 0.9) rectangle (\x + \t * 0.28 + .13, 0.7);
  }
  }


 % ESSIEUX 2

\def\y{0.2}
  \foreach \x in {3.64, -3.64}
   {
\foreach \t in {-1,1}
  {  %  montants supérieur
    \coordinate (A) at (\x + \t*1.5, 0.95) ;
    \coordinate (B) at (\x + \t*1.5, 0.85) ;

    \coordinate (C) at (\x + \t*1, 0.83) ;
    \coordinate (D) at (\x + \t*0.65, 0.6) ;

    \coordinate (E) at (\x + \t*0.6, 0.4) ;
    \coordinate (F) at (\x + \t*0.45, 0.4) ;

    \coordinate (G) at (\x + \t*0.55, 0.95) ;
 \fill[essieux] (A) -- (B) -- (C) -- (D) -- (E) -- (F) -- (G) -- cycle;


    %  montant inférieur
    \coordinate (A) at (\x + \t * 1.5,0.4) ;
    \coordinate (B) at (\x + \t * 1.5,0.35) ;

    \coordinate (C) at (\x + \t * 1.2,0.3) ;
    \coordinate (D) at (\x + \t * 1,0.3) ;

    \coordinate (E) at (\x + \t * 0.80,0.32) ;

    \coordinate (F) at (\x + \t * 0.65,0.2) ;
    \coordinate (G) at (\x + \t * 0.65,0.4) ;
 \fill[essieux] (A) -- (B) -- (C) -- (D) -- (E) -- (F) -- (G) -- cycle;

 \fill[essieux] (\x + \t*1.06, \hauteur) circle (0.17 cm);
 \fill[grisEssieux]
 (\x + \t*1.06 + 0.16, 0.4) rectangle (\x + \t*1.06 + -0.16, 0.32);
 \fill[essieux]
 (\x + \t*1.06 + 0.1, \hauteur + -0.1) rectangle (\x + \t*1.06 + -0.1, \hauteur + 0.1);
    \fill[essieux] (\x + \t*1.06, \hauteur) circle (0.09 cm);
    \fill[essieux] (\x + \t*1.06, \hauteur) circle (0.04 cm);

  }

    %  milieux
  %(\x - 0.65, 0.2) rectangle (\x + 0.65, 0.95);
    \coordinate (A) at (\x-0.22, 0.85) ;
    \coordinate (B) at (\x + 0.22, 0.85) ;
    \coordinate (C) at (\x + 0.3, 0.35) ;
    \coordinate (D) at (\x - 0.3, 0.35) ;
 \fill[essieux, rounded corners = 3pt] (A) -- (B) -- (C) -- (D) -- cycle;

    %  milieux inférieur 
  %(\x - 0.65, 0.2) rectangle (\x + 0.65, 0.95);
    \coordinate (A) at (\x - 0.62,0.45) ;
    \coordinate (B) at (\x + 0.62,0.45) ;
    \coordinate (C) at (\x + 0.62,0.2) ;
    \coordinate (D) at (\x - 0.62,0.2) ;
 \fill[grisEssieux] (A) -- (B) -- (C) -- (D) -- cycle;

\foreach \t in {-1,1}    % silent bloc
  {
  \fill[brown!30!black!80!]
 (\x + \t * 0.38 - 0.1, 0.55) rectangle (\x + \t * 0.38 + 0.1, 0.35);
  \fill[essieux]
 (\x + \t * 0.38 - 0.1, 0.4) rectangle (\x + \t * 0.38 + 0.1, 0.25);
  }
  }

% BAS DE CAISSE


 % BUTOIR GAUCHE
  \fill[color=gray!80!black] % 
 (-6.1, 0.8) rectangle (-6.4, 0.95);
  \fill[color=black!80!gray] % 
 (-6.4, 0.75) rectangle (-6.45, 1);

 % BUTOIR DROIT
  \fill[color=gray!80!black] % 
 (6.1, 0.8) rectangle (6.4, 0.95);
  \fill[color=black!80!gray] % 
 (6.4, 0.75) rectangle (6.45, 1);

  % RAIL
  \shade[bottom color=brown!20!gray!60!black, top color=brown!20!gray!40!black]
 (6.6, 0) rectangle (-6.6, -0.1);

  \end{scope}
%
%
\end{tikzpicture}
%

\end{center}


  \subsection{Principe de relativité}

Énoncé par galilée au {\footnotesize XVII} $^\text{e}$ siècle, il exprime le fait que les lois de la physique restent les mêmes dans les différents référentiels, en mouvement rectiligne uniforme les uns avec les autres.

    \subsubsection{Exemple}

On observe la chute d'une balle dans le champ de pesanteur. Elle est accélérée vers le bas et son mouvement est rectiligne. Cette trajectoire met en évidence la verticale du lieu ou l'on réalise l'expérience. Cette expérience s'interprète dans le paradigme de la mécanique classique à l'aide de la force gravitationnelle que la Terre exerce sur la balle et par la loi reliant la force au mouvement.

\begin{center}
\input{./relativiteMouvements/mouvement/chuteSimple.tex}
\end{center}

Si l'on reproduit cette expérience dans un train en mouvement rectiligne uniforme, le principe d'inertie nous dit que le mouvement doit être identique, puisque les lois dans ce référentiel doivent être identiques.

\begin{center}
\input{./relativiteMouvements/mouvement/chuteWagon.tex}
\end{center}

  \subsection{Relativité des mouvements}

Le mouvements d'un objet dépend du référentiel dans lequel le mouvement est observé.

    \subsubsection{Additivité des vitesses}

Un train roule à la vitesse de 40 km/h. Un voyageur marche dans le couloir de ce train.
% à la vitesse de 5 km/h


\begin{center}
\input{./relativiteMouvements/mouvement/marcheWagon.tex}
\end{center}

%Un voyageur marche dans le couloir d'un train à la vitesse de 5 km/h. Le train roulant à la vitesse de 100 km/h. 

Pour un observateur dans le référentiel terrestre, le train a une vitesse de 40 km/h et le voyageur a une vitesse de 45 km/h. 

Le voyageur a une vitesse de 5 km/h dans le référentiel du train.

    \subsubsection{Relativité des trajectoires}

Dans le référentiel du train, la chute d'une balle lachée par un voyageur est rectiligne.

Dans le référentiel terrestre, la trajectoire de la balle est une courbe.

% \subsection{}

%%%%%%%%%%%%%%%%%%%%%%%%%%%%%%%%%%%%%%%%%%%{\it }




%

%%%%%%%%%%%%%%%%%%%%%
\section{L'horloge a lumière}
%%%%%%%%%%%%%%%%%%%%%

À la fin du {\footnotesize XIX}$^\text{e}$ siècle, les physiciens identifient la lumière avec des ondes électromagnétiques. Les équations fournissent une valeur de la vitesse de la lumière. Dans le vide, cette valeur est constante et devait être la même dans tout les {\it référentiels galliléens}. L'indépendance de la vitesse de la lumière suivant les référentiels va révolutionner la vision de l'espace-temps des physiciens du début du {\footnotesize XX}$^\text{e}$ siècle.

Dans cette section, nous allons nous attacher à décrire une horloge, utilisant une propriété de la lumière, afin de montrer le changement introduit dans la physique par la vision de l'espace temps après Einstein.

\subsection{Définition}

La technologie fournit des horloges performantes. De façon générale, une horloge fait appel à un {\it phénomène périodique} (pendule, oscilateur à ressort, vibration atomique) et à un {\it compteur} (cadran à aiguille, électronique).

L'horloge à lumière est une horloge imaginaire dans laquel de la lumière effecturait des allers-retours entre deux miroirs, un dispositif permettant de compter ces allers-retours. L'affichage de l'horloge va nous permettre de mesurer des durées en "aller-retour".


\subsection{Application}

Imaginons une de ces horloges embarqué dans le train. Imaginons également que nous puissions enregistrer le chemin parcouru par la lumière.

Le chemin parcouru par la lumière entre les deux miroirs dans l'horloge en mouvement est plus grand que dans l'horloge immobile. Autrement dit, alors que l'horloge lié au référentiel terrestre affiche 10 allers-retours, l'horloge en mouvement affiche 9 allers-retours.

Les horloges en mouvement semblent prendre du retard sur les horloges immobiles. 

%%%%%%%%%%%%%%%%%%%%%%%%%%%%%%%%%%%%%%%%%%%%%%%%%%%%%%%%%%%%%%%%%%%%%%%%%%{\it }

%

%%%%%%%%%%%%%%%%%%%%%
\section{Paradigmes}
%%%%%%%%%%%%%%%%%%%%%

\subsection{Ondes ou corpuscules ?}

%Démocrite suppose une composition atomique de la matière, déterminé par le seul principe de causalité. Il préfigure la pensée scientifique, ainsi que celle d'épicure, de Lucrèce et de Descartes.
La question de l'existence de l'atome (grain de matière indivisible) ou d'une matière continue, divisible à l'infini, a alimenté le débat scientifique depuis Démocrite (460-380 av JC). La lumière quand à elle, était souvent décrite par la notion de rayon lumineux, parfois par une notion de propagation.

En formulant le principe de Huygens, Christiaan Huygens (1629-1695), établissait le lien entre la propagation d'onde et les trajets des rayons lumineux.
L'optique de Newton (1661-1727) était une théorie corpusculaire jusqu'à un certain point. Newton mélange aux concepts corpusculaires d'autres concepts ("accès de réflexion facile","accès de transmission facile").
La démonstration du principe d'interférences par Thomas Young (1773-1829), imposa une interprétation ondulatoire des phénomènes lumineux.

Parallèlement, les développements de la chimie confirmaient la nature granulaire de la matière

Ainsi, la fin du XIX$^\text{\,e}$ siècle, la lumière était considérée comme un phénomène continu, alors que la matière était considéré comme granulaire. Tout comme Aristote séparait la mécanique céleste de la mécanique terrestre, les scientifiques séparait une mécanique granulaire d'une optique ondulatoire et continue.

Et tout comme la révolution copernicienne avait réconciliée les deux mécaniques d'Aristote, la révolution quantique allait tenter de réconcilier la mécanique granulaire de l'optique ondulatoire.



\subsection{Mécanique Newtoniennne}
%
La trajectoire d'un corps est décrite par ses coordonnées évoluant au cours du temps, obéissant au lois de la mécanique.
%
\subsection{Mécanique quantique}

L'évolution au cours du temps d'un quanton est décrite par sa fonction d'onde donnant l'amplitude de probabilité de mesurer ce quanton (de l'observer). La fonction d'onde evolue suivant l'équation de Schrödinger.

\subsection{Théorie quantique des champs}

Les différents champs échangent de l'énergie entre eux de manière quantique.

%%%%%%%%%%%%%%%%%%%%%%%%%%%%%%%%%%%%%%%%%%%%%%%%%%%%%%%%%%%%%%%%%%%%%%%%%%%%%%%%%%%%%

%
%\chapter{Plan chronologique}
%%%%%%%%%%%%%%%%%%%%%%%%%%%%%%%%%%%%%%%%%%%%

\section{La physique avant Galilé}
  \subsection{Les grecs anciens}
  \subsection{Le système de Ptolémé}

\section{La révolution copernicienne}
  \subsection{Le système de Copernic}
  \subsection{Le principe de relativité}
  \subsection{La gravitation universelle}

\section{La physique classique}
  \subsection{Lumière et matière}
  \subsection{Théorie de la chaleur}
  \subsection{Théorie des champs}

\section{La révolution quantique}
  \subsection{Matière et lumière}
  \subsection{Physique quantique}
  \subsection{Physique statistique}
\section{La révolution einsteinienne}
  \subsection{Mouvement et lumière}
  \subsection{Généralisation du principe de relativité}

\section{La physique contemporaine}
  \subsection{La relativité générale}
  \subsection{La théorie quantique des champs}

%%%%%%%%%%%%%%%%%%%%%%%%%%%%%%%%%%%%%%%%%%%




%

%
%\input{./theorieDesChamps/.tex}
%
%\input{./physiqueQuantique/physiqueQuantique/.tex}
%
%%
\chapter{Thermodynamique et physique statistique}
%

%%%%%%%%%%%%%%%%%%%%%%%%%%%%%%%%%%%%%%%%%%%%

\section{Thermodynamique classique}

 \subsection{Doctrine du phlogistique}

Développée au {\footnotesize XVII}$^\text{e}$ siècle, 


Modèle : le phlogiston est décrit comme une substance {\it fluide}, contenue dans les matières combustibles et libérée	 lors de la combustion.

Le volume de cendre restant d'un volume de bois brulé justifie ce modèle. La masse supérieur, des oxydes métalliques obtenus par combustion des métaux était expliqué par l'attribution d'une masse négative au phlogiston.

 \subsection{Révolution énergétique}

Au {\footnotesize XVIII}$^\text{e}$ siècle, Antoine-Laurent Lavoisier conclut à l'inexistence du phlogiston grace à l'utilisation systématique de la balance dans l'étude des réactions chimiques.

Plusieurs modèles de la chaleur s'affrontent : la chaleur est une substance, avec ou sans masse. La chaleur est un type de mouvement, une vibration.

\texttt{ Émilio Segré, p 222 et suivantes}

Nouveau paradigme : la chaleur est une forme microscopique de l'énergie.

Nouveau principe : l'énergie se conserve, de l'énergie thermique peut se convertir en énergie mécanique, de l'énergie mécanique peut se convertir en énergie thermique, 

 \subsection{Entropie}
L'irréversibilité des transformations thermodynamique conduit à l'invention d'une nouvelle grandeur physique, l'entropie, ainsi qu'à l'énoncé du second principe de la thermodynamique : au cours d'une transformation d'un système isolé, l'entropie augmente.


 % \subsubsection{}

%%%%%%%%%%%%%%%%%%%%%%%%%%%%%%%%%%%%%%%%%%%





\section{Physique statistique}
%\newpage

Le principe "Les mêmes causes produisent les mêmes effets" est retrouvé grâce à la physique statistique : il ne s'agit plus d'un principe fondamental, que l'on choisit de poser, mais d'un théorème que l'on démontre. 

%
%%%%%%%%%%%%%%%%%%%%%
\section{Paradigmes}
%%%%%%%%%%%%%%%%%%%%%

\subsection{Ondes ou corpuscules ?}

%Démocrite suppose une composition atomique de la matière, déterminé par le seul principe de causalité. Il préfigure la pensée scientifique, ainsi que celle d'épicure, de Lucrèce et de Descartes.
La question de l'existence de l'atome (grain de matière indivisible) ou d'une matière continue, divisible à l'infini, a alimenté le débat scientifique depuis Démocrite (460-380 av JC). La lumière quand à elle, était souvent décrite par la notion de rayon lumineux, parfois par une notion de propagation.

En formulant le principe de Huygens, Christiaan Huygens (1629-1695), établissait le lien entre la propagation d'onde et les trajets des rayons lumineux.
L'optique de Newton (1661-1727) était une théorie corpusculaire jusqu'à un certain point. Newton mélange aux concepts corpusculaires d'autres concepts ("accès de réflexion facile","accès de transmission facile").
La démonstration du principe d'interférences par Thomas Young (1773-1829), imposa une interprétation ondulatoire des phénomènes lumineux.

Parallèlement, les développements de la chimie confirmaient la nature granulaire de la matière

Ainsi, la fin du XIX$^\text{\,e}$ siècle, la lumière était considérée comme un phénomène continu, alors que la matière était considéré comme granulaire. Tout comme Aristote séparait la mécanique céleste de la mécanique terrestre, les scientifiques séparait une mécanique granulaire d'une optique ondulatoire et continue.

Et tout comme la révolution copernicienne avait réconciliée les deux mécaniques d'Aristote, la révolution quantique allait tenter de réconcilier la mécanique granulaire de l'optique ondulatoire.



\subsection{Mécanique Newtoniennne}
%
La trajectoire d'un corps est décrite par ses coordonnées évoluant au cours du temps, obéissant au lois de la mécanique.
%
\subsection{Mécanique quantique}

L'évolution au cours du temps d'un quanton est décrite par sa fonction d'onde donnant l'amplitude de probabilité de mesurer ce quanton (de l'observer). La fonction d'onde evolue suivant l'équation de Schrödinger.

\subsection{Théorie quantique des champs}

Les différents champs échangent de l'énergie entre eux de manière quantique.

%%%%%%%%%%%%%%%%%%%%%%%%%%%%%%%%%%%%%%%%%%%%%%%%%%%%%%%%%%%%%%%%%%%%%%%%%%%%%%%%%%%%%


%%%%%%%%%%%%%%%%%%%%%%%%%%%%%%%%%%%%%%%%%%%%

\section{Science et nature}

La science se préoccupe de ce qu'il advient, pas de ce qui est.

 \subsection{Propriétés des gaz}

Qu'advient-il lorsque l'on transperce une {\it enveloppe de caoutchouc} contenant un {\it gaz sous pression} ?




 % \subsubsection{}

%%%%%%%%%%%%%%%%%%%%%%%%%%%%%%%%%%%%%%%%%%%




%
%

%
%\chapter{Sommaire}
%
\chapter{Plan thématique}
%%%%%%%%%%%%%%%%%%%%%%%%%%%%%%%%%%%%%%%%%%%%

\section{Paradigmes et évolution scientifique}

Historiquement, la progression des sciences s'est réalisée par une succession de cycle :
\begin{center}
Science normale $\to$ crise $\to$ révolution $\to$ science normale.
\end{center}
L'évolution des sciences ne consiste pas tant en une accumulation de connaissance qu'à une remise en cause de ses paradigmes.

  \subsection{Discipline scientifique}
Une discipline scientifique est définie par l'ensemble des phénomènes qu'elle étudie, un ou plusieurs paradigmes, ainsi qu'un ensemble d'expériences.
 Il définissent un ensemble de phénomènes étudiés, qui constitue le cadre de leur discipline.
  \subsection{Paradigme et expérience}
Un paradigme scientifique est un ensemble de modèle, de lois et de principe. Un ensemble d'expériences est associé au paradigme, ces expériences vérifiant la mise en application du paradigme.
  \subsection{Science normale}
En période de sciences normale, une discipline scientifique est généralement dominée par un paradigme. L'ensemble des résultats expérimentaux s'acroit.
  \subsection{Révolution scientifique}
    \subsubsection{Crise}
Une expérience particulière, censé être décrite dans le paradigme, aboutit à un résultat en contradiction avec le paradigme.
Ou bien deux paradigmes contradictoires se trouvent en concurrence.
    \subsubsection{Résolution}
Soit l'expérience peut être remise en question, auquel cas le paradigme peut être conservé. Soit l'expérience remet en question le paradigme, auquel cas il fini par être changé.

\section{La physique classique}
  \subsection{Mouvement}
    \subsubsection{Notion de référentiel}
Un référentiel est le lieu ou l'on observe un mouvement. L'observateur dispose d'une horloge et de mesure de la position de l'objet dont on étudie le mouvement
    \subsubsection{Notion de forces et d'énergie}
Le mouvement d'un objet est influencé par les forces. Les lois de Newton permettent de prévoir la trajectoire des objets si l'on connait les forces s'exerçant sur eux.
Il découle de ces lois des lois de {\it conservations} : conservation de l'{\it énergie mécanique} et conservation de la {\it quantité de mouvement}.
  \subsection{Lumière et matière}
Le modèle de l'atome est un modèle {\it granulaire} de la matière. Dans le modèle ondulatoire, la lumière est décrite par des fonctions {\it continue}.

  \subsection{Théorie de la chaleur}
La chaleur est de l'énergie, l'entropie mesure le désordre microscopique, la lumière (décrite comme ondulatoire et continue) échange de l'énergie avec la matière (décrite comme atomique, quantifiable).

  \subsection{Théorie des champs}
Les charges électriques (objets granulaire, quantifiable) créent des champs électriques et magnétique (objet étendus dans l'espace et décrit par des fonctions continues)

\section{Les révolutions scientifiques en physique}

  \subsection{La révolution copernicienne}
    \subsubsection{Les grecs anciens}
Modèle : La Terre est fixe, la Lune, le Soleil, les planètes et les étoiles tournent autour de la Terre suivant des cercles.
Loi : le cercle est le mouvement naturel.
    \subsubsection{Le système de Ptolémé}
L'observation précise des trajectoires des planètes montre que celles-ci ont une trajectoire plus compliquée.
L'invention des épicycles conduit à un nouveau modèle, les planètes décrivent des cercles autour de point décrivant eux-même des cercles autour de la Terre. La loi est sauvée.
    \subsubsection{Le système de Copernic}
Le Soleil est fixe, les étoiles également, la Terre et les planètes tournent autour du Soleil, La lune toune autour de la Terre.
    \subsubsection{Galilé et Newton}
Principe de relativité, loi de la gravitation, loi de la mécanique.

  \subsection{La révolution quantique}
    \subsubsection{Matière et lumière}
Certaines expériences semblent ne pouvoir être interprété que si la lumière possède une nature granulaire, quantifiable, comme les constituant des atomes (électron, noyau)
Certaines expériences montrent que les électrons et les neutrons ont, comme la lumière, une nature ondulatoire
    \subsubsection{Physique quantique}
Le mouvement des quantons est décrit par les fonctions d'ondes (fonctions continues), les interactions entre les quantons sont quantifiable (granulaire, atomique).
    \subsubsection{La théorie quantique des champs}
Les champs sont décrit par des fonctions continues. Ils échangent entre eux de l'énergie, ces échanges sont quantifiés (discret, granulaire). Au cours de ces échanges, l'énergie est conservée ainsi que la quantité de mouvement, la charge électrique est conservée, sont aussi conservé des grandeurs propre à la dynamique des quarks (constituants des protons et des neutrons).

  \subsection{La révolution einsteinienne}
    \subsubsection{Mouvement et lumière}
L'identification de la lumière avec les ondes électromagnétiques introduit une contradiction au sein des équations de Newton.
%Les lois de la mécanique newtonienne sont en contradiction avec la description électromagnétique de la lumière.
 L'électromagnétisme et la mécanique sont deux paradigmes solidement établis par la conformité du résultat des expériences à leur prévisions.
    \subsubsection{Le principe de relativité}
Le principe de relativité énoncé par Galilé est conservé mais l'espace et le temps deviennent intimement liés. La transformation de Lorentz conduit à un changement de paradigme.
    \subsubsection{La relativité générale}
La généralisation du principe de relativité conduit à intégrer la gravitation au nouveau paradigme, la loi newtonienne de la gravitation devient une approximation de ce nouveau paradigme.


 % \subsubsection{}

%%%%%%%%%%%%%%%%%%%%%%%%%%%%%%%%%%%%%%%%%%%




%
\chapter{Plan chronologique}
%%%%%%%%%%%%%%%%%%%%%%%%%%%%%%%%%%%%%%%%%%%%

\section{La physique avant Galilé}
  \subsection{Les grecs anciens}
  \subsection{Le système de Ptolémé}

\section{La révolution copernicienne}
  \subsection{Le système de Copernic}
  \subsection{Le principe de relativité}
  \subsection{La gravitation universelle}

\section{La physique classique}
  \subsection{Lumière et matière}
  \subsection{Théorie de la chaleur}
  \subsection{Théorie des champs}

\section{La révolution quantique}
  \subsection{Matière et lumière}
  \subsection{Physique quantique}
  \subsection{Physique statistique}
\section{La révolution einsteinienne}
  \subsection{Mouvement et lumière}
  \subsection{Généralisation du principe de relativité}

\section{La physique contemporaine}
  \subsection{La relativité générale}
  \subsection{La théorie quantique des champs}

%%%%%%%%%%%%%%%%%%%%%%%%%%%%%%%%%%%%%%%%%%%




%

%
%%
\chapter{Sciences normales et révolutions scientifiques}
%

%%%%%%%%%%%%%%%%%%%%%
\section{Définitions}
%%%%%%%%%%%%%%%%%%%%%
%
%\subsection{Ondes et particules}
\subsection{Ondes et corpuscules}
%\subsection{Lumière et matière}

Un corpuscule est un "grain de matière". Le mouvement d'un corpuscule peut être décrit par sa position en fonction du temps. Au cours de son mouvement, un corpuscule conserve une nature ponctuelle.

%un corpuscule peut être considérée peut supposer les particules ponctuelles (leur taille est petite).
%. Ce mouvement peut se traduire comme un transfert d'énergie
%Une onde est un transfert d'énergie sans transport de matière.

Une onde a besoin d'un milieux matériel pour se propager. Le mouvement de l'onde consiste en une déformation de ce milieux. Une onde est nécessairement étendue dans l'espace. Au cours du temps, une onde s'étend, elle tend à remplir tout l'espace.

%Le mouvement d'une onde est décrit par la donnée de son "amplitude".

Décrire la matière comme constituée de particules revient à supposer qu'elle a une nature granulaire : la matière n'est pas divisible à l'infini, il existe une limite où l'on observe des grains indivisibles, des particules élémentaires. Cette description s'oppose à une vision "continue" de la matière.

%\subsubsection{Exemple microscopique}

%Les électrons sont des particules chargés électriquement.'interaction entre deux électrons

\subsection{Quanton}
%

La physique moderne ne décrit plus les particules élémentaires comme des corpuscules.
Un quanton n'est pas une onde et n'est pas un corpuscule. C'est un objet dont le mouvement est décrit par l'équation de schrödinger.

Un quanton se déplace comme une onde et se dévoile comme un corpuscule.

%%%%%%%%%%%%%%%%%%%%%%%%%%%%%%%%%%%%%%%%%%%%%%%%%%%%%%%%%%%%%%%%%%%%%%%%%%%%%%%%

\input{./evolution/normale.tex}

%%%%%%%%%%%%%%%%%%%%%%%%%%%%%%%%%%%%%%%%%%%%

\section{Crise}

  \subsection{Science normale}

En période de sciences normale, une discipline scientifique est généralement dominée par un paradigme. L'ensemble des résultats expérimentaux s'acroit.

Ce à quoi la science s'occupe

Expérimentation

Recherche



 % \subsubsection{}

%%%%%%%%%%%%%%%%%%%%%%%%%%%%%%%%%%%%%%%%%%%





%%%%%%%%%%%%%%%%%%%%%%%%%%%%%%%%%%%%%%%%%%%%

\section{Révolution scientifique}

L'évolution des sciences ne consiste pas tant en une accumulation de connaissance qu'à une remise en cause de ses paradigmes.


Historiquement, la progression des sciences s'est réalisée par une succession de cycle :
\begin{center}
Science normale $\to$ crise $\to$ révolution $\to$ science normale.
\end{center}


  \subsection{Crise}

Une expérience particulière, censé être décrite dans le paradigme, aboutit à un résultat en contradiction avec le paradigme.
Ou bien deux paradigmes contradictoires se trouvent en concurrence.

  \subsection{Résolution}


Soit l'expérience peut être remise en question, auquel cas le paradigme peut être conservé. Soit l'expérience remet en question le paradigme, auquel cas il fini par être changé.

 % \subsubsection{}

%%%%%%%%%%%%%%%%%%%%%%%%%%%%%%%%%%%%%%%%%%%




%
%

%
%
%%
\begin{appendix}
%

%%%%%%%%%%%%%%%%%%%%%
\section{Définitions}
%%%%%%%%%%%%%%%%%%%%%
%
%\subsection{Ondes et particules}
\subsection{Ondes et corpuscules}
%\subsection{Lumière et matière}

Un corpuscule est un "grain de matière". Le mouvement d'un corpuscule peut être décrit par sa position en fonction du temps. Au cours de son mouvement, un corpuscule conserve une nature ponctuelle.

%un corpuscule peut être considérée peut supposer les particules ponctuelles (leur taille est petite).
%. Ce mouvement peut se traduire comme un transfert d'énergie
%Une onde est un transfert d'énergie sans transport de matière.

Une onde a besoin d'un milieux matériel pour se propager. Le mouvement de l'onde consiste en une déformation de ce milieux. Une onde est nécessairement étendue dans l'espace. Au cours du temps, une onde s'étend, elle tend à remplir tout l'espace.

%Le mouvement d'une onde est décrit par la donnée de son "amplitude".

Décrire la matière comme constituée de particules revient à supposer qu'elle a une nature granulaire : la matière n'est pas divisible à l'infini, il existe une limite où l'on observe des grains indivisibles, des particules élémentaires. Cette description s'oppose à une vision "continue" de la matière.

%\subsubsection{Exemple microscopique}

%Les électrons sont des particules chargés électriquement.'interaction entre deux électrons

\subsection{Quanton}
%

La physique moderne ne décrit plus les particules élémentaires comme des corpuscules.
Un quanton n'est pas une onde et n'est pas un corpuscule. C'est un objet dont le mouvement est décrit par l'équation de schrödinger.

Un quanton se déplace comme une onde et se dévoile comme un corpuscule.

%%%%%%%%%%%%%%%%%%%%%%%%%%%%%%%%%%%%%%%%%%%%%%%%%%%%%%%%%%%%%%%%%%%%%%%%%%%%%%%%

%
\newpage
%

%%%%%%%%%%%%%%%%%%%%%
\section{Paradigmes}
%%%%%%%%%%%%%%%%%%%%%

\subsection{Ondes ou corpuscules ?}

%Démocrite suppose une composition atomique de la matière, déterminé par le seul principe de causalité. Il préfigure la pensée scientifique, ainsi que celle d'épicure, de Lucrèce et de Descartes.
La question de l'existence de l'atome (grain de matière indivisible) ou d'une matière continue, divisible à l'infini, a alimenté le débat scientifique depuis Démocrite (460-380 av JC). La lumière quand à elle, était souvent décrite par la notion de rayon lumineux, parfois par une notion de propagation.

En formulant le principe de Huygens, Christiaan Huygens (1629-1695), établissait le lien entre la propagation d'onde et les trajets des rayons lumineux.
L'optique de Newton (1661-1727) était une théorie corpusculaire jusqu'à un certain point. Newton mélange aux concepts corpusculaires d'autres concepts ("accès de réflexion facile","accès de transmission facile").
La démonstration du principe d'interférences par Thomas Young (1773-1829), imposa une interprétation ondulatoire des phénomènes lumineux.

Parallèlement, les développements de la chimie confirmaient la nature granulaire de la matière

Ainsi, la fin du XIX$^\text{\,e}$ siècle, la lumière était considérée comme un phénomène continu, alors que la matière était considéré comme granulaire. Tout comme Aristote séparait la mécanique céleste de la mécanique terrestre, les scientifiques séparait une mécanique granulaire d'une optique ondulatoire et continue.

Et tout comme la révolution copernicienne avait réconciliée les deux mécaniques d'Aristote, la révolution quantique allait tenter de réconcilier la mécanique granulaire de l'optique ondulatoire.



\subsection{Mécanique Newtoniennne}
%
La trajectoire d'un corps est décrite par ses coordonnées évoluant au cours du temps, obéissant au lois de la mécanique.
%
\subsection{Mécanique quantique}

L'évolution au cours du temps d'un quanton est décrite par sa fonction d'onde donnant l'amplitude de probabilité de mesurer ce quanton (de l'observer). La fonction d'onde evolue suivant l'équation de Schrödinger.

\subsection{Théorie quantique des champs}

Les différents champs échangent de l'énergie entre eux de manière quantique.

%%%%%%%%%%%%%%%%%%%%%%%%%%%%%%%%%%%%%%%%%%%%%%%%%%%%%%%%%%%%%%%%%%%%%%%%%%%%%%%%%%%%%

%
%
\end{appendix}
%

%
%\printindex
%
%====================== INCLUSION DE LA BIBLIOGRAPHIE ======================
%
%récupérer les citation avec "/footnotemark" : 
\nocite{*}
%
% choix du style de la biblio
\bibliographystyle{plain}
%
% inclusion de la biblio
\cleardoublepage
\addcontentsline{toc}{chapter}{Bibliographie}
\bibliography{bibliographie.bib}
%
\end{document}
%%%%%%%%%%%%%%%%%%%%%%%%%%%%%%%%%%%%%%%%%%%%%%%%%%%%%%%%%%%%%%%%%%%%%%%%%%%%%%%%%
