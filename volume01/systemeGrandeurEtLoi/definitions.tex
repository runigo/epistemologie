
%%%%%%%%%%%%%%%%%%%%%%%%%%%%%%%%%%%%%%%%%%%%

\section{Définitions}

	\begin{itemize}[leftmargin=1cm, label=\ding{32}, itemsep=1pt]
		\item {\bf Système physique} : Ensemble de corps matériels bien définis.
		\item {\bf Grandeur physique} : Mesure sur un système physique. Caractéristique, {\it observables}...
		\item {\bf Loi physique} : Relation entre des grandeurs physiques, valable pour un certain ensemble de systèmes physiques.
	\end{itemize}

\section{Exemples}

\subsection{Système solaire et gravitation}

Le système solaire est défini par l'ensemble du soleil et des planètes gravitant autour. Il est possible d'y ajouter les satellites (Lunes des planètes), les ceintures d'astéroïdes, les comètes. Dans un soucis de simplification, nous nous contenterons d'une définition simple du système solaire :
\begin{center}
Système solaire $=$ \{ Soleil, Mercure, Vénus, Terre, Mars, Jupiter, Saturne, Uranus, Neptune, Pluton \}
\end{center}

À ce système physique nous pouvons associer  des grandeurs permettant de le caractériser : masse du soleil et des planètes, distance moyenne au soleil, durée de révolution.
Ces grandeurs sont déterminées par des observations et des mesures à partir d'étalon définissant des unitées.

\begin{tabular}{ c c c }
 1 & 0 & 5 \\ 2 & -1 & 6 \\ 3 & 4 & 7 \\
\end{tabular}





De nouvelles grandeurs peuvent être définies par des formules mathématiques : vitesse des planètes, ...

\begin{tabular}{ c c c }
 1 & 0 & 5 \\ 2 & -1 & 6 \\ 3 & 4 & 7 \\
\end{tabular}

	\begin{itemize}[leftmargin=1cm, label=\ding{32}, itemsep=1pt]
		\item  :
		\item Pendule : corps suspendu à un fil rigide
		\item Système thermodynamique : Eau + vapeur dans une cocotte minute
	\end{itemize}
\subsection{Exemple de grandeur physique}
	\begin{itemize}[leftmargin=1cm, label=\ding{32}, itemsep=1pt]
		\item Masse, position, date (temps)
		\item Énergie, entropie, fréquence
		\item pression, température, volume
	\end{itemize}
\subsection{Exemple de loi physique}
	\begin{itemize}[leftmargin=1cm, label=\ding{32}, itemsep=1pt]
		\item Loi des aires
		\item E $=\hbar\omega$
		\item P.V $=$ n.R.T
	\end{itemize}





%%%%%%%%%%%%%%%%%%%%%%%%%%%%%%%%%%%%%%%%%%%
