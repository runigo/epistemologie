
%%%%%%%%%%%%%%%%%%%%%%%%%%%%%%%%%%%%%%%%%%%%

\section{Exemple de la gravitation}

  \subsection{Galilé}
\begin{itemize}[leftmargin=1cm, label=\ding{32}, itemsep=1pt]
  \item Systèmes : ensemble des corps macroscopique à proximité de la Terre
  \item Grandeurs : durée de la chute et hauteur de la chute.
  \item Loi : la hauteur de chute est proportionnel au carré de la durée.
\end{itemize}

  \subsection{Kepler}
\begin{itemize}[leftmargin=1cm, label=\ding{32}, itemsep=1pt]
  \item Systèmes : ensemble des planêtes
  \item Grandeurs : durée et distance au soleil
  \item Loi : loi des aires
\end{itemize}
\vspace{0.24cm}
{\footnotesize D'après https://www.futura-sciences.com/sciences/definitions/physique-deuxieme-loi-kepler-5104/}
\vspace{0.31cm}

{\it Enoncé de la deuxième loi de Kepler ou "Loi des aires" : le rayon-vecteur reliant une planète au Soleil balaie des aires égales en des temps égaux. Le rayon-vecteur est une droite imaginaire reliant le Soleil, situé à un des foyers de l'ellipse, à la planète située sur l'ellipse.

Cette deuxième loi permet d'affirmer qu'une planète se déplacera plus rapidement sur son orbite au périhélie qu'à l'aphélie.}


  \subsection{Newton}
\begin{itemize}[leftmargin=1cm, label=\ding{32}, itemsep=1pt]
  \item Systèmes : ensemble des corps possédant une masse
  \item Grandeurs : masses, distance
  \item Loi : F = G m m' / d$^2$
\end{itemize}
\[
\overrightarrow{\mt{F}} = \mt{G}\frac{\mt{m}_1\mt{m}_2}{\mt{d}^2}\hat{u}
\]
%%%%%%%%%%%%%%%%%%%%%%%%%%%%%%%%%%%%%%%%%%%



