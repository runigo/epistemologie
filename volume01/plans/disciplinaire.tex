\chapter{Plan disciplinaire}
%%%%%%%%%%%%%%%%%%%%%%%%%%%%%%%%%%%%%%%%%%%%

%\part{La physique classique}

\chapter{Mécanique}
  \section{Les grecs anciens}

  \section{Le système de Ptolémé}
Épicycle
  \section{Galilé}
Principe de relativité
  \section{Newton}
Gravitation universelle
  \section{Hamilton}
Conservation de l'énergie mécanique

\chapter{Optique}
  \section{Le rayon lumineux}
  \section{Huyghens}
La lumière est un phénomène ondulatoire
  \section{Planck}
La lumière est un phénomène quantique (granulaire)

\chapter{Thermodynamique}
  \section{Le phlogistique}
La chaleur est un fluide
  \section{Carnot}
La chaleur est une forme de l'énergie
  \section{Boltzmann}
L'entropie mesure le désordre microscopique

\chapter{Électromagnétisme}
  \section{Le magnétisme}
La bousole
  \section{L'électricité}
La triboélectricité et la pile électrique
  \section{La théorie des champs}
  \section{Maxwell}
Unification

%\part{La physique moderne}

\chapter{La théorie quantique}
  \section{Planck}
  \section{De Broglie}

\chapter{La théorie relativiste du temps}
  \section{L'horloge à lumière}
  \section{La transformation de Lorentz}

\chapter{La théorie relativiste de la gravitation}
  \section{Mouvement et lumière}
\section{Généralisation du principe de relativité}

\chapter{La théorie quantique des champs}
  \section{Les diagrammes de Feynman}

%%%%%%%%%%%%%%%%%%%%%%%%%%%%%%%%%%%%%%%%%%%



