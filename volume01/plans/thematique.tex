\chapter{Plan thématique}
%%%%%%%%%%%%%%%%%%%%%%%%%%%%%%%%%%%%%%%%%%%%

\section{Paradigmes et évolution scientifique}

Historiquement, la progression des sciences s'est réalisée par une succession de cycle :
\begin{center}
Science normale $\to$ crise $\to$ révolution $\to$ science normale.
\end{center}
L'évolution des sciences ne consiste pas tant en une accumulation de connaissance qu'à une remise en cause de ses paradigmes.

  \subsection{Discipline scientifique}
Une discipline scientifique est définie par l'ensemble des phénomènes qu'elle étudie, un ou plusieurs paradigmes, ainsi qu'un ensemble d'expériences.
 Il définissent un ensemble de phénomènes étudiés, qui constitue le cadre de leur discipline.
  \subsection{Paradigme et expérience}
Un paradigme scientifique est un ensemble de modèle, de lois et de principe. Un ensemble d'expériences est associé au paradigme, ces expériences vérifiant la mise en application du paradigme.
  \subsection{Science normale}
En période de sciences normale, une discipline scientifique est généralement dominée par un paradigme. L'ensemble des résultats expérimentaux s'acroit.
  \subsection{Révolution scientifique}
    \subsubsection{Crise}
Une expérience particulière, censé être décrite dans le paradigme, aboutit à un résultat en contradiction avec le paradigme.
Ou bien deux paradigmes contradictoires se trouvent en concurrence.
    \subsubsection{Résolution}
Soit l'expérience peut être remise en question, auquel cas le paradigme peut être conservé. Soit l'expérience remet en question le paradigme, auquel cas il fini par être changé.

\section{La physique classique}
  \subsection{Mouvement}
    \subsubsection{Notion de référentiel}
Un référentiel est le lieu ou l'on observe un mouvement. L'observateur dispose d'une horloge et de mesure de la position de l'objet dont on étudie le mouvement
    \subsubsection{Notion de forces et d'énergie}
Le mouvement d'un objet est influencé par les forces. Les lois de Newton permettent de prévoir la trajectoire des objets si l'on connait les forces s'exerçant sur eux.
Il découle de ces lois des lois de {\it conservations} : conservation de l'{\it énergie mécanique} et conservation de la {\it quantité de mouvement}.
  \subsection{Lumière et matière}
Le modèle de l'atome est un modèle {\it granulaire} de la matière. Dans le modèle ondulatoire, la lumière est décrite par des fonctions {\it continue}.

  \subsection{Théorie de la chaleur}
La chaleur est de l'énergie, l'entropie mesure le désordre microscopique, la lumière (décrite comme ondulatoire et continue) échange de l'énergie avec la matière (décrite comme atomique, quantifiable).

  \subsection{Théorie des champs}
Les charges électriques (objets granulaire, quantifiable) créent des champs électriques et magnétique (objet étendus dans l'espace et décrit par des fonctions continues)

\section{Les révolutions scientifiques en physique}

  \subsection{La révolution copernicienne}
    \subsubsection{Les grecs anciens}
Modèle : La Terre est fixe, la Lune, le Soleil, les planètes et les étoiles tournent autour de la Terre suivant des cercles.
Loi : le cercle est le mouvement naturel.
    \subsubsection{Le système de Ptolémé}
L'observation précise des trajectoires des planètes montre que celles-ci ont une trajectoire plus compliquée.
L'invention des épicycles conduit à un nouveau modèle, les planètes décrivent des cercles autour de point décrivant eux-même des cercles autour de la Terre. La loi est sauvée.
    \subsubsection{Le système de Copernic}
Le Soleil est fixe, les étoiles également, la Terre et les planètes tournent autour du Soleil, La lune toune autour de la Terre.
    \subsubsection{Galilé et Newton}
Principe de relativité, loi de la gravitation, loi de la mécanique.

  \subsection{La révolution quantique}
    \subsubsection{Matière et lumière}
Certaines expériences semblent ne pouvoir être interprété que si la lumière possède une nature granulaire, quantifiable, comme les constituant des atomes (électron, noyau)
Certaines expériences montrent que les électrons et les neutrons ont, comme la lumière, une nature ondulatoire
    \subsubsection{Physique quantique}
Le mouvement des quantons est décrit par les fonctions d'ondes (fonctions continues), les interactions entre les quantons sont quantifiable (granulaire, atomique).
    \subsubsection{La théorie quantique des champs}
Les champs sont décrit par des fonctions continues. Ils échangent entre eux de l'énergie, ces échanges sont quantifiés (discret, granulaire). Au cours de ces échanges, l'énergie est conservée ainsi que la quantité de mouvement, la charge électrique est conservée, sont aussi conservé des grandeurs propre à la dynamique des quarks (constituants des protons et des neutrons).

  \subsection{La révolution einsteinienne}
    \subsubsection{Mouvement et lumière}
L'identification de la lumière avec les ondes électromagnétiques introduit une contradiction au sein des équations de Newton.
%Les lois de la mécanique newtonienne sont en contradiction avec la description électromagnétique de la lumière.
 L'électromagnétisme et la mécanique sont deux paradigmes solidement établis par la conformité du résultat des expériences à leur prévisions.
    \subsubsection{Le principe de relativité}
Le principe de relativité énoncé par Galilé est conservé mais l'espace et le temps deviennent intimement liés. La transformation de Lorentz conduit à un changement de paradigme.
    \subsubsection{La relativité générale}
La généralisation du principe de relativité conduit à intégrer la gravitation au nouveau paradigme, la loi newtonienne de la gravitation devient une approximation de ce nouveau paradigme.


 % \subsubsection{}

%%%%%%%%%%%%%%%%%%%%%%%%%%%%%%%%%%%%%%%%%%%



