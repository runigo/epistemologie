\chapter{Plan du volume 1}
%%%%%%%%%%%%%%%%%%%%%%%%%%%%%%%%%%%%%%%%%%%%

\section{Systèmes du monde}
%%%%%%%%%%%%%%%%%%%%%%%%%%%

Notions de cette section : Modèles.

    \subsection{Les antiques}
Un seul modèle : terre plate au centre de l'univers.

Principes et lois : les dieux ont créé cet univers, le mouvement des astres influençent les hommes.

    \subsection{Les grecs}
Modèles concurrents : Terre plate, Terre sphérique, au centre de l'univers ou en mouvement.


    \subsection{Révolution copernicienne}

Modèle héliocentrique et trajectoires éliptiques.

\section{Mouvement et relativité}
%%%%%%%%%%%%%%%%%%%%%%%%%%%%%%%%%

Notion introduite dans cette section : lois et principes.

    \subsection{Les grecs}

Principes concurents : un élément : l'eau; quatre éléments  : l'eau, l'air, la terre, le feu.

Principes aristotéliciens : sur Terre, le mouvement naturel est la ligne droite vers le lieu de son être (la pierre tombe vers le bas, le feu monte vers le haut). Dans le ciel, le mouvement naturel est le cercle (les astres tournent autour de la Terre)

    \subsection{Révolution galiléenne}

Principes galiléens : Le mouvement naturel est la ligne droite, et il est relatif.

    \subsection{Révolution einsteinienne}

Principes einsteinien : la vitesse de la lumière est constante, 

\section{Physique quantique}
%%%%%%%%%%%%%%%%%%%%%%%%%%%%

Notion de cette section : l'unification.

  \subsection{Lumière et matière}

Modèle : la matière est granulaire (constituée de corpuscules infiniments petits et indivisibles), la lumière est une onde électromagnétique.

Principe : la lumière interagit avec la matière.


  \subsection{Révolution quantique}

Les particules élémentaires ont un caractère ondulatoire, la lumière interagit de manière granulaire.

Unification : lumière et matière sont constituées de quantons, lumière et matière obéissent au même lois (mouvement à caractère ondulatoire et interaction ponctuel et granulaire)

\section{théorie des champs}
%%%%%%%%%%%%%%%%%%%%%%%%%%%%

Notion de cette section : l'unification.

    \subsection{Électrostatique}
Force et champs
    \subsection{Magnétisme}
Force et champs
    \subsection{Unification}
Expérience d'oersted

  \subsection{Lumière et matière}
Équation de maxwell : lois décrivant la force électromagnétique.

\section{Physique statistique}
%%%%%%%%%%%%%%%%%%%%%%%%%%%%%%

  \subsection{Théorie de la chaleur}
    \subsection{Thermodynamique}
    \subsection{Déterminisme et hasard}


\section{Physique contemporaine}
%%%%%%%%%%%%%%%%%%%%%%%%%%%%%%%%

Cette section expose les visions modernes de l'énergie et du temps.

  \subsection{Espace et temps}
Modèle : L'espace-temps est un espace à quatre dimension.
  \subsection{Energie et mouvement}
Principe : au cours d'une transformation d'un système isolé, l'énergie et la quantité de mouvement se conservent.
  \subsection{Intrication quantique}
Un système {\it lié} de plusieurs quantons est un quantons
  \subsection{Diagramme de Feynman}


\section{Structure des révolutions}
%%%%%%%%%%%%%%%%%%%%%%%%%%%%%%%%%%%

Cette section expose quelques notions d'épistémologie.

  \subsection{Science normale}
  \subsection{Crise}
  \subsection{Révolution}


%%%%%%%%%%%%%%%%%%%%%%%%%%%%%%%%%%%%%%%%%%%



