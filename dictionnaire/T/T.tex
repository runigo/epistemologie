\chapter{T}
%%{\bf }{\it }{\bf --} « » {\footnotesize X}$^\text{e}$
\section{Temps}

La causalité quantique et l'irréversibilité thermodynamique
donnent deux {\it visions} du temps.

Pour le physicien, la durée est une grandeur physique, le
temps est l' « axe mathématique » de cette grandeur.

temps propre, durée de vie (de demi-vie)

{\footnotesize
\begin{itemize}[leftmargin=1cm, label=\ding{32}, itemsep=1pt]
\item {\bf \textsc{Étymologie} :} latin {\it },.
\item {\bf \textsc{Sens ordinaire} :} .
\item {\bf \textsc{Spiritualisme} :} .
\end{itemize}

\begin{itemize}[leftmargin=1cm, label=\ding{32}, itemsep=1pt]
\item {\bf \textsc{Terme voisin} : durée} .
\item {\bf \textsc{Terme opposé} :} .
\item {\bf \textsc{Corrélats} :} .
\end{itemize}
}
