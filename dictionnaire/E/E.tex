\chapter{E}
%%{\bf }{\it }{\bf --} « » {\footnotesize X}$^\text{e}$
\section{Électron}
Particule élémentaire
{\footnotesize
\begin{itemize}[leftmargin=1cm, label=\ding{32}, itemsep=1pt]
\item {\bf \textsc{Étymologie} :} latin {\it electricus}, de {\it electrum}, du grec {\it êlektron}, ambre jaune, d'après sa propriété d'attirer les corps légers quand on l'a frotté.
% {\bf électron} 1829, Boiste, Stoney.
\item {\bf \textsc{Corrélats} :} proton, quanton.
\end{itemize}
}
\section{Énergie}
Grandeur physique.
{\footnotesize
\begin{itemize}[leftmargin=1cm, label=\ding{32}, itemsep=1pt]
\item {\bf \textsc{Étymologie} :} latin {\it energia}, du grec {\it energeia}, force en action.
% {\bf énergétique} du grec  {\bf energetikos}
%\item {\bf \textsc{Sens ordinaire} :} .
%\item {\bf \textsc{Spiritualisme} :} .
\item {\bf \textsc{Corrélats} :} temps.
\end{itemize}
}
