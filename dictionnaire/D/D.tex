\chapter{}
%%{\bf }{\it }{\bf --} « » {\footnotesize X}$^\text{e}$
\section{Durée}

Grandeur physique mesuré par un chronomètre.


Pour le physicien, le {\it temps} est l' « axe mathématique »
de la grandeur {\it durée}.



{\footnotesize
\begin{itemize}[leftmargin=1cm, label=\ding{32}, itemsep=1pt]
\item {\bf \textsc{Étymologie} :} latin {\it },.
\item {\bf \textsc{Sens ordinaire} :} .
\item {\bf \textsc{Spiritualisme} :} .
\end{itemize}

\begin{itemize}[leftmargin=1cm, label=\ding{32}, itemsep=1pt]
\item {\bf \textsc{Terme voisin} :} .
\item {\bf \textsc{Terme opposé} :} .
\item {\bf \textsc{Corrélats} :} .
\end{itemize}
}
