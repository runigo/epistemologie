\chapter{D}
%%{\bf }{\it }{\bf --} « » {\footnotesize X}$^\text{e}$
\section{Dogme}

Ce que l'on ne peut pas remettre en question.

{\footnotesize
\begin{itemize}[leftmargin=1cm, label=\ding{32}, itemsep=1pt]
\item {\bf \textsc{Étymologie} :} latin {\it dogma}, du grec {\it dogma}, opinion.
\item {\bf \textsc{Terme voisin} :} croyance, foi.
\item {\bf \textsc{Terme opposé} :} paradigme.
\item {\bf \textsc{Corrélats} :} science, religion.
\end{itemize}
}

\section{Dualité onde/corpuscule}

%Chronologiquement, l'Homme découvre le phénomène de la vibration, puis le caractère vibratoire au niveau quantique. Ce n'est pas parceque la matière est fondamentalement vibratoire mais c'est parceque notre entendement ne nous permet pas de comprendre la quantique.
Notre entendement nous permet de distinguer le granulaire de l'ondulatoire.
%Ni le granulaire ni l'ondulatoire ne sont vrai, la réalité est autre.
Si la théorie quantique parle d'onde et de corpuscule,
c'est parceque nous n'avons pas de meilleur image pour décrire le comportement des quantons

{\footnotesize
\begin{itemize}[leftmargin=1cm, label=\ding{32}, itemsep=1pt]
\item {\bf \textsc{Étymologie} :} latin {\it dualis}, composé de deux.
\end{itemize}
}

\section{Durée}

Grandeur physique mesurée par un chronomètre.

Pour le physicien, le {\it temps} est l' « axe mathématique »
de la grandeur {\it durée}.

{\footnotesize
\begin{itemize}[leftmargin=1cm, label=\ding{32}, itemsep=1pt]
\item {\bf \textsc{Étymologie} :} latin {\it d\~{u}r\~{a}re}, durer.
\item {\bf \textsc{Corrélats} :} temps.
\end{itemize}
}
