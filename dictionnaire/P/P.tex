\chapter{P}
%%{\bf }{\it }{\bf --} « » {\footnotesize X}$^\text{e}$
%%%%%%%%%%%%%%%%%%%%%%%%%%%%%%%
\section{Paradigme}
%%%%%%%%%%%%%%%%%%%%%%%%%%%%%%%
Ensemble (constituée de lois, de modèles et d'expériences) que l'on peut remettre en question.

La science est constituée de plusieurs paradigmes, certain concurrent entre eux, certains incompatibles entre eux.

{\footnotesize
\begin{itemize}[leftmargin=1cm, label=\ding{32}, itemsep=1pt]
\item {\bf \textsc{Étymologie} :} latin {\it parradigma}, (grec {\it paradeigma}), exemple,
de {\it deiknumi}, montrer.
% 1484, Chuquet.
\item {\bf \textsc{Terme voisin} :} théorie.
\item {\bf \textsc{Terme opposé} :} dogme.
\item {\bf \textsc{Corrélats} : science, révolution scientifique} .
\end{itemize}
}

%%%%%%%%%%%%%%%%%%%%%%%%%%%%%%%
\section{Particule élémentaire}
%%%%%%%%%%%%%%%%%%%%%%%%%%%%%%%
Se dit d'un quanton indivisible, ou du moins dont on ne connait pas de structure interne.
{\footnotesize
\begin{itemize}[leftmargin=1cm, label=\ding{32}, itemsep=1pt]
\item {\bf \textsc{Étymologie} :} latin {\it particula}, de {\it pars}, {\it partis}, partie.
% 1484, Chuquet.
{\bf élémentaire} 1390, Conty, du latin {\it elementarius}.
{\bf élément} latin {\it elementum}, principe, élément.
\item {\bf \textsc{Corrélats} : atome, quanton} .
\end{itemize}
}

%%%%%%%%%%%%%%%%
\section{Photon}
%%%%%%%%%%%%%%%%
Particule élémentaire, {\it quantum} de la lumière.
Le photon est électriquement neutre et sa masse est nulle.
{\footnotesize
\begin{itemize}[leftmargin=1cm, label=\ding{32}, itemsep=1pt]
\item {\bf \textsc{Étymologie} :} grec {\it phôs}, {\it phôtos}, lumière. 1923, Louis de Broglie.
\item {\bf \textsc{Corrélats} : lumière, électromagnétisme} .
\end{itemize}
}

%%%%%%%%%%%%%%%%
\section{Proton}
%%%%%%%%%%%%%%%%
Particule pas si élémentaire (constitué de trois quarks).
Le proton possède une charge électrique positive.
Sa masse est proche de celle du neutron.
C'est un des constituants du noyau des atomes.
{\footnotesize
\begin{itemize}[leftmargin=1cm, label=\ding{32}, itemsep=1pt]
\item {\bf \textsc{Étymologie} :} grec {\it prôton}, neutre de {\it prôtos}, premier. 1920, Rutherford.
\item {\bf \textsc{Corrélats} : neutron, électron, quanton} .
\end{itemize}
}
