\chapter{A}
%%{\bf }{\it }{\bf --}{\footnotesize X}$^\text{e}$

\section{Analogie}

{\footnotesize
\begin{itemize}[leftmargin=1cm, label=\ding{32}, itemsep=1pt]
\item {\bf \textsc{Étymologie} :} latin {\it analogia},
emprunté au grec {\it analogia}, {\it analogos}, proportionnel.
\end{itemize}
}

\section{Atome}

Pour Démocrite, entité indivisible, {\it que l'on ne peut pas
couper}.
Pour les physiciens,
système électriquement neutre constitué d'un noyau (lui-même
constitué de protons et de neutrons) et d'électrons.
{\footnotesize
\begin{itemize}[leftmargin=1cm, label=\ding{32}, itemsep=1pt]
\item {\bf \textsc{Étymologie} :} latin {\it atomus}, du grec
{\it atomos}, de {\it a}, privatif et {\it temnein}, couper.
\end{itemize}
}
%%%%%%%%%%%%%%%%%%%%%%%%%%%%%%%%%%%%%%%%%%%%%%%%%%%%%%%%%
