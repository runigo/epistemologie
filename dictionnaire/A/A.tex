\chapter{A}
%%{\bf }{\it }{\bf --}{\footnotesize X}$^\text{e}$
\section{Analogie}





\subsection{Incertitude}

Sécuriser son territoire, réduire l'incertitude, prendre une assurance,
car il y aurait un principe d'incertitude macroscopique.
Par analogie, on utilise abusivement ce terme en physique quantique

\subsection{Vibration}

Chronologiquement, l'Homme découvre le phénomène de la vibration.
Puis le caractère vibratoire au niveau quantique. Ce n'est pas parceque
la matière est fondamentalement vibratoire mais c'est parceque notre
entendement ne nous permet pas de comprendre la quantique. Notre
entendement nous mermet de distinguer le granulaire de l'ondulatoire.
Ni le granulaire ni l'ondulatoire ne sont vrai, la réalité est autre.
Si la théorie quantique parle d'onde et de corpuscule, c'est parceque
nous n'avons pas les capacités d'aller plus loin






{\footnotesize
\begin{itemize}[leftmargin=1cm, label=\ding{32}, itemsep=1pt]
\item {\bf \textsc{Étymologie} :} latin {\it },.
\item {\bf \textsc{Sens ordinaire} :} .
\item {\bf \textsc{Spiritualisme} :} .
\end{itemize}

\begin{itemize}[leftmargin=1cm, label=\ding{32}, itemsep=1pt]
\item {\bf \textsc{Terme voisin} :} .
\item {\bf \textsc{Terme opposé} :} .
\item {\bf \textsc{Corrélats} :} .
\end{itemize}
}
