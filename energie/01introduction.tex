\chapter{Introduction}
\section{Définitions}
\subsection{}
On peut distinguer un sens commun et un sens scientifique au mot énergie (sens I et II).
%Chacun de ces deux sens peuvent être distingués

\begin{enumerate}[leftmargin=3cm, label=\Roman*{\bf .}]
\item {\bf Sens commun}
  \begin{enumerate}[label=\arabic*{\bf .}]
  \item Sens {\bf courant} : force, vigueur, vivacité (éthymologie : force en action)
  \item Sens {\bf spiritualiste} : principe, substance.
  \end{enumerate}
\item {\bf Sens scientifique}
  \begin{enumerate}[label=\arabic*{\bf .}]
  \item Sens {\bf économique} : source d'énergie exploitable économiquement (pétrole, charbon, gaz, éolien, hydrolique, solaire)
  \item Sens {\bf physique} : grandeur conservée lors d'une transformation d'un système isolé.
  \end{enumerate}
\end{enumerate}

\subsection{Sens économique}
Le pétrole, le charbon, le gaz sont des sources d'énergie dont le transport est un enjeux économique. Ainsi, il existe des pétroliers, des trains, des gazoducs qui les acheminent. On peut lire dans certains articles : {\it Le transport des énergies est un enjeux économique} (ne devrait-on pas dire "le transport du pétrole, du charbon et du gaz est un enjeux économique" ou "le transport des énergies fossiles est un enjeux économique" ?).

Également, on peut lire dans d'autres articles : {\it Le transport de l'énergie électrique est un enjeux économique}. On devrait plutôt dire "le transport de l'électricité est un enjeux économique".

\subsection{Sens physique}
La physique distingue le système et les grandeurs. En particulier, une onde électromagnétique est un système physique auquel on associe des grandeurs (longueur d'onde, amplitude, énergie). On peut alors lire dans certains article "une onde électromagnétique transporte de l'amplitude", ce qui est impropre, il faut plutôt dire "une onde électromagnétique possède une amplitude".

\subsection{Sens éthymologique}
Il faudrait préciser ce que l'on entend par "en action" : est-ce par opposition à "au repos" (il existerait des {\it forces en action} et des {\it forces au repos}), ou bien de façon plus philosophique par opposition à la puissance (il existerait des forces {\it en acte} et des forces {\it en puissance}). (en philosophie, {\it puissance} signifie {\it possibilité, faculté}, en physique, la puissance est une {\it énergie par unité de temps}, c'est différent).

\subsection{Sens spiritualiste}
Dans certains discours, des énergies s'opposent :
\begin{itemize}[leftmargin=1cm, label=\ding{32}, itemsep=1pt]
\item Des énergies positives s'opposent à des énergies négatives.
\item Des énergies bénéfiques s'opposent à des énergies maléfiques.
\item Des énergies vibratoires s'opposeraient à des énergies non-vibratoires.
\item Des énergies immatérielles s'opposeraient à des énergies matérielles.
\end{itemize}
On peut reconnaître dans certains de ces discours des {\it jugements de valeurs}, un discours sur le {\it bien et le mal}.

Dans ces discours, le mot énergie peut alors être remplacé par d'autre mots (vibration, onde, longueur d'onde) sans changer le sens du discours :
\begin{itemize}[leftmargin=1cm, label=\ding{32}, itemsep=1pt]
\item Énergies positives $=$ ondes positives, bonnes vibrations...
\item Énergies négatives $=$ vibrations négatives, mauvaises ondes...
\item Énergies vibratoires $=$ vibrations énergétiques...
\item Énergies bénéfiques $=$ "bonne longueur d'onde" $=$ "bonne synchronisation".
\item Énergies immatérielles $=$ indéfinissable.
\item Énergies matérielles $=$ ici-bas.
\end{itemize}

\subsection{Epistémologie spiritualiste}
\subsubsection{Paradigme vitaliste}

Il existe des ouvrages dont le titre est "l'énergie vitale". Le vitalisme est un courant philosophique. D'autres ouvrages portant sur "le développement de soi" peuvent faire référence à un certain paradigme vitaliste. Dans ce paradigme est utilisé la notion d'énergie vitale.

\subsubsection{Paradigme vibratoire}

Existe-t-il des ouvrages dont le titre est "l'énergie vibratoire" ? Existerait-il un paradigme vibratoire ? Une révolution spiritualiste aurait-elle eu lieu ?

