\chapter{Thermodynamique}

\section{Variables d'état}
L'état d'un système thermodynamique est décrit par des variables d'états. Ces variables peuvent être fonction les unes des autres. En générale, trois variables d'états indépendante sont suffisante pour déterminer complètement l'état du système, les autres variables étant fonction des trois premières.

{\bf Exemples : }
Énergie U, entropie S, température T,  pression P, volume V, enthalpie H, énergie libre F, enthalpie libre G...

Lorsqu'un système thermodynamique est à l'équilibre, ses variables d'états sont constantes au cours du temps.

\section{Principes}

Les trois principes de la thermodynamique permettent de déterminer l'état d'équilibre d'un système thermodynamique.

La variation élémentaire d'énergie interne est égale à la somme des quantités élémentaires de chaleur et de travail apporté au système par l'extérieur :
\[
\tag{1}dU = \delta Q + \delta W
\]

La variation élémentaire d'entropie est égale à la quantité élémentaire de chaleur apporté par l'extérieur divisé par la température :
\[
\tag{2}dS = \frac{\delta Q}{T}
\]

L'entropie s'annule si la température est nulle :
\[
\tag{3}S = 0 \mt{ si } T=0
\]

\section{Théorème}
\begin{minipage}[c]{.45\linewidth}
Des principes, on tire le théorème TdS :
\end{minipage}
\hfill
\begin{minipage}[c]{.45\linewidth}
\[
dU = TdS - pdV
\]
\end{minipage}

\section{Définitions}
On définie les fonctions suivantes :

L'énergie libre : 
\[
F=U-TS
\]

L'enthalpie :
\[
H=U-PV
\]

L'enthalpie libre :
\[
G = U-TS-PV
\]

