
\begin{appendix}

\chapter{Système et grandeurs}

\section{Exemples}

\subsection{Système solaire}
Le système solaire est constitué du soleil et des planètes. On associe à ce système des grandeurs : 
\begin{itemize}[leftmargin=1cm, label=\ding{32}, itemsep=1pt]
\item La position des astres
\item Leur vitesse
\item Leur masse
\end{itemize}

\subsection{Pendule}
Un pendule pesant est un système physique constitué par un corps suspendu à un fil rigide, placé dans un champ de gravitation. On associe à ce système des grandeurs :
\begin{itemize}[leftmargin=1cm, label=\ding{32}, itemsep=1pt]
\item La masse du corps
\item La longueur du fil
\item L'accélération du champ de pesanteur
\item La position du pendule et sa vitesse
\end{itemize}

\subsection{Système mécanique}
Les deux exemples précédent sont des systèmes mécaniques. En générale, l'état d'un système est déterminé par la donnée des positions et des vitesses de ses composants.

\section{Loi physique}
L'objet de la physique est de déterminer l'évolution des systèmes au cours du temps : connaissant l'état d'un système à un instant, quel est son état l'instant suivant.

Les lois physiques sont des relations permettant de déterminer cette évolution.


\chapter{Dérivée et intégrale}

Exemple graphique : on considère la fonction suivante :
\begin{center}
\begin{tikzpicture}
\draw[->] (0,-2) -- (0,2) node[right] {$f(x)$};
\draw[color=red,domain=-3:3,samples=100] plot ({\x},{1/(1+\x*\x)});
\draw[->] (-3,0) -- (3,0) node[right] {$x$};
\end{tikzpicture} 
\end{center}

Elle possède une dérivée et une intégrale :
\vspace{0.5cm}

\begin{minipage}[c]{.45\linewidth}
\begin{center}
Dérivée de f(x) :

\begin{tikzpicture}
\draw[->] (0,-2) -- (0,2) node[right] {$f'(x)$};
\draw[color=red,domain=-3:3,samples=100] plot ({\x},{-2*\x/(1+\x*\x)/(1+\x*\x))});
\draw[->] (-3,0) -- (3,0) node[right] {$x$};
\end{tikzpicture} 
\end{center}
\end{minipage}
\hfill
\begin{minipage}[c]{.45\linewidth}
\begin{center}
Intégrale de f(x)

\begin{tikzpicture}
\draw[->] (0,-2) -- (0,2) node[right] {$\int\!\!f(x)$};
\draw[color=red,domain=-3:3,samples=100] plot ({\x},{0.9+0.6*rad(atan(2*\x))});
\draw[->] (-3,0) -- (3,0) node[right] {$x$};
\end{tikzpicture} 
\end{center}
\end{minipage}

\end{appendix}

