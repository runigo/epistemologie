\chapter{Vocabulaire}
%Lorsque des scientifiques découvrent quelque chose de nouveau, ils le nomment. Le nom choisi peut être un nom propre, hommage ou clin d'œuil au découvreur ou à l'un des précursseur à 

\section{Larousse éthymologique}
{\bf Énergie }{\footnotesize XV}$^\text{e}$ s. {\it Jardin de santé}, du bas latin {\it energia} (saint Jérôme), emprunté au grec {\it energeia}, force en action. || {\bf énergique} fin {\footnotesize XVI}$^\text{e}$ s. || {\bf énergétique} 1768, {\it Encycl.}, « qui paraît avoir une énergie innée »; sens actuel, fin {\footnotesize XIX}$^\text{e}$ s. (1909, L. M.); du grec {\it energetikos}.

{\bf Force} 1080, {\it Roland}, du bas latin {\it fortia}, pl. neutre subst. de {\it fortis}, courageux puis fort. || {\bf forcer} {\footnotesize XIII}$^\text{e}$ s. {\it Chr d'Antioche}, du lat. pop. {\it fortiare}, de {\it fortia}. [...]% || {\it forçage} {\footnotesize XII}$^\text{e}$ s.


\section{Vocabulaire de la philosophie}
{\bf Force} — \si{Vulg.} {\bf 1.} (Souvent opp. {\it droit}$^1$). Contrainte :
« Céder à la force ». — {\bf 2.} Puissance d'action : « Les forces morales ».
{\it Idée-force} (Fouillée) : la représentation$^1$ considérée comme poussant
à l’action$^2$. — {\bf 3.} Agent$^1$ physique : « Les forces naturelles ».

— \si{Math.} {\bf 4.} En mécanique : tout ce qui est capable de modifier l’état de
repos ou de mouvement d’un corps (cf. {\it Inertie}$^{\,2}$). — {\bf 5.}
{\it Force vive} (syn. : énergie actuelle) : demi-produit de la masse par le
carré de la vitesse (1/2 mv$^2$).

— \si{Méta.} {\bf 6.} L’énergie*, considérée comme le principe indéfinissable
qui produit les phénomènes de l’univers : « Force et matière » (Büchner) ;
« La persistance de la force » (Spencer). — {\bf 7.} {\it Chez Leibniz} :
voir {\it Substance}$^1$.

{\bf Énergétique} — \si{Phys.} {\bf 1.} Science des
propriétés générales de l’énergie$^1$,
abstraction faite des caractères$^2$
particuliers propres à chacune des
formes sous lesquelles elle apparaît.
— {\bf 2.} Théorie physique fondée sur
le principe de la conservation* de
l'énergie et sur le principe de moindre
action* : « Le système$^2$ énergétique
a pris naissance à la suite de la
découverte du principe
de la conservation de l'énergie. C’est
Helmholtz qui lui a donné sa forme
définitive » (H. Poincaré).

{\bf Énergétisme} — \si{Méta.} \fsb{S. norma.} La théorie
énergétique$^2$ érigée en système métaphysique qui fait de l'{\it énergie}
la substance même du monde : « L’énergétisme d'Ostwald ».
% 66

{\bf Énergie} — \si{Phys.} L'{\it énergie} d'un système de corps se mesure
par le travail$^1$ mécanique qu’il est capable
de produire. {\it Énergie actuelle} ou
{\it cinétique} : celle qui se manifeste par
le mouvement (égale à la somme
des forces$^5$ vives du système).
{\it Énergie potentielle} : celle qui n'existe
qu’en puissance, {\it i. e.} travail que les
forces intérieures du système effectueraient si les corps qui le composent obéissaient à l'action de ces
forces.

