\documentclass[12pt, a4paper]{report}
%\documentclass[11pt, a4paper]{article}

%====================== PACKAGES ======================
\usepackage[french]{babel}

\frenchbsetup{StandardLists=true}
\usepackage{enumitem}
\usepackage{pifont}

\usepackage[utf8x]{inputenc}
%\usepackage[latin1]{inputenc}

%pour gérer les positionnement d'images
\usepackage{float}
\usepackage{amsmath}
\usepackage{amssymb}
\usepackage{amsfonts}
\DeclareMathOperator{\dt}{dt}
\usepackage{graphicx}
%\usepackage{tabularx}
\usepackage[colorinlistoftodos]{todonotes}
\usepackage{url}

%pour les informations sur un document compilé en PDF et les liens externes / internes
\usepackage[pdfborder=0]{hyperref}
\hypersetup{
	colorlinks = true
	}

%pour la mise en page des tableaux
\usepackage{array}
\usepackage{tabularx}
\usepackage{multirow}
\usepackage{multicol}
\setlength{\columnsep}{50pt}

%pour utiliser \floatbarrier
%\usepackage{placeins}
%\usepackage{floatrow}

%espacement entre les lignes
\usepackage{setspace}

%modifier la mise en page de l'abstract
\usepackage{abstract}

%police et mise en page (marges) du document
\usepackage[T1]{fontenc}
\usepackage[top=2cm, bottom=2cm, left=2cm, right=2cm]{geometry}

%Pour les galerie d'images
\usepackage{subfig}

\usepackage{pdfpages}

\usepackage{tikz}
\usetikzlibrary{trees}
\usetikzlibrary{decorations.pathmorphing}
\usetikzlibrary{decorations.markings}
\usetikzlibrary{decorations.pathreplacing,calligraphy}
%\usetikzlibrary{decorations}
\usetikzlibrary{angles, quotes}
\usepackage{verbatim}

\usepackage{appendix}

\usepackage{comment}

\usepackage{xcolor}

%\PreviewEnvironment{tikzpicture}
%\setlength\PreviewBorder{0pt}%

%====================== INFORMATION ET REGLES ======================

%rajouter les numérotation pour les \paragraphe et \subparagraphe
\setcounter{secnumdepth}{4}
\setcounter{tocdepth}{4}

\hypersetup{							% Information sur le document
pdfauthor = {Stephan Runigo},			% Auteurs
pdftitle = {Documentation},			% Titre du document
pdfsubject = {Documentation},		% Sujet
pdfkeywords = {Document},	% Mots-clefs
pdfstartview={FitH}}	% ajuste la page à la largeur de l'écran
%pdfcreator = {MikTeX},% Logiciel qui a crée le document
%pdfproducer = {} % Société avec produit le logiciel

%======================== DEFINITION COMMANDES ========================
\newcommand{\fsb}[1]{\textsf{\textbf {\footnotesize #1}}}
\newcommand{\bi}[1]{\textbf{\textit {#1}}}
\newcommand{\si}[1]{\textsf{\textit {#1}}}
\newcommand{\lb}{$\lozenge$\ }
\newcommand{\bl}{$\blacklozenge$\ }
%======================== DEBUT DU DOCUMENT ========================
%
\begin{document}
%
% Table des matières
\tableofcontents
\thispagestyle{empty}
\setcounter{page}{0}
%
%====================== INCLUSION DES CHAPITRES ======================
%
~
\thispagestyle{empty}
%recommencer la numérotation des pages à "1"
\setcounter{page}{0}
\newpage
\chapter{Dictionnaires}
\section{Larousse éthymologique}
{\bf énergie }{\footnotesize XV}$^\text{e}$ s. {\it Jardin de santé}, du bas latin {\it energia} (saint Jérôme), emprunté au grec {\it energeia}, force en action. || {\bf énergique} fin {\footnotesize XVI}$^\text{e}$ s. || {\bf énergétique} 1768, {\it Encycl.}, « qui paraît avoir une énergie innée »; sens actuel, fin {\footnotesize XIX}$^\text{e}$ s. (1909, L. M.); du grec {\it energetikos}.

{\bf Force} 1080, {\it Roland}, du bas latin {\it fortia}, pl. neutre subst. de {\it fortis}, courageux puis fort. || {\bf forcer} {\footnotesize XIII}$^\text{e}$ s. {\it Chr d'Antioche}, du lat. pop. {\it fortiare}, de {\it fortia}. [...]% || {\it forçage} {\footnotesize XII}$^\text{e}$ s.

\newpage
\section{Petit Larousse}

{\bf ÉNERGIE} {\sf n.f.} (gr {\it energeia}, force en action). {\bf I.1.} Force morale, fermeté, puissance, vigueur. {\it L'énergie du désespoir}. {\bf 2.} Vigueur dans la manière de s'exprimer. {\it Parler avec énergie}. {\bf 3.} Force physique, vitalité. {\it Un être plein d'énergie}. {\bf 4.} \fsb{PSYCHAN.} {\it Énergie psychique} ; libido. {\bf II.1.} \fsb{PHYS.} {\bf a.} Grandeur caractérisant un système et exprimant sa capacité à modifier l'état d'autres systèmes avec lesquels il entre en interaction (unité SI : le {\it joule}). {\bf b.} Chacun des modes que peut présenter un tel système. {\it Énergie mécanique, magnétique, nucléaire}. {\bf 2.} {\it Sources d'énergie} : ensemble des matières premières ou des phénomènes naturels utilisés pour la production d'énergie (charbon, hydrocarbures, uranium, cours d'eau, marées, vent, etc.).

$\blacksquare$ Outre l'énergie mécanique ({\it énergie potentielle} d'un poids soulevé, d'un ressort comprimé, etc., et {\it énergie cinétique} d'une masse en mouvement), on distingue les énergies {\it chimique, électrique, nucléaire, calorifique, rayonnante}. L'énergie est un concept de base de la physique car un système isolé a une énergie totale constante. Il ne peut donc y avoir création ou disparition d'énergie, mais seulement transformation d'une forme d'énergie en une autre ou transfert d'énergie d'un système à un autre. Toute conversion d'énergie s'accompagne de pertes. Celles-ci sont particulièrement importantes dans la conversion d'énergie thermique en énergie mécanique.

\vspace{0.24cm}
{\footnotesize 
{\bf ÉNERGÉTICIEN, ENNE} {\sf n.} Spécialiste d'énergétique.

{\bf 1. ÉNERGÉTIQUE} {\sf adj.} (angl. {\it energetic}). Relatif à l'énergie, aux sources d'énergie. \lb {\it Aliment énergétique :} aliment nécessaire à l'organisme pour réparer ses dépenses d'énergie et ses pertes de matière. {\bf —} {\it Apport énergétique :} apport d'énergie fourni à un organisme par un aliment, une boisson.

{\bf 2. ÉNERGÉTIQUE} {\sf n.f.} Science et technique de la production de l'énergie, de ses emplois et des conversions de ses différentes formes.

{\bf ÉNERGIQUE} {\sf adj.} {\bf 1.} Qui agit fortement ; efficace. {\it Un remède énergique}. {\bf 2.} Qui est plein d'énergie, qui manifeste de l'énergie. {\it Visage énergique}.

{\bf ÉNERGIQUEMENT} {\sf adv.} avec énergie.

{\bf ÉNERGISANT, E} {\sf adj. n.m.} Se dit d'un produit qui donne de l'énergie ; stimulant.

{\bf ÉNERGIVORE} {\sf adj. Fam.} Qui consomme beaucoup d'énergie.}
\vspace{0.31cm}

{\bf FORCE} {\sf n.f.} (bas lat. {\it fortia}, pl. neutre de {\it fortis}, courageux). {\bf I.1.} Énergie, vigueur physique. {\it Elle a beaucoup de force. \lb À force de :} à la longue, par des efforts répétés. {\it Force de la nature :} persone qui a beaucoup d'endurance, de résistance ou qui est pleine de vitalité. Tour de force : exercice corporel exigeant une grande force physique ; {\sf fig.}, résultat qui suppose une habileté, un effort exceptionnels. {\bf 2.} {\it Force de travail :} dans la terminologie marxiste, ensemble des facultés physiques et intellectuelles de l'homme, à l'aide desquelles il produit des choses utiles. {\bf 3.} Courage, capacité de résister aux épreuves. {\it Force d'âme, de caractère}. {\bf 4.} Degré d'aptitude dans le domaine intellectuel, niveau ; habileté. {\it Ils sont de la même force en maths}. {\bf 5.} Degré d'intensité (d'un sentiment). {\it La force du désir}. {\bf II.1.a} Emploi de moyen violents pour contraindre une ou plusieurs personnes. {\it Céder à la force. Employer la force. Coup de force.} \lb {\it De force, de vive force :} en employant la violence, la contrainte. {\bf —} {\it Par force :} sous l'effet de la contrainte. {\bf b.} {\it Épreuve de force :} situation résultant de l'echec des négociations entre deux groupes antagonistes et où la solution ne dépend plus que de la supériorité éventuelle de l'un sur l'autre. {\bf 2.a.} Ensemble de personnes armées et organisées, chargées d'une tâche de protection de défense ou d'attaque. \lb {\it Force publique :} ensemble des formations de la police, de la gendarmerie et des armées qui sont à la disposition du gouvernement pour assurer le respect de la loi et le maintien de l'ordre. {\bf b.} {\it Force de frappe, de dissuasion} ou, en France {\it force nucléaire stratégique :} force militaire aux ordres directs de la plus haute instance politique d'un État, rassemblant la totalité de ses armements nucléaires stratégiques. {\bf 3.} Pouvoir de ce qui incite ou oblige à se comporter d'une certaine manière. {\it La force de l'habitude} \lb {\it Par force :} par nécessité. {\bf —} {\it (Cas de) force majeure :} évènement imprévisible, contraignant ; nécessité qui impose à qqn sa conduite. {\bf 4.} Autorité, ascendant, pouvoir effectif. {\it La force des lois. Avoir force de loi.} {\bf —} \fsb{DR.} {\it Force exécutoire :} qualité d'un acte ou d'un jugement qui permet si besoin est, le recours à la force publique pour son exécution. {\bf III.} \fsb{PHYS}. Toute cause capable de déformer un corps, d'en modifier l'état de repos ou de mouvement. {\it Force d'inertie.} {\bf —} {\it Force électromotrice}, caractéristique d'une source d'énergie électrique qui crée un courant dans un circuit et détermine l'intensité de ce courant.{\bf —} {\it Force contre-électromotrice} $\to$ \bi{contre-électromotrice.} {\bf —} {\it Force d'un électrolyte :} mesure de son énergie de dissociation. {\bf IV.1.} Degré de puissance, d'intensité d'un agent physique. {\it La force d'un courant. Vent force 7.} {\bf 2.} Degré d'efficacité, de rendement de qqch. {\it La force d'un médicament. La force d'une machine.} {\bf V.1.} Importance numérique, quantité. \lb {\it Être en force :} être nombreux. {\bf 2.} {\it Faire force de rames, de voiles :} faire en sorte que les rames, les voiles déploient leur maximum de force. 
\bl {\sf pl.} {\bf 1.} {\it Forces (armées) :} ensemble des formations militaire d'un état. {\it Forces aériennes, navales, terrestre.} {\bf 2.} Ensemble des personnes unies par une même volonté, et œuvrant à sa réalisation. {\it Les forces de progrès.} 
$\blacksquare$ 
La notion de force tend à être remplacée par celle d'interaction. Un nombre restreint d'interactions fondamentales permet en effet de rendre compte de la complexité des phénomènes physiques. Outre l'interaction {\it gravitationnelle}, s'exerçant sur tous les systèmes possédant une masse, on distingue l'interaction {\it électrofaible}, rendant compte des phénomènes électromagnétiques et des phénomènes radioactifs, et l'interaction {\it forte}, responsable de la cohésion du noyau atomique.

\vspace{0.24cm}
{\footnotesize 
FORCÉ, E {\sf adj.} {\bf 1.a.} Qui manque de naturel. {\it Un rire forcé.} {\bf b.} Qui est imposé, que l'on fait contre sa volonté. {\it Atterrissage forcé.}{\bf —} {\it Avoir la main forcée :} agir malgré soi sous la pression d'autrui. {\bf 2.} {\it Marche forcée :} marche dont la durée et la rapidité dépassent celles des marches ordinaires. {\bf 3.} {\sf Fam.} Inévitable. {\it Elle gagnera, c'est forcé.} {\bf 4.} {\it Culture forcée :} culture de plantes soumises au forçage.
}
\vspace{0.31cm}

\newpage
\section{Petit Robert}

{\bf ÉNERGIE} {\sf n.f.} ( v. 1500 ; bas lat. {\it energia}, gr. {\it energeia} « force en action »).

{\bf I.} {\it cour.} \bl {\bf 1°} {\it Vieilli}. Pouvoir, efficacité (d'un agent quelconque). \lb Force , vigueur (dans l'expression, dans l'art). « {\it Quelle fraîcheur de coloris, quelle énergie d'expression} » ({\sc Rousseau}). « Une énergie singulière, un pittoresque effrayant » ({\sc Hugo}).
%
\bl {\bf 2°}(Fin {\footnotesize XVIII}$^\text{e}$). Force et fermeté dans l'action qui rend capable de grands effets. {\bf V. Dynamisme, ressort, volonté.} « {\it Cette énergie sublime qui fait faire les choses extraordinaires} » ({\sc Sthendal}). « {\it La quantité d'énergie ou de volonté que chacun de nous possède} » ({\sc Balzac}). « {\it Galvaniser nos énergie} » ({\sc Gide}). {\it Une énergie indomptable, farouche} « {\it L'internationale... avait perdu sa vitalité, tout en confisquant l'énergie du prolétariat} » ({\sc Romains}). {\it Regain d'énergie} {\bf V. Souffle} (second souffle). \lb Force, vitalité physique. {\it Se sentir plein d'énergie, frotter avec énergie.} « {\it Je le battis avec l'énergie obstinée des cuisiniers qui veulent attendrir un beafsteak} » ({\sc Baudelaire}).

{\bf II.} {\it Sc.} \bl {\bf 1°} {\it Phys.} (1875 ; angl. {\it energy}, 1852 ; sens plus vague 1807). Ce que possède un système s'il est capable de produire du travail. {\it Les différentes forme de l'énergie et leurs transformation. Énergie mécanique potentielle d'un corps} (Travail pouvant être produit en raison de la position d'un corps) ; {\it énergie cinétique} (acquise du fait de sa vitesse). {\it Énergie thermique} {\bf V. chaleur, thermodynamique.} {\it Énergie électrique, solaire} {\bf (V. rayonnement,)} chimique, nucléaire  {\bf (V. Radiation ; fission, fusion)}. {\it Principe de la conservation de l'énergie. Énergie interne,} en thermodynamique, somme des énergie potentielle et cinétique inhérentes à un système. {\it Les variations de l'énergie interne d'un système ne dépendent que de ses états initial et final.} \bl {\bf 2°} Énergie chimique potentielle de l'être vivant. {\it Énergie physiologique minimale} (ou métabolisme de base), dépense énergétique de l'organisme au repos complet.

$\boxdot$ {\sc Antonyme} \si{Indolence, inertie, mollesse, paresse.}

\vspace{0.24cm}
{\footnotesize 
{\bf ÉNERGÉTICIEN} {\it n. m.} (v. 1970 ; de énergétique). {\it Sc., techn.} Spécialiste de l'énergétique.

{\bf ÉNERGÉTIQUE} {\it adj.} et {\it n. f.} ( 1898 « qui parait avoir une énergie innée »; angl. energetic, gr {\it energêtikos}). \bl {\bf 1°} {\it Adj.} Relatif à l'énergie, aux grandeurs, aux unité, liées à l'énergie sous toute ses formes. {\it Puissance énergétique. Théorie énergétique}, système remplaçant en mécanique la notion de force par celle d'énergie. \lb Relatif à l'énergie utilisé industriellement. {\it Les ressources énergétique d'un pays.} \lb Physiol. {\it Dépense énergétique}, énergie qu'utilise l'organisme pour une action ou une fonction déterminée. {\it Aliments énergétiques}, qui fournissent beaucoup d'énergie à l'organisme. \bl {\bf 2°} {\it N. f.} Théorie énergétique. Science traitant des diverses manifestations de l'énergie.

{\bf ÉNERGIQUE} {\it adj.} (fin {\footnotesize XVI}$^\text{e}$; de {\it énergie}). \bl {\bf 1°} Actif, efficace. \bl {\it Un remède énergique.} {\bf —} Plein d'énergie (dans l'expression). {\bf V. Vigoureux.} \bl {\bf 2°}(fin {\footnotesize XVIII}$^\text{e}$). Qui a de l'énergie, de la volonté. {\bf V. Ferme, fort, mâle, résolu.} « {\it Un homme énergique n'a jamais peur en face du danger pressant} » ({\sc Maupassant}). {\bf —} qui exprime, marque de l'énergie. {\it Un visage énergique.} « {\it Une intervention énergique de la police} » ({\sc Martin du Gard}). \lb Fort, puissant (dans l'ordre physique). « {\it Un coup de pied... assez énergique pour briser les homoplates} » ({\sc Baudelaire}) $\boxdot$ {\sc ANTONYME} \si{Indolence, inertie, mollesse, paresse.}

{\bf ÉNERGIQUEMENT} {\it adv.} (1584; de énergique) Avec énergie. {\bf V. Fermement, résolument.} « {\it Fais énergiquement ta longue et lourde tâche} » ({\sc Vigny}). {\it Résister, protester énergiquement.} \lb Avec force « {\it Je serrai énergiquement cette main} » ({\sc Jaloux}).

{\bf ÉNERGISANT, E} {\it adj.} et {\it n.} (v. 1970 ; calque de l'angl. {\it energizing}). {\it Méd.} \bl {\bf 1°} {\it adj.} Qui stimule, donne de l'énergie. {\it L'action énergisante d'un médicament.} \bl {\bf 2°} {\it N. m.} Médicament qui stimule l'activité psychique. {\it Prendre des énergisant}. {\bf V. Antidépresseur, psychotonique, psychotrope.}
}
\vspace{0.31cm}

{\bf FORCE} {\sf n.f.} (1080 ;bas lat. {\it fortia}, plur. neutre substantivé de {\it fortis}. {\bf V. Fort ; forcer}).

{\bf I.} {\it La force de quelqu'un}. \bl {\bf 1°} Puissance d'action physique (d'un être, d'un organe). {\it Force physique ; force musculaire.} {\bf V. Résistance, robustesse, vigueur.} {\it La force du lion. Force de colosse, d'athlète. Avoir de la force.} {\bf V. fort.} {\it Ne plus avoir la force de marcher, de parler. Ne pas sentir sa force :} frapper, pousser, etc., trop fortsans s'en rendre compte. {\it Lutter à force égales, à égalité de forces.} « {\it Elle serrait la rampe avec tant de force que le bois grinçait} » ({\sc Green}). « {\it Patience et longueur de temps Font plus que force ni que rage} » ({\sc La Fontaine}) {\it Être à bout de force, sans force}. \lb {\it (Au plur.)} Ensemble, concours d'énergie. {\it Ménager ses forces. Ce travail est au-dessus de ses forces. Ses forces l'ont trahi. Reprendre des forces. Aliment qui redonne des forces :} fortifie, réconforte. {\it De toute ses forces :} en rassemblant et en utilisant toute ses forces ; {\it par ext.} le plus fort possible. {\it Il tapait, il criait de toutes ses forces.} \lb ({\it opposé} à adresse, souplesse) {\sc En force}, opposé à « en souplesse ». Courir, nager en force. {\bf —} {\sc De force} : qui exige de la force. {\it Tour de force. Épreuve de force. Travailleur de force :} personne dont le métier exige une grande dépense de force physique. {\it Travail, exercice de force.} \lb Mar. {\it Faire force} exercer ou imposer l'effort maximum. {\bf V. forcer.} {\it Faire force de rames :} ramer de toute ses forces. \lb Loc. {\it Dans la force de l'âge :} au moment ou un homme est le plus fort (maturité). \bl {\bf 2°} Capacité de l'esprit ; possibilités intellectuelles et morales. {\it La science des géomètres qui } « {\it exerce la force de l'esprit} » ({\sc Suarès}). « {\it Et consultez longtemps votre esprit et vos forces} » ({\sc Boileau}). Dans l'ordre moral. {\bf V. Constance, courage, cran, détermination, énergie, fermeté, volonté.} {\it Force morale ; force de caractère. La force d'âme des héros cornéliens.} « {\it Elle avait la force devant qui les autres plient : le calme} » ({\sc R. Rolland}). « {\it Elle me résistait avec une force de volonté qui voulait maîtriser la mienne} » ({\sc Loti}). {\it Ce sacrifice est au dessus de mes forces.} \bl {\bf 3°}  {\sc De} (telle ou telle) {\sc force.} {\it Ils sont de la même force} (physique morale). {\bf —} {\it Spécialement} (sur le plan intellectuel ou de l'habileté) {\it Ce joueur n'est pas de force. Ils sont de la même force au tennis, aux échecs, en mathématiques.} {\bf V. niveau.} \bl {\bf 4°} {\it Faire la force de qqn. :} constituer sa supériorité. « {\it Ce qui fait ma force c'est que je fais tout moi-même} » ({\sc Romains}).

{\bf II.} {\it La force d'un groupe, de qqch.} \bl {\bf 1°} Pouvoir, puissance. {\it La force de l'Église, d'un parti. Force militaire d'un pays.} Par ext. {\it La force publique :} les agents armés du gouvernement {\bf V. Police.} {\it La force armée :} les troupes. {\bf —} (1959) {\it Force de frappe :} ensemble des moyens militaires modernes (fusées, armes atomiques) destinés à écraser rapidement l'ennemi. Fig. (1961) Autorité, force, puissance. {\bf —} {\it Forces de dissuasion*. } {\bf —} {\sc prov.} {\it L'union fait la force.} \lb {\sc En force.} {\it Être en force ; arriver, attaquer en force :} en nombre, avec des effectifs considérables. \bl {\bf 2°} {\it Plur.} ({\footnotesize XVII}$^\text{e}$). Ensemble des armées. {\bf V. Armée, troupe}. {\it Les forces armées françaises. Forces navales, aériennes (1939). Forces terriennes} (F.T.A.). {\it Regrouper, concentrer ses forces. Les forces de police, les forces de l'ordre} (dans le langage gouvernemental, police et gendarmerie intervenant en cas d'émeutes). {\bf —} Par ext. {\it Forces politiques, syndicales.} « {\it Rallier les forces d'opposition} » ({\sc Martin du Gard}). \bl {\bf 3°} Résistance d'un objet. {\bf V. Résistance, robustesse, solidité.} {\it Force d'un mur, d'une barre.} Spécialt. {\it jambe de force}, ou {\it force :} pièce de charpente qui sert à soulager la portée des longues poutres. \bl {\bf 4°} Intensité ou pouvoir d'action d'une chose ; caractère de ce qui est fort (III). {\it La force du vent. Force d'un coup, d'un choc. Diminuer la force d'un son. La force d'un acide. Force d'un médicament.} {\bf V. Activité, efficacité.} \lb (Choses abstraites) {\it La force d'un sentiment, d'un désir :} son intensité.  {\bf V. violence.} « {\it un muscle perd sa vigueur, un désir sa force} » ({\sc Colette}). {\bf —} {\it Force d'un argument, d'une idée. Ici,} « {\it le mensonge a autant de force que la vérité} » ({\sc Green}). {\bf —} Loc. {\it Dans toute la force du mot, du terme :} dans l'acception la plus signifiante. {\bf —} {\it Force du style.} {\bf V. Couleur, vie vigueur.} {\it S'exprimer avec force.} {\bf V. Éloquence, feu, véhémence.} \bl {\bf 5°} Typogr. {\it Force de corps* d'un caractère :} mesuré en points. {\it Un corps de force 6} (du 6).

{\bf III.} ({\footnotesize XII}$^\text{e}$). Pouvoir de contrainte. \bl {\bf 1°} En parlant d'une personne, d'un groupe. {\bf V. Contrainte, oppression, violence.} {\it Employer alternativement la force et la douceur. Céder, obéir à la force.} {\bf —} {\it La force et la justice, et le droit. La force prime le droit}, mot attribué à Bismarck. {\it Le gouvernement menace de recourir à la force} (en employant des {\it forces} de police, la {\it force} publique ; Cf. {\it ci-dessus}, II). \lb {\sc De force}. {\it Coup de force.} {\bf —} Pouvoir de contraindre donné par la supériorité militaire. {\it Situation de force. Épreuve de force}, tout espoir de conciliation étant écarté. {\bf —} {\it Maison centrale de force :} prison d'État où sont les condamnés aux travaux forçés et à la réclusion. {\bf V. Forçat.} {\bf —} {\it Camisole de force.} \bl {\bf 2°} {\it La force de} (qqch.) : son caractère irrésistible. {\it La force de l'évidence :} devant laquelle on s'incline. {\it Faire qqch. par la force de l'habitude :} machinalement. {\bf —} {\it La force des choses} : la nécessité qui résulte d'une situation. {\bf V. Nécessité, obligation.} « {\it C'est précisément parceque la force des choses tend toujours à détruire l'égalité, que la force de la législation doit toujours tendre à la maintenir} » ({\sc Rousseau}). {\it Par la force des choses :} obligatoirement, inévitablement. {\bf —} {\it Force d'une loi}, son caractère obligatoire. {\bf V. Autorité.} {\it V. Autorité.} {\it Avoir force de loi :} être assimilable à une loi, en avoir le caractère obligatoire. \lb {\it Force majeure} (dr.) : évènement imprévisible, inévitable et irrésistible qui libère d'une obligation. Cour. C'est un cas de force majeur. \lb {\it Force est de... :} il faut, on ne peut éviter de... « {\it Force lui fût de reconnaître qu'... il avait opté pour le plus facile} » ({\sc Martin du Gard}). \bl {\bf 3°} {\it Loc. adv} {\sc De force} : en faisant effort pour surmonter une résistance. {\it Faire entrer de force une chose dans une autre. Prendre, enlever de force qqch. à qqn.} {\bf V. Arracher, extorquer} {\it Il obéira de gré ou de force :} qu'il le veuille ou non. {\bf —} {\it Loc. adv.} {\sc Par force} : en recourant à la force ; en cédant à la force. {\it Prendre, obtenir qqch. par force. Il n'a pas accepté de son plein gré, mais par force :} parce que les évènements l'y contraignaient. {\bf —} {\sc À toute force} : en dépit de tout les obstacles, de toute les résistances. {\bf V. absolument.} {\it Il voulait à toute force que nous l'accompagnions :} à tout prix, coûte que coûte.

{\bf IV.} Principe d'action physique ou morale. \bl {\bf 1°} Énergie, travail ({\it vx. en science}). \lb {\it Mod.} Toute cause capable de déformer un corps, ou d'en modifier le mouvement, la direction, la vitesse. {\it La mécanique, science de l'équilibre des forces et des mouvementqu'elles engendrent. Représentation vectorielle d'une force} (direction, sens, point d'application, intensité). {\it Résultante de deux forces. Équilibre des forces. Force d'inertie,} résistance qu'oppose une mobile à ce qui peut le mettre en mouvement. {\it Moment d'une force par rapport à un point. {\bf —} Spécialt.} Produit de la masse d'un corps par l'accélération que le corps subit (F $=$ m $\Gamma$). {\it Force vive d'un corps,} produit de la masse d'un corps par le carré de sa vitesse. {\it Force centrifuge, centripète. Force ascensionnelle d'un ballon. L'erg, unité de force dans le système C.G.S, le newton dans le système M.K.S.A. Force de contact} opposé à {\it forces de champ}, à distance. \lb {\it Lignes de force d'un champ électrique, magnétique.} Fig. {\it Les lignes de force d'une œuvre. {\bf —} Forces de gravitation, électromagnétiques, nucléaires. {\bf —} Force électromotrice*. {\bf —} (Électr.)} Courant électrique, et {\it spécialt.} courant électrique triphasé. {\it Faire installer la force chez soi.} \bl {\bf 2°} Principe d'action, cause quelconque de mouvement, dechangement. « {\it Notre volonté est une force qui commande à toutes les autres forces} » ({\sc Buffon.}) {\bf —} {\it Idées-force :} opinions ou idées capable d'influencer l'évolution d'un individu ou d'une nation, d'une époque. \lb {\it Les forces aveugles, mystérieuse, occultes de l'univers, du destin.} {\bf —} Fig. {\it C'est une force de la nature,} se dit d'une personne doué d'une vitalité irrésistible.

{\bf V.} {\sc À force.} {\it adv} (Vx). {\bf V. beaucoup, extrêmement.} « {\it Ne vois-tu pas le sang, lequel dégoutte à force...} » ({\sc Ronssard}). \lb Mod. {\sc À force de} {\it (Loc. prép.) :} par beaucoup de grâce à beaucoup de. {\it À force de patience, il finira par réussir.} {\bf V. Avec.} « {\it À force de plaisir, notre bonheur s'abîme} » ({\sc Cocteau}). {\bf —} Suivi d'un verbe, exprime la répétition, l'intensité. « {\it Quels cheveux sans couleur, à force d'être blonds !} » ({\sc Stendhal}). « {\it À force de penser à Marthe, j'y pensai de moins en moins} » ({\sc Radiguet}). {\bf —} {\it Ellipt.} {\sc À force.} {\it loc. adv.} (Fam.) À force, il a fini par y arriver.

$\boxdot$ {\sc Antonyme} \si{Affaiblissement, asthénie, débilité, faibless, fatigue. Apathie, inertie, molesse. Impuissance. Inefficacité. Douceur, persuasion.} {\bf —} {\sc Homonyme} \si{Forces.} Formes de forcer

2. {\bf Force} {\it adv.} (1337 ; du précéd.). {\it Vx} ou {\it littér.} Beaucoup. « {\it J'ai dévoré force moutons} » ({\sc La Fontaine}). « {\it Nous nou séparâmes à la porte avec force poignées de main} » ({\sc Daudet}). $\boxdot$ {\sc Homonyme} {\bf V. Force (1)}

{\bf FORCÉ, ÉE}. {\it adv.} ({\footnotesize XVI}$^\text{e}$, « involontaire ». {\bf V. Forcer}). \bl {\bf 1°} Qui est imposé par la force des hommes ou des choses. {\it Conséquence forcé}. {\bf V. Inéluctable, inévitable, nécessaire.} {\it Le mariage forçé}, comédie de Molière (1664). {\it Emprunt forçé}. {\bf V. obligatoire.} {\it Cours forçé d'une monnaie. Bagnard qui purge sa peine de travaux forçés} {\bf (V. forçat).} {\it L'avion a dû faire un atterrissage forçé. Un bain forçé.} {\bf V. involontaire.} {\it vente forçée. \lb Fam.} (Pour marquer le caractère nécessaire d'un évenement passé ou futur) {\it C'est forcé.} {\bf V. Évident, inévitable.} {\it Il perdra, c'est forcé !} {\bf V. forcément.} \bl {\bf 2°} {\it Vieilli.} Qui manque de sincérité ou de naturel. {\bf V. Affecté, artificiel, contraint, embarassé.} « {\it Vous vous moquez, me dit-il d'un air forcé} » ({\sc Marivaux}). Mod. {\it Un rire, un sourire forcé.} \bl {\bf 3°} Qui s'écarte du vrai ou du naturel. {\it Une comparaison forcée} (Cf. Tirée par les cheveux). {\it Effet forcé :} mal amené, trop recherché. $\boxdot$ {\sc ANTONYME} \si{Facultatif, libre, naturel, vrai.}


%
\begin{appendix}
%

%%%%%%%%%%%%%%%%%%%%%
\chapter{Glossaire}
%%%%%%%%%%%%%%%%%%%%%

\begin{itemize}[leftmargin=1cm, label=\ding{32}, itemsep=2pt]
\item {\bf application} : en mathématique, synonyme de fonction.
\item {\bf } :
\item {\bf } :
\item {\bf quanton} : particule élémentaire satisfaisant à l'équation de schrödinger.
\item {\bf } :
\item {\bf } :
\item {\bf } :
\end{itemize}


%%%%%%%%%%%%%%%%%%%%%%%%%%%%%%%%%%%%%%%%%%%%%%%%%%%%%%%%%%%%%%%%%%%%%%%%%%%%%%%%%%%%%

%

%%%%%%%%%%%%%%%%%%%%%
\chapter{Espace vectoriel}
%%%%%%%%%%%%%%%%%%%%%

%%%%%%%%%%%%%%%%%%%%%%%%%
\section{Ensemble et application}
%%%%%%%%%%%%%%%%%%%%%%%%%
%$\mathcal{}$
Un ensemble est une collection d'objets. Ces objets sont appelés éléments (a) de l'ensemble ($\mathcal{A}$) :
\[
 a \in \mathcal{A}
\]


Une application ($f$) met en relation chaque élément ($a$) d'un ensemble ($\mathcal{A}$, dit de départ) avec un élément ($b$) d'un autre ensemble ($\mathcal{B}$, dit d'arrivé) :
\begin{align*}
f :\ \ \ \ \ \ \ \ \ \mathcal{A} \ \  & \rightarrow \ \ \ \mathcal{B} \\
a \ \ & \mapsto \ \ b = f(a)
\end{align*}

Une loi de composition est une application qui associe deux éléments (éventuellement du même ensemble) à un troisième élément. 
\begin{align*}
f :\ \ \ \ \ \ \ \ \ \mathcal{A} \times \mathcal{B} \ \  & \rightarrow \ \ \ \mathcal{C} \\
(a,b) \ \ & \mapsto \ \ c = f(a,b)
\end{align*}

Une loi de composition est dite interne si $\mathcal{A} = \mathcal{B} = \mathcal{C}$, externe sinon.



%%%%%%%%%%%%%%%%%%%%%%%%%
\section{Espace vectoriel}
%%%%%%%%%%%%%%%%%%%%%%%%%
%
Un espace vectoriel est un ensemble (ses éléments sont appelés vecteur), possédant une loi de composition interne (la somme de deux vecteurs d'un espace vectoriel appartient à cet espace) et une loi de composition externe (la multiplication par un scalaire d'un vecteur d'un espace vectoriel appartient à cet espace).


%%%%%%%%%%%%%%%%%%%%%%%%%%%%%%%%%%%%%%%%%%%%%%%%%%%%%%%%%%%%%%%%%%%%%%%%%%%%%%%%%%%%%

%

%%%%%%%%%%%%%%%%%%%%%
\chapter{Transformation de fourier}
%%%%%%%%%%%%%%%%%%%%%

%%%%%%%%%%%%%%%%%%%%%%%%%
\section{Série de fourier}
%%%%%%%%%%%%%%%%%%%%%%%%%
%
Une fonction périodique (de période $T$) est égale à une somme discrète de sinusoïde :
\[
f_T(x)=a_0 + \sum_{n=1}^\infty \left( a_n \cos \frac{2 n \pi x}{T} + b_n \sin \frac{2 n \pi x}{T} \right)
\]
$a_n$ et $b_n$ sont les coefficient de fourier de $f_T(x)$.

%%%%%%%%%%%%%%%%%%%%%%%%%
\section{Transformé de fourier}
%%%%%%%%%%%%%%%%%%%%%%%%%
%
Une fonction quelconque est égale à une somme continue de sinusoïde :
\[
f(x) = \int_{-\infty}^\infty e^{2 i \pi \nu x}\widehat{f}(\nu) d\nu
\]
$\widehat{f}(\nu)$ est la transformé de fourier de $f(x)$.

%%%%%%%%%%%%%%%%%%%%%%%%%%%%%%%%%%%%%%%%%%%%%%%%%%%%%%%%%%%%%%%%%%%%%%%%%%%%%%%%%%%%%

%
%\newpage
%
\end{appendix}
%

%
%====================== INCLUSION DE LA BIBLIOGRAPHIE ======================
%
%récupérer les citation avec "/footnotemark" : 
\nocite{*}
%
% choix du style de la biblio
\bibliographystyle{plain}
%
% inclusion de la biblio
\cleardoublepage
\addcontentsline{toc}{chapter}{Bibliographie}
\bibliography{bibliographie.bib}
%
%====================== FIN DU DOCUMENT ======================
%
\end{document}
%%%%%%%%%%%%%%%%%%%%%%%%%%%%%%%%%%%%%%%%%%%%%%%%%%%%%%%%%%%%%%%%%%%%%%%%%%%%%%%%%
