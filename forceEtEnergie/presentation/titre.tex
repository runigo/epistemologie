\begin{titlepage}
%
%~\\[1cm]
%régler l'épaisseur des lignes
\newcommand{\HRule}{\rule{\linewidth}{0.5mm}}

\begin{center}
%\includegraphics[scale=1.25]{./presentation/deuxDes}
\end{center}

\textsc{\Large }\\[0.5cm]

% Title \\[0.4cm]
\HRule

\begin{center}
{\huge \bfseries  Force et\\
énergie\\[0.4cm] }
\end{center}

\HRule \\[1.5cm]


\vfill

\begin{minipage}{0.4\textwidth}
%\begin{flushleft}
{\bf énergie }{\footnotesize XV}$^\text{e}$ s. {\it Jardin de santé}, du bas latin {\it energia} (saint Jérôme), emprunté au grec {\it energeia}, force en action. || {\bf énergique} fin {\footnotesize XVI}$^\text{e}$ s. || {\bf énergétique} 1768, {\it Encycl.}, « qui paraît avoir une énergie innée »; sens actuel, fin {\footnotesize XIX}$^\text{e}$ s. (1909, L. M.); du grec {\it energetikos}.
%\end{flushleft}
\end{minipage}
\hfill
\begin{minipage}{0.4\textwidth}
\begin{flushleft}
{\bf force} 1080, {\it Roland}, du bas latin {\it fortia}, pl. neutre subst. de {\it fortis}, courageux puis fort. || {\bf forcer} {\footnotesize XIII}$^\text{e}$ s. {\it Chr d'Antioche}, du lat. pop. {\it fortiare}, de {\it fortia}. [...]% || {\it forçage} {\footnotesize XII}$^\text{e}$ s.
\end{flushleft}
\begin{flushright} \large
Larousse éthymologique
\end{flushright}
\end{minipage}

\vfill
{\sf \footnotesize
\begin{itemize}[leftmargin=1cm, label=\ding{32}, itemsep=1pt]
\item {\bf Objet : } Définir les forces et les énergies.
\item {\bf Contenu : } Extraits de dictionnaires.
\item {\bf Public concerné : } Tous.
\end{itemize}
}
\vfill

\begin{flushright}
Numérisation : Stephan Runigo
\end{flushright}
\vfill

% Bottom of the page
{\large \today}

\end{titlepage}
