\section{Action, lagrangien et énergie}
Éthymologiquement, l'énergie est une {\it force en action} (opposé à une {\it force au repos}). C'est le sens premier d'{\it énergie}.

La théorie mécanique dévellopée à partir de la mécanique newtonienne et des théorie ondulatoire conduit à introduire de nouvelles grandeurs : l'{\it action}, le {\it lagrangien}, l'{\it énergie}. Ce sont des fonctions mathématiques des grandeurs déja connue (position, vitesse, masse, ...). Ces nouvelles grandeurs sont pertinentes car elles permettent d'unifier des lois physiques.


