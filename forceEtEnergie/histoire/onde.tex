\section{Onde et particule}
La physique classique (De Newton à Maxwell) distingue les phénomènes ondulatoires des phénomènes particulaires. Une onde se propage, s'étend, est susceptible d'interférence, une particule conserve sa forme, est susceptible d'en choquer une autre, suit une trajectoire.

\begin{center}
{\sf Figure : diffraction, interférence, choc, trajectoire.}
\end{center}

La nouvelle physique (quantique) unifie ces phénomènes, il n'y a plus lieu de distinguer des ondes et des particules, il n'y a plus que des quantons (ou des champs dans les théorie les plus récentes $\to$ Feynmann).
