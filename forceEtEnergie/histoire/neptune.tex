\section{Solution ontologique et solution législative}

%https://www.sciencepresse.qc.ca/blogue/cerveau-niveaux/2019/05/14/faire-quand-survient-ecart-entre-theorie-observation
%https://fr.linkedin.com/pulse/r%C3%A9solution-ontologique-ou-l%C3%A9gislative-deli%C3%A8ge-builder-of-influencers
%https://forums.futura-sciences.com/archives/822692-solution-ontologique-legislative.html
%https://www.researchgate.net/publication/360773689_La_matiere_noire_une_enigme_de_la_cosmologie_contemporaine

L'observation du mouvement des planètes de notre système solaire s'est affinée au cours des temps. Les observations anciennes ont conduit à la mécanique aristotélicienne : les planètes ont un mouvement circulaire autour de la terre immobile. Des observations plus rigoureuse ont montré des mouvements plus compliqué que de simple cercle : cela à conduit à la théorie des épicycles.

Dix siècles plus tard, La révolution copernicienne conduit à l'invention de la mécanique newtonienne, qui s'impose face à la mécanique aristotélicienne.
%Face à des observations précises, une loi plus générale

La mécanique newtonienne permet de prédir les trajectoires des planètes de façon plus précise que la mécanique aristotélicienne.

Mais l'histoire ne s'arrète pas là. Des observations toujours plus précise vont conduire à découvrir que les trajectoires d'Uranus et de Mercure diffère des trajectoires calculées à partir de la nouvelle mécanique. Afin d'expliquer ces différences, des solutions vont finir par s'imposer.

Dans le cas d'Uranus, la solution va aboutir à la découverte d'une nouvelle planète : Neptune. Dans le cas de Mercure, la solution va aboutir à la découverte d'une nouvelle théorie : la relativité générale.

%La mécanique newtonienne permet de déduire la trajectoire des planètes de notre système solaire.
