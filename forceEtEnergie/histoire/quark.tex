\section{Les quarks}

Les quarks sont les constituants des hadrons (proton, neutron, ...). Un neutron, comme un proton, est constitué de trois quark. Les quarks sont soumis à la force forte. Cette force, décrite dans la seconde moitié du {\footnotesize XX}$^\text{e}$ s., maintient les quarks entre eux.

Comme la force gravitationnelle maintient les planètes autour du soleil, la force électromagnétique maintient les électrons autour du noyau, la force forte maintient les quarks dans les protons et les neutrons. la force forte maintient également les protons et les neutrons dans les noyaux.

L'invention (la découverte) de la force forte a nécessité de baptiser de nouvelles grandeurs physiques.


\begin{center}
{\large\begin{tabular}{ccccc}
{\sf Interaction} & : & {\bf gravitationnelle} & {\bf électromagnétique} & {\bf forte} \\
{\sf remarque} & : & toujours attractive & attractive ou répulsive & toujours attractive ? \\
{\sf grandeur} & : & {\bf masse gravitationnelle} & {\bf charge électrique} & {\bf saveur} \\
{\sf remarque} & : & toujours positive & positive ou négative & 6 saveurs \\
\end{tabular}}
\end{center}

