\section{Vocabulaire}

Au fur et à mesure des découvertes théoriques, les physiciens se sont servie dans le vocabulaire courant afin de nommer les nouvelles grandeurs. Les succés de la physique théorique ont conduit à un certain phagocytage du sens courant de certain mots, en particulier l'énergie (mais aussi l'atome, le temps, ...).

Ontologie : la physique se débarasse des problèmes ontologique en séparant le système des grandeurs et des lois. Le système est et les grandeurs associées (les observables) sont reliées par des relations mathématiques (les lois).

Dans un paradigme un ensemble d'expérience sont décrites. Elles sont censés vérifier les lois du paradigme

Philosophie de l'histoire, lois de l'histoire, Hegel


Newton était alchimiste, ...
