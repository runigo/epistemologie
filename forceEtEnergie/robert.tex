\section{Petit Robert}

{\bf ÉNERGIE} {\sf n.f.} ( v. 1500 ; bas lat. {\it energia}, gr. {\it energeia} « force en action »).

{\bf I.} {\it cour.} \bl {\bf 1°} {\it Vieilli}. Pouvoir, efficacité (d'un agent quelconque). \lb Force , vigueur (dans l'expression, dans l'art). « {\it Quelle fraîcheur de coloris, quelle énergie d'expression} » ({\sc Rousseau}). « Une énergie singulière, un pittoresque effrayant » ({\sc Hugo}).
%
\bl {\bf 2°}(Fin {\footnotesize XVIII}$^\text{e}$). Force et fermeté dans l'action qui rend capable de grands effets. {\bf V. Dynamisme, ressort, volonté.} « {\it Cette énergie sublime qui fait faire les choses extraordinaires} » ({\sc Sthendal}). « {\it La quantité d'énergie ou de volonté que chacun de nous possède} » ({\sc Balzac}). « {\it Galvaniser nos énergie} » ({\sc Gide}). {\it Une énergie indomptable, farouche} « {\it L'internationale... avait perdu sa vitalité, tout en confisquant l'énergie du prolétariat} » ({\sc Romains}). {\it Regain d'énergie} {\bf V. Souffle} (second souffle). \lb Force, vitalité physique. {\it Se sentir plein d'énergie, frotter avec énergie.} « {\it Je le battis avec l'énergie obstinée des cuisiniers qui veulent attendrir un beafsteak} » ({\sc Baudelaire}).

{\bf II.} {\it Sc.} \bl {\bf 1°} {\it Phys.} (1875 ; angl. {\it energy}, 1852 ; sens plus vague 1807). Ce que possède un système s'il est capable de produire du travail. {\it Les différentes forme de l'énergie et leurs transformation. Énergie mécanique potentielle d'un corps} (Travail pouvant être produit en raison de la position d'un corps) ; {\it énergie cinétique} (acquise du fait de sa vitesse). {\it Énergie thermique} {\bf V. chaleur, thermodynamique.} {\it Énergie électrique, solaire} {\bf (V. rayonnement,)} chimique, nucléaire  {\bf (V. Radiation ; fission, fusion)}. {\it Principe de la conservation de l'énergie. Énergie interne,} en thermodynamique, somme des énergie potentielle et cinétique inhérentes à un système. {\it Les variations de l'énergie interne d'un système ne dépendent que de ses états initial et final.} \bl {\bf 2°} Énergie chimique potentielle de l'être vivant. {\it Énergie physiologique minimale} (ou métabolisme de base), dépense énergétique de l'organisme au repos complet.

$\boxdot$ {\sc antonyme} \textit{\textsf{Indolence, inertie, mollesse, paresse.}}

\vspace{0.24cm}
{\footnotesize 
{\bf ÉNERGÉTICIEN} {\it n. m.} (v. 1970 ; de énergétique). {\it Sc., techn.} Spécialiste de l'énergétique.

{\bf ÉNERGÉTIQUE} {\it adj.} et {\it n. f.} ( 1898 « qui parait avoir une énergie innée »; angl. energetic, gr {\it energêtikos}). \bl {\bf 1°} {\it Adj.} Relatif à l'énergie, aux grandeurs, aux unité, liées à l'énergie sous toute ses formes. {\it Puissance énergétique. Théorie énergétique}, système remplaçant en mécanique la notion de force par celle d'énergie. \lb Relatif à l'énergie utilisé industriellement. {\it Les ressources énergétique d'un pays.} \lb Physiol. {\it Dépense énergétique}, énergie qu'utilise l'organisme pour une action ou une fonction déterminée. {\it Aliments énergétiques}, qui fournissent beaucoup d'énergie à l'organisme. \bl {\bf 2°} {\it N. f.} Théorie énergétique. Science traitant des diverses manifestations de l'énergie.

{\bf ÉNERGIQUE} {\it adj.} (fin {\footnotesize XVI}$^\text{e}$; de {\it énergie}). \bl {\bf 1°} Actif, efficace. \bl {\it Un remède énergique.} {\bf —} Plein d'énergie (dans l'expression). {\bf V. Vigoureux.} \bl {\bf 2°}(fin {\footnotesize XVIII}$^\text{e}$). Qui a de l'énergie, de la volonté. {\bf V. Ferme, fort, mâle, résolu.} « {\it Un homme énergique n'a jamais peur en face du danger pressant} » ({\sc Maupassant}). {\bf —} qui exprime, marque de l'énergie. {\it Un visage énergique.} « {\it Une intervention énergique de la police} » ({\sc Martin du Gard}). \lb Fort, puissant (dans l'ordre physique). « {\it Un coup de pied... assez énergique pour briser les homoplates} » ({\sc Baudelaire}) $\boxdot$ {\sc ANTONYME} \textit{\textsf{Indolence, inertie, mollesse, paresse.}}

{\bf ÉNERGIQUEMENT} {\it adv.} (1584; de énergique) Avec énergie. {\bf V. Fermement, résolument.} « {\it Fais énergiquement ta longue et lourde tâche} » ({\sc Vigny}). {\it Résister, protester énergiquement.} \lb Avec force « {\it Je serrai énergiquement cette main} » ({\sc Jaloux}).

{\bf ÉNERGISANT, E} {\it adj.} et {\it n.} (v. 1970 ; calque de l'angl. {\it energizing}). {\it Méd.} \bl {\bf 1°} {\it adj.} Qui stimule, donne de l'énergie. {\it L'action énergisante d'un médicament.} \bl {\bf 2°} {\it N. m.} Médicament qui stimule l'activité psychique. {\it Prendre des énergisant}. {\bf V. Antidépresseur, psychotonique, psychotrope.}
}
\vspace{0.31cm}

{\bf FORCE} {\sf n.f.} (1080 ;bas lat. {\it fortia}, plur. neutre substantivé de {\it fortis}. {\bf V. Fort ; forcer}).

{\bf I.} {\it La force de quelqu'un}. \bl {\bf 1°} Puissance d'action physique (d'un être, d'un organe). {\it Force physique ; force musculaire.} {\bf V. Résistance, robustesse, vigueur.} {\it La force du lion. Force de colosse, d'athlète. Avoir de la force.} {\bf V. fort.} {\it Ne plus avoir la force de marcher, de parler. Ne pas sentir sa force :} frapper, pousser, etc., trop fortsans s'en rendre compte. {\it Lutter à force égales, à égalité de forces.} « {\it Elle serrait la rampe avec tant de force que le bois grinçait} » ({\sc Green}). « {\it Patience et longueur de temps Font plus que force ni que rage} » ({\sc La Fontaine}) {\it Être à bout de force, sans force}. \lb {\it (Au plur.)} Ensemble, concours d'énergie. {\it Ménager ses forces. Ce travail est au-dessus de ses forces. Ses forces l'ont trahi. Reprendre des forces. Aliment qui redonne des forces :} fortifie, réconforte. {\it De toute ses forces :} en rassemblant et en utilisant toute ses forces ; {\it par ext.} le plus fort possible. {\it Il tapait, il criait de toutes ses forces.} \lb ({\it opposé} à adresse, souplesse) {\sc En force}, opposé à « en souplesse ». Courir, nager en force. {\bf —} {\sc De force} : qui exige de la force. {\it Tour de force. Épreuve de force. Travailleur de force :} personne dont le métier exige une grande dépense de force physique. {\it Travail, exercice de force.} \lb Mar. {\it Faire force} exercer ou imposer l'effort maximum. {\bf V. forcer.} {\it Faire force de rames :} ramer de toute ses forces. \lb Loc. {\it Dans la force de l'âge :} au moment ou un homme est le plus fort (maturité). \bl {\bf 2°} Capacité de l'esprit ; possibilités intellectuelles et morales. {\it La science des géomètres qui } « {\it exerce la force de l'esprit} » ({\sc Suarès}). « {\it Et consultez longtemps votre esprit et vos forces} » ({\sc Boileau}). Dans l'ordre moral. {\bf V. Constance, courage, cran, détermination, énergie, fermeté, volonté.} {\it Force morale ; force de caractère. La force d'âme des héros cornéliens.} « {\it Elle avait la force devant qui les autres plient : le calme} » ({\sc R. Rolland}). « {\it Elle me résistait avec une force de volonté qui voulait maîtriser la mienne} » ({\sc Loti}). {\it Ce sacrifice est au dessus de mes forces.} \bl {\bf 3°}  {\sc De} (telle ou telle) {\sc force.} {\it Ils sont de la même force} (physique morale). {\bf —} {\it Spécialement} (sur le plan intellectuel ou de l'habileté) {\it Ce joueur n'est pas de force. Ils sont de la même force au tennis, aux échecs, en mathématiques.} {\bf V. niveau.} \bl {\bf 4°} {\it Faire la force de qqn. :} constituer sa supériorité. « {\it Ce qui fait ma force c'est que je fais tout moi-même} » ({\sc Romains}).

{\bf II.} {\it La force d'un groupe, de qqch.} \bl {\bf 1°} Pouvoir, puissance. {\it La force de l'Église, d'un parti. Force militaire d'un pays.} Par ext. {\it La force publique :} les agents armés du gouvernement


