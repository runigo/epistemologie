\section{Petit Robert}

{\bf ÉNERGIE} {\sf n.f.} ( v. 1500 ; bas lat. {\it energia}, gr. {\it energeia} « force en action »).

{\bf I.} {\it cour.} \bl {\bf 1°} {\it Vieilli}. Pouvoir, efficacité (d'un agent quelconque). \lb Force , vigueur (dans l'expression, dans l'art). « {\it Quelle fraîcheur de coloris, quelle énergie d'expression} » ({\sc Rousseau}). « Une énergie singulière, un pittoresque effrayant » ({\sc Hugo}).
%
\bl {\bf 2°}(Fin {\footnotesize XVIII}$^\text{e}$). Force et fermeté dans l'action qui rend capable de grands effets. {\bf V. Dynamisme, ressort, volonté.} « {\it Cette énergie sublime qui fait faire les choses extraordinaires} » ({\sc Sthendal}). « {\it La quantité d'énergie ou de volonté que chacun de nous possède} » ({\sc Balzac}). « {\it Galvaniser nos énergie} » ({\sc Gide}). {\it Une énergie indomptable, farouche} « {\it L'internationale... avait perdu sa vitalité, tout en confisquant l'énergie du prolétariat} » ({\sc Romains}). {\it Regain d'énergie} {\bf V. Souffle} (second souffle). \lb Force, vitalité physique. {\it Se sentir plein d'énergie, frotter avec énergie.} « {\it Je le battis avec l'énergie obstinée des cuisiniers qui veulent attendrir un beafsteak} » ({\sc Baudelaire}).

{\bf II.} {\it Sc.} \bl {\bf 1°} {\it Phys.} (1875 ; angl. {\it energy}, 1852 ; sens plus vague 1807). Ce que possède un système s'il est capable de produire du travail. {\it Les différentes forme de l'énergie et leurs transformation. Énergie mécanique potentielle d'un corps} (Travail pouvant être produit en raison de la position d'un corps) ; {\it énergie cinétique} (acquise du fait de sa vitesse). {\it Énergie thermique} {\bf V. chaleur, thermodynamique.} {\it Énergie électrique, solaire} {\bf (V. rayonnement,)} chimique, nucléaire  {\bf (V. Radiation ; fission, fusion)}. {\it Principe de la conservation de l'énergie. Énergie interne,} en thermodynamique, somme des énergie potentielle et cinétique inhérentes à un système. {\it Les variations de l'énergie interne d'un système ne dépendent que de ses états initial et final.} \bl {\bf 2°} Énergie chimique potentielle de l'être vivant. {\it Énergie physiologique minimale} (ou métabolisme de base), dépense énergétique de l'organisme au repos complet.

$\boxdot$ {\sc Antonyme} \si{Indolence, inertie, mollesse, paresse.}

\vspace{0.24cm}
{\footnotesize 
{\bf ÉNERGÉTICIEN} {\it n. m.} (v. 1970 ; de énergétique). {\it Sc., techn.} Spécialiste de l'énergétique.

{\bf ÉNERGÉTIQUE} {\it adj.} et {\it n. f.} ( 1898 « qui parait avoir une énergie innée »; angl. energetic, gr {\it energêtikos}). \bl {\bf 1°} {\it Adj.} Relatif à l'énergie, aux grandeurs, aux unité, liées à l'énergie sous toute ses formes. {\it Puissance énergétique. Théorie énergétique}, système remplaçant en mécanique la notion de force par celle d'énergie. \lb Relatif à l'énergie utilisé industriellement. {\it Les ressources énergétique d'un pays.} \lb Physiol. {\it Dépense énergétique}, énergie qu'utilise l'organisme pour une action ou une fonction déterminée. {\it Aliments énergétiques}, qui fournissent beaucoup d'énergie à l'organisme. \bl {\bf 2°} {\it N. f.} Théorie énergétique. Science traitant des diverses manifestations de l'énergie.

{\bf ÉNERGIQUE} {\it adj.} (fin {\footnotesize XVI}$^\text{e}$; de {\it énergie}). \bl {\bf 1°} Actif, efficace. \bl {\it Un remède énergique.} {\bf —} Plein d'énergie (dans l'expression). {\bf V. Vigoureux.} \bl {\bf 2°}(fin {\footnotesize XVIII}$^\text{e}$). Qui a de l'énergie, de la volonté. {\bf V. Ferme, fort, mâle, résolu.} « {\it Un homme énergique n'a jamais peur en face du danger pressant} » ({\sc Maupassant}). {\bf —} qui exprime, marque de l'énergie. {\it Un visage énergique.} « {\it Une intervention énergique de la police} » ({\sc Martin du Gard}). \lb Fort, puissant (dans l'ordre physique). « {\it Un coup de pied... assez énergique pour briser les homoplates} » ({\sc Baudelaire}) $\boxdot$ {\sc ANTONYME} \si{Indolence, inertie, mollesse, paresse.}

{\bf ÉNERGIQUEMENT} {\it adv.} (1584; de énergique) Avec énergie. {\bf V. Fermement, résolument.} « {\it Fais énergiquement ta longue et lourde tâche} » ({\sc Vigny}). {\it Résister, protester énergiquement.} \lb Avec force « {\it Je serrai énergiquement cette main} » ({\sc Jaloux}).

{\bf ÉNERGISANT, E} {\it adj.} et {\it n.} (v. 1970 ; calque de l'angl. {\it energizing}). {\it Méd.} \bl {\bf 1°} {\it adj.} Qui stimule, donne de l'énergie. {\it L'action énergisante d'un médicament.} \bl {\bf 2°} {\it N. m.} Médicament qui stimule l'activité psychique. {\it Prendre des énergisant}. {\bf V. Antidépresseur, psychotonique, psychotrope.}
}
\vspace{0.31cm}

{\bf FORCE} {\sf n.f.} (1080 ;bas lat. {\it fortia}, plur. neutre substantivé de {\it fortis}. {\bf V. Fort ; forcer}).

{\bf I.} {\it La force de quelqu'un}. \bl {\bf 1°} Puissance d'action physique (d'un être, d'un organe). {\it Force physique ; force musculaire.} {\bf V. Résistance, robustesse, vigueur.} {\it La force du lion. Force de colosse, d'athlète. Avoir de la force.} {\bf V. fort.} {\it Ne plus avoir la force de marcher, de parler. Ne pas sentir sa force :} frapper, pousser, etc., trop fortsans s'en rendre compte. {\it Lutter à force égales, à égalité de forces.} « {\it Elle serrait la rampe avec tant de force que le bois grinçait} » ({\sc Green}). « {\it Patience et longueur de temps Font plus que force ni que rage} » ({\sc La Fontaine}) {\it Être à bout de force, sans force}. \lb {\it (Au plur.)} Ensemble, concours d'énergie. {\it Ménager ses forces. Ce travail est au-dessus de ses forces. Ses forces l'ont trahi. Reprendre des forces. Aliment qui redonne des forces :} fortifie, réconforte. {\it De toute ses forces :} en rassemblant et en utilisant toute ses forces ; {\it par ext.} le plus fort possible. {\it Il tapait, il criait de toutes ses forces.} \lb ({\it opposé} à adresse, souplesse) {\sc En force}, opposé à « en souplesse ». Courir, nager en force. {\bf —} {\sc De force} : qui exige de la force. {\it Tour de force. Épreuve de force. Travailleur de force :} personne dont le métier exige une grande dépense de force physique. {\it Travail, exercice de force.} \lb Mar. {\it Faire force} exercer ou imposer l'effort maximum. {\bf V. forcer.} {\it Faire force de rames :} ramer de toute ses forces. \lb Loc. {\it Dans la force de l'âge :} au moment ou un homme est le plus fort (maturité). \bl {\bf 2°} Capacité de l'esprit ; possibilités intellectuelles et morales. {\it La science des géomètres qui } « {\it exerce la force de l'esprit} » ({\sc Suarès}). « {\it Et consultez longtemps votre esprit et vos forces} » ({\sc Boileau}). Dans l'ordre moral. {\bf V. Constance, courage, cran, détermination, énergie, fermeté, volonté.} {\it Force morale ; force de caractère. La force d'âme des héros cornéliens.} « {\it Elle avait la force devant qui les autres plient : le calme} » ({\sc R. Rolland}). « {\it Elle me résistait avec une force de volonté qui voulait maîtriser la mienne} » ({\sc Loti}). {\it Ce sacrifice est au dessus de mes forces.} \bl {\bf 3°}  {\sc De} (telle ou telle) {\sc force.} {\it Ils sont de la même force} (physique morale). {\bf —} {\it Spécialement} (sur le plan intellectuel ou de l'habileté) {\it Ce joueur n'est pas de force. Ils sont de la même force au tennis, aux échecs, en mathématiques.} {\bf V. niveau.} \bl {\bf 4°} {\it Faire la force de qqn. :} constituer sa supériorité. « {\it Ce qui fait ma force c'est que je fais tout moi-même} » ({\sc Romains}).

{\bf II.} {\it La force d'un groupe, de qqch.} \bl {\bf 1°} Pouvoir, puissance. {\it La force de l'Église, d'un parti. Force militaire d'un pays.} Par ext. {\it La force publique :} les agents armés du gouvernement {\bf V. Police.} {\it La force armée :} les troupes. {\bf —} (1959) {\it Force de frappe :} ensemble des moyens militaires modernes (fusées, armes atomiques) destinés à écraser rapidement l'ennemi. Fig. (1961) Autorité, force, puissance. {\bf —} {\it Forces de dissuasion*. } {\bf —} {\sc prov.} {\it L'union fait la force.} \lb {\sc En force.} {\it Être en force ; arriver, attaquer en force :} en nombre, avec des effectifs considérables. \bl {\bf 2°} {\it Plur.} ({\footnotesize XVII}$^\text{e}$). Ensemble des armées. {\bf V. Armée, troupe}. {\it Les forces armées françaises. Forces navales, aériennes (1939). Forces terriennes} (F.T.A.). {\it Regrouper, concentrer ses forces. Les forces de police, les forces de l'ordre} (dans le langage gouvernemental, police et gendarmerie intervenant en cas d'émeutes). {\bf —} Par ext. {\it Forces politiques, syndicales.} « {\it Rallier les forces d'opposition} » ({\sc Martin du Gard}). \bl {\bf 3°} Résistance d'un objet. {\bf V. Résistance, robustesse, solidité.} {\it Force d'un mur, d'une barre.} Spécialt. {\it jambe de force}, ou {\it force :} pièce de charpente qui sert à soulager la portée des longues poutres. \bl {\bf 4°} Intensité ou pouvoir d'action d'une chose ; caractère de ce qui est fort (III). {\it La force du vent. Force d'un coup, d'un choc. Diminuer la force d'un son. La force d'un acide. Force d'un médicament.} {\bf V. Activité, efficacité.} \lb (Choses abstraites) {\it La force d'un sentiment, d'un désir :} son intensité.  {\bf V. violence.} « {\it un muscle perd sa vigueur, un désir sa force} » ({\sc Colette}). {\bf —} {\it Force d'un argument, d'une idée. Ici,} « {\it le mensonge a autant de force que la vérité} » ({\sc Green}). {\bf —} Loc. {\it Dans toute la force du mot, du terme :} dans l'acception la plus signifiante. {\bf —} {\it Force du style.} {\bf V. Couleur, vie vigueur.} {\it S'exprimer avec force.} {\bf V. Éloquence, feu, véhémence.} \bl {\bf 5°} Typogr. {\it Force de corps* d'un caractère :} mesuré en points. {\it Un corps de force 6} (du 6).

{\bf III.} ({\footnotesize XII}$^\text{e}$). Pouvoir de contrainte. \bl {\bf 1°} En parlant d'une personne, d'un groupe. {\bf V. Contrainte, oppression, violence.} {\it Employer alternativement la force et la douceur. Céder, obéir à la force.} {\bf —} {\it La force et la justice, et le droit. La force prime le droit}, mot attribué à Bismarck. {\it Le gouvernement menace de recourir à la force} (en employant des {\it forces} de police, la {\it force} publique ; Cf. {\it ci-dessus}, II). \lb {\sc De force}. {\it Coup de force.} {\bf —} Pouvoir de contraindre donné par la supériorité militaire. {\it Situation de force. Épreuve de force}, tout espoir de conciliation étant écarté. {\bf —} {\it Maison centrale de force :} prison d'État où sont les condamnés aux travaux forçés et à la réclusion. {\bf V. Forçat.} {\bf —} {\it Camisole de force.} \bl {\bf 2°} {\it La force de} (qqch.) : son caractère irrésistible. {\it La force de l'évidence :} devant laquelle on s'incline. {\it Faire qqch. par la force de l'habitude :} machinalement. {\bf —} {\it La force des choses} : la nécessité qui résulte d'une situation. {\bf V. Nécessité, obligation.} « {\it C'est précisément parceque la force des choses tend toujours à détruire l'égalité, que la force de la législation doit toujours tendre à la maintenir} » ({\sc Rousseau}). {\it Par la force des choses :} obligatoirement, inévitablement. {\bf —} {\it Force d'une loi}, son caractère obligatoire. {\bf V. Autorité.} {\it V. Autorité.} {\it Avoir force de loi :} être assimilable à une loi, en avoir le caractère obligatoire. \lb {\it Force majeure} (dr.) : évènement imprévisible, inévitable et irrésistible qui libère d'une obligation. Cour. C'est un cas de force majeur. \lb {\it Force est de... :} il faut, on ne peut éviter de... « {\it Force lui fût de reconnaître qu'... il avait opté pour le plus facile} » ({\sc Martin du Gard}). \bl {\bf 3°} {\it Loc. adv} {\sc De force} : en faisant effort pour surmonter une résistance. {\it Faire entrer de force une chose dans une autre. Prendre, enlever de force qqch. à qqn.} {\bf V. Arracher, extorquer} {\it Il obéira de gré ou de force :} qu'il le veuille ou non. {\bf —} {\it Loc. adv.} {\sc Par force} : en recourant à la force ; en cédant à la force. {\it Prendre, obtenir qqch. par force. Il n'a pas accepté de son plein gré, mais par force :} parce que les évènements l'y contraignaient. {\bf —} {\sc À toute force} : en dépit de tout les obstacles, de toute les résistances. {\bf V. absolument.} {\it Il voulait à toute force que nous l'accompagnions :} à tout prix, coûte que coûte.

{\bf IV.} Principe d'action physique ou morale. \bl {\bf 1°} Énergie, travail ({\it vx. en science}). \lb {\it Mod.} Toute cause capable de déformer un corps, ou d'en modifier le mouvement, la direction, la vitesse. {\it La mécanique, science de l'équilibre des forces et des mouvementqu'elles engendrent. Représentation vectorielle d'une force} (direction, sens, point d'application, intensité). {\it Résultante de deux forces. Équilibre des forces. Force d'inertie,} résistance qu'oppose une mobile à ce qui peut le mettre en mouvement. {\it Moment d'une force par rapport à un point. {\bf —} Spécialt.} Produit de la masse d'un corps par l'accélération que le corps subit (F $=$ m $\Gamma$). {\it Force vive d'un corps,} produit de la masse d'un corps par le carré de sa vitesse. {\it Force centrifuge, centripète. Force ascensionnelle d'un ballon. L'erg, unité de force dans le système C.G.S, le newton dans le système M.K.S.A. Force de contact} opposé à {\it forces de champ}, à distance. \lb {\it Lignes de force d'un champ électrique, magnétique.} Fig. {\it Les lignes de force d'une œuvre. {\bf —} Forces de gravitation, électromagnétiques, nucléaires. {\bf —} Force électromotrice*. {\bf —} (Électr.)} Courant électrique, et {\it spécialt.} courant électrique triphasé. {\it Faire installer la force chez soi.} \bl {\bf 2°} Principe d'action, cause quelconque de mouvement, dechangement. « {\it Notre volonté est une force qui commande à toutes les autres forces} » ({\sc Buffon.}) {\bf —} {\it Idées-force :} opinions ou idées capable d'influencer l'évolution d'un individu ou d'une nation, d'une époque. \lb {\it Les forces aveugles, mystérieuse, occultes de l'univers, du destin.} {\bf —} Fig. {\it C'est une force de la nature,} se dit d'une personne doué d'une vitalité irrésistible.

{\bf V.} {\sc À force.} {\it adv} (Vx). {\bf V. beaucoup, extrêmement.} « {\it Ne vois-tu pas le sang, lequel dégoutte à force...} » ({\sc Ronssard}). \lb Mod. {\sc À force de} {\it (Loc. prép.) :} par beaucoup de grâce à beaucoup de. {\it À force de patience, il finira par réussir.} {\bf V. Avec.} « {\it À force de plaisir, notre bonheur s'abîme} » ({\sc Cocteau}). {\bf —} Suivi d'un verbe, exprime la répétition, l'intensité. « {\it Quels cheveux sans couleur, à force d'être blonds !} » ({\sc Stendhal}). « {\it À force de penser à Marthe, j'y pensai de moins en moins} » ({\sc Radiguet}). {\bf —} {\it Ellipt.} {\sc À force.} {\it loc. adv.} (Fam.) À force, il a fini par y arriver.

$\boxdot$ {\sc Antonyme} \si{Affaiblissement, asthénie, débilité, faibless, fatigue. Apathie, inertie, molesse. Impuissance. Inefficacité. Douceur, persuasion.} {\bf —} {\sc Homonyme} \si{Forces.} Formes de forcer

2. {\bf Force} {\it adv.} (1337 ; du précéd.). {\it Vx} ou {\it littér.} Beaucoup. « {\it J'ai dévoré force moutons} » ({\sc La Fontaine}). « {\it Nous nou séparâmes à la porte avec force poignées de main} » ({\sc Daudet}). $\boxdot$ {\sc Homonyme} {\bf V. Force (1)}

{\bf FORCÉ, ÉE}. {\it adv.} ({\footnotesize XVI}$^\text{e}$, « involontaire ». {\bf V. Forcer}). \bl {\bf 1°} Qui est imposé par la force des hommes ou des choses. {\it Conséquence forcé}. {\bf V. Inéluctable, inévitable, nécessaire.} {\it Le mariage forçé}, comédie de Molière (1664). {\it Emprunt forçé}. {\bf V. obligatoire.} {\it Cours forçé d'une monnaie. Bagnard qui purge sa peine de travaux forçés} {\bf (V. forçat).} {\it L'avion a dû faire un atterrissage forçé. Un bain forçé.} {\bf V. involontaire.} {\it vente forçée. \lb Fam.} (Pour marquer le caractère nécessaire d'un évenement passé ou futur) {\it C'est forcé.} {\bf V. Évident, inévitable.} {\it Il perdra, c'est forcé !} {\bf V. forcément.} \bl {\bf 2°} {\it Vieilli.} Qui manque de sincérité ou de naturel. {\bf V. Affecté, artificiel, contraint, embarassé.} « {\it Vous vous moquez, me dit-il d'un air forcé} » ({\sc Marivaux}). Mod. {\it Un rire, un sourire forcé.} \bl {\bf 3°} Qui s'écarte du vrai ou du naturel. {\it Une comparaison forcée} (Cf. Tirée par les cheveux). {\it Effet forcé :} mal amené, trop recherché. $\boxdot$ {\sc ANTONYME} \si{Facultatif, libre, naturel, vrai.}
