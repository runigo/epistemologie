\section{Petit Larousse}

{\bf ÉNERGIE} {\sf n.f.} (gr {\it energeia}, force en action). {\bf I.1.} Force morale, fermeté, puissance, vigueur. {\it L'énergie du désespoir}. {\bf 2.} Vigueur dans la manière de s'exprimer. {\it Parler avec énergie}. {\bf 3.} Force physique, vitalité. {\it Un être plein d'énergie}. {\bf 4.} \fsb{PSYCHAN.} {\it Énergie psychique} ; libido. {\bf II.1.} \fsb{PHYS.} {\bf a.} Grandeur caractérisant un système et exprimant sa capacité à modifier l'état d'autres systèmes avec lesquels il entre en interaction (unité SI : le {\it joule}). {\bf b.} Chacun des modes que peut présenter un tel système. {\it Énergie mécanique, magnétique, nucléaire}. {\bf 2.} {\it Sources d'énergie} : ensemble des matières premières ou des phénomènes naturels utilisés pour la production d'énergie (charbon, hydrocarbures, uranium, cours d'eau, marées, vent, etc.).

$\blacksquare$ Outre l'énergie mécanique ({\it énergie potentielle} d'un poids soulevé, d'un ressort comprimé, etc., et {\it énergie cinétique} d'une masse en mouvement), on distingue les énergies {\it chimique, électrique, nucléaire, calorifique, rayonnante}. L'énergie est un concept de base de la physique car un système isolé a une énergie totale constante. Il ne peut donc y avoir création ou disparition d'énergie, mais seulement transformation d'une forme d'énergie en une autre ou transfert d'énergie d'un système à un autre. Toute conversion d'énergie s'accompagne de pertes. Celles-ci sont particulièrement importantes dans la conversion d'énergie thermique en énergie mécanique.

\vspace{0.24cm}
{\footnotesize 
{\bf ÉNERGÉTICIEN, ENNE} {\sf n.} Spécialiste d'énergétique.

{\bf 1. ÉNERGÉTIQUE} {\sf adj.} (angl. {\it energetic}). Relatif à l'énergie, aux sources d'énergie. \lb {\it Aliment énergétique :} aliment nécessaire à l'organisme pour réparer ses dépenses d'énergie et ses pertes de matière. {\bf —} {\it Apport énergétique :} apport d'énergie fourni à un organisme par un aliment, une boisson.

{\bf 2. ÉNERGÉTIQUE} {\sf n.f.} Science et technique de la production de l'énergie, de ses emplois et des conversions de ses différentes formes.

{\bf ÉNERGIQUE} {\sf adj.} {\bf 1.} Qui agit fortement ; efficace. {\it Un remède énergique}. {\bf 2.} Qui est plein d'énergie, qui manifeste de l'énergie. {\it Visage énergique}.

{\bf ÉNERGIQUEMENT} {\sf adv.} avec énergie.

{\bf ÉNERGISANT, E} {\sf adj. n.m.} Se dit d'un produit qui donne de l'énergie ; stimulant.

{\bf ÉNERGIVORE} {\sf adj. Fam.} Qui consomme beaucoup d'énergie.}
\vspace{0.31cm}

{\bf FORCE} {\sf n.f.} (bas lat. {\it fortia}, pl. neutre de {\it fortis}, courageux). {\bf I.1.} Énergie, vigueur physique. {\it Elle a beaucoup de force. \lb À force de :} à la longue, par des efforts répétés. {\it Force de la nature :} persone qui a beaucoup d'endurance, de résistance ou qui est pleine de vitalité. Tour de force : exercice corporel exigeant une grande force physique ; {\sf fig.}, résultat qui suppose une habileté, un effort exceptionnels. {\bf 2.} {\it Force de travail :} dans la terminologie marxiste, ensemble des facultés physiques et intellectuelles de l'homme, à l'aide desquelles il produit des choses utiles. {\bf 3.} Courage, capacité de résister aux épreuves. {\it Force d'âme, de caractère}. {\bf 4.} Degré d'aptitude dans le domaine intellectuel, niveau ; habileté. {\it Ils sont de la même force en maths}. {\bf 5.} Degré d'intensité (d'un sentiment). {\it La force du désir}. {\bf II.1.a} Emploi de moyen violents pour contraindre une ou plusieurs personnes. {\it Céder à la force. Employer la force. Coup de force.} \lb {\it De force, de vive force :} en employant la violence, la contrainte. {\bf —} {\it Par force :} sous l'effet de la contrainte. {\bf b.} {\it Épreuve de force :} situation résultant de l'echec des négociations entre deux groupes antagonistes et où la solution ne dépend plus que de la supériorité éventuelle de l'un sur l'autre. {\bf 2.a.} Ensemble de personnes armées et organisées, chargées d'une tâche de protection de défense ou d'attaque. \lb {\it Force publique :} ensemble des formations de la police, de la gendarmerie et des armées qui sont à la disposition du gouvernement pour assurer le respect de la loi et le maintien de l'ordre. {\bf b.} {\it Force de frappe, de dissuasion} ou, en France {\it force nucléaire stratégique :} force militaire aux ordres directs de la plus haute instance politique d'un État, rassemblant la totalité de ses armements nucléaires stratégiques. {\bf 3.} Pouvoir de ce qui incite ou oblige à se comporter d'une certaine manière. {\it La force de l'habitude} \lb {\it Par force :} par nécessité. {\bf —} {\it (Cas de) force majeure :} évènement imprévisible, contraignant ; nécessité qui impose à qqn sa conduite. {\bf 4.} Autorité, ascendant, pouvoir effectif. {\it La force des lois. Avoir force de loi.} {\bf —} \fsb{DR.} {\it Force exécutoire :} qualité d'un acte ou d'un jugement qui permet si besoin est, le recours à la force publique pour son exécution. {\bf III.} \fsb{PHYS}. Toute cause capable de déformer un corps, d'en modifier l'état de repos ou de mouvement. {\it Force d'inertie.} {\bf —} {\it Force électromotrice}, caractéristique d'une source d'énergie électrique qui crée un courant dans un circuit et détermine l'intensité de ce courant.{\bf —} {\it Force contre-électromotrice} $\to$ \bi{contre-électromotrice.} {\bf —} {\it Force d'un électrolyte :} mesure de son énergie de dissociation. {\bf IV.1.} Degré de puissance, d'intensité d'un agent physique. {\it La force d'un courant. Vent force 7.} {\bf 2.} Degré d'efficacité, de rendement de qqch. {\it La force d'un médicament. La force d'une machine.} {\bf V.1.} Importance numérique, quantité. \lb {\it Être en force :} être nombreux. {\bf 2.} {\it Faire force de rames, de voiles :} faire en sorte que les rames, les voiles déploient leur maximum de force. 
\bl {\sf pl.} {\bf 1.} {\it Forces (armées) :} ensemble des formations militaire d'un état. {\it Forces aériennes, navales, terrestre.} {\bf 2.} Ensemble des personnes unies par une même volonté, et œuvrant à sa réalisation. {\it Les forces de progrès.} 
$\blacksquare$ 
La notion de force tend à être remplacée par celle d'interaction. Un nombre restreint d'interactions fondamentales permet en effet de rendre compte de la complexité des phénomènes physiques. Outre l'interaction {\it gravitationnelle}, s'exerçant sur tous les systèmes possédant une masse, on distingue l'interaction {\it électrofaible}, rendant compte des phénomènes électromagnétiques et des phénomènes radioactifs, et l'interaction {\it forte}, responsable de la cohésion du noyau atomique.

\vspace{0.24cm}
{\footnotesize 
FORCÉ, E {\sf adj.} {\bf 1.a.} Qui manque de naturel. {\it Un rire forcé.} {\bf b.} Qui est imposé, que l'on fait contre sa volonté. {\it Atterrissage forcé.}{\bf —} {\it Avoir la main forcée :} agir malgré soi sous la pression d'autrui. {\bf 2.} {\it Marche forcée :} marche dont la durée et la rapidité dépassent celles des marches ordinaires. {\bf 3.} {\sf Fam.} Inévitable. {\it Elle gagnera, c'est forcé.} {\bf 4.} {\it Culture forcée :} culture de plantes soumises au forçage.
}
\vspace{0.31cm}
