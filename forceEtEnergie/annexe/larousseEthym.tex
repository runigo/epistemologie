\section{Larousse éthymologique}
{\bf énergie }{\footnotesize XV}$^\text{e}$ s. {\it Jardin de santé}, du bas latin {\it energia} (saint Jérôme), emprunté au grec {\it energeia}, force en action. || {\bf énergique} fin {\footnotesize XVI}$^\text{e}$ s. || {\bf énergétique} 1768, {\it Encycl.}, « qui paraît avoir une énergie innée »; sens actuel, fin {\footnotesize XIX}$^\text{e}$ s. (1909, L. M.); du grec {\it energetikos}.

{\bf Force} 1080, {\it Roland}, du bas latin {\it fortia}, pl. neutre subst. de {\it fortis}, courageux puis fort. || {\bf forcer} {\footnotesize XIII}$^\text{e}$ s. {\it Chr d'Antioche}, du lat. pop. {\it fortiare}, de {\it fortia}. [...]% || {\it forçage} {\footnotesize XII}$^\text{e}$ s.
