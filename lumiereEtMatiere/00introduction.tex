\chapter{Introduction}


L'étude physique de l'interaction entre la lumière et la matière à l'aide des paradigmes classiques conduit à des difficultés, provoquant la crise scientifique de la fin du XIX ème. De cette crises émergera, au début du XX ème siècle, un nouveau paradigme (la physique quantique) qui réalisera une unification de la lumière et de la matière.

Ce document ce propose de retracer cette révolution scientifique.

Les deux premiers chapitres traiteront séparement de la lumière et de la matière. Ils évoqueront l'évolution des paradigmes antiques et médiévaux vers les paradigmes classiques.

Les chapitres suivant évoqueront l'interaction lumière -- matière en physique classique et 

Paradigme de la physique quantiques

\subsection{}\subsection{}
\begin{center}
\end{center}

PARADIGMES ANTIQUES ET MÉDIÉVAUX
	Atomisme : la matière est constituée d'atome (atome = insécable, indivisible)
	Quatre éléments : l'eau, l'air, la terre et le feu sont élémentaire, les autres matériaux sont des mélanges de ces quatres éléments.

PARADIGMES MODERNES
	Optique géométrique = rayon lumineux, principe de Fermat ( 1657 )

	Optique ondulatoire = diffraction et interférence, principe de Fresnel-Huyghens ( 1678 - 1815)

	Électromagnétisme =  équations de Maxwel-Lorentz ( 1875 ) = Comportement des champs électromagnétique et interaction avec la matière.

	Mécanique = équations de Newton ( 1687 )

	Mécanique analytique = théorie lagrangienne ( 1788 )

	Thermodynamique classique = température, conservation de l'énergie et accroissement de l'entropie.

	Physique statistique = Boltzmann et argument en faveur de l'atomisme (existence des atomes).


PARADIGMES CONTEMPORAINS
	Mécanique quantique = équation de schrödinger, 

GÉNÉALOGIE

Au XIXème, les équations de maxwel couronnent la physique classique. Associées aux équations de la mécanique newtonienne, elles semblent expliquer l'intégralité des interactions entre la lumière et la matière.

Dans le paradigme de la physique classique (Maxwell + Newton) on distingue les ondes des corpuscules : La lumière est une onde électromagnétique et la matière est constituée de corpuscule (éventuellement chargé électriquement).

Les corpuscules obéissent aux lois de la mécanique de Newton (ils ont une masse, une trajectoire, une position, ...). Leur mouvement est déterministe (la bonne connaissance de l'état d'un système de corpuscule à un instant permet de prévoir l'état de ce système plus tard).

Les ondes électromagnétiques obéissent aux équations de Maxwell (également déterministe).

Enfin, la force de Lorentz exprime le couplage entre le champ électromagnétique et les corpuscules (électron et proton = les constituants des atomes)




\subsection{}\subsection{}
\begin{center}
\end{center}

