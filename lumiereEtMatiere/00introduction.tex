\chapter{Introduction}

Qu'est-ce que la lumière ? On ne voit pas la lumière, on ne voit que les objets éclairés. Notre esprit conçoit une image des objets matériels grâce à la lumière, notre esprit construit une vision de la matière grâce à la lumière.

Ainsi, notre perception visuel de l'univers dépend de l'interaction entre la lumière et la matière. Nous ne voyons pas l'air, mais nous pouvons le sentir (quand il y a du vent), comme nous pouvons sentir la chaleur de la lumière lors d'un bain de soleil.

Qu'est-ce que la matière ? Nous voyons les corps opaques, nous ne voyons pas les corps transparents (le sens du toucher nous permet de distinguer les corps mous des corps durs, les corps chauds des corps froids, mais ce qui nous interesse ici c'est lumière et matière). 

Généalogiquement, les physiciens ont produit des {\it paradigmes} conduisant à différentes {\it visions} de la lumière et de la matière.
Les deux premiers chapitres traiteront séparement de la lumière et de la matière. Ils présenteront l'évolution des paradigmes antiques et médiévaux vers les paradigmes classiques.
La présentation des paradigmes classiques de l'interaction entre la lumière et la matière sera l'objet du troisième chapitre.

L'étude de l'interaction entre la lumière et la matière à l'aide des paradigmes classiques conduit à des difficultés, provoquant la crise scientifique de la fin du {\footnotesize XIX}$^\text{e}$ siècle. Des incompatibilités entre les différents paradigmes apparaissent, des écarts entre les résultats théorique et les résultats expérimentaux apparaissent.

De cette crise émergera, au début du {\footnotesize XX}$^\text{e}$, un nouveau paradigme (la physique quantique) qui réalisera une {\it unification} de la lumière et de la matière.
Le quatrième chapitre traitera de cette révolution scientifique.

\subsection{Paradigme}
Dans ce document, un paradigme est un ensemble contenant une modélisation d'un certain nombre d'expériences physiques et un certain nombre de lois physiques.

\subsection{}
\begin{center}
\end{center}

\begin{comment}
\end{comment}

PARADIGMES MODERNES
	Optique géométrique = rayon lumineux, principe de Fermat ( 1657 )

	Optique ondulatoire = diffraction et interférence, principe de Fresnel-Huyghens ( 1678 - 1815)

	Électromagnétisme =  équations de Maxwel-Lorentz ( 1875 ) = Comportement des champs électromagnétique et interaction avec la matière.

	Mécanique = équations de Newton ( 1687 )

	Mécanique analytique = théorie lagrangienne ( 1788 )

	Thermodynamique classique = température, conservation de l'énergie et accroissement de l'entropie.

	Physique statistique = Boltzmann et argument en faveur de l'atomisme (existence des atomes).


PARADIGMES CONTEMPORAINS
	Mécanique quantique = équation de schrödinger, 

GÉNÉALOGIE

Au XIXème, les équations de maxwel couronnent la physique classique. Associées aux équations de la mécanique newtonienne, elles semblent expliquer l'intégralité des interactions entre la lumière et la matière.

Dans le paradigme de la physique classique (Maxwell + Newton) on distingue les ondes des corpuscules : La lumière est une onde électromagnétique et la matière est constituée de corpuscule (éventuellement chargé électriquement).

Les corpuscules obéissent aux lois de la mécanique de Newton (ils ont une masse, une trajectoire, une position, ...). Leur mouvement est déterministe (la bonne connaissance de l'état d'un système de corpuscule à un instant permet de prévoir l'état de ce système plus tard).

Les ondes électromagnétiques obéissent aux équations de Maxwell (également déterministe).

Enfin, la force de Lorentz exprime le couplage entre le champ électromagnétique et les corpuscules (électron et proton = les constituants des atomes)




\subsection{Résumé}

LUMIÈRE
	Fluide emplissant l'espace
	Ensemble de rayon
	Ensemble de photon
	Onde
	Onde électromagnétique
MATIÈRE
	Ensemble de molécules, d'atomes
	Ensemble de charges en interaction électromagnétique
	

Pour qui à joué au billard ou à la pétanque (éventuellement au billes), et qui aime à regarder la mer, distingue naturellement les ondes des corpuscules.


\subsection{}
\begin{center}
\end{center}

