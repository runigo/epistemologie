\chapter{Atomisme}

\section{Quelques révolutions scientifiques qui ont changé notre vision de la matière}

\subsection{Les éléments chimique}
\begin{center}
Paradigme des 4 éléments : l'eau, l'air, la terre, et le feu sont \it{élémentaire}.
\end{center}

Les progrès de la chimie décompose l'eau en deux éléments (hydrogène et oxygène) puis l'air en deux éléments (oxygène et azote). C'est une crise pour le paradigme des 4 éléments. Il s'en suit un nouveau paradigme :

\begin{center}
Paradigme du tableau périodique des éléments : il y a 92 éléments.
\end{center}

\subsection{Les atomes}

Les progrès de la physique décrivent la matière comme constitué d'atomes. Ces atomes sont constitués de particules chargés électriquements (électron et proton) et neutre (neutron). La classification périodique des éléments est liée à la répartition des électrons dans les atomes, le paradigme de l'atomisme semble recevoir ici une confirmation.
\begin{center}
Modèle de l'atome : noyau chargé positivement et électrons chargés négativement.
\end{center}

\begin{center}
Paradigme de l'électrostatique : force de Coulomb (entre les charges électriques).

Paradigme de la magnétostatique : aimant et bousole.
\end{center}

La force de Coulomb explique la cohésion de la matière. Son expression mathématique permet la définition du champ électrique et les progrès de l'électricité conduisent à un paradigme unifiant les champs électrique et magnétique.

\begin{center}
Paradigme de l'électromagnétisme : Équations de Maxwell-Lorentz
\end{center}

\subsection{}\subsection{}
\begin{center}
\end{center}

