\chapter{Onde et corpuscule}

\section{Onde, interférence et propagation}
%%%%%%%%%%%%%%%%%%%%%%%%%%%%%%%%%
\subsection{Définitions et propriétés}
Le phénomène d'interférence est caractéristique des ondes.
\begin{center}
Une onde est étendue dans l'espace.
\end{center}
\begin{center}
Une onde est une perturbation qui se propage.
\end{center}

\begin{itemize}[leftmargin=1cm, label=\ding{32}, itemsep=1pt]
\item {\bf Les vagues : ondes de surface (perturbation de la surface entre l'air et l'eau).}
\item {\bf Le son : perturbation de la pression}
\end{itemize}
Exemples : Les vagues sont des 

\section{Corpuscule, force et collision}
%%%%%%%%%%%%%%%%%%%%%%%%%%%%%%%%%%%%
\subsection{Définitions et propriétés}
Le phénomène de collision est caractéristique des corpuscules.
\begin{center}
Un corpuscule est localisé dans l'espace.
\end{center}
\subsection{Exemples}
\begin{itemize}[leftmargin=1cm, label=\ding{32}, itemsep=1pt]
\item {\bf Le billard : }
\item {\bf Le billard : }
\end{itemize}



\section{Révolution Quantique}
\subsection{Photon et électron}



\subsection{Quanton}



\subsection{}\subsection{}
\begin{center}
\end{center}

