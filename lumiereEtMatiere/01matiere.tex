\chapter{Matière}

\section{Paradigmes antiques et médiévale}

\begin{comment}
\end{comment}
\subsection{Physique chinoise antique}

\begin{center}
5 {\it agents} : la terre, le feu, le metal, l'eau et le bois.

2 {\it principe} : yin (obscur, froid, humide, féminin, impair) et yang (lumineux, chaud, sec, masculin, pair)
\end{center}

La nature est un équilibre entre le yin et le yang. L'alternance des jours et de la nuit, des étés et des hivers, du soleil et de la lune, montrent les deux principes qui se succèdent sans se détruire.

\subsection{Physique hellène (de -600 à -300)}

{\bf Élément premier}

\begin{itemize}[leftmargin=1cm, label=\ding{32}, itemsep=5pt]
\item Thalès : l'élément premier est l'eau.
\item Anaximandre : l'élément premier est l'apeiron ( = cause matérielle = infini et indéfini, illimité et indéterminé)
\item Anaximène : l'air est le principe universel. En se dilatant, il produit le feu, en se condensant il produit l'eau, la pierre est son plus haut niveau de condensation.
\end{itemize}
%(211-212)

{\bf Atomisme}

Deux grands principe servent de base à la physique d'Anaxagore :

\begin{itemize}[leftmargin=1cm, label=\ding{32}, itemsep=5pt]
\item Dans ce qui est petit, il n'y a pas un dernier degré de petitesse, mais il y a toujours un plus petit.
\item Rien ne naît ni ne périt, mais des choses déja existantes se combinent, puis se séparent de nouveau.
\end{itemize}
%(217)

Le premier de ces principes est évidemment incompatible avec toute physique du type atomistique (la matière est constituée d'atome, {\it atome au sens insécable, indivisible}.

Le pythagorisme (toute chose procède du nombre entier) implique la discontinuité de la matière et semble annoncer l'atomisme.

Empédocle formule la théorie des quatre éléments : l'eau, l'air, la terre et le feu sont élémentaire, les autres matériaux sont des mélanges de ces quatres éléments.

\begin{comment}
\item Platon : les éléments derniers de la matière sont les corps simples.
\end{comment}

PARADIGMES ANTIQUES ET MÉDIÉVAUX
	Atomisme
	Quatre éléments : 

\subsection{Physique hellénistique et romaine}

\subsection{Les éléments chimiques}
\begin{center}
Paradigme des 4 éléments : l'eau, l'air, la terre, et le feu sont \it{élémentaire}.
\end{center}

Les progrès de la chimie décompose l'eau en deux éléments (hydrogène et oxygène) puis l'air en deux éléments (oxygène et azote). C'est une crise pour le paradigme des 4 éléments. Il s'en suit un nouveau paradigme :

\begin{center}
Paradigme du tableau périodique des éléments : il y a 92 éléments.
\end{center}

\subsection{Les atomes}

Les progrès de la physique décrivent la matière comme constitué d'atomes. Ces atomes sont constitués de particules chargés électriquements (électron et proton) et neutre (neutron). La classification périodique des éléments est liée à la répartition des électrons dans les atomes, le paradigme de l'atomisme semble recevoir ici une confirmation.
\begin{center}
Modèle de l'atome : noyau chargé positivement et électrons chargés négativement.
\end{center}

\begin{center}
Paradigme de l'électrostatique : force de Coulomb (entre les charges électriques).

Paradigme de la magnétostatique : aimant et bousole.
\end{center}

La force de Coulomb explique la cohésion de la matière. Son expression mathématique permet la définition du champ électrique et les progrès de l'électricité conduisent à un paradigme unifiant les champs électrique et magnétique.

\begin{center}
Paradigme de l'électromagnétisme : Équations de Maxwell-Lorentz
\end{center}

\subsection{}\subsection{}
\begin{center}
\end{center}

