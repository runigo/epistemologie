\section{La crise scientifique de la fin du XIX$^\mt{ème}$ siècle}

Les succès technologiques et théorique de la physique classique donnent une impression de conclusion, il semble qu'elle soit arrivée à bout des voiles d'isis. Les théories classiques mathématisent et expliquent la quasi-totalité des phénomènes physiques.

Quelques phénomènes restent inexpliqués, certaines théories montrent quelques incompatibilités, mais une grande partie des physiciens estiment que ce ne sont que des détails qui finiront par s'expliquer à l'aide des théories classiques.

\subsection{Électromagnétisme}
Les équations de maxwell-Lorentz ont été découvertes grâce à l'étude de la matière (électrostatique, magnétisme, mouvement des charges électriques) et à l'introduction des champs (E et B). Elles ont été écrites sans aucunes références aux phénomènes lumineux. L'étude de la matière a conduit aux équations de maxwell-Lorentz, l'étude de la lumière a conduit aux paradigmes de l'optique géométrique et de l'optique ondulatoire.

Au {\footnotesize XIX}$^\text{e}$ siècle, les physiciens montrent que les équations de maxwell-Lorentz conduisent à l'existence d'onde, susceptible de se propager dans le vide. Ces ondes peuvent être créées par des charges électriques en mouvement et peuvent mettre des charges électriques en mouvement.

Ces ondes électromagnétiques sont identifiés à la lumière. La lumière visible correspond à des ondes électromagnétique dont la longueur d'onde s'étend entre 400 et 800 nm, les ondes de longueurs d'ondes supérieures correspondent aux infrarouge puis aux ondes radio, les ondes de longueurs d'ondes inférieures correspondent aux ultraviolet, puis au rayon gamma.

Le débat scientifique de l'époque porte sur le milieu de propagation de ces ondes. L'air est le milieu de propagation des ondes sonnores, la mer est le milieu de propagation des vagues. Le milieu de propagation des ondes électromagnétiques sera appelé éther. La recherche des caractéristiques de l'éther et du mouvement de la matière à travers l'éther conduira à remettre en question le paradigme d'espace-temps newtonien.

%Si la lumière est une onde qui se propage comme une déformation d'un milieu, ce milieu doit exister et posséder certaines propriétées. L'étude de la lumière et de son interaction avec la matière conduit à donner à ce milieu des propriétés compliqués et relativement extravagante. L'idée de l'existence de ce milieu dénommé {\it éther} par les physicien du XIX$^\text{e}$ sera abandonné au XX$^\text{e}$.

Les équations de Maxwell ne sont pas compatibles avec la description de l'espace à trois dimension indépendant d'un temps absolu.

\subsection{Rayonnement du corps noir}
(On utilise ici le terme de rayonnement pour qualifié la lumière, la lumière est un rayonnement électromagnétique).

La thermodynamique impose aux corps en équilibre thermodynamique d'être à la même température. Le rayonnement est un système thermodynamique, comme les corps matériels (liquide, solide ou gaz). La matière et le rayonnement échange de l'énergie, l'équilibre thermodynamique entre le rayonnement et la matière peut être étudié. L'hypothèse d'un échange {\it continu} de l'énergie entre le rayonnement et la matière conduit à un écart flagrant avec l'expérience, cet écart est supprimé dans l'hypothèse d'un échange {\it discontinu} de l'énergie entre le rayonnement et la matière.

Les principes de la thermodynamique ne sont pas compatibles avec une nature continue de l'interaction lumière-matière. Ils sont compatibles avec une nature granulaire de la lumière.

\begin{center}

\end{center}

\fbox{%
\begin{minipage}{0.75\textwidth}

Considérons une enceinte isolé thermiquement contenant un gaz. Lorsque l'équilibre est ateint, les parois de l'enceinte sont à la même température que le gaz à l'intérieur.

L'expérience montre qu'un corps à une certaine température émet un rayonnement électromagnétique. Ainsi, dans l'enceinte il y a aussi ce rayonnement sans cesse émis par les parois et sans cesse absorbé par ces mêmes parois.


\end{minipage}
}


\subsection{Milieu de propagation de la lumière}

\begin{center}

\end{center}

