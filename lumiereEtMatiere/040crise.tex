\chapter{}

\section{La crise scientifique de la fin du XIX$^\mt{ème}$ siècle}

Les succès technologiques et théorique de la physique classique donnent une impression de conclusion, il semble qu'elle soit arrivée à bout des voiles d'isis. Les théories classiques mathématisent et expliquent la quasi-totalité des phénomènes physiques.

Quelques phénomènes restent inexpliqués, certaines théories montrent quelques incompatibilités, mais une grande partie des physiciens estiment que ce ne sont que des détails qui finiront par s'expliquer dans les paradigmes du XIX$^\mt{ème}$.

\begin{center}
Crise du corps noir
\end{center}

La thermodynamique suppose que les corps en équilibre thermodynamique d'être à la même température. 

\begin{center}
Loi de Snell-Descartes : mathématise le phénomène de réfraction, optique géométrique
\end{center}

Le modèle du rayon lumineux et la loi de Snell-Descartes constituent le paradigme de l'optique géométrique. Les progrès de la physique permettent d'observer le phénomène de diffraction puis d'interférence. Ces phénomènes sont décrits par le paradigme de l'optique ondulatoire

\begin{center}
Principe de fresnel puis de Huyghens : mathématise le phénomène de diffraction et d'interférence, la lumière est une onde
\end{center}





\subsection{}
\begin{center}
\end{center}

