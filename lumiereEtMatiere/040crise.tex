\chapter{}

\section{La crise scientifique de la fin du XIX$^\mt{ème}$ siècle}

Les succès technologiques et théorique de la physique classique donnent une impression de conclusion, il semble qu'elle soit arrivée à bout des voiles d'isis. Les théories classiques mathématisent et expliquent la quasi-totalité des phénomènes physiques.

Quelques phénomènes restent inexpliqués, certaines théories montrent quelques incompatibilités, mais une grande partie des physiciens estiment que ce ne sont que des détails qui finiront par s'expliquer à l'aide des théories classiques.

\subsection{Rayonnement du corps noir}

La physique statistique impose aux corps en équilibre thermodynamique d'être à la même température.

\begin{center}

\end{center}

\fbox{%
\begin{minipage}{0.75\textwidth}

Considérons une enceinte isolé thermiquement contenant un gaz. Lorsque l'équilibre est ateint, les parois de l'enceinte sont à la même température que le gaz à l'intérieur.

L'expérience montre qu'un corps à une certaine température émet un rayonnement électromagnétique. Ainsi, dans l'enceinte il y a aussi ce rayonnement sans cesse émis par les parois et sans cesse absorbé par ces mêmes parois.


\end{minipage}
}


\subsection{Milieu de propagation de la lumière}

Si la lumière est une onde qui se propage comme une déformation d'un milieu, ce milieu doit exister et posséder certaines propriétées. L'étude de la lumière et de son interaction avec la matière conduit à donner à ce milieu des propriétés compliqués et relativement extravagante. L'idée de l'existence de ce milieu dénommé {\it éther} par les physicien du XIX$^\text{e}$ sera abandonné au XX$^\text{e}$.

\begin{center}

\end{center}

