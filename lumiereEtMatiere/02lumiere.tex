
\section{La lumière}

\subsection{Physique hellénistique et romaine}

\vspace{0.24cm}
{\footnotesize
La lumière est pour Homère une forme du feu, commme elle le sera pour Héraclite, Empédocle et Platon. La nature ignée de la lumière est affirmée. Elle est lancée par ses sources comme un projectile. Elle est « pénétrante » ; son rayonnement est « infatiguable », fort et continu ; il est perceptible à de grande distances. Il n'est pas sur qu'Homère ait reconnu la nature de la lumière réfléchie. Les reflets du soleil sur les surfaces polies des armes sont présentés comme une lumière jaillissant du métal lui-même ; l'airain des armures lance des éclairs même pendant la nuit.

Les yeux des êtres vivant contiennent une matière ignée qui est rayonnée dans le regard, et la vision s'opère par ce rayonnement, à condition que les objets soit plongés dans la clarté du jour ou des astres de la nuit. Ce mécanisme de la perception visuelle est, de tout les traits de la physique archaïque d'Homère celui qui est promis à la carrière la plus longue

\begin{flushright}
Histoire de sciences PUF
\end{flushright}
 }
\vspace{0.31cm}
%(209-210)

L'oeil émet des rayons visuels qui se propagent en ligne droite à très grande vitesse. Ne sont visible que les corps lumineux ou qui sont éclairés par des rayons lumineux. Ces derniers, comme les visuels, se propagent en ligne droite, mais ne doivent pas être confondus avec eux.

Lois de la reflexion : rayon incident et rayon réfléchi sont dans un même plan, l'angle d'incidence est égale à l'angle de réflexion.

Lois de la refraction : rayon incident et rayon réfracté sont dans un même plan, l'angle d'incidence et l'angle de réflexion sont inégaux. 
%(350)

\subsection{Optique géométrique}
Le modèle du rayon lumineux et la loi de Snell-Descartes constituent le paradigme de l'optique géométrique.

\fbox{%
\begin{minipage}{0.85\textwidth}
Modèle du rayon lumineux : la lumière est constituée de rayon lumineux

Loi de Snell-Descartes : mathématise le phénomène de réfraction, optique géométrique
\end{minipage}
}

\subsection{Optique ondulatoire}
Les progrès de la physique permettent d'observer le phénomène de diffraction puis d'interférence. Ces phénomènes sont décrits par le paradigme de l'optique ondulatoire.

\fbox{%
\begin{minipage}{0.87\textwidth}
Modèle ondulatoire : la lumière est une onde

Principe de Huyghens-Fresnel : mathématise le phénomène de diffraction et d'interférence
\end{minipage}
}

