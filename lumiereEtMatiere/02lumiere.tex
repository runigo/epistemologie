\chapter{Lumière}

\section{Quelques paradigmes qui ont changé notre vision de la lumière}

\begin{center}
Modèle du rayon lumineux : la lumière se propage en ligne droite
\end{center}

\begin{center}
Loi de Snell-Descartes : mathématise le phénomène de réfraction, optique géométrique
\end{center}

Le modèle du rayon lumineux et la loi de Snell-Descartes constituent le paradigme de l'optique géométrique. Les progrès de la physique permettent d'observer le phénomène de diffraction puis d'interférence. Ces phénomènes sont décrits par le paradigme de l'optique ondulatoire

\begin{center}
Principe de fresnel puis de Huyghens : mathématise le phénomène de diffraction et d'interférence, la lumière est une onde
\end{center}





\subsection{}\subsection{}
\begin{center}
\end{center}

