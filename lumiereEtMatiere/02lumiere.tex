\chapter{Lumière}

\section{Quelques paradigmes qui ont changé notre vision de la lumière}

\subsection{Physique hellénistique et romaine}

La lumière est une forme du feu. Elle est lancée par ses sources comme un projectile. L'oeil contient une matière ignée qui est rayonnée dans le regard et la vision s'opère par ce rayonnement à condition que les objets soit plongés dans la clarté du jour ou des astres de la nuit.
%(209-210)
L'oeil émet des rayons visuels qui se propagent en ligne droite à très grandes vitesse. Ne sont visible que les corps lumineux ou qui sont éclairés par des rayons lumineux. Ces derniers, comme les visuels, se propagent en ligne droite, mais ne doivent pas être confondus avec eux.

Lois de la reflexion : rayon incident et rayon réfléchi sont dans un même plan, l'angle d'incidence est égale à l'angle de réflexion.

Lois de la refraction : rayon incident et rayon réfracté sont dans un même plan, l'angle d'incidence et l'angle de réflexion sont inégaux. 


\subsection{Optique géométrique}
Le modèle du rayon lumineux et la loi de Snell-Descartes constituent le paradigme de l'optique géométrique.

\fbox{%
\begin{minipage}{0.85\textwidth}
Modèle du rayon lumineux : la lumière se propage en ligne droite

Loi de Snell-Descartes : mathématise le phénomène de réfraction, optique géométrique
\end{minipage}
}
\begin{center}
\end{center}

\begin{center}
\end{center}

\subsection{Optique ondulatoire}
Les progrès de la physique permettent d'observer le phénomène de diffraction puis d'interférence. Ces phénomènes sont décrits par le paradigme de l'optique ondulatoire.

\fbox{%
\begin{minipage}{0.87\textwidth}
Modèle ondulatoire : la lumière est une onde

Principe de Huyghens-Fresnel : mathématise le phénomène de diffraction et d'interférence
\end{minipage}
}


\subsection{}\subsection{}
\begin{center}
\end{center}

