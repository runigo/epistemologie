\chapter{Physique Quantique}

\section{Onde et particule}
%%%%%%%%%%%%%%%%%%%%%%%%%%%%%%%%%
\subsection{Ondes}
Le phénomène d'interférence est caractéristique des ondes.
\begin{center}
Une onde est étendue dans l'espace.
\end{center}
\begin{center}
Une onde est une perturbation qui se propage.
\end{center}

\begin{itemize}[leftmargin=1cm, label=\ding{32}, itemsep=1pt]
\item {\bf Vagues : ondes de surface (perturbation de la surface entre l'air et l'eau).}
\item {\bf Son : perturbation de la pression de l'air}
\end{itemize}

\subsection{Particule}
Le phénomène de collision est caractéristique des particules.
\begin{center}
Une particule est localisée dans l'espace.
\end{center}
\begin{itemize}[leftmargin=1cm, label=\ding{32}, itemsep=1pt]
\item {\bf Billard :} solide macroscopique
\item {\bf Molécules d'un gaz :} corpuscules microscopique
\end{itemize}

\section{Révolution Quantique}
%%%%%%%%%%%%%%%%%%%%%%%%%%%%%%%%%%%%
%\subsection{Photon et électron}

\subsection{Quanton}

En 1900, les physiciens pouvaient dire : 

\begin{itemize}[leftmargin=1cm, label=\ding{32}, itemsep=3pt]
\item La lumière est une onde.
\item La lumière se comporte parfois comme une onde parfois comme des corpuscules.
\item La lumière est constituée de photons.
\end{itemize}

Les photons comme les électrons sont des quantons. Un quanton n'est ni une onde ni un corpuscule.

Le paradigme de la physique quantique unifie la lumière et la matière : les constituants de la matière et les constituants de la lumière ont un comportement analogue, un comportement quantique.

\subsection{Intrication}
L'expérience d'Alain Aspect à tranché un débat sur la quantique. Cette expérience cruciale ne vient pas contredire le paradigme de la physique quantique.


On pourrait dire "la lumière est constitué de quanton", "la lumière est un ensemble de quanton".
\subsection{}
\begin{center}
\end{center}

