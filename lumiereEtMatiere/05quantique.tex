\chapter{Physique Quantique}

\section{Onde et particule}
%%%%%%%%%%%%%%%%%%%%%%%%%%%%%%%%%
\subsection{Ondes}
Le phénomène d'interférence est caractéristique des ondes.
\begin{center}
Une onde est étendue dans l'espace.
\end{center}
\begin{center}
Une onde est une perturbation qui se propage.
\end{center}

\begin{itemize}[leftmargin=1cm, label=\ding{32}, itemsep=1pt]
\item {\bf Les vagues : ondes de surface (perturbation de la surface entre l'air et l'eau).}
\item {\bf Le son : perturbation de la pression de l'air}
\end{itemize}

\subsection{Particule}
Le phénomène de collision est caractéristique des particules.
\begin{center}
Une particule est localisée dans l'espace.
\end{center}
\begin{itemize}[leftmargin=1cm, label=\ding{32}, itemsep=1pt]
\item {\bf Le billard : solide macroscopique}
\item {\bf Les molécules d'un gaz : }
\end{itemize}

\section{Révolution Quantique}
%%%%%%%%%%%%%%%%%%%%%%%%%%%%%%%%%%%%
\subsection{Photon et électron}

\subsection{Quanton}

En 1900, les physiciens pouvaient dire : 

\begin{itemize}[leftmargin=1cm, label=\ding{32}, itemsep=3pt]
\item La lumière est une onde
\item La lumière se comporte parfois comme une onde parfois comme des corpuscules.
\end{itemize}

Les photons comme les électrons sont des quantons. Un quanton n'est ni une onde ni un corpuscule.

\subsection{}\subsection{}
\begin{center}
\end{center}

