
\section{Relativité restreinte}
%%%%%%%%%%%%%%%%%%%%%%%%%%%%%%%%%
En mécanique newtonienne, les corps se déplacent au cours du temps dans l'espace à trois dimensions. La position est mesurée par rapport à un corps de référence


%Une régle graduée permet de mesurer la longueur de la tige : A étant placé en face de la graduation 0, la graduation en face de B nous donne la longueur.

%\subsection{Paradigme de la relativité restreinte}
En physique relativiste, les évenements sont des points de l'espace-temps à 4 dimensions.
Plus précisemment, un évènement est modélisé par un vecteur d'un espace vectoriel de dimension 4.

Exemple : l'émission d'un photon par une source de lumière et sa réception par un détecteur définissent deux évènements.



Postulat : la vitesse de la lumière est une constante fondamentale.

Définition : la lumière est une onde électromagnétique obéissant aux équations de Maxwell, 

Conséquences : transformation de Lorentz, relativité de la simultanéité, 

On peut alors définir l'évenement E$_1$ : A est en face de C, et l'évenement E$_2$ : B est en face de D, et la définition devient : la tige et le tube sont de même longueur si E$_1$ et E$_2$ sont {\bf simultané} (ils ont lieu au même moment). Afin de vérifier cette simultanéité, on doit placer à chaque extrémités du tube, un observateur munit d'une horloge.

\subsection{Relativité de la simultanéité}

A présent la tige est en mouvement. Elle se déplace à la vitesse {\bf V}. Les deux observateurs sont en place (en C et en D) et relèvent le moment, l'instant (l'heure de leur horloge) des évènements E$_1$ et E$_2$. On compare les résultats observés et on constate que E$_1$ a lieu avant E$_2$ ! Autrement dit, il semble que la tige est plus courte que le tube. Ce phénomène est généralement nommé "contraction des longueurs" mais il vaut mieux l'appeler "relativité de la simultanéité" : 
\begin{center}
\fbox{%
\begin{minipage}{0.75\textwidth}
Deux évenements, espacé par une certaine distance, étant simultané dans un certain référentiel ${\mc R}$, ne sont pas simultanés dans un référentiel ${\mc R'}$ en mouvement par rapport à ${\mc R}$
\end{minipage}
}
\end{center}

Le formalisme de la relativité permet d'obtenir une expression de la contraction des longueurs : 
\[
l'=l\sqrt{1-v^2/c^2}
\]


\subsection{Chevauchons le photon}
Que se passerait-il si nous nous déplacions à la vitesse de la lumière ? que verrions nous ? Comment un photon nous apparaitrait-il ? Comment l'univers nous apparaitrait-il ? 
Explorons cette question à partir de la formule exprimant la contraction des longueurs
\[
l'=l\sqrt{1-v^2/c^2}
\]
Lorsque nous sommes sur le photon, nous observons l'univers "contracté". Tellement contracté que celui ci est un plan perpendiculaire à la direction de propagation du photon. En particulier, les points d'émission et de réception du photon sont confondus. Dans le référentiel du photon la distance parcourue est nulle.



\[
t' = \frac{t}{\sqrt{1-v^2/c^2}}
\]


\subsection{Embarquons dans une super fusée}

Imaginons que nous ayons une super fusée, permettent d'atteindre une vitesse proche de la lumière, et observons la lumière issue du soleil. Et bien nous observerions cette lumière se propager à la vitesse c ! la même que si l'on était resté sur Terre !

%Autrement dit, nous sommes Achile et nous voyons la tortue gagner la course.
On a beau courir après la lumière, celle-ci a toujours cette vitesse c par rapport à nous. On a beau accélérer, augmenter notre vitesse, la lumière aura toujours cette vitesse c par rapport à nous.

\begin{itemize}[leftmargin=1cm, label=\ding{32}, itemsep=1pt]
\item {\bf :}
\end{itemize}

