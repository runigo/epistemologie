\chapter{Interaction lumière -- matière}

\section{Quelques définitions}
\subsection{Milieu opaque}
Un objet, mat et opaque, reçoit de la lumière. Une partie de la lumière va être absorbée, une partie va être  diffusée. L'énergie de la lumière absorbé est convertie en énergie thermique. La perception de la couleur de l'objet est donnée par la lumière diffusée.

\begin{tikzpicture}
\definecolor{green}{rgb}{0.0,0.50,0.0}
\tikzset{>={Straight Barb[angle'=80, scale=1.1]},
    photon/.style={decorate, decoration={snake, segment length=3pt, amplitude=0.8pt}, draw=red}}

\draw[->] (0, 3) -- ++(0.5, -1);\draw[] (0.5, 2) -- ++(0.5, -1);
\draw[->] (0.5, 3) -- ++(0.5, -1);\draw[] (1, 2) -- ++(0.5, -1);
\draw[->] (1, 3) -- ++(0.5, -1);\draw[] (1.5, 2) -- ++(0.5, -1);
\draw (0, 3.3) node {Lumière incidente};

\draw[thick,->,green] (1.5,1) -- ++(0, 1);\draw[thick,green] (1.5, 2) -- ++(0, 0.5);
\draw[thick,->,green] (1.5,1) -- ++(0.65, 0.85);\draw[thick,green] (2.15, 1.85) -- ++(0.325, 0.425);
\draw[thick,->,green] (1.5,1) -- ++(1.25, 0.5);\draw[thick,green] (2.75, 1.5) -- ++(0.625, 0.25);

\draw[thick,->,green] (1.5,1) -- ++(-0.65, 0.85);\draw[thick,green] (0.85, 1.85) -- ++(-0.325, 0.425);
\draw[thick,->,green] (1.5,1) -- ++(-1.25, 0.5);\draw[thick,green] (0.25, 1.5) -- ++(-0.625, 0.25);
\draw (3.3, 2.7)[green] node {Lumière diffusée};

\shade[thin, left color=green!20, right color=green!60, draw=none,
  shift={(0.2, 0.7)},scale=0.5]
  (0, 0) to[out=10, in=140] (3.3, -0.8) to [out=60, in=190] (5, 0.5)
    to[out=130, in=60] cycle;

\draw[->,photon]  (1.8,1.3) -- ++(0, -0.5);
\draw[->,photon]  (1.4,1.3) -- ++(0, -0.5);
\draw (1.5, 0.25)[red] node {Lumière absorbée};
\end{tikzpicture}

%Énergie et longueur d'onde

\subsection{Milieu transparent}
Lumière incidente, lumière réfléchie, lumière réfractée

\subsection{}

\section{Électromagnétisme}
Les équations de maxwell-Lorentz ont été découvertes grâce à l'étude de la matière (électrostatique, magnétisme, mouvement des charges électriques) et à l'introduction des champs (E et B). Elles ont été écrites sans aucunes références aux phénomènes lumineux. L'étude de la matière a conduit aux équations de maxwell-Lorentz, l'étude de la lumière a conduit aux paradigmes de l'optique géométrique et de l'optique ondulatoire.

Fin {\footnotesize XIX}$^\text{e}$e, les physiciens montrent que les équations de maxwell-Lorentz conduisent à l'existence d'onde, susceptible de se propager dans le vide. 
\begin{center}
\end{center}

\section{Thermodynamique}
La thermodynamique est la science de la température et de la chaleur.

\fbox{%
\begin{minipage}{0.87\textwidth}

Loi de la conservation de l'énergie, loi de l'accroissement de l'entropie.
\end{minipage}
}
\subsection{}
\subsection{}

\subsection{}
\begin{center}
\end{center}

